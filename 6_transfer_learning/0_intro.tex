\chapter{Dynamic Classification Ensembles for Handling Imbalanced
Multiclass Drifted Data Streams}
\label{chapter:6_transfer_learning}
Transfer learning is a powerful approach for addressing the challenges of dynamic data streams and concept drift. Its primary objective is to improve learning performance in a target domain by utilizing knowledge from one or more source domains. The source domain typically comprises data or tasks with abundant labeled information, whereas the target domain often faces constraints such as limited or unlabeled data. This distinction underscores the significance of transfer learning in bridging knowledge gaps and enhancing learning in resource-constrained environments \cite{pan2009survey, wang2019characterizing}.  
The field of transfer learning focuses on maximizing positive knowledge transfer while minimizing negative transfer. To achieve this, researchers have developed strategies such as instance re-weighting and feature matching to reduce the domain gap \cite{zadrozny2004learning, cortes2008sample, pan2010domain, sun2016return}. Additionally, mitigating negative knowledge transfer often involves down-weighting irrelevant data sources to prevent the incorporation of misleading information \cite{wang2019characterizing}. While much of the research has been dedicated to static environments where data distributions remain stable, real-world scenarios like financial forecasting, energy demand prediction, and climate analysis often feature dynamic environments. These scenarios present the concept drift problem, where evolving data distributions challenge traditional models \cite{li2015learning, cao2019learning}.  
In dynamic environments, transfer learning must go beyond static methodologies to adapt continuously to changing data. A notable technique is Dynamic Ensemble Selection (DES), which employs multiple classifiers to make predictions. DES dynamically evaluates and selects the most competent classifiers for the prevailing data conditions, ensuring adaptability and resilience against concept drift. It excludes underperforming classifiers while progressively improving overall performance \cite{cruz2017meta, jackowski2014improved, kuncheva2000clustering}.  
Another critical approach is Adaptive Windowing (ADWIN), which detects concept drift by monitoring statistical changes, such as mean or variance, within dynamically adjusted sliding windows. When substantial shifts occur, ADWIN reconfigures the ensemble by retraining or introducing classifiers suited to the updated data distribution \cite{madkour2023historical}. Similarly, the Streaming Ensemble Algorithm (SEA) manages real-time data streams by continuously adapting classifier ensembles, maintaining robustness and performance even as data conditions shift and new classes emerge \cite{gama2004learning, adams2023explainable, madkour2023historical}.  
This study introduces efficient algorithms for classifying data streams in real-time, specifically addressing the challenges posed by heterogeneous transfer learning in non-stationary environments. The proposed framework integrates DES, ADWIN, and SEA to tackle dynamic data streams effectively. Its objectives include achieving high classification performance, adapting to evolving data distributions, and minimizing computational complexity. By leveraging these advanced techniques, the framework provides a robust, scalable, and efficient solution to the challenges of heterogeneous transfer learning, ensuring reliable performance in dynamic real-world scenarios.
The subsequent sections of this paper adhere to a well-structured organization. Section \ref{sec:6_2_motivation} presents the motivations and contributions. Section \ref{sec:proposed_methodology} introduces the third proposed approach, providing intricate explanations of its constituents, including dynamic classifier ensembles, concept drift handling, and heterogeneous multisource transformation. Section \ref{sec:4_5_Expsetup} outlines the experimental setup and presents the results, providing details regarding the employed datasets, evaluation metrics, and procedures. Finally Section \ref{sec:6_summary} presents a summary of this chapter.

\section{Motivations and Contributions of this Chapter} \label{sec:6_2_motivation}
This paper presents a novel contribution to real-time streaming scenarios by addressing the challenges of heterogeneous multisource streams, with key contributions that can be summarized as follows:
\begin{enumerate}[nosep]
  \setlength{\itemsep}{0pt}
  \setlength{\parskip}{0pt}
  \item The primary innovation involves incorporating a concept drift detection method in conjunction with an ensemble classifier, enabling real-time adaptation and refinement of the third proposed approach in response to transfer learning in non-stationary environments. This methodology ensures continuous evolution of the classification model in accordance with the changing data landscape.
  \item The second significant advancement is the introduction of a precise weighting method to assess the significance of each local classifier within the ultimate classifier.
 \item The third significant contribution is the development of an innovative approach that employs the eigenvector technique to facilitate the transfer of knowledge from heterogeneous source domains to the target domain.
  \end{enumerate} 
 
   

