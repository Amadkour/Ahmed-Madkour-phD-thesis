\chapter{Dynamic Classification Ensembles for Handling Imbalanced
Multiclass Drifted Data Streams}
\label{chapter:6_transfer_learning}

Transfer learning plays a pivotal role in addressing the intricate challenges posed by dynamic data streams and inherent concept drifts. Transfer learning aims to enhance a model's learning performance within a target domain by leveraging the knowledge gleaned from the source domains \cite{pan2009survey, wang2019characterizing}. This research domain focuses extensively on addressing target tasks that grapple with limited or unlabeled data, with a dual emphasis on amplifying positive knowledge transfer and alleviating negative knowledge transfer \cite{wang2019characterizing}. To facilitate the transfer of valuable knowledge, researchers have developed diverse techniques, including methodologies aimed at reducing the domain gap, such as instance re-weighting \cite{zadrozny2004learning, cortes2008sample, pan2010domain} and feature matching \cite{sun2016return, pan2010domain}. Conversely, strategies for mitigating negative knowledge transfer often involve down weighting irrelevant data sources \cite{wang2019characterizing}.
Although a substantial body of transfer learning research has focused on static environments, where data in each domain are assumed to conform to the same distribution, real-world scenarios, including financial data analysis, energy demand prediction, and climate data analysis, often involve dynamic environments. Within these dynamic contexts, the concept drift problem \cite{li2015learning, cao2019learning} emerges as data distributions evolve over time. Dynamic environment transfer learning requires continuous adaptation to concept drift and adaptive utilization of valuable knowledge from source domains within each distinct environment. These new challenges in transfer learning, particularly within dynamic environments, represent a profound departure from traditional transfer learning paradigms that do not inherently address concept drift and the dynamic nature of evolving data.
To overcome these challenges, Dynamic Ensemble Selection (DES) \cite{cruz2017meta, jackowski2014improved, kuncheva2000clustering}, ADWIN, and Streaming Ensemble Algorithm (SEA) have become prominent in the scientific literature \cite{gama2004learning, adams2023explainable, madkour2023historical} Dynamic ensemble selection employs multiple classifiers within the realm of machine learning to make collective predictions or classifications, and leverage data characteristics. Its key feature is its dynamic adaptability, as it continuously evaluates the performance of individual classifiers and selects a subset that demonstrates the highest competence within the prevailing data conditions. This adaptability empowers the ensemble to progressively enhance its performance by incorporating best-fitting classifiers for the existing data context. Moreover, ineffective classifiers can be excluded from the ensemble in scenarios marked by concept drift, thereby protecting the overall performance from detrimental effects.
Adaptive Windowing (ADWIN) is another widely used method designed to address the challenges of concept drift \cite{madkour2023historical}. Concept drift refers to the phenomenon in which the statistical properties of data change over time, leading to a decline in the learning algorithm performance. ADWIN monitors incoming data by maintaining sliding windows of variable sizes and evaluating statistical attributes, such as the mean or variance within these windows. Upon detecting substantial shifts or drifts, the ADWIN initiates adjustments to the ensemble or classifier configuration. These adjustments may involve retraining classifiers with fresh data, or incorporating new classifiers that are better suited to the updated data distribution. Additionally, the Streaming Ensemble Algorithm (SEA) is an integral component for addressing these dynamic challenges \cite{gama2004learning, adams2023explainable, madkour2023historical}. SEA is specifically designed to manage data streams and inherent concept drifts. This is achieved by adapting the ensemble of classifiers in real time as new data arrives. SEA's adaptability of SEA ensures that the ensemble remains robust and accurate even in the face of shifting data conditions and the emergence of new classes.
This study proposes efficient algorithms for classifying data streams in real-time scenarios, focusing on addressing the challenges posed by heterogeneous transfer learning in data distributions. The goal is to develop a novel algorithm (Heterogeneous Transfer Learning) that can effectively handle non-stationary data streams, achieve high classification accuracy, and minimize computational complexity. 
The subsequent sections of this paper adhere to a well-structured organization. Section 2 comprehensively reviews the relevant literature on concept drift and transfer learning. Section 3 introduces the proposed approach, providing intricate explanations of its constituents, including dynamic classifier ensembles, concept drift handling, and heterogeneous multisource transformation. Section 4 outlines the experimental setup and presents the results, providing details regarding the employed datasets, evaluation metrics, and procedures. Finally, in Section 5, we offer concluding remarks, summarizing the key findings, discussing their implications, and proposing potential avenues for future research.
\section{Motivations and Contributions} \label{sec:4_2_motivation}
This paper presents a novel contribution to real-time streaming scenarios by addressing the challenges of heterogeneous multisource streams, with key contributions that can be summarized as follows:
\begin{enumerate}[nosep]
  \item The primary innovation involves incorporating a concept drift detection method in conjunction with an ensemble classifier, enabling real-time adaptation and refinement of the proposed approach in response to transfer learning in non-stationary environments. This methodology ensures continuous evolution of the classification model in accordance with the changing data landscape.
  \item The second significant advancement is the introduction of a precise weighting method to assess the significance of each local classifier within the ultimate classifier.
 \item The third significant contribution is the development of an innovative approach that employs the eigenvector technique to facilitate the transfer of knowledge from heterogeneous source domains to the target domain.
  \end{enumerate} 
 
   

