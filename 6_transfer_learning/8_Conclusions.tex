%%%%%%%%%%%%%%%%%%%%%%%%%%%%%%%%%%%%%%%%%%%%%%%
%
%   Conclusions
%
%%%%%%%%%%%%%%%%%%%%%%%%%%%%%%%%%%%%%%%%%%%%%%%

%%%%%%%%%%%%%%%%%%%%%%%% Here it is needed to re-frame the conclusions

\section{Conclusion and Future Works}
\label{sec:4_8_Conclusions}

Our study in this chapter presents a comprehensive methodology designed to facilitate incremental learning in drifted streams, with a particular focus on addressing the challenges associated with imbalanced data streams including minority and overlapping classes. The proposed methodology integrates our proposed oversampling method, concept drift detection strategies, and Dynamic Ensemble Selection (DES) to select the most suitable ensemble classifier. Extensive experimentation across various datasets, including benchmark datasets, real application streams, and synthetic data, validated the effectiveness of this contribution. The critical role of the concept drift detector in our approach lies in its capacity to promptly detect concept drifts, allowing our methodology to adapt by training new base classifiers to maintain relevance and performance in real-time scenarios. Oversampling techniques were utilized to mitigate the minority class problem, and the KNN algorithm was employed to prevent the generation of overlapped class instances. The DES technique was utilized to intelligently select the best base classifiers to ensure optimal performance. Our evaluation approach employs various performance measures, showcasing its proficiency in addressing multiclass imbalanced stream problems, particularly its exceptional performance in classifying data streams with evolving class distributions. In addition to its performance and performance, our approach has several advantages, including adaptability, efficiency, and scalability. Dynamic model updates driven by incoming data instances enable continuous adaptation to changing data distributions, thereby ensuring reliability and relevance in real-time scenarios. However, it is essential to recognize the specific limitations of the proposed approach. One limitation is the extended time required to generate nonoverlapping synthetic instances. Moreover, the efficacy of our contribution depends on the performance of the MLSMOTE and MLSOL techniques. Consequently, future research efforts should be directed toward enhancing our contributions. Potential avenues for improvement include the development of advanced oversampling techniques to avoid generating overlapping synthetic instances. Additionally, meta-learning methods have been explored to calculate imbalanced multiclass ratios and minority classes.