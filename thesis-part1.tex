% ----------------------------------------------------------------------
% ----------------------------------------------------------------------
%              Latex PhD template for the University of Deusto
% ----------------------------------------------------------------------


%: Style file for Latex
% Most style definitions are in the external file PhDthesisPSnPDF.
% In this template package, it can be found in ./Latex/Classes/
\documentclass[twoside,12pt]{Latex/Classes/PhDthesisPSnPDF}
\usepackage{enumitem, kantlipsum}
\usepackage{mathtools}
\usepackage{amssymb}
\usepackage[warn]{textcomp}
\usepackage{amsbsy}
\usepackage{pifont}
\usepackage{numprint}
\usepackage{lscape}
\usepackage{booktabs}
\usepackage{tikz}
\usepackage{notoccite}
\usepackage[export]{adjustbox}
\usepackage[ruled,vlined]{algorithm2e}
\usepackage{amsmath}
\usepackage{threeparttable, tablefootnote}
\usepackage{tabularx}
\usepackage[table]{xcolor}
\usepackage{array}

\DeclarePairedDelimiter\ceil{\lceil}{\rceil}
\DeclarePairedDelimiter\floor{\lfloor}{\rfloor}
\newcommand\scalemath[2]{\scalebox{#1}{\mbox{\ensuremath{\displaystyle #2}}}}
\newcommand{\cmark}{\ding{51}}
\newcommand{\xmark}{\ding{55}}
\newcommand{\ssep}{;}
\newcommand{\RFCOM}{\raisebox{2pt}{\tikz{\draw[-,black!40!black,solid,line width = 0.9pt](0,0) -- (5mm,0);}}}
\newcommand{\RFSM}{\raisebox{2pt}{\tikz{\draw[-,black!40!black,dashed,line width = 0.9pt](0,0) -- (5mm,0);}}}

\newcommand*\circled[1]{\tikz[baseline=(char.base)]{
            \node[shape=circle,draw,inner sep=1pt] (char) {#1};}}

\newcommand\diag[4]{%
  \multicolumn{1}{p{#2}|}{\hskip-\tabcolsep
  $\vcenter{\begin{tikzpicture}[baseline=0,anchor=south west,inner sep=#1]
  \path[use as bounding box] (0,0) rectangle (#2+2\tabcolsep,\baselineskip);
  \node[minimum width={#2+2\tabcolsep},minimum height=\baselineskip+\extrarowheight] (box) {};
  \draw (box.north west) -- (box.south east);
  \node[anchor=south west] at (box.south west) {#3};
  \node[anchor=north east] at (box.north east) {#4};
 \end{tikzpicture}}$\hskip-\tabcolsep}}


%: Macro file for Latex
% Macros help you summarise frequently repeated Latex commands.
% Here, they are placed in an external file /Latex/Macros/MacroFile1.tex
% An macro that you may use frequently is the figuremacro (see introduction.tex)
\include{Latex/Macros/Macros}

%: ----------------------------------------------------------------------
%:                  TITLE PAGE: name, degree,..
% ----------------------------------------------------------------------
% 201803015 LED: I create some metadata to have a unique source for these fields
\def\myauthor{Ahmed Hamdy Madkour} % Author
%\def\mytitle{Pedestrian Localization using Wrist-Worn Smart Devices} % title
\def\mytitle{Learning in Non-stationary Envioronment } % title
\def\myadvisor{Dr. Amgad Monir and Prof. Hatem Mohamed} % Supervisor

% if output to PDF then put the following in PDF header
\ifpdf
    \pdfinfo { /Title  (\mytitle)
               /Creator (TeX)
               /Producer (pdfTeX)
               /Author (\myauthor)
               /CreationDate (D:201803150940)  %format D:YYYYMMDDhhmmss
               /ModDate (D:YYYYMMDDhhmm)
               /Subject (Pedestrian Localization)
               /Keywords (Auto machine learning, Concept drift, Imbalanced Stream, Transfer Learning, Emerging new Classes, and Dynamic Ensemble Selection.) }
    \pdfcatalog { /PageMode (/UseOutlines)
                  /OpenAction (fitbh)  }
    % 201803015 LED: \pdfinfo does not work in combination with hyperref
    \hypersetup{
    	pdfinfo={
    			Title={\mytitle},
                Author={\myauthor},
                Creator={TeX},
                Producer={pdfTeX},
                CreationDate={D:201803150940},
                ModDate={D:\pdfdate},
                Subject={Transportation},
                Keywords={Auto machine learning, Concept drift, Imbalanced Stream, Transfer Learning, Emerging new Classes, and Dynamic Ensemble Selection.}
    	}
}
\fi


% ----------------------------------------------------------------------

% turn of those nasty overfull and underfull hboxes
\hbadness=10000
\hfuzz=50pt



\begin{document}

%\selectlanguage{british}
\selectlanguage{english}

% sets line spacing
\renewcommand\baselinestretch{1.2}
\baselineskip=18pt plus1pt

% As abstract contains various languages we set the main language again
%\selectlanguage{british}
\selectlanguage{english}


%: ----------------------- contents ------------------------

\setcounter{secnumdepth}{5} % organisational level that receives a numbers
\setcounter{tocdepth}{5}    % print table of contents for level 3


%%You can also add extra lines to the ToC or to force extra unnumbered section headings to be included. For example, if you wanted to add an entry called Preface, and you didn't want the Preface to be numbered, you'd use these commands:
%\ subsection*{Preface}
%\addcontentsline{toc}{subsection}{Preface}

% \tableofcontents            % print the table of contents
% % levels are: 0 - chapter, 1 - section, 2 - subsection, 3 - subsection

% %: ----------------------- list of figures/tables ------------------------

%: ----------------------- glossary ------------------------
% 20180315 LED: for WinEdt it is necessary to install the Nomenclature.zip that is available in the WindEdt website

% Tie in external source file for definitions: /0_frontmatter/glossary.tex
% Glossary entries can also be defined in the main text. See glossary.tex

%\begin{multicols}{2} % \begin{multicols}{#columns}[header text][space]
%\begin{footnotesize} % scriptsize(7) < footnotesize(8) < small (9) < normal (10)

% Parts of the thesis are included below. Rename the files as required.
% But take care that the paths match. You can also change the order of appearance by moving the include commands.

% 20180315 LED:
%\include consideres the file as a section, introducing a page break after it and placing all the figures inside that section;
%\input only concatenates the content of the files, without assuming anything else

%: ----------------------- introduction ------------------------
%\part{Proposal}
% introduction

% this file is called up by thesis.tex
% content in this file will be fed into the main document


% this file is called up by thesis.tex
% content in this file will be fed into the main document

%: ----------------------- introduction file header -----------------------


\begin{savequote}[50mm]
Success is the sum of small efforts, repeated day-in and day-out.
\qauthor{Robert Collier}
\end{savequote}

\chapter{Introduction}
\label{cha:1_Introduction}

% the code below specifies where the figures are stored
\ifpdf
    \graphicspath{{1_introduction/figures/PNG/}{1_introduction/figures/PDF/}{1_introduction/figures/}}
\else
    \graphicspath{{1_introduction/figures/EPS/}{1_introduction/figures/}}
\fi


%-------------------------------------------------------------------------
%Chapter 1 contents:
%- Motivation of the research field: Context-aware systems -> LBS -> GNSS limitation -> Positioning techniques -> DR -> inertial PDR -> inertial PDR + wearables
%- Problem identification: smartphone not a wearable -> potentiality of wrist-worn wearables -> Problem: no wrist-worn PDRS
%- Goal of the thesis: tackle the problem -> how? Splitting it into sub-problems
%- Structure of the thesis
%-------------------------------------------------------------------------

Governments and companies are producing vast streams of data and require effective data analytics and machine learning methods to assist in making predictions and decisions promptly. One crucial aspect is the machine learning pipeline, which involves training a prepared dataset to construct a model and subsequently utilizing this model to predict new instance outputs. As depicted in Fig. (1.1), the process entails fetching historical data from the database during the training phase to construct the machine learning model. Then, the system can input new instances from the database to predict the output.

\begin{figure}[!ht]
    \centering
    \includegraphics[width=.9\textwidth]{1_introduction/figures/PNG/machine_flow.png}
    \caption{The research methodology of the thesis.}
    \label{ch1:research-emthodo}
\end{figure}

Nevertheless, when endeavoring to forecast outcomes for fresh instances sourced from an alternative database, as illustrated in Fig. (1.2), there frequently emerges a conspicuous decline in accuracy. This disparity accentuates the imperative for model developers to intervene and rectify the issue. Addressing this, developers must adjust and retrain the model utilizing datasets from the new environment to ameliorate accuracy. This iterative process aims to refine the model's precision and ensure its efficacy across diverse contexts, thereby bolstering the reliability of decision-making and predictive capabilities. To confront this challenge, the field of auto machine learning endeavors to facilitate online updates to the model without necessitating direct intervention from developers for modification.

\begin{figure}[!ht]
    \centering
    \includegraphics[width=.8\textwidth]{1_introduction/figures/PNG/wrong_machine_flow_1.png}\\
    (a) \\
    \includegraphics[width=.8\textwidth]{1_introduction/figures/PNG/wrong_machine_flow_2.png}\\
    (b)
    \caption{The research methodology of the thesis.}
    \label{ch1:research-emthodo}
\end{figure}



In recent years, the surge in high-speed data streams has posed notable challenges for machine learning models, particularly in the context of streaming data analysis. These data streams, characterized by continuous, dynamic, and high-volume data arrivals, demand adaptive learning algorithms that can effectively cope with their evolving nature [1] [2] [3]. Within these evolving data environments, two paramount challenges have emerged: concept drift, class imbalance, Emerging new class, and heterogenous transfer learning.

Concept drift, a phenomenon defined by the evolving statistical properties of a data generation process over time [4] [5]. introduces a dynamic element to the data, necessitating continuous adaptation of machine learning models. This shift can manifest as changes in underlying concepts, relationships between variables, or alterations in data distribution. Traditional models trained on historical data may suffer diminished accuracy or become inadequate when confronted with new data influenced by concept drift, highlighting the need for effective concept drift detection mechanisms. Addressing concept drift involves the utilization of concept drift detectors, which are methods capable of identifying changes in data stream distributions. These detectors rely on information related to classifier performance or incoming data items to signal the need for model updates, retraining, or even replacing the old model with a new one. The dynamic nature of concept drift necessitates ongoing monitoring and adaptation to maintain the model's efficacy.

Data streams also present challenges related to class imbalance, a condition characterized by uneven distribution among different classes [6] [7]. This scenario, especially prevalent in multi-class settings, poses a significant challenge for traditional classifiers. The risk of misclassifying minority class samples due to their limited representation demands specialized techniques to ensure accurate classification without sacrificing the performance of the majority class [8] [9] [10] [11]. To tackle class imbalance, three primary methods are commonly employed: sampling methods, algorithm adaptation methods, and hybrid methods. Sampling methods involve undersampling the majority class or oversampling the minority class to balance class distribution. Algorithm adaptation methods modify existing algorithms to handle imbalanced data [12] [13] [14], while hybrid methods combine data preprocessing with classification techniques, often utilizing ensemble classifiers to effectively mitigate class imbalance and enhance overall classifier performance [15] [16] [17] [18].

Another challenge arising in the context of class imbalance is class overlap, where instances from different classes share the same region in data space [17] [18]. This overlap complicates the task of distinguishing between representative instances of different classes, leading to performance challenges for traditional classifiers referred to as overlapping problems. Recent research introduces class-overlap undersampling methods to address this issue, leveraging local similarities among minority instances to identify potentially overlapping majority instances.

Therefore, both class imbalance and class overlap present significant hurdles in the realm of data stream analysis. Consequently, addressing class imbalance has become crucial in multi-class learning, leading to research efforts focusing on both concept drift and class imbalance challenges. Researchers have explored dynamic ensemble selection (DES) and multi-class oversampling techniques to tackle these issues. Dynamic classifier ensembles offer a unique ability to adapt their composition based on data characteristics, making them valuable in situations with evolving data conditions [19]. Researchers focus on the overproduce-and-select approach for classifier ensemble selection methods. The objective of classifier ensemble selection is to choose an optimal subset of classifiers from a larger ensemble. The selection process is guided by various criteria, including individual performance measures, diversity metrics, meta-learning techniques, and performance estimation approaches. This optimization is particularly important in scenarios where a balance between accuracy and computational resource constraints is critical. There are two distinct approaches: static and dynamic selection. Static selection involves assigning classifiers to predefined feature partitions, while dynamic selection adaptively selects classifiers based on their competency [20]. Dynamic selection offers two choices: individual models, known as Dynamic Classifier Selection (DCS), and ensemble models, called Dynamic Ensemble Selection (DES). DCS algorithms enable the selection of the most appropriate classifier for each data point based on its local competencies. In contrast, DES focuses on selecting the optimal classifiers for each instance based on their competence within localized regions [21] [22] [23] Competency assessment relies on a dynamic selection dataset (DSEL) containing labeled samples. Moreover, innovative techniques like the Randomized Reference Classifier introduce randomness into class supports to enhance adaptability in addressing challenges related to imbalanced data.

Additionally, transfer learning assumes a pivotal role in addressing the intricate challenges posed by dynamic data streams and inherent concept drift. This domain of research focuses on enhancing a model's learning performance within a target domain by harnessing knowledge gleaned from source domains [4] [5].Techniques in transfer learning include reducing domain gaps through instance re-weighting and feature matching, along with strategies to mitigate negative knowledge transfer by down-weighting irrelevant source data.

Lastly, in the study, the focus extends to the specific scenario of Streams with Emerging New Classes (SENC). This refers to situations where new classes, not present during the initial training of a learning model, emerge in the data stream. Traditional learning approaches, designed for fixed or predefined class distributions, face challenges in effectively recognizing and adapting to these novel classes in real-time. The need for adaptive learning mechanisms that can handle the emergence of new classes underscores the complexity of real-world data stream scenarios.
     

In this chapter, the motivation for this research along with the research questions
that naturally arise are discussed in  Section~\ref{sec:1_1_motivation}. After this, the objectives and contributions are presented in Sections \ref{sec:1_2_opportunity} and \ref{sec:1_3_1_goal}, respectively. Next, the
research methodology is summarised in Section \ref{sec:1_3_automl_and_tf}. Finally,
the research context and the outline of this thesis are presented in Sections \ref{ch1:research-context} and
\ref{sec:1_3_2_DissertationStructure}, respectively.
\section{Challenges}
\label{sec:1_introduction_challange}
Within the swiftly evolving realm of machine learning, the proliferation of high-speed data streams presents a significant hurdle — the proficient management of non-stationary data environments. This challenge is marked by dynamic fluctuations in statistical attributes, fundamental concepts, and data distributions over time. The crux of the issue emerges as models, initially trained on historical data, experience a decline in accuracy when confronted with new data influenced by concept drift. This includes scenarios such as:
\begin{itemize}
    \setlength{\itemsep}{0pt}
    \setlength{\parskip}{0pt}
    \item Multi-class imbalanced data streams
    \item Overlapping classes
    \item Emergence of new classes
    \item Heterogeneous transfer learning
\end{itemize}
However, addressing these challenges yields substantial benefits. By effectively managing non-stationary data environments, organizations can enhance the adaptability and resilience of their machine learning systems. This enables more robust decision-making processes, improved predictive capabilities, and heightened performance across diverse contexts. Moreover, it fosters innovation and agility, empowering organizations to stay ahead in dynamic markets and evolving scenarios.

\section{Thesis Motivations}
\label{sec:1_introduction_motivation}

The swift progress in machine learning, particularly in real-time data streams, introduces new challenges that demand innovative solutions. This thesis seeks to address the limitations of traditional models in dynamic, non-stationary data environments. The motivations behind this research are as follows:
\begin{itemize}
    \setlength{\itemsep}{0pt}
    \setlength{\parskip}{0pt}
    \item \textbf{Adapting to Emerging Classes in Real-Time:} New classes within data streams can cause a decline in accuracy. The goal is to develop adaptive mechanisms that quickly incorporate new classes, maintaining system relevance.
    \item \textbf{Proactive Management of Concept Drift:} Concept drift degrades model performance as data properties change. This research focuses on strategies to detect and manage concept drift to maintain model accuracy.
    \item \textbf{Dynamic Optimization of Classifier Ensembles:} To address emerging classes and shifting distributions, the goal is to develop techniques for the real-time optimization of classifier ensembles.
    \item \textbf{Addressing Multi-Class Imbalance:} Imbalanced multi-class distributions often lead to biased classification. The thesis aims to create methods that address class imbalances dynamically, ensuring fair classification.
    \item \textbf{Enhancing Transfer Learning in Non-Stationary Environments:} Transfer learning can suffer from negative transfer in non-stationary settings. The research focuses on frameworks that minimize negative transfer and enhance knowledge sharing across diverse domains.
\end{itemize}

\section{Thesis Objectives}
\label{sec:1_introduction_objectives}
The thesis objectives for our research stems from the imperative need to address the following key motivations:
\begin{itemize}
    \item \textbf{Objective 1}. Thandling Imbalanced Multiclass Drifted Data and overlapping classes streams.
    \item \textbf{Objective 2}. Addressing Emerging New Classes in Incremental Streams via Concept Drift and K-means Techniques.
    \item \textbf{Objective 3}. Addressing Heterogeneous Transfer Learning Problem in data streams via Concept Drift and Eigenvector Techniques.
\end{itemize}

\section{Thesis Questions}
\label{sec:1_introduction_questions}

\begin{itemize}
    % \setlength{\itemindent}{-.5in}
       \item $\pmb{Q_1}$. What is the impact of imbalanced stream on the performance of ML models in non-stationary Environment?
        \item $\pmb{Q_2}$. What is the impact of reducing overlapping classes stream on the performance of ML models in non-stationary environment?
        \item $\pmb{Q_3}$.How does the emergence of new classes in data streams affect the stability and performance of ML models?
        \item $\pmb{Q_4}$. how to employee concept drift to solve the emerging new classes problem?
        \item $\pmb{Q_5}$. How does the Heterogeneous sources in data streams affect the stability and performance of Transfer Learning Technique?
        \item $\pmb{Q_6}$. What approaches or techniques can be employed in heterogeneous transfer learning to facilitate knowledge transfer across diverse sources and domains?
    \end{itemize}
    
\section{Thesis Contributions}
\label{sec:1_introduction_contribution}

Our research focuses on developing advanced frameworks to manage non-stationary data streams with high accuracy and minimal computational complexity. The key contributions include:

\begin{itemize}
    \setlength{\itemsep}{0pt}
    \setlength{\parskip}{0pt}
    \item \textbf{Concept Drift Detection and Ensemble Classifier:} Integrating concept drift detection with ensemble classifiers for real-time adaptation in transfer learning.
    \item \textbf{Dynamic Classifier Ensemble for Emerging Classes:} A framework that dynamically adjusts classifiers to address emerging class issues in non-stationary streams.
    \item \textbf{Precise Weighting for Local Classifiers:} A novel method to refine local classifier contributions, enhancing overall ensemble accuracy.
    \item \textbf{Eigenvector-Based Framework for Heterogeneous Transfer Learning:} A framework using eigenvectors to facilitate knowledge transfer across diverse domains, improving performance.
    \item \textbf{Dynamic Adjustment for Multi-class Imbalanced Data:} A method combining drift detection and optimized ensemble selection to improve accuracy in imbalanced streams.
    \item \textbf{Adaptive Class Imbalance Method:} A dynamic approach to select oversampling methods, addressing class overlap in drifted data streams.
\end{itemize}

% 
\section{Publications}
\label{sec:1_introduction_publication}
During the research activities of this thesis, several international peer-reviewed
journal articles were published to disseminate the obtained results. The publications can be found in Table \ref{ch1.publication.list}. 

\begin{table}[!ht]
 \centering
 \caption{Publications in journals and conferences conducted during this thesis.}
\label{ch1.publication.list}
\renewcommand{\arraystretch}{1.1}
\begin{tabular}{l}
\hline
 \textbf{Title:} Vertical and Horizontal Data Partitioning for Classifier Ensemble Learning.\\
 \textbf{Authors:} AM Mohammed, E. ~Onieva, M.~Wo{\'{z}}niak.\\
 \textbf{Congress:} The 11th International Conference on Computer Recognition Systems,\\$\quad \quad \quad \quad$ 2019, Poland. \\ \line(1,0){400}\\
 \textbf{Title:} Training set selection and swarm intelligence for enhanced integration in \\ $\quad \quad \quad$ multiple classifier systems.\\
\textbf{Authors:} AM Mohammed, E. ~Onieva, M.~Wo{\'{z}}niak.\\
 \textbf{Journal:} Applied Soft Computing (Impact Factor $=$ 5.472 $\rightarrow$ Q1).\\
 \textbf{Status:} Published. \\ \line(1,0){400}\\

\textbf{Title:} Selective Ensemble of Classifiers Trained on Selective Samples.\\
 \textbf{Authors:} AM Mohammed, E. ~Onieva, M.~Wo{\'{z}}niak.\\
 \textbf{Journal:} Neurocomputing (Impact Factor $=$ 4.438 $\rightarrow$ Q1).\\
 \textbf{Status:} Under review. \\ \line(1,0){400}\\
 
\textbf{Title:} An Analysis of Heuristic Metrics For Classifier Ensemble Pruning Based on\\$\quad \quad \quad$ Ordered Aggregation.\\
\textbf{Authors:} AM Mohammed, E. ~Onieva, M.~Wo{\'{z}}niak, G Mart\'{i}nez-Mu\~{n}oz.\\
 \textbf{Journal:} Pattern Recognition (Impact Factor $=$ 7.196 $\rightarrow$ Q1).\\
 \textbf{Status:} Under review. \\ \line(1,0){400}\\ 
 
 
\end{tabular}
\end{table}
% \section{Thesis Plan}
\label{sec:1_introduction_methodology}
\begin{figure}[!ht]
    \centering
    \includegraphics[width=.8\textwidth]{1_introduction/figures/fig_research-methodo.pdf}
    \caption{The Research Plan of Thesis.}
    \label{ch1:research-emthodo}
\end{figure}

The research in this thesis is progressing rapidly due to technological advancements and continuous contributions in machine learning (ML). An iterative research methodology was followed, where each cycle builds upon the knowledge gained in the previous phase, leading to increasingly effective and original solutions as shown in Fig. \ref{ch1:research-emthodo}. The phases of this research methodology are as follows:

\begin{enumerate}
    \setlength{\itemsep}{0pt}
    \setlength{\parskip}{0pt}
    \item \textbf{Review of the current state-of-the-art:} Investigate existing research to identify challenges and inform the design of a solution.
    \item \textbf{Design and development:} Design a novel solution using updated knowledge to address the identified challenges.
    \item \textbf{Experimentation and evaluation:} Test the solution through experimentation, using established criteria for comparison.
    \item \textbf{Results analysis and comparison:} Analyze and compare results with state-of-the-art to determine the effectiveness of the solution, and disseminate the findings.
\end{enumerate}



\section{Structure of the Dissertation}
\label{sec:1_introduction_organizations}
The structure of the remainder of this thesis dissertation is outlined below.
\begin{description}	
	\item \textbf{Chapter~\ref{cha:2_background}} reviews background about concept drift, concept drift types, concept drift components, adapting types.  
	
	\item \textbf{Chapter~\ref{cha:3_State-of-the-art}} reviews state-of-the-art  concept drift, classifier ensemble selection, imbalanced data streams, Streams with Emerging New Classes (SENC), and transfer learning.  
	
	\item \textbf{Chapter~\ref{chapter:4_Imbalanced_Multiclass}
	} presents our first proposal to build an effective proposed approch for  handling Imbalanced Multi-class Drifted Data streams.
	
	\item \textbf{Chapter~\ref{chapter:4_Imbalanced_Multiclass}} provides our second proposal to adressing emerging new classes in incremental streams via concept drift techniques. 
	
	\item \textbf{Chapter~\ref{chapter:4_Imbalanced_Multiclass}} presents our third proposal to addressing heterogeneous transfer learning problem  in incremental streams via concept drift techniques.
	
	\item \textbf{Chapter~\ref{cha:7_Conclusions}} revisits the main goal and specific objectives posted earlier. In this chapter, we summarise the main contributions of this thesis and outline possible future research.


\end{description}

% background
% this file is called up by thesis.tex
% content in this file will be fed into the main document



% this file is called up by thesis.tex
% content in this file will be fed into the main document

%: ----------------------- introduction file header -----------------------


\begin{savequote}[50mm]
The good thing about science is that it's true whether or not you believe in it. 
\qauthor{Neil deGrasse Tyson}
% The beginning is the most important part of the work.
% \qauthor{Plato}
\end{savequote}


\chapter{Background}
\label{cha:2_background}

Amidst the surge of vast streaming data, governments and businesses find themselves in an urgent need for sophisticated data analysis and machine learning analytics approaches. These tools are indispensable for anticipating future trends and making well-informed decisions. However, the perpetual emergence of new goods, markets, and consumer behaviors introduces a formidable challenge known as concept drift [24]. This phenomenon involves the variation of statistical parameters of the target variable over time in unexpected ways, posing a substantial obstacle to accurate forecasting and optimal decision-making. The patterns derived from historical data may become obsolete when applied to new and evolving datasets.
The impact of concept drift extends across data-driven information systems, including decision support and early warning systems, diminishing their overall effectiveness. In the dynamic realm of big data, where data types and distributions are inherently unpredictable, the challenge of concept drift becomes even more pronounced. In response to this challenge, the field introduces a new subject: adaptive data-driven prediction/decision systems.

\section{Data Preprocessing}
\label{sec:2_1_DP}

Data preprocessing \cite{garcia2015} includes data preparation; (e.g., integration, cleaning, normalization, transformation) and data reduction; (e.g., instance selection, feature selection, discretization). The desired result is to get a cleaned, relevant, manageable, and meaningful dataset ready for analysis. Usually, there is a trade-off between time-complexity and accuracy to get the prepared data, which keeps the research ongoing for this area. 
\vspace{7mm}

\noindent \textbf{Data Preparation:}
Refers to a set of techniques to prepare data as an input for a certain DM algorithm. Usually, it is ignored by inexperienced practitioners, which may cause the model runtime crash. Even if the algorithm works, the expected results will not be optimistic. A set of preliminary steps can be followed before model training as described in \cite{garcia2015}: 
\begin{itemize}
    \item[-] \textbf{Data Cleaning:} The process of detecting and correcting (or removing) inaccurate records from data. Other tasks could be to detect irrelevant data fragments that do not make sense. The result will be a consistent, accurate, meaningful dataset \cite{rahm2000d}.
    
    \item[-] \textbf{Data Transformation and Data Integration:} Data transformation is the process to convert and consolidate the data to another format to improve model efficiency. This process is composed of sub-tasks; feature construction, feature aggregation, normalization, and more. While data integration is recommended for merging data that come from multiple data sources. The aim is to detect conflicts and to remove redundant and inconsistent data \cite{lenzerini2002,doan2012}. 
    \item[-] \textbf{Data Normalization:} This process is dedicated to unify the measurement unit to all the attributes. Under this schema, all the attributes are equally weighted. Statistically, to align the entire probability distribution of the adjusted values, to reduce the effects of certain gross influences \cite{pochon2008,pyle1999}.
    
   \item[-] \textbf{Missing Data Imputation and Noise Identification:} Training a model with a dataset that has many missing values can have a dramatic effect on the quality of the machine learning model. The imputation strategies can be (Mean/Median values, most frequent value, k-nearest neighbors, using deep learning, multivariate imputation by Chained Equation) \cite{buuren2010}. Noise identification is known as smoothing, to detect variances or random errors in a measured variable \cite{wang2010}. 
\end{itemize}


\noindent \textbf{Data Reduction:}
Not like data preparation, data reduction is an optional step. It provides a set of methods for obtaining a reduced version of the original data. It is the process of downsizing the data while maintaining the integrity of the complete dataset. However, it could be a crucial step as data preparation, to enable the DM algorithm when the data size exceeds. Following are some representative approaches: 

\begin{itemize}
    \item[-] \textbf{Feature Selection:} Data reduction can be accomplished by removing irrelevant or redundant attributes. The aim is to use the least number of features while keeping the output of the classification as similar as possible as if we were using the full feature set. Regarding that, the training speed can be boosted and the model performance can be elevated \cite{li2017, dash1997}. 
    \item[-] \textbf{Feature Extraction:} Lessening the amount of resources required for the representation of a large array of data. Analysis with a wide range of variables usually involves a large amount of memory and computational resources. In addition,  the classification algorithm could generalize to new samples badly. Feature extraction is a set of techniques that create combinations of variables to fix these issues, but also reflecting the data with sufficient precision. Many machine learning specialists believe that the properly configured extraction of features is the key to a successful model building \cite{geron2019}. 
    
    \item[-] \textbf{Instance Selection:} The process of reducing or eliminating samples intelligently without affecting the DM application. This process can be guided by heuristic rules to select horizontal subsets of data \cite{olvera2010,fu2013}. In addition, the process can be applied to adapt with a particular DM algorithm; like "selection of support vectors for support vector machine algorithm" \cite{liu2017}.   
      \item[-] \textbf{Discretization:} The mechanism by which quantitative data is converted into qualitative data; via converting numerical variables into discrete or nominal variables. For that, a huge spectrum of numeric values can be compacted or reduced into a subset of discrete values \cite{ding2010}.  
\end{itemize}

Finally, the benefits of data preprocessing can be among the following; (1) Adaptation to a particular machine learning algorithm. (2) Increasing predictive accuracy. (3) Enabling: data mining algorithms are negatively affected by the data size, and data reduction provides a solution for data choking. (4) Cleaning noisy, missing, and redundant data to improve data quality. (5) Focusing: to focus on relevant data instead of all available information.


%%%%%%%%%%%%%%%%%%%%%%%%%%%%%%%%%%%%%%%%%%%%%%%
%
%   Brief history of TF
%
%%%%%%%%%%%%%%%%%%%%%%%%%%%%%%%%%%%%%%%%%%%%%%%
\section{Data Mining}
\label{sec:2_2_SML}
A closely related field to machine learning is Data Mining, which is defined in \cite{lantz2013} as, "\textit{the generation of novel insights from large databases}". While in \cite{garcia2015}; data mining is defined as, "\textit{solving problems by analyzing data present in real databases}". While others \cite{chakrabarti2006,han2011,nisbet2009,clifton2010} view DM as, "\textit{the main steps of knowledge discovery in databases}". The main distinction between ML and DM is that; ML is dedicated to teach computers how to use data to solve problems, while DM is dedicated to teach computers how to identify patterns like a human to solve problems. It is interesting to mention that every DM involves the use of ML, but not all ML involves DM. Next, a brief description of data mining categories is presented:
 
\begin{itemize}
    \item [-] \textbf{Unsupervised Learning}: Those techniques deal with an unlabeled dataset, with a minimum of human intervention \cite{fisher2014}. Data with no pre-existing labels at our disposal and the purpose is to find associations, relationships, regularities, and similarities in the data. Cluster analysis is among the common models used in unsupervised learning. Cluster analysis is the process to segment/group datasets with shared attributes \cite{kaufman2009}. Unlabeled examples are given an implied cluster label from the relationships within the data entirely. Sometimes, the clustering task is referred to as "unsupervised classification", because it classifies unlabeled examples.   
    
    
\item[-] \textbf{Supervised Learning:}   Supervised learning is popular in the field of DM, commonly knows as prediction methods. In supervised learning, a relationship is to be learned between input space and target space. Based on the predicted target type, there are two common tasks; regression and classification. In regression, the numerical output to be predicted falls in a certain interval. Contrary to classification, the domain of the target is finite and categorical.   
\end{itemize}

    



Next, all the subsequent sections will be dedicated to the classification task in matching with the core of this thesis. In the classification problem, the input attributes (features) and the target attribute (class) are transparent. The aim is to learn a function that maps inputs to outputs. The learned function is called a model, and it is inferred from labeled training data. Let, 

\begin{equation}\label{eq:classification}
\textbf{x}=\left[x^{(1)},..., x^{(d)}\right]^T,\textrm{ and } \textbf{x}\in \mathcal{X}=\mathcal{X}^{(1)}\times ... \times \mathcal{X}^{(d)}
\end{equation}
 
\noindent where $\mathcal{X} $ denotes feature space and \textbf{x} is the sample, i.e., \textbf{x} is the so-called feature vector which informs about attribute values. We will assume that we have $d$ attributes at our disposal. The supervised classification model will assign a given object described by its features, \textbf{x}, into one of the predefined categories, also called labels. Let $\mathcal{M}=\{1,..., M\}$ stands for the set of class labels (decision regions). The classification algorithm (discrimination algorithm) is any learned function $\Psi$ with domain $\mathcal{X}$ and codomain $\mathcal{M}$ as clarified in Equation (\ref{mapper}). Where the target values in codomain are finite and categorical.  

\begin{equation}\label{mapper}
\Psi\: :\: \mathcal{X} \to \mathcal{M}.
\end{equation}

\begin{figure}[!ht]
    \centering
    \includegraphics[width=\textwidth]{2_Background/figures/fig_SML_models.pdf}
    \caption{Supervised classification algorithms.}
    \label{ch2:cl.sml}
\end{figure}

There is a large number of models with different inferring strategies to discover the hidden patterns from data. Next, I provide a short review of popular classification algorithms according to the division in Figure \ref{ch2:cl.sml}, inspired by \cite{garcia2015}.  

\begin{itemize}
    \item Regression Models: Logistic Regression (LR) is a statistical model, that returns a probability for each class level. The cutoff value is embedded to separate the upper and the lower probabilities to work as a binary classifier (binomial LR). While multinomial LR deals with more than two classes. In order to be enabled, the missing values should be handled. In addition, the high correlation among the predictors (variables) should be minimized. LR has been praised for its robustness and simplicity \cite{friedman2001}.  \item Artificial Neural Networks (ANNs): The excessive and growing formulations of ANNs from the theoretical and algorithmic depth \cite{carlini2017}, made them more influence on the field of pattern recognition. ANNs handle large and complex tasks due to their nested non-linear structure. While non-explanation is one of the pitfalls of those black-box models.  The computations of those models are based on the definition of neurons. ANNs are unstable and more sensitive to small changes in training data. Similar to LR, they require no missing values.  
    \item Bayesian Learning: Based on the probability theory to get rational decisions \cite{saritas2019}. The Na\"ive Bayes is the most popular algorithm in this category. The posterior probability for each class label is calculated, then the decision is promoted upon the maximum probability returned. Those methods only work with categorical attributes, cannot work with missing values, and are very sensitive to redundancy. The "Na\"ivety" comes from the assumption of the conditional independence between features. It can be viewed as an explainable model to give reasoning about the decisions. 
    \item Instance-based Learning: The prediction of a new unknown sample is based on a distance function with the past stored samples. Also called memorization techniques and lazy learners \cite{lopes2015}. The performance of those methods is affected by the used distance function, neighborhood size, and decision aggregation mechanism. K-Nearest Neighbor (KNN) is the most popular method in this category. Pitfalls of those methods can be mentioned as: high memory space for storage, delayed prediction response, sensitivity to noise.
    \item Support Vector Machines (SVM): Learning algorithms which are based on maximizing gab separation (margin) between different class samples to get correct decisions. They are suitable to work as linear and non-linear data separation. Only the class borders (support vectors) are important to optimize the margin where internal points can be removed to improve the efficiency \cite{nalepa2019}. They require no missing values and are commonly robust against noise. \item Rule Learning: Called divide-conquer algorithms \cite{furnkranz2012}. The data parts are divided based on one rule, then recursively conquer the divided parts. Those models are transparent or explainable to nonexperts in the form of logical structures. Available features are analyzed to find homogeneous groups, then an additional rule is built to drill down more. Small changes in the training data results in decision change. Rule learning techniques are affected by missing values, noisy samples, and outliers.
    \item Decision Trees (DT): This is a kind of indirect rule learning. Uses structure branching decisions to model the relationships among the features and the predicted class value. They are widely used and can model any type of data. The human-readable model is appropriate in applications where legal reasoning is required. Those models are vulnerable to overfitting, and the internal parameters should be tuned \cite{geron2019}. Unstable like ANNs and sensitive to change in the training data. Usually, they are biased towards the splits on features.               
\end{itemize}

For unseen pattern $\textbf{x}$, a class membership values are calculated as in Equation (\ref{membership}), then the labeling choice will be connected to the highest score.
\begin{equation}\label{membership}
\Psi(\textbf{x})=\text{Max}\{g_1(\textbf{x}),g_2(\textbf{x}),\dots,g_M(\textbf{x})\}
\end{equation}


 The decision region $R_1$ for the $1^{st}$ class, is the set of points for which $g_1(\textbf{x})$ has the highest score \cite{kuncheva2014}. While the \textit{classification boundaries} contain data points for which the membership values tie. If the decision region $R_1$ contains data points from the $2^{nd}$ class, then we have overlapped classes. From Figure \ref{ch2:overlapping}. (c) and as shown in \cite{galbusera2019}, the regions are nonoverlapping as the model learns all the details about the data. This case is known as overfitting, and the model will not perform properly to predict unseen samples. While  Figure \ref{ch2:overlapping}. (a) shows the optimal class separation boundary that guarantees minimum possible error with the future samples. Finally,  Figure \ref{ch2:overlapping}. (b) shows the the underfitting case when the model fails to capture relationships between a dataset’s features and a target variable during training. 
 
 
 \begin{figure}[!ht]
    \centering
    \includegraphics[width=\textwidth]{2_Background/figures/fig_overlapping.pdf}
    \caption{Trade-off between overlapping and overfitting, taken from \cite{galbusera2019}.}
    \label{ch2:overlapping}
\end{figure}


Each model has an accompanied error, we need to understand the different sources that cause this error. Equation (\ref{Gen.error}) represents the compound generalization error $E_G$ of a classifier $\Psi$ that is trained on dataset $D$. 


\begin{equation}\label{Gen.error}
E_G(\Psi,D)=E_A(D)+ E_M+E_B
\end{equation}

\noindent where $E_A(D)$ is the "approximation error" represents the variance due to using different training data, or non-deterministic training algorithm. Clarified as, the hyper-parameters of the model that affect its performance. The second term $E_M$ is the "model error" which represents the bias due to selecting a model in preference of another. The last term $E_B$ is the "irreducible error" coming from the insufficient representation of the data. This is commonly known as Bias/Variance Tradeoff \cite{geron2019,kuncheva2014}. Increasing a model’s complexity will typically increase its variance and reduce its bias.
Conversely, reducing a model’s complexity increases its bias and reduces its variance.
This is why it is called a tradeoff.    
%%%%%%%%%%%%%%%%%%%%%%%%%%%%%%%%%%%%%%%%%%%%%%%
%
%   Machine Learning Modelling Approaches of Traffic Forecasting
%
%%%%%%%%%%%%%%%%%%%%%%%%%%%%%%%%%%%%%%%%%%%%%%%
\section{Ensemble Data Mining}
\label{sec:2_3_EDM}

Ensemble learning is the strategy of using multiple learning algorithms in order to obtain greater predictive precision than all of the constituent learning algorithms alone \cite{polikar2006,polikar2012,rokach2010,dietterich2002,zhang2012,opitz1999,sagi2018,oza2001,krawczyk2017}. In addition, Wozniak. et al. defined ensemble models as hybrid intelligent systems \cite{wozniak2014} with the potentiality to cope with ambiguity, uncertainty, and complex problems. Thanks to their capabilities, ensembles received great attention in the applications related to data mining. For unsupervised learning, clustering performance could be significantly improved by ensemble methods \cite{zhou2006,vega2011,cornuejols2018}. Furthermore, ensembles are employed for unsupervised anomaly detection \cite{zhao2015}. While, for supervised learning, those systems are widely popular for regression tasks \cite{mendes2012,ren2015} and for classification tasks \cite{wozniak2014,sagi2018}.


Ensemble systems for pattern classification have been expanded in the literature under creative names as: consensus aggregation \cite{benediktsson1992}, stacked generalization \cite{naimi2018}, committees of neural networks \cite{cirecsan2011}, mixture of experts \cite{jacobs1991,jordan1994}, classifier ensembles \cite{yu2014,rodriguez2006}, classifier selection \cite{ruta2005,cruz2018}, multiple classifier systems \cite{wozniak2014, catal2017}, classifier fusion \cite{liu2017cl} and more.  Theoretical and empirical studies prove that ensemble systems are more accurate than any random classifier \cite{catal2017, tahir2012,polikar2006,fernandez2014}. The final decision is the accumulative decisions of all the classifier set. Let $\Pi$ denotes a pool of $T$ base classifiers $\Pi=\{\Psi_1, \Psi_2,\Psi_k ..., \Psi_T\}$ to be grouped by a combination function. The ensemble output $\hat{\Psi}$ is determined on the basis of the outputs of the base classifiers, i.e.,

\begin{equation}
\label{eq-generalfusion}
\hat{\Psi}(\textbf{x})=F(\Psi_1(\textbf{x}), \Psi_2(\textbf{x}), ..., \Psi_T(\textbf{x}))
\end{equation}

\textit{Intuitively, any classifier ensemble is in fact a classifier} (L. Kuncheva \cite{kuncheva2014}).


% \begin{figure}[!ht]
%    \centering
%    \includegraphics[width=.7\textwidth]{2_Background/figures/fig_ensemble-top.pdf}
 %   \caption{The simple topology for combining classifiers, taken from \cite{kuncheva2014}.}
  %  \label{ch2:ensemble-topology}
%\end{figure}










Each learning algorithm, Section \ref{sec:2_2_SML}, has a limit to discovering the hidden pattern. According to Wolpert's \emph{no free lunch} theorem \citep{wolpert2002}, there is no best classifier suitable for all problems, but each model has its own area of competence giving the design assumptions. For that, a set of learning models solving the same problem can be consolidated to generate a better composite global model \cite{ksieniewicz2018}. L. Kuncheva \cite{kuncheva2014} stated \textit{"the improvement of the ensemble over the single best classifier or even on the average of individual classifier accuracies is not guaranteed"}. From our perspective, the proper design of ensemble is conditioned by outperformance over the best individual classifier in the group.  



\subsection{Are we pursuing complexity?}
Why we accept complex systems instead of depending on a single classification algorithm?. The answer to this question is highlighted in the following points.
\begin{itemize}[nosep]
    \item[-] Ensemble models are the solution to deal with uncertainty. A solution from a single classifier can be boosted and trusted by aggregating a group of predictors (\textit{wisdom of crowds} \cite{rokach2010}).
    \item[-] There is no guide to design universal approximators, perfect model, i.e. it is difficult to set up ANNs or reaching their optimistic parameters.

    \item[-] Ensemble selection or pruning is an interesting research topic that aims to reduce ensemble complexity without deterioration in the performance.
\end{itemize}

In addition, a promising solution via ensemble learning can be achieved for the following scenarios:

\begin{itemize}[nosep]
    \item Imperfect Learning: Non-deterministic classifier can be considered as a local optimizer in terms of the training error, a safer option is to group several models that cover the solution space properly.
    \item Too much data: We are surrounded by too much data. In this case, data is split into chunks where similar or different learning algorithms can be trained on each part independently. Ensemble learning support parallelization and distributed computing for handling this scenario efficiently.
    \item Small-size data: In data shortage, stratified sampling with replacement can be applied, where several data replications will be obtained to train individual classifiers inside the ensemble. 
    \item Data fusion: The data pattern can be identified differently based on the data source. The availability of sensors strengthens decision making by analyzing different features. Instead of fusing all features and building a single classifier, it could be better to build a single classifier for each feature space and combine their decisions.  
    \item Complex hypothesis: Complex classification boundary can be approximated by combining several base classifiers.
    
\end{itemize}


\subsection{A Taxonomy of Classifier Ensemble Methods } \label{ch2.taxonomy}

A comprehensive review of the classifier ensemble methods, thorough discussion, and the development of further knowledge in this area was the core of many articles \cite{gonzalez2020,polikar2006,cruz2018, britto2014,re2012,sewell2008}, with a proposed taxonomy by L. Rokach \cite{rokach2009} who indicated the five dimensions to design this kind of powerful models. Figure \ref{ch2:ensemble-taxonomy} shows our perspective for the Multiple Classifier System (MCS) taxonomy as it has been proposed by L. Kuncheva \cite{kuncheva2014}, R. Cruz et al. \cite{cruz2018}, and L. Rokach \cite{rokach2009}. The two main phases in classifier ensembles are; Generation and Integration, while the selection is an intermediate/optional phase.   

\begin{figure}[!ht]
    \centering
    \includegraphics[width=.95\textwidth]{2_Background/figures/fig-taxonomy.pdf}
    \caption{The Taxonomy of Multiple Classifier System (MCS).}
    \label{ch2:ensemble-taxonomy}
\end{figure}



\subsubsection{Generation} \label{sub.generation}
The goal in this phase is to generate a pool of classifiers that are both diverse and accurate. This phase discusses strategies to handle data horizontally/vertically. In addition, what classifier type to accommodate, how to build the classifiers: dependent/independent manner, and how many classifiers to train (ensemble pool size). 

 \textbf{1) Diversity:}
Diversity is one of the main reasons for the effectiveness of ensemble methods \cite{zhou2012,gonzalez2020}. Diversified classifiers cause uncorrelated errors, which lead to improved classification accuracy \cite{hu2001}. In general, this can be achieved through the following six wide subcategories to promote diversity during the generation phase.
\begin{itemize}
    \item[-] \textbf{Different Parameters/Initialization:} \textit{Algorithm-level diversity}; via different parameters, the base classifiers can be generated by modifying the hyper-parameters of the learning algorithm; for example, controlling the number of $k$ neighbors, distance function in KNN model, and controlling the confidence level parameter in the decision tree. While, via different initialization and if the training process is initialization dependent, the model will be sensitive. In neural networks, different initial configurations of weights result in different decisions \cite{hansen1990}.        
  \item[-] \textbf{Different Architectures:}  \textit{Algorithm-level diversity}; this strategy is more suitable to multi-layer perceptron neural networks, where the number of hidden layers, the number of neurons, and the network topology affect the classifier domain space. For example; in the Addemup algorithm \cite{opitz1996}, the genetic algorithm is used to choose the required network topology to compose the ensemble according to a measure of diversity.     
  \item[-]\textbf{Different Classifier types:} \textit{Algorithm-level diversity}; each classifier model build its inference with a capability to discover a hidden pattern differently. Each model, Section \ref{sec:2_2_SML}, contains explicit or implicit bias that leads to a better generalization accuracy through combination. In addition, as in \cite {woods1997}, the divide and conquer mechanism is best implemented by calculating the local accuracy in the feature space to choose between four different classifier types.  
  \item[-]\textbf{Data Partitioning:} \textit{Data-level diversity}; Partitioning means the division into smaller disjoint components of the initial training. A different classifier will be trained on each part to gain different and accurate decision \cite{maimon2005}. This mechanism enables data mining algorithms to handle massive datasets by managing memory size and computational resources perfectly. The data can be partitioned horizontally or vertically. \textit{In horizontal partitioning}, several parts will be formed; each part will contain sub-samples while sharing the complete feature set. Sub-samples can reflect the entire dataset by selecting the instances from all the formed clusters of the dataset; known as cluster-based concurrent decomposition (CBCD) \cite{rokach2005}. The mixture of experts (ME) \cite{nowlan1991}   splits the input space into several subspaces and assign an expert, classifier, to each subspace. In \cite{cohen2007} a decision tree framework is employed to divide the input space into mutual exclusive partitions, then a new unseen pattern will be classified by a dedicated classifier that is learned from the space to which the instance belongs. \textit{In vertical partitioning}, The feature space is partitioned into several subsets keeping the same number of samples, then each classifier can be trained on a different projection. This mechanism is more suitable for a high-dimensional dataset without affection by the feature selection drawbacks \cite{tumer2003}, and the accuracy could be improved by the less correlation among classifiers. This division results in a high-speed classification algorithm \cite{bryll2003}, and solves the problem of class under-representation that exist with instance-based sampling.      
  
  \item[-]\textbf{Resampling:} \textit{Data-level diversity}; the diversity could be promoted by manipulating the training set. Each base model is trained over a different sample from the training. Bagging \cite{breiman1996,skurichina1998} and Boosting \cite{freund1997,freund1999} are the popular strategies in that paradigm. Bagging ensembles achieve a reasonable diversity level by creating different bootstrap samples to train each base model independently, then the final decision is adopted by a simple majority voting-based aggregation. Moreover, the non-sensitivity of bagging and robustness under diverse noise conditions makes it more attractive \cite{dietterich2000}. Contrary to sequential ensembles, \textit{Boosting}, the individual members are generated in the sequential schema by the learning algorithm \cite{freund1997}. The sequential mechanism of boosting encourages the complementariness between ensemble members, by focusing on previously misclassified samples. However, the performance in boosting is more sensitive to noisy samples \cite{dietterich2000,caruana2006} and sometimes overfitting can be observed for large pool size \cite{ratsch2001}. 
  
  \item[-]\textbf{Differnet label targets:} \textit{Data-level diversity}; manipulating the target attribute is very interesting mechanism to promote diversity. Instead of building complex classifier, several classifiers with usually simpler representations, about the target attribute, will be trained. For instance, to handle the multi-class dataset, the original target attribute can be replaced by a simpler and smaller target domain. Among those strategies, One versus all (OVA) \cite{anand1995} which divides the $M$-class classification problem into $M$ two-class classification tasks. While one versus one (OVO) \cite{lu1999} divides the $M$-class classification problem into $M(M-1)/2$ two-class classification tasks, so the complex decision boundary can be simplified. Minimal classification method (MCM) \cite{sivalingam2005} converts the $M$-class classification task to the minimal binary classification tasks. MCM requires $\log_2(M)$ classification tasks in the form of a separation between groups of multiple classes. Furthermore, the error-correcting output coding (ECOC) algorithm, uses a code matrix to decompose the multiclass problem into multiple binary problems \cite{dietterich1994}. In addition, label switching algorithms \cite{breiman2000,martinez2005} change the labels of samples picked randomly. 
\end{itemize}


\textbf{2) Building:}
This part discusses the building schema to generate a pool of classifiers, whether to be dependent or independent. In the dependent framework (sequential/incremental), the performance of a classifier affects the creation of the next classifier in the chain. Instead, in the independent framework (simultaneous), the classifiers are generated independently. 

\begin{itemize}
    \item[-]\textbf{Simultaneous:} The independent methods to generate a set of classifiers mainly depends on forming different training samples from the original training set. The training samples could be mutually exclusive (disjointed) or overlapping. The reason behind preferring this schema is to improve the predictive accuracy or to speed up the generation step as this schema supports parallel implementation. \textit{Bagging} (Bootstrap AGGregatING) \cite{breiman1996} is the popular method in this category, with a replacement from the original training set, each classifier is trained on data samples. The sample size will be equivalent to the original training size; randomly some samples will be duplicated and some samples will be ignored.  
     \item[-]\textbf{Incremental:} Sometimes called dependent, there is a kind of interconnection during the learning process. The model generation depends on the accuracy of the previous model/committee. The training process is done in iterative form, where the learning process will be directed to focus on the previously misclassified samples. \textit{Boosting} \cite{freund1996} (also known as arcing- Adaptive Resampling and Combining) is the popular method in this category. The training process mainly depends on assigning weights to all the training samples. In the beginning, all the samples will be equally weighted (having the same importance), but throughout the iterations, the weights of correctly classified samples are lowered while the weights of misclassified samples are raised. As a consequence, the base model is directed to focus on the hard samples in the training set.          
\end{itemize}


\textbf{3) Universality:}
This part discusses the universality of the ensemble model. Some ensembles techniques could be designed to work for any classifier, while other ensembles have been designed to work with specific classifiers. In other meaning, the relation between the ensemble technique and the used classifier type. 

\begin{itemize}
    \item[-]\textbf{Specific Inducer:} Known as inducer-dependent ensembles, where the effectiveness of the ensemble could be degraded if applied for other classifier types. For example, \cite{hansen1990,lu1999} those ensembles are explicitly designed for neural networks. Additional schemas \cite{tao2006} are perfectly suited for SVM.
    
     \item[-]\textbf{Any Inducer:} Known as inducer-independent ensembles, Those implementations can be extended to a wide variety of classifier types without affecting the generalization accuracy. 
    
\end{itemize}


\textbf{4) Ensemble Size:}
This part discusses the aspect of pool size. How many classifiers should be trained?. The main four factors that concern the ensemble size as determined by L. Rokach \cite{rokach2009} are; (A) \textit{Sufficient Accuracy}; the appropriate accuracy of the ensemble could be reached by aggregating 10 models \cite{hansen1990}, while other empirical studies show that this level of accuracy is correlated with large-size ensembles containing 25 models \cite{opitz1999}. (B) \textit{Computational Cost}; the more classifiers are generated the more computational resources are consumed. As a consequence, the user may predetermine the pool size to match the computational cost limits. (C) \textit{Nature of Task}; the ensemble size could be problem-dependent, as we stated before with the ensemble size of OVO and OVA strategies for handling multi-class classification tasks. (D) \textit{Number of processor cores}; for independent ensembles, the number of internal cores can be the upper bound to control how many classifiers can be trained in parallel mode.                      

\begin{itemize}
    \item[-]\textbf{Fixed in Advance:} This is the simple form to predetermine how many iterations should be considered. Most Bagging and Boosting software packages give the flexibility to control the number of iterations.  
    
     \item[-]\textbf{Determined during Training:} The best ensemble size can be determined during train time. The contribution of a new classifier to the ensemble performance is to be checked if it is significant or not. To estimate the unbiased error of the test sample, Random Forests uses the out-of-bag (OOB) procedure. The OOB error estimation is used in \cite{banfield2006} to determine the sufficient number of classifiers. As the maximum accuracy no longer increases, the training procedure stops.       
    
\end{itemize}



\subsubsection{Selection}
There is no value from combining similar classifiers' decisions. The effectiveness of the ensemble is conditioned by the diversity and the correctness of the base classifiers. Ensemble Selection is one of the strategies that can be used to handle this challenge. Ensemble Selection has been known in the literature as ensemble pruning \cite{lu2010,zhang2006,ykhlef2017}, ensemble thinning \cite{banfield2005} and ensemble reduction \cite{diao2013,zhang2018}. ES can be considered as an intermediate process between building the ensemble and aggregating the decisions. Specifically, ES is the strategy of optimizing and selecting the number and the type of individual classifiers in-advance. Collecting the decisions from a reduced number of models speeds up the classification systems and relieves memory storage. In the literature, the selection process can be performed offline, static selection \cite{tsoumakas2009}, or online, dynamic selection \cite{cruz2018}.
 
 
 \begin{itemize}
    \item[-]\textbf{Static Ensemble Selection:} Known in the literature review as ensemble pruning. This process is done during the training and before the real test. A subset of classifiers that optimize a predefined function/metric are selected, and the estimation of that metric is done over a pruning set. The most common metrics/selection criteria are; diversity \cite{aksela2003,brown2010}, classification accuracy \cite{dos2009,ruta2005}, instance margin \cite{yang2012, guo2013,hernandez2008}. However, it is not trivial to find the optimal subset of classifiers from a large ensemble as the complexity increases exponentially with the pool size. Researchers agreed in common that ES is a combinatorial search problem with $2^{T}-1$ nonempty subsets to be evaluated from pool size, $T$, to find the best subset \cite{tamon2000, tsoumakas2009}. To handle this complexity, several attempts ranging from optimized search \cite{diao2013,adnan2016,ruta2001b}, clustering techniques \cite{onan2017,zyblewski2020}, and greedy algorithms \cite{cao2018,martinez2009,margineantu1997, partalas2010, mao2011} have been applied over decades. More details about ordering-based pruning will be discussed in Section \ref{ch5_ordering_pruning}.        
    
     \item[-]\textbf{Dynamic Ensemble Selection:} This process is done during the test time. A single classifier or subset of classifiers are selected based on the unseen sample. In addition, dynamic weights are assigned according to the competition among individual classifiers over the local region of an unknown sample $\textbf{x}$. As a consequence, the selected subset is changeable for each test pattern \cite{KO2008}. This process consists of three steps; (1) Definition of the local region surrounding the query sample $\textbf{x}$, (2) Determination of the selection criteria to estimate the competence level of the classifiers, and (3) Determination of the selection mechanism, whether to select a single classifier or ensemble of classifiers. Without debate, dynamic selection techniques can outperform the static selection methods as experimentally been proved in \cite{cruz2018} since the selection is optimized for each test sample independently. The rationale in dynamic selection is that each classifier is an expert in a different local region within the feature space. However, in the dynamic selection, there is a computational overhead for selecting subensemble for each test sample. Besides, those techniques flood the memory space as all individual classifiers have to be retained in memory. Additionally, the dynamic selection is affected by the outlier instances around the query sample in the feature space \cite{CRUZ2015}.
    
\end{itemize}

Cluster and pick \cite{kuncheva2000} is a poor variant of dynamic selection. Initially, the input space is split into disjoint regions via clustering the training samples. The best classifier is then defined and chosen for each cluster. Cluster and select is in-between static and dynamic strategies; the classifiers are selected dynamically depending on which region the input sample belongs to, but the regions are static, determined in advance during the training \cite{ruta2005}.

 
 
 
\subsubsection{Integration}
Also known as the combiner function. The combiner balances the deviation between the diversity and the bias, also alleviates the errors that certain models have made \cite{SOARES2013}. This process concerns the methodology by which the outputs of the base classifiers will be aggregated.  A vital phase in building a group of classifiers is the use of a suitable fusion strategy to aggregate response decisions \cite{zhou2012,kuncheva2014}. The responses of individual classifiers restrict the fusion method and enhance or degrade the composite prediction. Those responses may be on the \textit{abstract level} \cite{xu1992}, where each classifier specifies a class name as a decision. Additionally, the responses may be \textit{ranked levels} \cite{PARKER2001}, where each classifier outputs a ranked subset of class labels, or even  \textit{measurement values} \cite{kuncheva2014a,niu2007}, where each classifier specifies a posterior probability.


\begin{itemize}
    \item[-]\textbf{Non-trainable:} 
    The combination function does not require any training from the classifiers' decision space. Many combination rules have been proposed; for measurement values (Sum, Product, Maximum, Minimum, Median, and Decision Templates \cite{Kuncheva2001}), for abstract level (Majority voting \cite{Kittler1998}, Behavior Knowledge Space \cite{Huang1995}, and Naive Bayes \cite{xu1992}), and for ranked levels (Borda count \cite{Ho1994}). The majority voting, Equation(\ref{majority}), will be effective if the base classifiers are independent. While, the weighted sum rule, Equation(\ref{weighted.sum}), produces good results if the classifiers perform the same task and have comparable success or when we would like to avoid over-fitting or long training time \cite{rokach2009}. The weights, $w_k$, are calculated to be proportional to the individuals' performance over a validation dataset.  In addition, weighted voting is preferred for highly imbalanced datasets \cite{kuncheva2014a,krawczyk2014}.  
    
\begin{equation}
\label{majority}
\hat{\Psi}(\textbf{x})=\mathop{\arg\max}\limits_{c \in \mathcal{M}} \mathop{\sum}\limits_{k=1}^T \left[\Psi_k(\textbf{x})=c\right] 
\end{equation}    

\begin{equation}
\label{weighted.sum}
\hat{\Psi}(\textbf{x})=\mathop{\arg\max}\limits_{c \in \mathcal{M}} \mathop{\sum}\limits_{k=1}^T \left[\Psi_k(\textbf{x})=c\right]w_k
\end{equation}
    
    
     \item[-]\textbf{Trainable:}  
The combination function is to be configured specifically to the classification task. The aggregation function will be trained over a validation dataset from the domain, \textit{base classifiers outputs}, and co-domain, \textit{real attribute output}. For example; the classifiers' fusion weights can be optimized by evolutionary algorithms (EA).  In \cite{krawczyk2014}, the authors tuned the weights for the selection and fusion of multiple cost-sensitive decision trees to handle imbalanced data using the evolutionary genetic algorithm (GA) \cite{mitchell1998}. Each chromosome from the genetic population simulates a weighted ensemble as $\left[w_1, w_2, \ldots, w_k, \ldots w_T\right]^T$, $ \forall w_k \in [0,1]$. The ensemble performance is estimated for each chromosome, and genetic operators evolve the next generations. In addition, GA has been applied in \cite{sikora2015} to tune the weights of heterogeneous classifiers in consideration of the above chromosome encoding. While, a bird-flock based optimization algorithm has been applied in \cite{cordeiro2011, kausar2010} to enhance the fusion process.          
      
\item[-]\textbf{Meta-classifier:}   
Meta-learner is another important fusion mechanism. It is a trainable method, where the aggregation function is to be learned based on the base classifiers outputs. \textit{Stacking} \cite{Wolpert1992} is the most popular meta-learning method, where the predictions from the base classifiers become meta-features/inputs to a new classifier. The original target attribute from the training set remains as it is. Stacking
is generally used to combine heterogeneous models, the term refers to stacking layers of classifiers. In stacking, the correctness of the base classifiers will be learned indirectly through the meat-learner. In \textit{Grading} \cite{seewald2001}, the predictions of base classifiers are transformed into true or false. After that, one meta-classifier will be trained on the transformed decisions of one base model to learn when it errors. Therefore, grading can be seen as a generalization of cross-validation selection, by using only those classifiers that correctly predict specific instances. 



\item[-]\textbf{Dynamic weighting:} Similar to dynamic ensemble selection. The classifiers' fusion weights are determined dynamically based on the local competence of the classifier in the region where the unknown $\textbf{x}$ is located. A higher weight value is assigned to the most competent classifier  \cite{cruz2018}. For example, dynamic integration of classifiers in \cite{Tsymbal2008}. In addition, the dynamic ensemble selection and the dynamic ensemble weighting can be hybrid as in \cite{KO2008}.       
\end{itemize}


Finally, Figure \ref{ch2:four-levels} shows the questions that should be answered during the four levels of constructing MCS \cite{kuncheva2014}.     

\begin{figure}[!ht]
    \centering
    \includegraphics[width=\textwidth]{2_Background/figures/fig-four-levels.pdf}
    \caption{ Four level questions while building MCS.}
    \label{ch2:four-levels}
\end{figure}


\subsection{Diversity and Uncorrelated Errors}
Indeed the two basic facets of enhancing the efficacy of the committee are often referred to as decision optimization and coverage optimization \cite{ho2002}. \textit{Decision optimization} refers to methods for choosing and optimizing the combination method. \textit{Coverage optimization} refers to the techniques used to construct a diverse classifier set, assuming a fixed combiner. In Section \ref{sub.generation}, the prospective methods to generate diverse base models were reviewed. 
If a classifier makes errors on some objects, then it is better to complement it with another classifier that compensates that error. The ensemble performance is restricted by a compromise between an individual's accuracy and group diversity. Confirmed as, neither the individual's accuracy \cite{rogova1994} nor the diversity \cite{ruta2001a} on their own provide reliable ensemble to outperform the best individual classifier. Here, we agree with  L. Kuncheva \cite{kuncheva2014}, ''\textit{good estimate can be obtained from arbitrarily inaccurate estimators as long as their deviations cancel one another}''. However, it is so difficult to engineer this ensemble design.   


There is no benefit from combining redundant classifiers, and the system will be more complex. A large number of classifiers increase the ambiguity, risk, and add more complexity to the selection procedure which could lead to weak generalization \cite{ruta2005}. While for critical systems the useful evidence is so important and should not be wasted. Regarding that, more attempts have been made to select classifiers based on diversity measures. In \cite{giacinto2001}, the authors used a double fault measure to cluster classifier outputs, then they picked a single classifier from each cluster. While in \cite{kim1997} a similarity measure is used to pick a classifier from a pool of five classifiers. Many diversity measures have been presented \cite{kuncheva2003,ruta2001a} with the conclusion that diversity guided search could be invaluable. In addition, the diversity has been intensively used with the combiner performance \cite{ruta2005,li2012,brown2010} to improve the efficiency. The system design does not end with the selection process, as the selected classifiers should be next combined. For that, there should be a correlation between diversity measure value and committee performance. Diversity initiatives that represent at least some clear association trends, with generalization, have the potential to become appropriate criteria for selection. Diversity measurement is not a trivial task in ensemble methods. 


Matti et.al \cite{aksela2006} stated that diversity can be measured from two perspectives, Figure \ref{ch2:diversity-measures}, based on the population. (a) \textit{The data-based approach:} we have $N$ populations with $T$ objects. The ensemble diversity of the $T$ classifiers is calculated for each sample $\textbf{x}_i$, then the overall average is considered over $N$ samples. Here the diversity of all member classifiers is evaluated simultaneously (Non-Pairwise measurement). (b) \textit{The classifier-based approach:} in this case we have $T$ populations each contain $N$ objects. The diversity is to be calculated based on the classifiers' outputs for all input data. Each pair of classifiers are considered for measuring the diversity, then the overall diversity is computed by averaging over the number of pairs (Pairwise measurement).       

\begin{figure}[!ht]
    \centering
    \includegraphics[width=.5\textwidth]{2_Background/figures/fig-diveristy-measure.pdf}
    \caption{Diversity measure approaches for MCS: (a) data-based, and (b) classifier-based, taken from \cite{aksela2006}.}
    \label{ch2:diversity-measures}
\end{figure}




Next, a set of popular measures are presented based on the correct/incorrect outputs. First, the output of $\Psi_k$ can be represented as N-dimensional binary vector $y_k= \left[y_{1k}, y_{2k}, \ldots, y_{Nk}\right]^T$, such that $y_{ik}$=1, if $\Psi_k$ recognizes $\textbf{x}_i$ correctly, and 0 otherwise, $k=1,2,\dots,T$. Moreover, the relationship between a pair of classifiers ( $\Psi_j$, $\Psi_{k}$ ) can be represented as in Table \ref{classifier.relation}. 

  \begin{table}[!ht]
 \centering \scriptsize
 \caption{2 $\times$ 2 table of the relationship between a pair of classifiers.}
\label{classifier.relation}
\begin{tabular}{l|cc}
\hline
 & $\Psi_k$ correct (1) & $\Psi_k$ wrong (0)  .\\ \hline

$\Psi_{j}$ correct (1)  & $N^{11}$ & $N^{10}$  \\
$\Psi_{j}$ wrong (0)  & $N^{01}$& $N^{00}$ \\ \hline
\multicolumn{2}{c}{Total, $N=N^{00} + N^{01} + N^{10}+ N^{11} $} & \\

\hline
\end{tabular}
\end{table}   


\subsubsection{Pairwise diversity measures}

\begin{enumerate}
    \item \textit{The Q statistics :} Is a various statistic measure to measure the similarity of two classifier outputs. The \textit{Q} statistics for two classifiers, $\Psi_j$ and $\Psi_{k}$ is: 

\begin{equation}
\label{Q.statistics}
 Q_{j,k}=\frac{N^{11}N^{00}-N^{01}N^{10}}{N^{11}N^{00}+N^{01}N^{10}}
\end{equation}

Symbols are explained in Table \ref{classifier.relation}. \textit{Q} varies between -1 and 1. Classifiers that agree on the same objects will have positive \textit{Q} values, and those that make mistakes on different objects will have negative \textit{Q} values. The averaged statics over all pairs of classifiers is:


\begin{equation}
\label{averaged.qstatistics}
Q_{avg}=\frac{2}{T(T-1)} \mathop{\sum}\limits_{j=1}^{T-1} \mathop{\sum}\limits_{k=j+1}^T Q_{j,k}   \end{equation}
    
 \item \textit{The correlation coefficient 	$\rho$ :}  
 The correlation between two binary classifier outputs can be calculated as:   
   

\begin{equation}
\label{correlation.cofficient}
 \rho_{j,k}=\frac{N^{11}N^{00}-N^{01}N^{10}}{\sqrt{(N^{11} + N^{10})(N^{01} + N^{00})(N^{11} + N^{01})(N^{10} + N^{00})}}
\end{equation}    
    
For any two classifiers, $Q$ and $\rho$ have the same sign, and it can be proved that $|\rho|$ $\le |Q|$. 
  
  
\item \textit{The disagreement measure :}    
This measure has been used by HO \cite{ho1998} to measure the diversity in decision forests. It is the ratio between the number of samples on which one classifier disagree with another, to the total number of samples.

\begin{equation}
\label{disagreement.metr}
 Dis_{j,k}=\frac{N^{01} + N^{10}}{N^{11}+ N^{10}+N^{01}+N^{00}}
\end{equation}  
    
    
\item \textit{The double-fault measure :}     
It is defined as the proportion of samples on which both classifiers makes error, i.e.,

\begin{equation}
\label{default.measure}
 DF_{j,k}=\frac{N^{00}}{N^{11}+ N^{10}+N^{01}+N^{00}}
\end{equation}  

    
For all pairwise measures, the averaged values are calculated similarly to Equation (\ref{averaged.qstatistics}).       
\end{enumerate}
 
 
 
 
\subsubsection{Non-Pairwise diversity measures}

\begin{enumerate}
   
 \item \textit{The entropy measure E :}    The highest diversity among classifiers for a particular $\textbf{x}_i \in \textbf{X}$ is proved by $\floor{T/2}$ of the votes for $\textbf{x}_i$ with the same value (0 or 1) and the other $T - \floor{T/2}$ with the alternative value. If all decisions are 0's or all are 1's, then there is no diverse. Denote by $t(\textbf{x}_i$) the number of classifiers that correctly classify $\textbf{x}_i$, i.e,  $t(\textbf{x}_i$)=$\sum_{k=1}^{T} \Psi_{ik}$. On the basis of this concept, the diversity can be measured as:
    
   \begin{equation}
\label{Entropy.measure}
 E=\frac{1}{N} \sum_{i=1}^{N} \frac{1}{(T-\ceil{T/2})} \text{min}\{t(\textbf{x}_i),T-t(\textbf{x}_i)\}
\end{equation}   
  
$E$ ranges between 0 and 1, where 0 implies no difference, whereas high diversity is measured by 1.  
  
\item \textit{Kohavi-Wolpert variance :} It measures the average variance between the binomial distributions of the outputs of each classifier. This measure can be simply calculated by :
 
   \begin{equation}
\label{Kohavi.measure}
 KW=\frac{1}{NT^2} \sum_{i=1}^{N} t(\textbf{x}_i)(T-t(\textbf{x}_i))
\end{equation}   
  
  
 \item \textit{Measurement of interrater agreement $\kappa$ :} It is used to measure the level of agreement while correcting the chance. let's denote $\Bar{p}$ as the average individual classification accuracy, i.e.,
 
\begin{equation}
\label{individual.accuracy}
 \Bar{p}=\frac{1}{NT} \sum_{i=1}^{N}\sum_{k=1}^{T} \Psi_{ik}
\end{equation}   
  
 
 then, the interrater agreement could be formulated as:
 
\begin{equation}
\label{interrater.agreement}
 \kappa=1-\frac{\sum_{i=1}^{N}t(\textbf{x}_i)(T-t(\textbf{x}_i))}{NT(T-1)\Bar{p}(1-\Bar{p})} 
\end{equation} 
    
\end{enumerate}





 
\subsubsection{Uncorrelated Errors } \label{ch2_uncorrelated}

Standard diversity measures do not take into account that for classification purposes, identical correct decisions are preferred over identical erroneous decisions. It may be useful to analyze, in particular, the errors made by committee members. In this case, we will focus on the error types and whether this error occurs or not by the base models. If two classifiers incorrectly classify a sample into two separate categories, then this case is known as \textit{uncorrelated errors/diversity of errors}, as the predicted class is not the same despite the fact that both make mistake. The most difficult samples are the cases where all classifiers agree to the same incorrect class. Diversity measures that capture the type of error are initiatives and powerful as they can solve the following challenges:


    


\begin{itemize}[nosep]
\item[-] There are a few correct recognition results in most recognition tasks, while it is hard for the combination function to predict the correct output from this whole set of incorrect predictions. This can be solved by knowing the class of error.
\item[-] Classifiers that agree on the correct outcome should be credited rather than eliminated when selecting classifiers, however, this is conflicting with the principle of diversity maximization. 
\end{itemize}
 
\noindent Naturally, It is best if all classifiers make a correct prediction and it is better if we have fewer classifiers make a mistake. Even, for those mistakes, it is exceedingly helpful if the errors are different as often as possible, i.e. maximum diverse errors. Hence, the oracle-level (binary) classifier outputs should be interconnected with the predicted class category to identify the diversity of errors.


As stated in the introduction of this part, the difference in the mistakes made by the member classifiers really affects the performance. Given the following notations; $N_{same}^{00}$ indicates the number of samples when both classifiers are inaccurate and suggest the same decision. $N_{diff}^{00}$ stands for the number of samples when both classifiers are incorrect, but suggest different decisions. Next, let's present metrics \cite{aksela2006} that could be used to measure the diversity of errors: 

\begin{enumerate}
   
 \item \textit{Same-fault measure :} In an extension of the Double-Fault measure, the simultaneous fault could be restricted to measure when both classifiers are inaccurate  and suggest the same output. This can be measured for two classifiers $\Psi_j$ and $\Psi_{k}$ as:
 
 
 \begin{equation}
\label{samefault.measure}
 SF_{j,k}=\frac{N_{same}^{00}}{N^{11}+ N^{10}+N^{01}+N^{00}}
\end{equation}  
 
Then the averaged pairwise measures could be calculated similarly to Equation (\ref{averaged.qstatistics}). The optimal classifier set is picked based on the minimum measure.


\item \textit{Weighted count of errors and correct results:} It is designed to consider information about correct prediction. The number of samples that match specific cases can be weighted; based on classifier outputs, "both correct" is favorable and classifiers of this type are highly selected via increasing the weights. While "both incorrect and same" are highly penalized by assigning high negative weight, this can be defined for two classifiers $\Psi_j$ and $\Psi_{k}$ as: 
 
 
 \begin{equation}
\label{weighted.count.errors}
WCEC_{j,k}=N^{11}+\frac{1}{2}(N^{10}+N^{01}) -N_{diff}^{00}-5N_{same}^{00}
\end{equation}   
 
Moreover, based on the combiner's performance over the training set, the weights above could be optimized. The averaged pairwise measures are calculated similarly to Equation (\ref{averaged.qstatistics}), and the optimal classifier set is picked based on the maximum measure.  

\item \textit{Exponential error count :} As more errors are encountered the classifier capability will be hindered. Here, this concept is emphasized by counting the number of errors and assigning a weight in an exponential fashion. Assume $\Psi_{same}^{k \times 0}$ denote the number of errors made by $k$ classifiers to the same class, then the measure can be defined as:

\begin{equation}
\label{exponential.error.count}
EEC_{(\Psi_1,\dots,\Psi_T)}= \frac{\sum_{k=1}^{T} (\Psi_{same}^{k \times 0})^k }{N^{T \times 1}+1}
\end{equation} 
 
This measure considers all classifiers set. In addition, the correct classification is considered by scaling the exponential sum with $N^{T \times 1}$ (the number of samples for which every member classifier was correct). The optimal classifier set is picked based on the minimum measure.    
 
 \end{enumerate}






%%%%%%%%%%%%%%%%%%%%%%%%%%%%%%%%%%%%%%%%%%%%%%%
%
%   Machine Learning Modelling Approaches of Traffic Forecasting
%
%%%%%%%%%%%%%%%%%%%%%%%%%%%%%%%%%%%%%%%%%%%%%%%
\section{Soft Computing}
\label{sec:2_4_soft-computing}
Soft computing is one of the pioneer computing paradigms which resembles the human mind's remarkable capacity to think, understand, and solve difficult real-life problems \cite{das2013,konar2018}. Soft computing exploits the tolerance for imprecision, partial truth, and uncertainty to achieve robustness, tractability, low solution cost, and very high-performance \cite{zhu2015}. Modern machinery is more complex to be controlled and stabilized. The reasons for this difficulty is the lack of numerical models that describe exactly how they work, and the existence of many nonlinear and time-variant plants. Soft computing methods support intelligent control, nonlinear programming, optimization, and decision making support. Among those methods; fuzzy logic, genetic algorithms, artificial neural network. Soft computing has become popular with their wide applications to many research fields as speech recognition, communication, pattern recognition, signal processing, automatic control engineering. In their connectivity with the area of MCS, we present several soft computing methods that have been used to improve efficiency:


 \textbf{Fuzzy Logic:}
In domains and environments that are realistic, incomplete evidence inevitably emerges. During experiments, noise corruption or instrument errors may give rise to data when a certain value is measured, leading to incomplete data. In other scenarios, collecting the correct information can be excessively costly or unviable. In addition, using extra information from an expert, which is usually given by fuzzy logic and fuzzy sets, can be useful. Typically, data has a certain degree of vagueness. If the imprecision is significant, then the imprecise values must be handled in all the phases of learning and classification. To address the uncertainty of decision trees with minor disruptions, fuzzy sets and their underlying approximate reasoning capacities have been paired with decision trees in \cite{yuan1995,olaru2003}. The resulting trees exhibit increased tolerance to noise, and extend applicability to ambiguous or unclear circumstances. In \cite{bonissone2010}, a fuzzy random forest was suggested to increase the diversity of the trees through randomness, with the versatility of handling incomplete data. The numeric attributes are discretized through a fuzzy partition, so each internal node in the fuzzy decision tree constructs a child node for each fuzzy set of the partition. Then the membership degree of the examples to different fuzzy sets is determined to optimize the node-split attribute. In addition, a fuzzy combination method for one-class classifiers has been proposed in \cite{wilk2012}.             

\textbf{Evolutionary Algorithms:} Evolutionary algorithms uses mechanisms inspired by biological evolution, such as reproduction, mutation, recombination, and selection to solve complex problems. For example, genetic algorithm (GA) \cite{Holland1992} emulates the natural and human evolution, where common genes from parents can be transferred to their children. GA evolves an initial population of chromosomes; each chromosome represents one solution in the search space. The algorithm depends on applying crossover and mutation operators to explore and exploit different regions from the search space, which probably contains promising solutions. The process continues for a specific number of generations or till reaching threshold error or if the algorithm is trapped in a local minimum without the ability to find more accurate solutions. Ensemble systems are very complex architectures that need to be optimized. GA has been applied in \cite{ruta2005} to control the ensemble size (classifier subset selection) via optimizing several criterias. While in \cite{sikora2015} GA has been considered as a trainable combiner to tune the classifiers' weights for the fusion process. Furthermore,  Kuncheva et al. \cite{kuncheva2000d} applied two versions of GA for selecting feature subsets to be used by base models. While tsymbal et al. \cite{tsymbal2005} applied GA to optimize the diversity of the best-collected feature subsets.            

\textbf{Artificial Neural Networks:}
Neural networks were proved to be universal approximators with unlimited flexibility.  In any number of dimensions, they could approximate any classification boundary. However, this capability comes at a price. There is a need to train large systems with a huge number of parameters. Then, an acceptable architecture with the tuned parameters will be trusted for all future classifications. In the face of several local minima, all global optimization methods yield "optimal" parameters ($w$) that differ significantly from one run of the algorithm to the next. This reveals a great deal of randomness arising from various initial weights ($w^0$) and various sequencing of training examples. This randomness appears to distinguish between network errors. The final weights correspond to various ways of identifying the training pattern. Hence, the collective decision created by several ANNs may, therefore, be much less fallible than any network. In \cite{kim2010}, the ANNs in the form of base classifiers were generated to form bagging and boosting ensembles to predict the bankruptcy. While in \cite{plesinger2018} a convolutional neural network is linked to a shallow neural network to classify arrhythmia in a Holter ECG signal.


\textbf{Swarm Intelligence:}  SI algorithms are defined as \textit{"nature-inspired algorithms that concern the collective, emerging behavior of multiple, interacting agents that follow some simple rules"} \cite{hassanien2018}. These algorithms mimic the social behavior of swarms/groups of creatures in nature.  In \cite{mirjalili2014}, the authors discussed the benefits of SI over evolutionary algorithms.  Most SI methods consider exploration and exploitation in their working mechanisms \cite{mirjalili2010}. The popularity of those algorithms returns to; the simplicity of inspiration, flexibility, derivative-free mechanism, and local optimum avoidance \cite{mirjalili2014}. Firefly algorithm has been applied in \cite{krawczyk2015} to combine the ensemble pruning with a weighted classifier fusion module. In addition, the Ant colony algorithm has been incorporated to optimize the decision forest \cite{kozak2015} to provide self-adaptability with the classification task. Recent SI algorithms will be discussed in Sections \ref{Sec:4_3_5-MFO}, \ref{GWOalgo}, and \ref{Sec:4_3_5-WOA}.










% state-of-the-art
\chapter{State-of-the-art}
\label{cha:3_State-of-the-art}

This chapter provides an in-depth review of recent advancements in stream classification, addressing key challenges in dynamic data environments, such as concept drift, imbalanced multiclass scenarios, class overlap, ensemble selection, new class emergence, and transfer learning. Section \ref{sec:3_1_concept_drift} explores methodologies for real-time concept drift detection and adaptation to maintain classification accuracy. Section \ref{sec:3_2_ensemble} focuses on dynamic classifier ensemble selection strategies to enhance performance in evolving data streams. Section \ref{sec:3_3_imbalanced} tackles imbalanced multiclass problems, discussing oversampling techniques to balance minority classes in drifted streams. Section \ref{sec:3_4_emergence} examines methods for integrating new classes into existing systems to improve adaptability, while Section \ref{sec:3_5_transfer_learning} highlights the role of transfer learning in leveraging knowledge from related tasks to improve performance, particularly in scenarios with limited labeled data. Section \ref{sec:3_6_comparsion} provides a comparative analysis of recent works, evaluating contributions, limitations, and research gaps, and Section \ref{sec:3_7_remartks} concludes the chapter by outlining critical challenges and directions for future research.



%%%%%%%%%%%%%%%%%%%%%%%%%%%%%%%%%%%%%%%%%%%%%%%
%
%   Taxonomies in the Traffic Forecasting Field
%
%%%%%%%%%%%%%%%%%%%%%%%%%%%%%%%%%%%%%%%%%%%%%%%
\section{Concept Drift}
\label{sec:3_1_concept_drift}
Concept drift refers to changes in the underlying data distribution over time, which can reduce the accuracy of previously trained machine learning models \cite{baena2006early, madkour2023historical, tan2022information}. Detecting and responding to concept drift is crucial for maintaining model performance. Several detection methods have been proposed to address this challenge. The Drift Detection Method (DDM) \cite{gama2004learning, bifet2009new} uses a statistical test to identify significant error rate increases, signaling concept drift. The Early Drift Detection Method (EDDM) \cite{gama2004learning, adams2023explainable} extends DDM by considering a moving window of recent data. ADWIN \cite{gama2004learning, adams2023explainable} employs a sliding window to monitor statistical differences and adjusts the window size to adapt to drift patterns. The Kolmogorov-Smirnov windowing method (KSWIN) \cite{adams2023explainable} calculates the Kolmogorov-Smirnov distance to detect drift, while Hoeffding's bounds with moving average test (HDDMA) and its variant HDDMW \cite{gama2004learning, bifet2009new} compute bounds for the true mean to detect distribution changes. Lastly, the Page-Hinkley method \cite{page1954continuous} tracks the cumulative sum of errors, detecting drift when the sum exceeds a threshold. These methods enable machine learning models to adapt to evolving data streams, enhancing model performance and robustness.

%%%%%%%%%%%%%%%%%%%%%%%%%%%%%%%%%%%%%%%%%%%%%%%
%
%   General-purpose Automated Machine Learning
%
%%%%%%%%%%%%%%%%%%%%%%%%%%%%%%%%%%%%%%%%%%%%%%%
\section{Classifier Ensemble Selection}

\label{sec:3_2_ensemble}
This study focuses on the overproduce-and-select approach for classifier ensemble selection methods [23] [88] [20]. The primary objective of classifier ensemble selection is to identify the optimal subset of classifiers from a larger ensemble, considering various criteria such as performance measures, diversity metrics, meta-learning techniques, and performance estimation approaches. This selection process aims to reduce computational complexity, enhance efficiency, and improve overall ensemble performance, making it highly valuable for real-world applications. By carefully selecting a smaller subset of classifiers, ensemble selection strikes a balance between accuracy and computational resources, adapting to the evolving nature of the data stream. This approach leverages the strengths of different classifiers and adjusts the ensemble composition to handle changing conditions effectively. The goal is to enhance the accuracy, robustness, and overall performance of classification models in dynamic and challenging scenarios. There are two main approaches to the selection process: static and dynamic selection. Static selection assigns classifiers to specific partitions of the feature space, while dynamic selection chooses a classifier specifically for each unknown data sample based on its local competencies. Dynamic Ensemble Selection (DES) is a widely recognized approach that selects the best classifiers for each test instance, considering their competence within the local region of competence. The Randomized Reference Classifier proposed by Woloszynski and Kurzynski [21] stands out among various approaches. This classifier introduces randomness through beta distribution, enhancing adaptability and robustness. By considering the stochastic nature of class supports, the Randomized Reference Classifier can potentially improve classification performance in concept drift scenarios. However, it is important to note that employing diversity measures during the classifier selection process, as demonstrated by Lysiak [22], may lead to smaller ensembles but does not necessarily enhance classification accuracy. Overall, the overproduce-and-select approach for classifier ensemble selection methods offers a comprehensive framework for addressing the challenges associated with concept drift. By dynamically adapting the ensemble composition and leveraging the competencies of individual classifiers, this approach aims to improve classification performance, efficiency, and adaptability in dynamic and challenging scenarios.
%%%%%%%%%%%%%%%%%%%%%%%%%%%%%%%%%%%%%%%%%%%%%%%
%
%   Machine Learning for Traffic Forecasting
%
%%%%%%%%%%%%%%%%%%%%%%%%%%%%%%%%%%%%%%%%%%%%%%%
\section{Imbalanced data Streams}
\label{sec:3_3_imbalanced}
In imbalanced data classification, three main approaches have been identified \cite{yin2022graph}, with this study emphasizing the first category, which addresses imbalanced data streams through sampling methods, particularly oversampling \cite{ren2023grouping}. This technique creates synthetic instances to achieve balanced class distributions \cite{nitesh2002smote, han2005borderline, bunkhumpornpat2009safe, maciejewski2011local}. Imbalances can arise in both binary and multi-class settings, with this research concentrating on multi-class oversampling strategies.

To address multi-class imbalances, Multi-Label SMOTE (MLSMOTE) \cite{charte2015mlsmote} extends SMOTE to multi-class learning by generating synthetic examples for minority class labels and ensuring their proper assignment. A more recent technique, Multi-Label Synthetic Oversampling based on Local Label Imbalance (MLSOL) \cite{yin2022graph}, improves upon MLSMOTE by targeting local imbalances within multi-class classification. MLSOL uses distinct sampling strategies for each label, offering superior performance in classification accuracy and other metrics. It generates synthetic samples from minority class instances in a restricted neighborhood, improving computational efficiency and reducing overfitting, making it a promising technique that outperforms MLSMOTE in several areas.
\section{Streams with Emerging New Classes (SENC)}
\label{sec:3_4_emergence}
Existing approaches have been proposed to detect and handle the emergence of new classes in streaming data. Clustering-based methods, such as SACCOS \cite{gao2020saccos}, ECSMiner \cite{masud2010classification}, and SAND \cite{haque2016sand}, employ clustering techniques to identify new class emergence. However, these methods require access to true labels for either parts or all instances, limiting their practical applicability. Similarly, SENC-MaS \cite{mu2017streaming} uses matrix sketches for detecting emerging new classes but assumes the availability of true label information for all instances. In contrast, tree-based methods like SENCForest \cite{mu2017classification} and SEEN \cite{zhu2020semi} utilize anomaly detection techniques to identify new classes, often with limited or no label information. However, these methods often suffer from high false positive rates and runtime inefficiencies. Another approach, SENNE \cite{cai2019nearest}, focuses on exploiting local information using the nearest neighbor ensemble for improved detection performance.  Nevertheless, the absence of an effective model retirement mechanism in SENNE results in longer runtimes than alternative methods. The k-nearest Neighbor Ensemble-based method (KNNENS) \cite{zhang2022knnens} method emerges as a promising solution for the challenges of streaming emerging new class problems. By effectively utilizing a k-nearest neighbor-based hypersphere ensemble and incorporating model updates, the KNNENS approach tackles the issues of new class detection and known class classification within a unified framework. It is worth noting that an explicit limitation of existing methods is their lack of utilization of concept drift techniques for detecting emerging new classes and retraining the classification model. This limitation highlights the need for approaches that can effectively handle concept drift while addressing the emergence of new classes in streaming data.

\section{Transfer Learning} 
\label{sec:3_5_transfer_learning}
Transfer learning has gained significant attention for addressing distribution disparities between source and target domains, with methods falling into two categories: instance re-weighting and feature matching \cite{long2013transfer}. Instance re-weighting methods, such as Kernel Mean Matching (KMM) \cite{long2014transfer}, Kullback-Leibler Importance Estimation Procedure (KLIEP) \cite{sun2011two}, and TrAdaBoost \cite{freund1996experiments}, adjust the weights of source instances to align with the target domain. These methods have been extended to multisource transfer learning (e.g., MsTrAdaBoost \cite{sun2016return}). Feature matching approaches, including Transfer Component Analysis (TCA) \cite{sun2016return} and CORAL \cite{rahman2020correlation}, focus on aligning feature representations between domains, often through transformations that minimize distribution differences. However, negative transfer, where transferred knowledge harms performance, is a challenge addressed by techniques like Transfer Joint Matching (TJM) \cite{zhong2009cross}. Additionally, methods such as HE-CDTL \cite{powers2020evaluation} handle concept drift in transfer learning by incorporating historical knowledge and class-wise weighted ensembles. Finally, online transfer learning techniques like Melanie \cite{sun2016return} manage non-stationary environments by dynamically adjusting model weights to accommodate concept drift.

% Define colors
\definecolor{headerColor}{HTML}{4F81BD}
\definecolor{rowColor1}{HTML}{B8CCE4}
\definecolor{rowColor2}{HTML}{DCE6F1}
\definecolor{textColor}{HTML}{000000}
\definecolor{highlightColor}{HTML}{00B0F0}
\section{Comparsion} 
\label{sec:3_6_comparsion}

This section presents a critical comparison of closely related works addressing the challenges of imbalanced multiclass streams \ref{sec:3_6_1_related_work_imbalanced}, the emergence of new classes \ref{sec:3_6_2_related_work_emergence}, and the integration of transfer learning \ref{sec:3_6_2_related_work_transfer} within streaming environments. The increasing complexity of real-world data streams necessitates advanced methodologies that can effectively manage the intricacies of these challenges. By examining various approaches in the literature, the goal is to highlight their contributions, strengths, and limitations in dealing with imbalanced data distributions, adapting to new class occurrences, and leveraging transfer learning techniques. This comparative analysis highlights the current state of research while emphasizing specific gaps and unresolved challenges, paving the way for more robust and adaptive solutions in streaming data classification.


\subsection{Imbalanced Stream}
\label{sec:3_6_1_related_work_imbalanced}

Addressing class imbalances is critical in multi-class classification. Multi-Label SMOTE (MLSMOTE) \cite{charte2015mlsmote} enhances classifier performance by generating synthetic examples for minority class labels using neighboring examples in the feature space. However, Multi-Label Synthetic Oversampling based on Local label imbalance (MLSOL) \cite{liu2022multi} improves upon this by employing tailored sampling strategies for each label to address local imbalances. Research shows that MLSOL outperforms MLSMOTE in classification accuracy and computational efficiency by generating synthetic samples from minority instances within restricted neighborhoods, resulting in a more compact and efficient dataset while reducing overfitting. Figure \ref{fig:mlsmote_mlsol} illustrates that MLSOL is more likely to select $x1$ as a seed instance because it is surrounded by more neighbors of the opposite class for $l3$. MLSMOTE assigns the label vector [0,1,0] to all synthetic instances based on their neighbors. In contrast, MLSOL creates more diverse instances by assigning labels according to their location. Moreover, synthetic instances $c2$ and $c3$ generated by MLSMOTE introduce noise, whereas MLSOL copies the labels of the nearest instance to the new examples. In summary, MLSMOTE tends to generate new instances biased toward the dominant class in the local area, whereas MLSOL effectively explores and exploits both the feature and label space.
\begin{figure*}[!ht]

    \begin{center}
      \includegraphics[width=1\textwidth]{3_State-of-the-art/fig/mlsmote_mlsol.png}
    \end{center}

    \caption{Comparsion of MLSMOTE and MLSOL Generated Instances \cite{liu2022multi}.\\ \textcolor{gray}{\fontsize{10}{0}\selectfont DOI: 110.1016/j.patcog.2021.108294}}
    \label{fig:mlsmote_mlsol}

    \end{figure*}
    
    Table \ref{table:imbalanced} compares MLSMOTE and MLSOL, two methods for addressing imbalanced data in multi-class classification. MLSMOTE generates synthetic examples for minority classes to balance distributions but may blur class distinctions and struggles with overlapping classes, leading to misclassification. MLSOL improves upon this by using localized sampling strategies to address local imbalances more effectively. However, both methods face limitations with overlapping class boundaries, which impact their overall classification accuracy. Despite MLSOL's precision in handling local imbalances, the challenge of overlapping classes remains significant for both approaches.
\begin{table*}[!ht]

    \centering
    \caption{Comparison of the MLSMOTE and MLSOL Methods.}
    \label{table:imbalanced}

    \small % Reduce font size
    \renewcommand{\arraystretch}{1} % Reduce cell padding
    \setlength{\tabcolsep}{4pt} % Reduce cell padding
    \setlength{\arrayrulewidth}{0.15mm}

    \begin{tabularx}{\textwidth}{|>{\centering\arraybackslash\bfseries}p{2cm}|
                                       >{\raggedright\arraybackslash}X|
                                       >{\raggedright\arraybackslash}X|
                                       >{\raggedright\arraybackslash}X|}
    \hline
    \textbf{Method} & \textbf{Theory} & \textbf{Advantages} & \textbf{Limitations} \\ 
    \hline
    \textbf{MLSMOTE \cite{charte2015mlsmote}} & 
    MLSMOTE significantly enhances classifier performance by generating synthetic examples for each minority class label. & 
    Generating synthetic examples for each minority class label. & 
    \begin{itemize}[leftmargin=*]
        \item Random synthetic samples may be related to the majority class.
        \item Overlapping classes.
    \end{itemize} \\ 
    \hline
    \textbf{MLSOL \cite{liu2022multi}} & 
    MLSOL systematically combats local imbalances within the domain of multi-class classification by employing distinct sampling strategies for each label. & 
    Generating synthetic examples for each minority class label within a restricted neighborhood. & 
    \begin{itemize}[leftmargin=*]
        \item Overlapping classes.
    \end{itemize} \\
    \hline
    \end{tabularx}
    \end{table*}

\subsection{Emergence of new classes}
\label{sec:3_6_2_related_work_emergence}


Detecting and adapting to new classes in streaming data is essential for maintaining classification accuracy. Tree-based methods like SENCForest \cite{mu2017classification} and SEEN \cite{zhu2020semi} use anomaly detection but suffer from high false positive rates and inefficiencies. SENNE improves detection using a nearest neighbor ensemble but has longer runtimes due to ineffective model retirement. The k-Nearest Neighbor Ensemble-based method (KNNENS) \cite{zhang2022knnens} enhances new class detection and known class classification through hypersphere ensembles and dynamic model updates. However, all these methods struggle to handle concept drift effectively, which is critical for detecting new classes and updating classification models.

Figures \ref{fig:SENCForest}, \ref{fig:SENNE}, and \ref{fig:KENNE} illustrate key approaches for classifying emerging and known classes. SENCForest divides the space into three regions (normal, outlying, and anomaly) and detects new classes using threshold path lengths. SENNE uses hyperplanes in three dimensions ($x1$, $x2$, and $x3$) to classify instances as emerging or known based on class rankings. KNNENS employs hyperplanes for all class samples and uses a voting mechanism to classify instances as emerging or known. These visualizations emphasize the differences in how SENNE and KNNENS handle class classification.

\begin{figure*}[!ht]
    \centering
    \includegraphics[width=0.80\textwidth]{3_State-of-the-art/fig/SENCForst.png}
    \caption{Overview of SENCForest Detection Flow Diagram \cite{mu2017classification}. \\
    \textcolor{gray}{\fontsize{10}{0}\selectfont DOI: 10.1109/TKDE.2017.2691702}}
    \label{fig:SENCForest}
\end{figure*}
    

\begin{figure*}[!ht]
    \begin{center}
        \includegraphics[width=.45\textwidth]{3_State-of-the-art/fig/senne0.png} 
        \includegraphics[width=.45\textwidth]{3_State-of-the-art/fig/senne.png} 
        (a)\hspace{6.5cm}(b)
    \end{center}
    \caption{Overview of the Stream Emerging Nearest Neighbor Ensemble (SENNE) \cite{zhu2020semi}.}
    \label{fig:SENNE}
    \end{figure*}
    \vline
    \begin{figure*}[!ht]    
        \begin{center}
            \includegraphics[width=.45\textwidth]{3_State-of-the-art/fig/kenne0.png} 
            \includegraphics[width=.45\textwidth]{3_State-of-the-art/fig/kenne.png}
            (a)\hspace{6.5cm}(b)
            \end{center}
    
        \caption{Overview of the k-nearest Neighbor Ensemble-based \cite{zhang2022knnens} (KENNE).}
        \label{fig:KENNE}
        \end{figure*}
        
        Table \ref{table:emerging} compares three emerging class detection methods: SENCForest, SENNE, and KNNENS. SENCForest uses iForest \cite{wang2010negative} for anomaly detection and a threshold path for identifying new classes, serving as both an unsupervised detector and supervised classifier, but it is prone to false positives and relies on a complex threshold mechanism. SENNE utilizes a nearest neighbor-based hypersphere ensemble to analyze local neighborhood information, effectively handling varying geometric distances between classes, though it assumes static class distributions and has lengthy update times. KNNENS improves by using a hypersphere ensemble for all classes, reducing false positives and enabling updates without true labels, but it shares SENNE's limitation of assuming unchanged known class distributions.
\begin{table*}[!ht]

    \centering
    \caption{Comparison of the SENCForest, SENNE, and KENNE methods.}
    \label{table:emerging}
    \small % Reduce font size
    \renewcommand{\arraystretch}{1} % Reduce cell padding
    \setlength{\tabcolsep}{4pt} % Reduce cell padding
    \setlength{\arrayrulewidth}{0.15mm}
    \begin{tabularx}{\textwidth}{|>{\centering\arraybackslash\bfseries}p{2cm}|
                                       >{\raggedright\arraybackslash}X|
                                       >{\raggedright\arraybackslash}X|
                                       >{\raggedright\arraybackslash}X|}
    \hline
    \textbf{Method} & \textbf{Theory} & \textbf{Advantages} & \textbf{Limitations} \\ 
    \hline
    \textbf{SENCForst \cite{mu2017classification}} & 
    employs anomaly detection method iForest for a new class detection and then applies threshold path to detect the anomalies. & 
    SENCForest serves as both an unsupervised anomaly detector and a supervised classifier.&
    \begin{itemize}[leftmargin=*]
        \item Potential for High false positives.
        \item Dependency on path length threshold (more complexity).
    \end{itemize} \\ 
    \hline
    \textbf{SENNE \cite{zhu2020semi}} & 
    nearest neighbor-based hypersphere of one class ensemble to explore local neighborhood information and sort distance to calculate distance. & 
    SENNE is able to handle both the low and high geometric distance between two classes in the feature space. & 
    \begin{itemize}[leftmargin=*]
        \item Assumes that the distribution of known classes remains unchanged.
        \item Take long time for update.
    \end{itemize} \\ 
    \hline
    \textbf{KENNE \cite{zhang2022knnens}} & 
    nearest neighbor-based hypersphere of all class ensemble to explore local neighborhood information. & 
    KNNENS to reduce false positives for the new class. KNNENS does not require true labels to update the model. & 
    \begin{itemize}[leftmargin=*]
        \item Assumes that the distribution of known classes remains unchanged.
    \end{itemize} \\
    \hline
    \end{tabularx}
    \end{table*}



\subsection{Transfer Learning}
\label{sec:3_6_2_related_work_transfer}

In transfer learning, CORAL, Melanie, and HE-CDTL are key approaches related to the third proposed method \ref{chapter:6_transfer_learning}. CORAL \cite{sun2016return} aligns sub-space bases through second-order statistics using a learned transformation matrix, minimizing domain discrepancies and negative transfer. Melanie \cite{dong2019multistream} employs an online ensemble learning strategy to address non-stationary environments by incrementally training models from source and target domains, dynamically adjusting weights, and combining models via a weighted-sum approach. HE-CDTL extends these concepts specifically for Concept Drift Transfer Learning (CDTL), leveraging historical and source domain knowledge through a class-wise weighted ensemble and AW-CORAL to reduce domain disparities. Experiments show HE-CDTL outperforms baseline methods, demonstrating its efficacy in managing transfer learning under concept drift.
Table \ref{table:transfer} compares CORAL, Melanie, and HE-CDTL for transfer learning. CORAL minimizes domain discrepancies and reduces negative transfer by projecting source data into the target domain using a transformation matrix and Singular Value Decomposition (SVD) but struggles with non-stationary and heterogeneous data. Melanie addresses non-stationary environments in online learning by dynamically training and combining models from source and target domains, yet faces challenges with the complexity of online learning and data heterogeneity. HE-CDTL reduces domain shifts by aligning second-order statistics and leveraging historical knowledge but depends on source domain quality and also encounters issues with heterogeneous data.
\begin{figure*}[!ht]

    \begin{center}
        \includegraphics[width=.80\textwidth]{3_State-of-the-art/fig/coral.png} 
    \end{center}
    \caption{Overview of CORrelation ALignment (CORAL)\cite{sun2016return}.}
    \label{coral_fig}
    \end{figure*}
    \begin{figure*}[!ht]    
        \begin{center}
            \includegraphics[width=.80\textwidth]{3_State-of-the-art/fig/cdtl.png} 
        \end{center}
        \caption{Overview of Concept Drift Transfer Learning (CDTL) \cite{sun2016return}.}
        \label{cdtl_fig}

        \end{figure*}

\begin{table*}[!ht]
    \centering
    \caption{Comparison of the CORAL, Malanie, and CDTL Methods.}
    \label{table:transfer}
    \small % Reduce font size
    \renewcommand{\arraystretch}{1} % Reduce cell padding
    \setlength{\tabcolsep}{4pt} % Reduce cell padding
    \setlength{\arrayrulewidth}{0.15mm}
    \begin{tabularx}{\textwidth}{|>{\centering\arraybackslash\bfseries}p{2cm}|
                                       >{\raggedright\arraybackslash}X|
                                       >{\raggedright\arraybackslash}X|
                                       >{\raggedright\arraybackslash}X|}
    \hline
    \textbf{Method} & \textbf{Theory} & \textbf{Advantages} & \textbf{Limitations} \\ 
    \hline
    \textbf{CORAL \cite{sun2016return}} & 
    Correlation Alignment (CORAL) uses a learned transformation matrix and Singular Value Decomposition (SVD)  to project the source instances into the target domain. & 
    CORAL can minimize domain discrepancy across
source and target domains, meanwhile reducing the negative
knowledge transfer. & 
    \begin{itemize}[leftmargin=*]
        \item Non-stationary environments.
        \item Heterogenous multisource.
    \end{itemize} \\ 
    \hline
    \textbf{Melanie \cite{dong2019multistream}} & 
    Multi-sourcE onLine TrAnsfer
learning for Non-statIonary Environments (Melanie). utilize the class-wise weighted . & 
It considers
an online problem in which the data in source and target
domains are generated from non-stationary environments. & 
    \begin{itemize}[leftmargin=*]
        \item Based on the online learning  only.
        \item Heterogenous multisource.
    \end{itemize} \\
    \hline
    \textbf{HE-CDTL \cite{sun2016return}} & 
    HE-CDTL uses the class-wise weighted and domain wise ensemble for historical knowledge and reduce the disparities between the source and target domains . & 
    HE-CDTL minimizes domain shift by aligning the second-order statistics of source and target distributions. & 
    \begin{itemize}[leftmargin=*]
        \item Depend on source domain quality.
        \item Heterogenous multisource.
    \end{itemize} \\
    \hline
    \end{tabularx}
    \end{table*}

\section{Related Works Challenges} 
\label{sec:3_7_remartks}

By comparing the literature on ensemble learning for classification tasks, the proposals in this thesis differ from other studies in several ways:

\begin{enumerate}
    
    \item [-] As evident from our literature review on imbalanced streams, most studies have concentrated on generating synthetic samples while ignoring class overlap. \textit{To address this challenge}, we propose an approach to generate non-overlapping classes in imbalanced streams.

    \item [-] Oversampling techniques often perform inefficiently in the presence of concept drift. \textit{To tackle this issue}, we introduce a methodology that selects the oversampling technique based on the current and historical distribution of the stream chunks.
    
    \item [-] Our literature review on non-stationary environments reveals that most works focus on detecting emerging new classes while overlooking distribution changes. \textit{To overcome this challenge}, we propose a combined approach utilizing Dynamic Ensemble Selection (DES) to select the best classifier for each chunk based on stream distribution, k-means clustering, and concept drift to address both emerging new class detection and distribution changes.
    
    \item [-] In our literature review on transfer learning, we observed that most studies focus on homogeneous multisource transfer and neglect heterogeneous multisources in non-stationary environments. \textit{To resolve this issue}, we propose a combined approach integrating Dynamic Ensemble Selection (DES), Concept Drift Transfer Learning (CDTL), eigenvector techniques, and concept drift to address heterogeneous transfer learning in non-stationary environments.

\end{enumerate}



% the back matter: appendix and references close the thesis
\backmatter
\include{0_frontmatter/summary-arabic}


%: ----------------------- appendix ------------------------

%\appendix


%: ----------------------- bibliography ------------------------



% The section below defines how references are listed and formatted
% The default below is 2 columns, small font, complete author names.
% Entries are also linked back to the page number in the text and to external URL if provided in the BibTex file.


% Original version:

% PhDbiblio-url2 = names small caps, title bold & hyperlinked, link to page
%\begin{multicols}{2} % \begin{multicols}{ # columns}[ header text][ space]
%\begin{tiny} % tiny(5) < scriptsize(7) < footnotesize(8) < small (9)
%
%\bibliographystyle{Latex/Classes/PhDbiblio-url2} % Title is link if provided
%\renewcommand{\bibname}{References} % changes the header; default: Bibliography
%
%\bibliography{9_backmatter/references} % adjust this to fit your BibTex file
%
%\end{tiny}
%\end{multicols}



% Show all bibliography entries
%\nocite*

% If we want bibliography backreference, use unsrt first and the desidered one after

%\bibliographystyle{unsrt} % Defines the bibliography style

%\bibliographystyle{alpha} % Defines the bibliography style

\bibliographystyle{Latex/StyleBST/IEEEtran} % Defines the bibliography style

%\bibliographystyle{apa-good} % Defines the bibliography style
%\bibliographystyle{natbib} % Defines the bibliography style

%\bibliographystyle{plainurl}

%\renewcommand{\bibname}{References} % changes the header; default: Bibliography

%To include the references/works cited/bibliography in your Table of Contents, right before the bibliography command, use the command
%\addcontentsline{toc}{section}{References}
%

\bibliography{11_backmatter/references} % adjust this to fit your BibTex file

% --------------------------------------------------------------
% Various bibliography styles exit. Replace above style as desired.

% in-text refs: (1) (1; 2)
% ref list: alphabetical; author(s) in small caps; initials last name; page(s)
%\bibliographystyle{Latex/Classes/PhDbiblio-case} % title forced lower case
%\bibliographystyle{Latex/Classes/PhDbiblio-bold} % title as in bibtex but bold
%\bibliographystyle{Latex/Classes/PhDbiblio-url} % bold + www link if provided

%\bibliographystyle{Latex/Classes/jmb} % calls style file jmb.bst
% in-text refs: author (year) without brackets
% ref list: alphabetical; author(s) in normal font; last name, initials; page(s)

%\bibliographystyle{plainnat} % calls style file plainnat.bst
% in-text refs: author (year) without brackets
% (this works with package natbib)


% --------------------------------------------------------------


%: Declaration of originality
%\include{9_backmatter/declaration}





\end{document}
