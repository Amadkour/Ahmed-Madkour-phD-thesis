
\section{Thesis Motivations}
\label{sec:1_introduction_motivation}

The swift progress in machine learning, particularly in real-time data streams, introduces new challenges that demand innovative solutions. This thesis seeks to address the limitations of traditional models in dynamic, non-stationary data environments. The motivations behind this research are as follows:
\begin{itemize}
    \setlength{\itemsep}{0pt}
    \setlength{\parskip}{0pt}
    \item \textbf{Adapting to Emerging Classes in Real-Time:} New classes within data streams can cause a decline in accuracy. The goal is to develop adaptive mechanisms that quickly incorporate new classes, maintaining system relevance.
    \item \textbf{Proactive Management of Concept Drift:} Concept drift degrades model performance as data properties change. This research focuses on strategies to detect and manage concept drift to maintain model accuracy.
    \item \textbf{Dynamic Optimization of Classifier Ensembles:} To address emerging classes and shifting distributions, the goal is to develop techniques for the real-time optimization of classifier ensembles.
    \item \textbf{Addressing Multi-Class Imbalance:} Imbalanced multi-class distributions often lead to biased classification. The thesis aims to create methods that address class imbalances dynamically, ensuring fair classification.
    \item \textbf{Enhancing Transfer Learning in Non-Stationary Environments:} Transfer learning can suffer from negative transfer in non-stationary settings. The research focuses on frameworks that minimize negative transfer and enhance knowledge sharing across diverse domains.
\end{itemize}
