\section{Contributions}
\label{sec:1_introduction_contribution}
Our research endeavors to create advanced frameworks designed for effectively managing non-stationary data streams while prioritizing high accuracy and minimizing computational complexity. The key contributions of our work can be summarized as follows:
\begin{itemize}
    \item \textbf{Incorporation of Concept Drift Detection and Ensemble Classifier:} Our primary innovation involves incorporating a concept drift detection method alongside an ensemble classifier. This integration enables real-time adaptation and refinement of our proposed framework in response to transfer learning in non-stationary environments. This methodology ensures the continuous evolution of our classification model in alignment with the changing data landscape.
    \item \textbf{Dynamic Classifier Ensemble Framework for Emergence Class Problem:} We introduce a classification framework that harnesses dynamic classifier ensemble techniques to address the emergence class problem. This framework enhances the performance of classification algorithms when dealing with non-stationary data streams featuring emerging new classes. By dynamically adjusting the ensemble based on data characteristics, our framework improves the accuracy and effectiveness of classification models, effectively tackling challenges associated with emerging new classes.
    \item \textbf{Introduction of Precise Weighting Method for Local Classifiers of the transfer learning framework:} A significant advancement in our research is the introduction of a precise weighting method to assess the significance of each local classifier within the ultimate classifier. This method enhances the accuracy of the overall classifier by providing a nuanced evaluation of individual classifier contributions.
    \item \textbf{Development of Innovative Framework using Eigenvector Technique for addressing heterogenous transfer learning problem:} A significant stride in our research involves the creation of an innovative framework, harnessing the power of the eigenvector technique. This framework is specifically designed to tackle the challenges posed by heterogeneous transfer learning problems. By leveraging the eigenvector technique, our framework enables the seamless transfer of knowledge from diverse source domains to the target domain, thereby enhancing adaptability and overall performance.
    \item \textbf{Dynamic Adjustment Method to Multi-class Imbalanced Data:} Our classification framework introduces a dynamic adjustment method tailored for multi-class imbalanced data scenarios. It seamlessly incorporates concept drift detection mechanisms and optimizes classifier ensemble selections, aiming to significantly improve classification accuracy in the context of multi-class imbalanced non-stationary streams.
    \item \textbf{Adaptive Method for Class Imbalance:} Additionally, we propose an adaptive method for addressing the class imbalance issue. This method considers data distributions and historical instances of class imbalance, especially in cases where class overlap occurs within multi-class and drifted data streams. The adaptive approach improves classification performance by selecting the most suitable oversampling method based on the unique characteristics of the data stream.
\end{itemize}
