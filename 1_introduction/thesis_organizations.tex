
\section{Structure of the Dissertation}
\label{sec:1_introduction_organizations}
The structure of the remainder of this thesis dissertation is outlined below:
\begin{itemize}	
	\setlength{\itemsep}{2pt}
    \setlength{\parskip}{2pt}
	\item \textbf{Chapter~\ref{cha:2_background}} reviews background about concept drift, concept drift types, concept drift components, adapting types.  
	\item \textbf{Chapter~\ref{cha:3_State-of-the-art}} reviews state-of-the-art  concept drift, classifier ensemble selection, imbalanced data streams, Streams with Emerging New Classes (SENC), and transfer learning.  
	\item \textbf{Chapter~\ref{chapter:4_Imbalanced_Multiclass}
	} presents the first proposal to build an effective proposed approch for  handling Imbalanced Multi-class Drifted Data streams.
	\item \textbf{Chapter~\ref{chapter:5_emerging}} provides the second proposal to adressing emerging new classes in incremental streams via concept drift techniques. 
	\item \textbf{Chapter~\ref{chapter:6_transfer_learning}} introduces a novel approach to addressing the heterogeneous transfer learning problem in incremental data streams using concept drift techniques.
	
	\item \textbf{Chapter~\ref{chapter:7_Conclusions}} revisits the main goal and specific objectives posted earlier. This chapter summarizes the main contributions of the thesis and outlines potential directions for future research.

\end{itemize}