\section{Challenges}
\label{sec:1_introduction_challange}
Within the swiftly evolving realm of machine learning, the proliferation of high-speed data streams presents a significant hurdle — the proficient management of non-stationary data environments. This challenge is marked by dynamic fluctuations in statistical attributes, fundamental concepts, and data distributions over time. The crux of the issue emerges as models, initially trained on historical data, experience a decline in accuracy when confronted with new data influenced by concept drift. This includes scenarios such as:
\begin{itemize}
    \setlength{\itemsep}{0pt}
    \setlength{\parskip}{0pt}
    \item Multi-class imbalanced data streams
    \item Overlapping classes
    \item Emergence of new classes
    \item Heterogeneous transfer learning
\end{itemize}
However, addressing these challenges yields substantial benefits. By effectively managing non-stationary data environments, organizations can enhance the adaptability and resilience of their machine learning systems. This enables more robust decision-making processes, improved predictive capabilities, and heightened performance across diverse contexts. Moreover, it fosters innovation and agility, empowering organizations to stay ahead in dynamic markets and evolving scenarios.