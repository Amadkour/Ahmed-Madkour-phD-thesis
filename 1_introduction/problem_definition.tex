\section{Problem Definition}
\label{sec:1_introduction_problem}
The rapid growth of high-speed data streams presents significant challenges for traditional machine learning models, particularly in non-stationary environments where data properties, concepts, and distributions evolve over time. These dynamic conditions cause models, initially trained on historical data, to experience a decline in accuracy when confronted with new data. The key challenges in managing non-stationary data streams are:
\begin{itemize}
    \setlength{\itemsep}{0pt}
    \setlength{\parskip}{0pt}
    \item \textbf{Concept Drift:} Changes in data distributions or relationships between variables over time, causing models to lose relevance and accuracy \cite{yang2021concept, dong2019multistream}.

    \item \textbf{Multi-class Imbalanced Data:} The uneven distribution of classes in multi-class data streams leads to biased predictions, with minority classes being misclassified \cite{wang2018systematic, sun2009classification}.

    \item \textbf{Class Overlap:} Instances from different classes occupying the same feature space, making it difficult for models to distinguish between them \cite{bhowan2012evolving, galar2011review}.

    \item \textbf{Emergence of New Classes:} The appearance of new classes that were not present during training, causing instability and degraded performance as traditional models struggle to adapt.

    \item \textbf{Heterogeneous Transfer Learning:} The challenge of transferring knowledge between domains with differing characteristics, risking negative knowledge transfer in non-stationary data environments \cite{pan2009survey, wang2018systematic}.
\end{itemize}
These challenges hinder the performance and adaptability of machine learning models in real-time data streams, necessitating the development of advanced frameworks to address these issues effectively.