\chapter{Background}\label{cha:2_background}
Governments and businesses increasingly rely on advanced data analysis and machine learning tools to predict trends and support decision-making in the face of vast streaming data. However, the dynamic nature of data, driven by the emergence of new markets, products, and behaviors, presents the challenge of concept drift \cite{widmer1996learning}. Concept drift refers to changes in the statistical properties of the target variable over time, which can render historical patterns ineffective, impacting the reliability of decision support and early warning systems. In the context of big data, with its unpredictable distributions and types, addressing concept drift necessitates adaptive, data-driven systems capable of detecting and responding to these changes. This chapter explores drifted streams, the sources, types, and components of concept drift, and discusses the application of the K-Nearest Neighbors (KNN) algorithm in such dynamic environments.
\section{Drifted Streams}
A drifted stream refers to a data stream in which the underlying patterns or relationships between the input features and the target variable change over time. This shift in the data distribution, known as concept drift, can manifest in several ways, including changes in the underlying data distributions, the appearance of new classes, or the disappearance of existing ones. Drifted streams are common in many real-world applications, such as financial markets, sensor networks, and online recommendation systems, where user behavior or environmental conditions evolve.

In a drifted stream, traditional machine learning models may struggle to maintain high predictive accuracy, as they are built on the assumption that the relationships within the data remain static. The challenge in such scenarios is that the model must adapt to the changes in the stream to continue providing reliable predictions. This necessitates the development of adaptive algorithms that can detect concept drift and adjust their learning accordingly.

To handle drifted streams effectively, various methods have been proposed, such as drift detection algorithms (e.g., ADWIN, DDM) and ensemble learning techniques that can dynamically update their models in response to drift. Additionally, algorithms like KNN may need to be adapted to account for these changes by giving more weight to recent data or by periodically retraining on updated portions of the data stream to ensure that the predictions remain relevant.
