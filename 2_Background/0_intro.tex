\chapter{Background}\label{cha:2_background}

Amidst the surge of vast streaming data, governments and businesses find themselves in an urgent need for sophisticated data analysis and machine learning analytics approaches. These tools are indispensable for anticipating future trends and making well-informed decisions. However, the perpetual emergence of new goods, markets, and consumer behaviors introduces a formidable challenge known as concept drift \cite{widmer1996learning}. This phenomenon involves the variation of statistical parameters of the target variable over time in unexpected ways, posing a substantial obstacle to accurate forecasting and optimal decision-making. The patterns derived from historical data may become obsolete when applied to new and evolving datasets.
The impact of concept drift extends across data-driven information systems, including decision support and early warning systems, diminishing their overall effectiveness. In the dynamic realm of big data, where data types and distributions are inherently unpredictable, the challenge of concept drift becomes even more pronounced. In response to this challenge, the field introduces a new subject: adaptive data-driven prediction/decision systems.
