\chapter{Background}\label{cha:2_background}

Amidst the surge of vast streaming data, governments and businesses find themselves in an urgent need for sophisticated data analysis and machine learning analytics approaches. These tools are indispensable for anticipating future trends and making well-informed decisions. However, the perpetual emergence of new goods, markets, and consumer behaviors introduces a formidable challenge known as concept drift \cite{widmer1996learning}. This phenomenon involves the variation of statistical parameters of the target variable over time in unexpected ways, posing a substantial obstacle to accurate forecasting and optimal decision-making. The patterns derived from historical data may become obsolete when applied to new and evolving datasets.
The impact of concept drift extends across data-driven information systems, including decision support and early warning systems, diminishing their overall effectiveness. In the dynamic realm of big data, where data types and distributions are inherently unpredictable, the challenge of concept drift becomes even more pronounced. In response to this challenge, the field introduces a new subject: adaptive data-driven prediction/decision systems.

\section{Drifted Streams}
A drifted stream refers to a data stream in which the underlying patterns or relationships between the input features and the target variable change over time. This shift in the data distribution, known as concept drift, can manifest in several ways, including changes in the underlying data distributions, the appearance of new classes, or the disappearance of existing ones. Drifted streams are common in many real-world applications, such as financial markets, sensor networks, and online recommendation systems, where user behavior or environmental conditions evolve.

In a drifted stream, traditional machine learning models may struggle to maintain high predictive accuracy, as they are built on the assumption that the relationships within the data remain static. The challenge in such scenarios is that the model must adapt to the changes in the stream to continue providing reliable predictions. This necessitates the development of adaptive algorithms that can detect concept drift and adjust their learning accordingly.

To handle drifted streams effectively, various methods have been proposed, such as drift detection algorithms (e.g., ADWIN, DDM) and ensemble learning techniques that can dynamically update their models in response to drift. Additionally, algorithms like KNN may need to be adapted to account for these changes by giving more weight to recent data or by periodically retraining on updated portions of the data stream to ensure that the predictions remain relevant.
