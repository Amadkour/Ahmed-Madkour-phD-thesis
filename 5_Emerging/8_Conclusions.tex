% Conclusion
\section{Conclusion And Future Directions}
\label{sec:5_8_Conclusions}
Our research introduces an extensive framework designed for incremental learning within data streams with emerging new classes. By integrating concept drift detection using the ADWIN algorithm, the K-means technique, Gaussian Naive Bayes (GNB) as the base classifier, and ensemble stratified bagging, our approach effectively addresses the challenges associated with this problem. Comprehensive testing on real-world application streams, benchmark datasets, and synthetic data demonstrates the effectiveness of our approach. The ADWIN algorithm is crucial in our approach, enabling timely detection of concept drifts and new classes, making it highly recommended for incremental streams due to its superior performance and lower runtime. This feature enables the proposed method to develop new base classifiers, maintaining their relevance and precision in real-time situations. The use of ensemble stratified bagging further enhances predictive performance and robustness by combining the outputs of various base classifiers. Dynamic Ensemble Selection (DES) is employed to select the best classifiers, ensuring optimal performance. Evaluated through various performance metrics, our approach effectively addresses SENC challenges, demonstrating exceptional performance in the precise classification of changeable streams with changing class distributions, especially when employing GNB as the base classifier. Beyond its notable performance, our approach provides key benefits such as adaptability, efficiency, and scalability. By updating the model dynamically in response to incoming data instances, our approach can continuously adjust to evolving data distributions, thereby maintaining its reliability and relevance in real-time situations. The integration of concept drift detection and ensemble stratified bagging techniques strengthens the framework’s ability to handle dynamic data environments and emerging new classes. Future work can explore further enhancements, such as investigating advanced concept drift detection algorithms, exploring alternative ensemble strategies, and incorporating deep learning techniques. Additionally, our framework can be expanded to address a broader range of real-world applications.