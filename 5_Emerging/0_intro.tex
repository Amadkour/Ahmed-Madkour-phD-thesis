 
\chapter{Addressing Emerging New Classes in Incremental Streams via Concept Drift Techniques}
  \label{chapter:5_emerging}
  
  In various real applications, data streams have introduced new challenges to learning algorithms. Data streams are continuous with high volumes of data arriving rapidly and dynamically. This data deluge poses unprecedented challenges for learning algorithms because they must adapt to the dynamic nature of the data environment \cite{yang2021concept}\cite{dong2019multistream}\cite{shan2018online}. Among the various research areas in machine learning, sequential learning under data Streams with Emerging New Classes (SENC) has garnered considerable attention because of its practical relevance and unique challenges \cite{da2014learning}, \cite{mu2017streaming}, \cite{zhu2020semi}. SENC refers to a scenario in which new classes that were not present during the initial training of a learning model emerged in the data stream. This poses a significant challenge for traditional learning approaches that are typically designed to handle fixed or predefined class distributions. The ability to effectively recognize and adapt to these novel classes in real-time is crucial for maintaining accurate and up-to-date models. Furthermore, the inherent limitations of data streams, such as limited memory and storage constraints, impose additional complexities in the learning process. Learning algorithms must operate efficiently within these resource constraints to ensure real-time processing and to avoid overwhelming computational overhead.
To address the challenges presented by data streams containing emerging new classes, dynamic ensemble selection (DES) \cite{cruz2017meta}\cite{jackowski2014improved}\cite{kuncheva2000clustering}. Dynamic ensemble selection involves utilizing multiple classifiers in machine learning to make collective predictions or classifications of data. Dynamic ensemble selection is distinguished by its ability to dynamically adapt the ensemble based on the characteristics of the data. Instead of relying on a fixed ensemble of classifiers, dynamic ensemble selection continuously assesses the performance of individual classifiers and selects the subset that demonstrates the highest competence for the current data. This adaptability enables the ensemble to enhance its performance over time by incorporating the most suitable classifiers for prevailing data conditions. Furthermore, if a classifier becomes ineffective owing to concept drift or the emergence of new classes, it can be excluded from the ensemble to prevent it from negatively affecting the overall performance.
Adaptive Windowing (ADWIN) is another widely employed method to address concept drift. Concept drift refers to the phenomenon in which the statistical properties of data change over time, leading to a decline in the learning algorithm performance \cite{gama2004learning}\cite{adams2023explainable}\cite{madkour2023historical}. ADWIN continuously monitors incoming data and detects changes or drifts in data distribution. It achieves this by maintaining sliding windows of variable sizes and monitoring statistical measures such as the mean or variance within these windows. Upon detecting a significant change or drift, the ADWIN triggers an update in the ensemble or classifier configuration. This update may involve retraining the classifiers with new data or incorporating new classifiers that are better suited to the updated data distribution. By adapting the ensemble to changing data conditions, ADWIN ensures that the classifier system remains accurate and up-to-date even in the presence of concept drift or the emergence of new classes. The primary objective of utilizing these approaches, such as dynamic ensemble selection and ADWIN, is to establish a flexible and effective classification system that is capable of handling data streams containing emerging new classes. By dynamically adjusting the ensemble based on the data characteristics and detecting and adapting to concept drift, these approaches enable the system to maintain accurate predictions over time. Adaptability and responsiveness are crucial for addressing the unique challenges posed by the dynamic nature of data streams and the emergence of new classes within them.

The subsequent sections of this paper adhere to a well-structured organization. Section \ref{sec:proposed_methodology} introduces the proposed framework, providing intricate explanations of its constituents, including dynamic classifier ensembles, concept drift handling, and emergency class identification, and the adaptive proposed method of the emerging pool size. Section \ref{sec:results} outlines the experimental setup and presents the results, providing details regarding the employed datasets, evaluation metrics, and procedures. Finally, in Section \ref{sec:5_8_Conclusions}, we offer concluding remarks, Highlighting the main discoveries, examining their significance, and suggesting possible directions for future investigations.
  
  
  \section{Motivations and Contributions} \label{sec:5_2_motivation}
  This research proposes efficient algorithms for classifying data streams in real-time scenarios, focusing on addressing the issues caused by emerging new classes in data distributions. We aim to develop novel algorithms that can effectively handle the emergence of new classes issue in dynamic data streams, achieve high accuracy of classification, and reduce the complexity of computations. The key contributions of this study are outlined as follows:
  \begin{enumerate}[nosep]
    \item Our first contribution involves utilizing the ADWIN, DES, and K-means techniques in combination with the ensemble stratified bagging technique to detect and adapt to emerging new classes. This approach allows us to dynamically update the classification model to accommodate an evolving data environment.
   \item The second contribution is the introduction of an adaptive method to adapt the emerging pool size depend on the stream distribution.
    \end{enumerate} 
   
     
  
  
 
   

