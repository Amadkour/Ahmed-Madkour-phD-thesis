 
\chapter{Addressing Emerging New Classes in Incremental Streams via Concept Drift Techniques}
  \label{chapter:5_emerging}
  
  In various real applications, data streams have introduced new challenges to learning algorithms. Data streams are continuous with high volumes of data arriving rapidly and dynamically. This data deluge poses unprecedented challenges for learning algorithms because they must adapt to the dynamic nature of the data environment \cite{yang2021concept, dong2019multistream, shan2018online}. Among the various research areas in machine learning, sequential learning under data Streams with Emerging New Classes (SENC) has garnered considerable attention because of its practical relevance and unique challenges \cite{da2014learning}, \cite{mu2017streaming}, \cite{zhu2020semi}. SENC refers to a scenario in which new classes that were not present during the initial training of a learning model emerged in the data stream. This poses a significant challenge for traditional learning approaches that are typically designed to handle fixed or predefined class distributions. The ability to effectively recognize and adapt to these novel classes in real-time is crucial for maintaining accurate and up-to-date models. Furthermore, the inherent limitations of data streams, such as limited memory and storage constraints, impose additional complexities in the learning process. Learning algorithms must operate efficiently within these resource constraints to ensure real-time processing and to avoid overwhelming computational overhead.
  To tackle challenges in data streams with emerging new classes, Dynamic Ensemble Selection (DES) and Adaptive Windowing (ADWIN) are widely used techniques. DES dynamically adapts ensembles of classifiers by continuously evaluating their performance and selecting the most competent subset for current data. This approach enhances adaptability and ensures improved performance over time by incorporating classifiers best suited to prevailing data conditions. Ineffective classifiers, affected by concept drift or the emergence of new classes, are excluded to avoid degrading overall system performance \cite{cruz2017meta, jackowski2014improved, kuncheva2000clustering}.  
  ADWIN addresses concept drift, where statistical properties of data evolve, causing a decline in classifier accuracy. It uses sliding windows of varying sizes to monitor statistical measures like mean or variance and detects significant changes in data distribution. When drift is detected, ADWIN updates the ensemble by retraining classifiers or incorporating new ones optimized for the updated data distribution. This adaptability ensures classifiers remain effective and accurate in changing conditions \cite{gama2004learning, adams2023explainable, madkour2023historical}. Both approaches aim to create flexible and responsive classification systems that maintain high accuracy over time, effectively handling dynamic data streams and emerging classes.

The subsequent sections of this paper adhere to a well-structured organization. Section \ref{sec:proposed_methodology} introduces the second proposed approach, providing intricate explanations of its constituents, including dynamic classifier ensembles, concept drift handling, and emergency class identification, and the adaptive proposed method of the emerging pool size. Section \ref{sec:results} outlines the experimental setup and presents the results, providing details regarding the employed datasets, evaluation metrics, and procedures. Finally,  Section \ref{sec:5_summary} presents a summary of this chapter.
  
  
  \section{Motivations and Contributions of this Chapter} \label{sec:5_2_motivation}
  This research proposes efficient algorithms for classifying data streams in real-time scenarios, focusing on addressing the issues caused by emerging new classes in data distributions. The goal is to develop novel algorithms that can effectively address the emergence of new classes in dynamic data streams, achieve high classification performance, and reduce computational complexity. The key contributions of this study are outlined as follows:
  \begin{enumerate}[nosep]
    \item Our first contribution involves utilizing the ADWIN, DES, and K-means techniques in combination with the ensemble stratified bagging technique to detect and adapt to emerging new classes. This approach allows us to dynamically update the classification model to accommodate an evolving data environment.
   \item The second contribution is the introduction of an adaptive method to adapt the emerging pool size depend on the stream distribution.
    \end{enumerate} 
   
     
  
  
 
   

