
\chapter{Dynamic Classification Ensembles for Handling Imbalanced
  Multiclass Drifted Data Streams}
  \label{chapter:4_Imbalanced_Multiclass}
  
  In recent years, the explosion of high-speed data streams has presented new challenges for machine learning models. Three critical
  issues that have emerged are concept drift, class imbalance, and class overlap. Concept drift refers to the phenomenon where the
  statistical properties of a data generation process change over time \cite{yang2021concept, dong2019multistream}. This signifies that the underlying concepts, relationships
  between variables, or data distribution can change, leading to a fundamental shift in the nature of data. Dealing with concept drift
  poses a fundamental challenge in machine learning and data mining. This can cause models trained on historical data to become
  inadequate when applied to new data affected by concept drift, leading to a decline in the model performance \cite{dong2019multistream}. To address this issue,
  concept drift detectors are used to identify changes in data stream distributions by leveraging information associated with classifier
  performance or the incoming data items themselves. These signals frequently prompt model updates, retraining, or substitution of an
  old model with a new one.
  In addition, class imbalance \cite{dong2019multistream, pan2009survey}, which is characterized by uneven class distribution, poses a challenge for traditional classifiers
  \cite{zhuang2020comprehensive}, particularly in multiclass scenarios where minority class samples are at risk of misclassification owing to their limited representation \cite{wang2018systematic}. Addressing imbalanced data classification requires specialized techniques to ensure accurate minority class classification without compromising the performance of the majority class \cite{sun2009classification, charte2015addressing, charte2015mlsmote}. This challenge becomes more daunting when minority class instances are scattered in unknown configurations, thereby increasing the likelihood of overfitting during the learning process. To
  address this issue, three primary methods are employed in the context of imbalanced data classification, which are effective in both
  binary and multiclass imbalance scenarios \cite{daniels2017addressing}. The first approach involves sampling methods that address class imbalance by either
  reducing the number of majority class instances (undersampling) or generating artificial minority class instances (oversampling)
  \cite{liu2018making,lopez2012analysis}. The second and third groups encompass adaptive algorithms and hybrid methods, respectively. Adaptive algorithms include
  one-class and cost-sensitive classification \cite{zhang2020towards}. Hybrid methods merge data preprocessing with classification techniques, often utilizing ensemble classifiers to effectively mitigate class imbalance and enhance classifier performance \cite{chawla2003smoteboost, wang2010negative, bhowan2012evolving}.
  Class overlap occurs when instances from different classes inhabit the same region in the data space \cite{galar2011review, cruz2018dynamic}. This overlap complicates the task of distinguishing between representative instances of various classes and posing performance challenges for traditional classifiers. This issue is commonly referred to as class overlap. Researchers have proposed class overlap undersampling
  techniques to address class imbalance problems \cite{kuncheva2000clustering}. These techniques aim to leverage local similarities among minority instances to
  identify potentially overlapping majority instances. Although these methods have demonstrated promising results in improving model
  performance on specific datasets, many of them rely heavily on nearest-neighbor approaches to detect overlapping regions around
  minority instances. This approach often neglects the global similarity within the overlapping domain, which can lead to local optimal
  values getting stuck during calculations. Furthermore, determining appropriate parameters for these models poses a significant
  challenge. If the parameter selection is too extensive, it may lead to the exclusion of valuable instances, while conservative parameters
  may overlook instances with overlap. This issue can significantly affect classifier performance, particularly when handling streams
  containing both a minority class and instances with class overlap. Therefore, both class imbalance and class overlap present significant
  challenges in the realm of data stream analyses. Consequently, addressing class imbalance is crucial in multiclass learning, leading to
  research efforts that focus on both concept drift and class imbalance challenges. Researchers have explored techniques such as DES and
  multiclass oversampling to address these issues.
  Dynamic classifier ensembles adapt their composition based on data characteristics, making them particularly valuable in evolving data conditions \cite{woloszynski2011probabilistic}. Classifier ensemble selection aims to identify the optimal subset of classifiers from a larger ensemble, often using the overproduce-and-select strategy. This process considers various criteria, including individual performance metrics, diversity measures, meta-learning techniques, and performance estimation methods, which are crucial for balancing accuracy with computational constraints. Ensemble selection methods are categorized as static or dynamic. Static selection assigns classifiers to predefined feature partitions, while dynamic selection adapts classifier choices based on their competency \cite{lysiak2014optimal}. Dynamic selection includes two approaches: Dynamic Classifier Selection (DCS) and Dynamic Ensemble Selection (DES). DCS selects the most suitable classifier for each data point by evaluating local competencies, whereas DES identifies the best-performing classifiers for instances within localized regions \cite{cruz2017meta, widmer1996learning, lu2016concept}.

  The main goal of this study is to formulate a precise classification approach that addresses changing conditions. Specifically, the
  first proposed approach aims to address scenarios where there is an uneven distribution among several classes, overlapping instances of
  classes, and instances where the fundamental concept of data evolves. To address these challenges, dynamic classifier
  ensembles are employed. These ensembles utilize oversampling techniques, implemented either on a global or local scale, as a preprocessing step to
  address class imbalance. Furthermore, multiclass learning techniques are enhanced to counteract class imbalance through method
  adaptation. 
  
  The remainder of this chapter is organized as follows: In Section \ref{sec:4_2_motivation},the motivations and the contributions are presented. The first proposed framework are discussed in detail in Section \ref{sec:4_first_proposed_approach}. The  experimental results and the discussion are presented in Sections \ref{sec:4_5_Expsetup}. Section \ref{sec:4_summary} presents a summary of this chapter.
  
  
  \section{Motivations and Contributions of this Chapter} \label{sec:4_2_motivation}
  \begin{enumerate}[nosep]
    \item A classification approach is introduced that dynamically adjusts to multiclass imbalanced data, incorporates mechanisms for
    detecting concept drift, and optimizes classifier ensemble selection. The objective is to enhance the classification performance, specifically for multiclass imbalanced non-stationary streams.
   \item Additionally, an adaptive method is proposed for the class imbalance issue, considering the data distributions and historical instances of class imbalance. This is particularly relevant in cases where class overlap occurs within the multiclass and drifted datan performance by selecting the most suitable oversampling method based on the unique characteristics of the data stream. 
    \end{enumerate}