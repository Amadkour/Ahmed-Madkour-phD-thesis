
% Thesis Abstract -----------------------------------------------------


%\begin{abstractslong}    %uncommenting this line, gives a different abstract heading


\begin{alwayssingle} \pagestyle{empty}\begin{center}
    \vspace*{1.5cm}
    {\Large \bfseries  Abstract}
  \end{center}      %this creates the heading for the abstract page
    This thesis addresses challenges in machine learning models when dealing with multi-class imbalanced data streams in non-stationary environments. These challenges result in biases, unreliable predictions, and diminished model performance. To overcome these issues, the thesis introduces innovative approaches that combines Dynamic Ensemble Selection (DES) technique, an adaptive method for handling imbalanced multi-class data streams, a concept drift detector, eigen vector technique, and the K-Nearest Neighbors (KNN) algorithm to address class overlap.
    The main objective is to improve the classification of imbalanced multi-class drifted data streams. The adaptive oversampling method generates synthetic samples to mitigate imbalanced data stream issues, with KNN ensuring non-overlapping generated samples. A drift detector assists in deciding whether to keep existing classifiers or create new ones to handle incoming data streams. Dynamic Ensemble Selection (DES) optimizes the classification task by selecting the most appropriate classifier for incoming data. The proposed method offers an effective solution for achieving accurate and resilient classification in the context of imbalanced multi-class drifted data streams. 
    Additionally, the thesis discusses challenges posed by emerging new classes in dynamic data streams for incremental learning. Existing approaches struggle with high false positive rates, long prediction times, and the impractical assumption of having access to true labels for all instances when new classes emerge. To address these challenges, the thesis proposes a novel framework that integrates concept drift detection using the ADWIN method and ensemble stratified bagging. By leveraging true labels during the learning process, the framework effectively detects concept drifts and adapts to accommodate emerging new classes. 
    Furthermore, the thesis discusses challenges faced by machine learning models in heterogeneous multisource streams and non-stationary conditions, which can introduce biases and unreliable predictions. In response, the thesis introduces a comprehensive framework that combines dynamic ensemble selection, eigenvector techniques, and a concept drift detector to enhance transfer learning of multi-class data streams subject to drift. During the training phase, the proposed weighted method assigns appropriate weights to each classifier, mitigating adverse learning effects. Eigenvectors handle heterogeneous multisource streams, and Dynamic Ensemble Selection (DES) determines the most fitting classifiers for incoming data. A concept drift detector identifies and addresses drifted chunks within the data stream.
    The experimental results on various datasets, encompassing benchmark datasets, a real application stream dataset, and synthetic data streams, consistently demonstrate the superior performance of the proposed approach in effectively addressing challenges inherent in such dynamic data streams. Through evaluations conducted on both real-world and synthetic datasets, the framework exhibits exceptional accuracy, precision, recall, geometric mean, and F1-measure, all while maintaining an efficient runtime.
    This robust performance positions the proposed approach as a frontrunner in enabling incremental learning within real-time scenarios, particularly in the context of emerging new classes. The framework provides a comprehensive solution to the intricate complexities associated with incremental learning in dynamic data streams. Moreover, the experimental findings underscore the efficacy and superiority of the proposed approach, particularly when confronted with the unique challenges posed by imbalanced drifted streams, emerging new classes, and heterogeneous multisource drifted data streams.
    Comparisons against state-of-the-art chunk-based methods consistently highlight the effectiveness of this approach in successfully managing the complexities inherent in these multifaceted data stream challenges. In essence, the proposed approach stands out as a powerful and versatile solution, showcasing its ability to outperform existing methods and addressing the evolving demands of imbalanced drifted streams and heterogeneous multisource streams.
    Keywords: - Auto machine learning, Concept drift, Imbalanced Stream, Transfer Learning, Emerging new Classes, and Dynamic Ensemble Selection.
    
\end{alwayssingle}

% \begin{resumen}        %this creates the heading for the abstract page
%\selectlanguage{spanish}
% Motivación

% \end{resumen}


%\begin{laburpena}        %this creates the heading for the abstract page
%\selectlanguage{basque}
%% Jarri zure laburpena hemen.
%%
%
%
%\end{laburpena}

%\end{abstractlongs}


% ----------------------------------------------------------------------
