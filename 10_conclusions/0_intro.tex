
\chapter{Conclusions and Future Work}
\label{cha:10_conclusions}

% the code below specifies where the figures are stored
\ifpdf
    \graphicspath{{10_conclusions/figures/PNG/}{10_conclusions/figures/PDF/}{10_conclusions/figures/}}
\else
    \graphicspath{{10_conclusions/figures/EPS/}{10_conclusions/figures/}}
\fi


% ----------------------------------------------------------------------
% Motivación de la tesis		
At the beginning of this thesis it was presented that, although GNSS systems are currently the main source of location information, due to the need for LOS with satellites, it is not possible to use them continuously.
This is especially true for pedestrians, as most of their time is spent in indoor environments.
The development of new LBS and systems under new paradigms such as ubiquitous computing requires seamless localization systems.
To this end, the role of DR-based localization systems was seen as key because they do not need additional infrastructure to be installed in the environment.
%In the case of people, it also requires the system to be comfortable and light so that it can be carried most of the time.

% Objetivo general
In this context, thanks to the current availability of smart devices for the wrist, such as smartwatches and smartbands, the technical capabilities they usually include, and the convenience they offer to the user, wrist-worn inertial PDR systems seem to be a great opportunity to make further progress towards seamless pedestrian localization systems.
However, despite their potential for developing pedestrian LBS, wrist-worn inertial PDR systems remain an open challenge.
And the existing literature was not enough to provide a complete overview of the operation and performance of this type of localization systems.
This indicated that there was a need to address wrist-worn inertial PDR systems in a specific and comprehensive manner. 
This was precisely the main goal of this thesis: 
% To contribute to the knowledge and understanding of the design and operation of wrist-worn inertial PDR systems.
to contribute to both the understanding and the improvement of the performance of wrist-worn inertial PDR systems.
%To this end, a set of specific objectives were defined which it was considered interesting to go into in greater depth, and this final chapter summarises the conclusions and results obtained.
After the revision of the state of the art literature about inertial PDR systems in general, and wrist-worn in particular, four specific objectives were defined for this research work, which were presented in the final section of Chapter~\ref{cha:3_pdr_sota}.
This set of objectives was defined in such a way as to provide a global vision of the different blocks that make up a PDR system and this final chapter summarises the conclusions and results obtained.

% Structure chapter
In this way, the Section~\ref{sec:10_1_specObj} presents the main conclusions reached during this doctoral thesis.
Subsequently, the Section~\ref{sec:10_2_future} proposes the main lines of future research in this field.
Finally, the Section~\ref{sec:10_3_publication} lists the articles that have been published in relation to the results of this thesis. 

\section{Conclusions}
\label{sec:10_1_specObj}
%%% Un párrafo largo de lo que se ha hecho por capítulo, por qué, conclusiones y contribuciones.
%%% CH1-SpecObj1
%First of all, in Chapter~\ref{cha:4_suitability} it has been done a deep analysis of the characteristics of the signals coming from different types of wrist-worn sensors and their variability with respect to factors such as the type of carrying mode, the walking speed or different users.
% This analysis, which corresponds to the \emph{specific objective 1}, aims to gain a more detailed knowledge about the characteristics of the existing patterns in the wrist-worn inertial signals, to understand what implications they may have when designing a PDR system and, therefore, to assess the feasibility to do it.
The research work began with a deep analysis of the characteristics of the signals coming from different types of wrist-worn sensors and their variability with respect to factors such as the type of carrying mode, the walking speed or different users.
This analysis, presented in Chapter~\ref{cha:4_suitability} and that is related to the \emph{specific objective~1}, aims to gain a more detailed knowledge about the characteristics of the existing patterns in the wrist-worn inertial signals, to understand what implications they may have when designing a PDR system and, therefore, to assess the feasibility to do it.
The different signals from sensors coupled to the pedestrian's COM when walking have been quite studied and used in the PDR field.
In contrast, little is known about the origins, motives and characteristics of the swinging of the arms when walking with them free.
Usually, PDR systems that consider swinging carrying mode assume an ideal swing and do not deepen their characteristics.
%However, the main conclusion of this analysis is that there is a great variability between users, both in the amplitude and the directions in which the arms swing. 
% In addition, walking speed also directly influences the amplitude of the swing.
% In a similar way, quasi-static carrying modes also showed some variability depending on the user and, especially, on the walking speed.
% These are the two main conclusions of this chapter and indicate the need to design highly adaptable algorithms in all PDR blocks and, additionally, to define calibration processes that help to adjust these algorithms to each user or scenario.
Therefore, one of the contributions of this analysis is the confirmation that different users or walking speeds strongly influence the patterns of the different signals, due, especially, to the variation of both the amplitude and the direction in which the arms swing. 
Even quasi-static carrying modes also showed some variability depending on the user and, especially, on the walking speed.
The main conclusion of this analysis, and therefore of Chapter~\ref{cha:4_suitability}, is that the presence of clear patterns related to the human walking on different types of signals during the main use cases confirms the feasibility of designing PDR systems on the wrist. 
However, it was also shown the need to design highly adaptive algorithms in all PDR blocks and, additionally, to define calibration processes that help to adjust these algorithms to each user or scenario.

%%% CH5-SpecObj3
During the review of the state of the art, a large number of SLE methods based on inertial sensors were found to exist, but it was difficult to obtain a clear and comparative view of the performance, strengths and weaknesses of each of them.
In addition to the lack of a methodology to guide the evaluation and testing of these methods, no review work was found to provide a complete overview of this field, which is applicable in several areas other than PDR systems.
Therefore, \emph{specific objective~2} was defined and Chapter~\ref{cha:5_sle_sota} presented a systematic and complete review that covered the whole workflow involved in the design, test, and evaluation of SLE methods based on inertial sensors.
The main conclusion offered by this review work is that the field of SLE methods is still in the research stage due to two main reasons. On the one hand, the lack of a standard testing and evaluation methodology that marks a common way of working for all researchers in the field and, on the other hand, the lack of availability of public datasets or competitions that facilitate the comparison of the different proposals in a fair and reliable way.
This work is, to the knowledge of the author, the first one in this field that offers a global vision and tries to serve as a starting point for the definition of future methodologies, to which we tried to contribute by offering a series of reflections on the matter.

%%% CH6-SpecObj3
One of the conclusions drawn in the previous Chapter~\ref{cha:5_sle_sota} was the important role that the ground truth source plays in the calibration, testing and evaluation of SLE methods based on inertial sensors.
Nevertheless, the technologies that are commonly used as ground truth do not offer a good balance between accuracy and coverage area.
On the one hand, technologies such as optical systems can achieve millimeter accuracies of the lengths of each step or stride. 
Unfortunately, due to LOS requirements, its use is limited to very small areas, which makes it difficult to collect data associated with realistic day-to-day activities.
On the other hand, other technologies that can be used in larger and outdoor areas are not able to measure the length of each individual step but the average after a walk.
Motivated by these facts, the UWB technology was identified as a possible candidate to be used as a ground truth source in the field of SLE.
This technology offers only decimeter-level ranging accuracy but provides larger coverage areas than optical or ultrasound systems, which can be useful for the training and evaluating PDR systems. 	 
Therefore, in Chapter~\ref{cha:7_sle_uwb}, a preliminary evaluation of the accuracy of UWB technology as a ground truth source for SLE related tasks was presented.
Four different methods were tested and, after analyzing their performances of the different SLE methods, a first observation can be highlighted: since the step is a relative measure between the foot positions, the use of inter-feet ranges has proven to be essential when it comes to reducing the estimation error.
The first results indicate that it is possible to obtain errors below 10 cm for more than 50\% of the steps, which makes us optimistic about the use of UWB as ground truth since it is possible to improve those results easily by installing a greater number of anchors to minimize NLOS situations and the use of filtering techniques for outliers ranges.
This exploratory work therefore supports future work on assessing whether the combination of the accuracy and coverage provided by UWB is really useful for SLE training of inertial PDR systems, specially in comparison to the use of the average step length as ground truth.

%%% Ch7-SpecObj4
The main advantage of wrist-worn inertial PDR systems over systems mounted on other parts of the body is the comfort and convenience offered by this body location to the users. 
At the same time, this comfort must be accompanied by a position estimation accuracy that, although a priori worse than that of systems in other parts of the body, can offer a good tradeoff.
However, the evaluations available in the literature did not provide a clear picture of the performance of this type of systems.
For this reason, and in line with the \emph{specific objective~3}, the Chapter~\ref{cha:8_eval} presents the results obtained by the implementation of a wrist-worn system described in the literature during two evaluation processes.
The first one corresponds to the IPIN Indoor Localization Competition 2016 held during that year’s IPIN conference, whose main characteristic is its challenging evaluation paths.
In order to participate, the PDR system had to be extended to provide 2.5D positioning.
The second evaluation stage focused on the comparison of different inertial pedestrian navigation systems simultaneously carried on the wrist, pocket and foot, and evaluated for several users.
In this way, an attempt was made to obtain a broader vision than that obtained in the competition, since in that competition each system was evaluated independently and by a single user, who was also different for each system.
The results of both evaluation stages show that, although the wrist-worn system implemented achieves worse performance than other types of systems, especially foot-mounted INS systems, the distance is not excessive and has even managed to outperform systems that, a priori, should be more accurate.
Specifically, the results obtained in the competition show that, once the error of using magnetometers in indoor environments has been corrected, the estimate of the trajectory obtained is quite acceptable.
In the second evaluation, however, two of the current weaknesses of the wrist-worn system became more visible: the detection of false positive steps and the difficulty of offering similar performances for different users, both with regard to the detection of steps and the estimation of their length.
These are the main conclusions of this evaluation chapter and will undoubtedly set the direction for future work.

%%% Ch8-SpecObj5
The accumulation of heading drift is a problem that affects all localization systems that use gyroscopes to estimate the direction of travel.
With the aim of trying to mitigate this problem, the \emph{specific objective~4} was defined for trying to adapt the iHDE heading reduction method so that can be used in PDR systems.
That adaptation was presented and tested in Chapter~\ref{cha:9_ihde}.
The advantage of this type of heuristic methods is that they need to know very little information about the building.
This facilitates its availability compared to other map-matching-based solutions, for which it is not always easy to obtain such information due to security, privacy or information formatting problems.
The results of the experimentation carried out show that the adaptation maintains the same corrective characteristics as the original iHDE and that it fits perfectly into a wrist-worn inertial PDR system.

% Contribuciones:
The main goal defined at the beginning of this thesis was to contribute to a better understanding of the wrist-worn inertial PDR systems and to try to provide proposals to improve their performance. 
Through the specific objectives that were defined, these are the main contributions:
\begin{itemize}
	\item An in-depth analysis of the characteristics of the signals coming from different types of wrist-worn sensors and their variability with respect to factors such as the type of carrying mode, the walking speed or the different users.
	\item A complete and systematic review of the SLE methods based on inertial sensors was presented. The lack of public datasets and a standard methodology for testing and evaluation were identified as important reasons that keep this field is still in the research phase.
	\item The use of UWB technology as a source of step length ground truth has been proposed and evaluated. The balance between accuracy and coverage offered by UWB has been identified as useful for calibrating and validating PDR systems.
	\item The performance of a wrist-worn inertial PDR system has been evaluated in realistic scenarios and compared to other state-of-the-art inertial pedestrian navigation systems.
	\item A heading drift reduction method originally designed for foot-mounted INS systems has been adapted for PDR systems, obtaining a reduction in the heading drift error.
\end{itemize}

% Luego un párrafo de valoración general y varios puntos con conclusiones más importantes.
% Tras la realización de las tareas asociadas a esta tesis, la principal conclusión que se ha extraído es que 
% los sistemas wrist-worn son posibles
% está claro que cuanto más alejado de los pies, más dificil pero no hay tanta distancia y hay capacidad de mejora
% Dos principales problemas a resolver:
% - Irregular o movimientos no previsibles
	% Genera mucho error porque cada paso mal detectado puede introducir un error de unos 70 cm, mayor que el error de SLE.
% - Variabilidad patrones
	% Incluso los casos de uso o movimientos identificados
	% it has been observed that the step patterns in the different signals are far from ideal, for several reasons: the person’s anatomy, the pose of the arm, as well as the walking speed, for instance.
	% adaptive algorithms will be necessary for implementing robust step/stride detectors from wrist-worn inertial sensors.
% % Poco uso de la saber posición y orientación del sensor sobre el cuerpo. Esto debe ser usado en futuro porque es información no aprovechada que puede resultar clave para conseguir mejores rendimientos.

After carrying out the tasks associated with this thesis, the main conclusion that has been drawn is that the wrist-worn inertial PDR system is an option that is still considered to have great potential to be a fundamental part of seamless pedestrian localization systems.
Clearly, the further away the inertial sensors are from the feet, the more complicated it is to design robust PDR systems as it increases the possibility of spurious signals and motions uncoupled to the walking.
However, the results obtained throughout the thesis indicate that the distance, in terms of estimation accuracy, with respect to systems in other parts of the body is not too great and, furthermore, there are still room for improvement.
The most important lines of future work are presented in the next section.		
%%%%%%%%%%%%%%%%%%%%%%%%
\section{Future Research Work}
\label{sec:10_2_future}
In the specific field of wrist-worn inertial PDR systems, two main problems have been identified which should be addressed first:
\begin{itemize}
	\item More robust SD methods: 
	The arms are used for many activities that can be performed both when the person is still and when walking.
	These situations generate signal shapes that can make the SD over-count (false positive) or miss (false negative) steps, significantly  reducing the accuracy of position estimation.
	Designing SD methods that offer better performance in these scenarios is key to developing wrist-worn inertial PDR systems that can be useful in real scenarios.
	Otherwise, they will always be very limited by this problem.
	Furthermore, the resolution of this problem is interesting for applications other than positioning, such as sport activity monitoring, energy consumption assessment, or prediction of human health status.	
	\item User calibration:
	Even when the user moves the arms in an expected way, it has been observed that the step patterns vary considerably due to factors such as different users, the pose of the arm or the walking speed.
	This indicates the need to design highly adaptive algorithms for all PDR blocks and the definition of calibration processes that help to adjust those algorithms to each user characteristics.	
\end{itemize}
For the resolution of these problems and, in general, of any related to wrist-worn systems, a possible source of further improvement may come from making use of the mechanical structure nature of the human body. 
Since the sensor is on a fixed and known position of that structure, which also can be known and modeled, this information may be useful to design new detection and estimation algorithms.

Finally, a general problem for all positioning systems based on DR is the lack of public available datasets and standard methodologies that guide the testing and evaluation.
Such methodologies should be specific to the characteristics of DR-based systems, for example, taking into account the effect of the trajectory on error metrics or considering testing and evaluation at both the system and PDR block levels.
In addition, new experimentation procedures and evaluation metrics should be define in order to provide a performance description of the tested system so that it would be possible to predict with some confidence the performance of a system in new scenario.
This kind of actions helps a research area to move forward reliably and to turn research prototypes into commercial products.
%%%%%%%%%%%%%%%%%%%%%%%%%%%%	
\section{Publications}
\label{sec:10_3_publication}
This section presents the different research works of this thesis that has been published in international journals and conferences. 
It also list the work carried out as part of the research group in which I collaborated as a coauthor.

\textbf{International Journals}
\begin{itemize}
	%\cite{diez_suitability_2018}
	\item L.~E. Díez, A.~Bahillo, J.~Otegui, and T.~Otim, ``Suitability {Analysis} of {Wrist}-{Worn} {Sensors} for {Implementing} {Pedestrian} {Dead} {Reckoning} {Systems},'' \emph{IEEE Sensors Journal}, vol.~18, no.~12, pp. 5098--5114, May 2018.
	%\cite{diez_step_2018-1}
	\item L.~E. Díez, A.~Bahillo, J.~Otegui, and T.~Otim, ``Step {Length} {Estimation} {Methods} {Based} on {Inertial} {Sensors}: {A} {Review},'' \emph{IEEE Sensors Journal}, vol.~18, no.~17, pp. 6908--6926, Sep. 2018.	
\end{itemize}	

\textbf{International Conferences}
\begin{itemize}
	%\cite{diez_signal_2015}
	\item L.~E. Díez, A.~Bahillo, A.~D. Masegosa, A.~Perallos, L.~Azpilicueta, F.~Falcone, J.~J. Astrain, and J.~Villadangos, ``Signal processing requirements for step detection using wrist-worn {IMU},'' in \emph{2015 {International} {Conference} on {Electromagnetics} in {Advanced} {Applications} ({ICEAA})}, Turin, Italy, Sep. 2015, pp. 1032--1035.
	%\cite{diez_enhancing_2016-1}
	\item L.~E. Díez, A.~Bahillo, S.~Bataineh, A.~D. Masegosa, and A.~Perallos, ``Enhancing improved heuristic drift elimination for step-and-heading based pedestrian dead-reckoning systems,'' in \emph{2016 38th {Annual} {International} {Conference} of the {IEEE} {Engineering} in {Medicine} and {Biology} {Society} ({EMBC})}, Aug. 2016, pp. 4415--4418.
	% \cite{diez_enhancing_2016}
	\item L.~E. Diez, A.~Bahillo, S.~Bataineh, A.~D. Masegosa, and A.~Perallos, ``Enhancing {Improved} {Heuristic} {Drift} {Elimination} for {Wrist}-{Worn} {PDR} {Systems} in {Buildings},'' in \emph{2016 {IEEE} 84th {Vehicular} {Technology} {Conference} ({VTC}-{Fall})}, Montreal, Canada, Sep. 2016, pp. 1--5.
	 % \cite{diez_step_2018}
	\item L.~E. Díez, A.~Bahillo, J.~Otegui, and T.~Otim, ``Step {Length} {Estimation} using {UWB} {Technology}: {A} {Preliminary} {Evaluation},'' Nantes, France, Sep. 2018, in press.
\end{itemize}

\textbf{Other Publications as Coauthor}
\begin{itemize}
	% \cite{bahillo_enabling_2015}
	\item A.~Bahillo, L.~E. Díez, A.~Perallos, and F.~Falcone, ``Enabling {Seamless} {Positioning} for {Smartphones},'' in \emph{{XXX} {Simposium} {Nacional} de  la {Unión} {Científica} {Internacional} de {Radio}, {URSI} 2015}, Pamplona,  Spain, Sep. 2015.
	%\cite{bataineh_conditional_2016}
	\item S.~Bataineh, A.~Bahillo, L.~E. Díez, E.~Onieva, and I.~Bataineh, ``{Conditional {Random} {Field}-{Based} {Offline} {Map} {Matching} for {Indoor} {Environments}},'' \emph{Sensors}, vol.~16, no.~8, p. 1302, Aug. 2016.
	%\cite{bahillo_blue_2017}
	\item A.~Bahillo, L.~E. Díez, and A.~Arambarri, ``{BLUE} {Care}: {A} {Cooperative} {Location} {Network} for {Handicapped} {Persons},'' \emph{Procedia Engineering}, vol. 178, no. Supplement C, pp. 67--75, 2017.
	%\cite{bataineh_using_2017}
	\item S.~Bataineh, A.~Bahillo, and L.~E. Díez, ``Using of behavioral information for enhancing {Conditional} {Random} {Field}-based map matching,'' in \emph{2017 {European} {Navigation} {Conference} ({ENC})}.\hskip 1em plus 0.5em minus 0.4em\relax Lausanne, Switzerland: IEEE, May 2017, pp. 206--212.
	%\cite{bousdar_performance_2017}
	\item D.~Bousdar, L.~E. Díez, and E.~Munoz~Diaz, ``Performance comparison of wearable-based pedestrian navigation systems in large areas,'' in \emph{2017 {International} {Conference} on {Indoor} {Positioning} and {Indoor} {Navigation} ({IPIN})}, Sapporo, Japan, Sep. 2017, pp. 1--7.
	%\cite{otegui_survey_2017}
	\item J.~Otegui, A.~Bahillo, I.~Lopetegi, and L.~E. Díez, ``A {Survey} of {Train} {Positioning} {Solutions},'' \emph{IEEE Sensors Journal}, vol.~17, no.~20, pp. 6788--6797, Oct. 2017.
	%\cite{otegui_evaluation_2018}
	\item J.~Otegui, A.~Bahillo, I.~Lopetegi, and L.~E. Díez, ``Evaluation of {Experimental} {GNSS} and 10-{DOF} {MEMS} {IMU} {Measurements} for {Train} {Positioning},'' \emph{IEEE Transactions on Instrumentation and Measurement}, pp. 1--11, Jun. 2018.	
\end{itemize}