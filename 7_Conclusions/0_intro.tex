\chapter{Conclusions and Future Work}
\label{chapter:7_Conclusions}
This chapter summarizes the key findings of this study on incremental learning in data streams and presents potential directions for future research to further improve and expand the methodology.

\section{Conclusions}
\label{section:7_1}
\begin{itemize}
    \setlength{\itemsep}{1.5pt}
    \setlength{\parskip}{1.5pt}
    \item This study introduces a novel approach integrating oversampling techniques, concept drift detection, and Dynamic Ensemble Selection (DES) to identify the most suitable ensemble classifiers.
    \item We demonstrated the effectiveness of the methodology through extensive experiments across various datasets, including benchmarks, real-world streams, and synthetic data.
    \item A key feature is the rapid detection of concept drift, enabling the system to adapt quickly, ensuring continued relevance and performance in real-time scenarios.
    \item We addressed the minority class problem using advanced oversampling and KNN algorithms to prevent overlapping class instances.
    \item The DES technique was pivotal in selecting the most effective classifiers, optimizing overall performance.
    \item Our methodology excels in handling multiclass imbalanced stream challenges, maintaining high performance as class distributions evolve.
    \item The approach is marked by its adaptability, efficiency, and scalability, offering dynamic model updates to ensure real-time reliability.
    \item Despite its advantages, the methodology has limitations such as the significant time required to generate non-overlapping synthetic instances, and its reliance on the performance of MLSMOTE and MLSOL techniques.
\end{itemize}

\section{Future Work}
\label{section:7_2}
\begin{itemize}
    \setlength{\itemsep}{1.5pt}
    \setlength{\parskip}{1.5pt}
    \item Future research should focus on developing more advanced oversampling techniques that avoid overlapping synthetic instances, thus reducing the computational burden.
    \item Exploring meta-learning methods to better assess imbalanced multiclass ratios and improve minority class identification could enhance the effectiveness of the approach.
    \item Further development of advanced concept drift detection algorithms is needed to improve the timely recognition of drift and better handle the evolving data distributions.
    \item The incorporation of alternative ensemble strategies and deep learning techniques could lead to further improvements in both drift detection and classifier performance.
    \item Expanding the methodology to address a broader range of real-world applications, including streaming from diverse domains, will enhance its applicability and robustness.
    \item Future work should also focus on refining classifier selection and prediction methods, especially for handling multisource domains in real-time environments.
    \item Investigating the use of deep learning methods to predict future drifts and optimize classifier performance is a promising direction for improving model adaptation and performance.
\end{itemize}


% In this chapter, we presented a comprehensive methodology for incremental learning in data streams, particularly addressing the complexities of imbalanced data, including minority and overlapping classes \ref{section:7_1}, Emerging new classes in the incremental drifted streams \ref{section:7_2} and the heterogeneous transfer learning in the drifted streams \ref{section:7_2}.

% \section{Imalanced multiclass stream}
% \label{section:7_1}

% This study integrates a novel oversampling technique, concept drift detection, and Dynamic Ensemble Selection (DES) to identify the most suitable ensemble classifier. Extensive experimentation across various datasets, including benchmarks, real-world streams, and synthetic data, demonstrated the effectiveness of our methodology.
% A key feature of our approach is the rapid detection of concept drifts, which allows the system to adapt quickly by training new base classifiers, ensuring continued relevance and accuracy in real-time scenarios. We addressed the minority class problem through advanced oversampling, while the KNN algorithm was employed to prevent the creation of overlapping class instances. The DES technique played a crucial role in selecting the most effective classifiers, optimizing overall performance.

% The results of our evaluation, based on multiple performance metrics, show that our methodology is particularly effective in handling multiclass imbalanced stream challenges. It excels in maintaining high accuracy as class distributions evolve. Beyond accuracy, the approach is marked by its adaptability, efficiency, and scalability. The capability for dynamic model updates driven by incoming data allows for continuous adaptation to changing data distributions, ensuring the model's reliability and relevance in real-time applications.

% However, our methodology does have certain limitations. The time required to generate non-overlapping synthetic instances is significant, and the approach's success heavily depends on the accuracy of MLSMOTE and MLSOL techniques. Future research should focus on developing more advanced oversampling techniques that avoid overlapping synthetic instances. Additionally, exploring meta-learning methods to better assess imbalanced multiclass ratios and minority class identification could further improve the approach.

% \section{Emerging new classes}
% \label{section:7_2}

% This section, a robust framework for incremental learning in data streams with emerging new classes. By integrating the ADWIN algorithm for concept drift detection, K-means clustering, and Gaussian Naive Bayes (GNB) as the base classifier within an ensemble stratified bagging framework, our method effectively tackles the challenges posed by such dynamic data streams. The ADWIN algorithm is particularly important, enabling the prompt detection of concept drifts and new classes, which is critical for training new classifiers and maintaining accuracy over time.

% The use of ensemble stratified bagging combined with DES significantly enhances predictive performance and robustness. Our evaluations confirm that the framework is highly effective in classifying data streams with evolving distributions, especially when GNB serves as the base classifier. In addition to its high accuracy, the framework's strengths lie in its adaptability, efficiency, and scalability. The dynamic updates to the model, based on incoming data, ensure continuous adaptation to shifting data patterns, supporting real-time reliability and relevance.

% Looking ahead, future work could explore several avenues for improvement. These include the development of advanced concept drift detection algorithms, the exploration of alternative ensemble strategies, and the incorporation of deep learning techniques. Additionally, expanding the framework to address a broader range of real-world applications and refining the selection and prediction methods for classifiers will be crucial for further enhancing the methodology's effectiveness.

% \section{Heterogeneous transfer learning}
% \label{section:7_3}

% This study introduces a comprehensive Approach for incremental learning in data streams, particularly focusing on heterogeneous transfer learning. The framework effectively addresses the challenges associated with this context by integrating concept drift detection through the ADWIN algorithm, employing eigenvectors, and utilizing ensemble classifiers. The efficacy of this approach was validated through extensive experiments on a variety of datasets, including benchmark datasets, real-world application streams, and synthetic data.

% A key strength of this approach is the pivotal role played by the ADWIN algorithm, which enables the timely detection of concept drifts and changes in data patterns. This allows the framework, referred to as HTL, to adapt by training new base classifiers, ensuring their continued relevance and accuracy in real-time scenarios. The use of ensemble classifiers further enhances predictive performance and robustness by aggregating predictions from multiple base classifiers. Dynamic Ensemble Selection (DES) is utilized to select the most suitable base classifiers, optimizing overall performance.

% Our thorough evaluation using various performance measures confirms the approach’s effectiveness in addressing the challenges of heterogeneous multisource domains. The framework excels in accurately classifying data streams with evolving class distributions, particularly when leveraging ADWIN as the drift detector, eigenvectors for knowledge transfer across heterogeneous domains, and DES for classifier selection.

% Beyond its strong accuracy and performance, the framework offers significant advantages in terms of adaptability, efficiency, and scalability. The ability to update the model dynamically, based on incoming data instances, ensures continuous adaptation to changing data distributions, maintaining the framework's reliability and relevance in real-time scenarios.

% Looking ahead, future research could focus on further enhancing this approach. Potential improvements include developing advanced concept drift detection methods, refining classifier selection by focusing on positive target and multisource classifiers, and incorporating deep learning techniques to predict future drifts and optimize classifier performance. These advancements could further bolster the framework’s capability to manage complex, evolving data streams in diverse real-world applications.