\documentclass[oneside,12pt]{Latex/Classes/PhDthesisPSnPDF}
\usepackage{enumitem, kantlipsum}
\usepackage{mathtools}
\usepackage{amssymb}
\usepackage[warn]{textcomp}
\usepackage{amsbsy}
\usepackage{pifont}
\usepackage{numprint}
\usepackage{lscape}
\usepackage{booktabs}
\usepackage{tikz}
\usepackage{notoccite}
\usepackage[export]{adjustbox}
\usepackage[ruled,vlined]{algorithm2e}
\usepackage{amsmath}
\usepackage{threeparttable, tablefootnote}
\usepackage{tabularx}
\usepackage[table]{xcolor}
\usepackage{array}

\DeclarePairedDelimiter\ceil{\lceil}{\rceil}
\DeclarePairedDelimiter\floor{\lfloor}{\rfloor}
\newcommand\scalemath[2]{\scalebox{#1}{\mbox{\ensuremath{\displaystyle #2}}}}
\newcommand{\cmark}{\ding{51}}
\newcommand{\xmark}{\ding{55}}
\newcommand{\ssep}{;}
\newcommand{\RFCOM}{\raisebox{2pt}{\tikz{\draw[-,black!40!black,solid,line width = 0.9pt](0,0) -- (5mm,0);}}}
\newcommand{\RFSM}{\raisebox{2pt}{\tikz{\draw[-,black!40!black,dashed,line width = 0.9pt](0,0) -- (5mm,0);}}}

\newcommand*\circled[1]{\tikz[baseline=(char.base)]{
            \node[shape=circle,draw,inner sep=1pt] (char) {#1};}}

\newcommand\diag[4]{%
  \multicolumn{1}{p{#2}|}{\hskip-\tabcolsep
  $\vcenter{\begin{tikzpicture}[baseline=0,anchor=south west,inner sep=#1]
  \path[use as bounding box] (0,0) rectangle (#2+2\tabcolsep,\baselineskip);
  \node[minimum width={#2+2\tabcolsep},minimum height=\baselineskip+\extrarowheight] (box) {};
  \draw (box.north west) -- (box.south east);
  \node[anchor=south west] at (box.south west) {#3};
  \node[anchor=north east] at (box.north east) {#4};
 \end{tikzpicture}}$\hskip-\tabcolsep}}


%: Macro file for Latex
% Macros help you summarise frequently repeated Latex commands.
% Here, they are placed in an external file /Latex/Macros/MacroFile1.tex
% An macro that you may use frequently is the figuremacro (see introduction.tex)
\include{Latex/Macros/Macros}

%: ----------------------------------------------------------------------
%:                  TITLE PAGE: name, degree,..
% ----------------------------------------------------------------------
% 201803015 LED: I create some metadata to have a unique source for these fields
\def\myauthor{Ahmed Hamdy Madkour} % Author
%\def\mytitle{Pedestrian Localization using Wrist-Worn Smart Devices} % title
\def\mytitle{Learning in Non-stationary Envioronment } % title
\def\myadvisor{Dr. Amgad Monir and Prof. Hatem Mohamed} % Supervisor

% if output to PDF then put the following in PDF header
\ifpdf
    \pdfinfo { /Title  (\mytitle)
               /Creator (TeX)
               /Producer (pdfTeX)
               /Author (\myauthor)
               /CreationDate (D:201803150940)  %format D:YYYYMMDDhhmmss
               /ModDate (D:YYYYMMDDhhmm)
               /Subject (Pedestrian Localization)
               /Keywords (Auto machine learning, Concept drift, Imbalanced Stream, Transfer Learning, Emerging new Classes, and Dynamic Ensemble Selection.) }
    \pdfcatalog { /PageMode (/UseOutlines)
                  /OpenAction (fitbh)  }
    % 201803015 LED: \pdfinfo does not work in combination with hyperref
    \hypersetup{
    	pdfinfo={
    			Title={\mytitle},
                Author={\myauthor},
                Creator={TeX},
                Producer={pdfTeX},
                CreationDate={D:201803150940},
                ModDate={D:\pdfdate},
                Subject={Transportation},
                Keywords={Auto machine learning, Concept drift, Imbalanced Stream, Transfer Learning, Emerging new Classes, and Dynamic Ensemble Selection.}
    	}
}
\fi

% below is to generate the title page with crest and author name

% Title of the dissertation
%\title{Pedestrian Localization using Wrist-Worn Smart Devices}
\title{\mytitle}
\author{\myauthor}
\advisor{\myadvisor}

% ----------------------------------------------------------------------
% This section below defines front covert (external and internal)
% Shield logo
%\crest{\includegraphics[width=2cm]{Deusto_Shield}}
\crest{\includegraphics[width=2cm]{faculty_image.png}}
% \crest{\includegraphics[width=2cm]{menoufia_logo.png}}
% Full logo
%\crest{\includegraphics[width=6cm]{UDeusto}}
\university{Menoufia University}
\degree{A dissertation submitted in partial fulfillment of the requirements for the Doctor of Philosophy degree, Faculty of Computers and Information, Menoufia University}
\textadvisor{Supervised by }
\textsignaturecandidate{The candidate}
\textsignatureadvisor{The supervisor}
\cityofbirth{Menoufia}
\degreedate{\monthname \ \the\year}
%\degreedate{Month de \the\year}


% ----------------------------------------------------------------------

% turn of those nasty overfull and underfull hboxes
\hbadness=10000
\hfuzz=50pt



\begin{document}

%\selectlanguage{british}
\selectlanguage{english}

% sets line spacing
\renewcommand\baselinestretch{1.2}
\baselineskip=18pt plus1pt

% Watermark
%\watermark{DRAFT	DRAFT	DRAFT	DRAFT	DRAFT	DRAFT	DRAFT	DRAFT	DRAFT}

% 
% Custom command to format text with two images in between and captions

\begin{titlepage}
    \begin{tabbing}
        \hspace{1cm} % Adjust this to control the space on the left side
        \= \kill % Set tab stops
        \begin{tabular}{ccc} % 3 columns with space between them
          \includegraphics[width=3cm]{0_frontmatter/figures/PNG/faculty_image.png} & \hspace{2.5cm} & \includegraphics[width=3cm]{0_frontmatter/figures/PNG/menoufia_logo.png} \\
          \multicolumn{1}{l}{Faculty of Computer and Information} & \hspace{2.5cm}  & \multicolumn{1}{l}{Menoufia  University} \\
        \end{tabular}\\
    \end{tabbing}
    \begin{center}
        \vspace{0.5cm}

        \textbf{\Huge \normalfont Learning in Nonstationary Environment:Concept Drift handling techniques} 
        \vspace{0.6cm}
        
        \textbf{\large \normalfont A thesis submitted to the Faculty of Computers and Information, Menoufia University, in partial fulfillment of the requirements for the degree of Master of Computers and Information} \\
        \vspace{0.4cm}
        
        \textbf{\large In} \\
        \vspace{0.3cm}
        
        \textbf{\Large [Information Systems]} \\
        \vspace{0.6cm}
        
        \textbf{\Large By} \\
        \vspace{0.3cm}
        
        \textbf{\LARGE  Ahmed Hamdy Madkour} \\
        \vspace{0.3cm}
        
        \textbf{\large \normalfont Teaching Assessment at Information Systems Department, Faculty of Computers and Information, Menoufia University} \\
        \vspace{0.7cm}
        
        \textbf{\Large Supervised By} \\

    \begin{tabbing}
        \hspace{1cm} % Adjust this to control the space on the left side
        \= \kill % Set tab stops
        \begin{tabular}{ccc} % 3 columns with space between them
            \textbf{Prof. Hatem Mohamed} && \textbf{Dr. Amgad Monir} \\
            Professor of Information Systems,  && Lecturer of Information Systems,\\
             Dean for Faculty of Computers and &&  Faculty of Computers and Information,
            \\
            Information, Menoufia University  && Menoufia University \\\\
            \hspace{0.1cm} [ \hspace{3.5cm} ] && \hspace{0.1cm} [ \hspace{3.5cm} ]
        \end{tabular}
    \end{tabbing}
    \vspace{0.4cm}
    \textbf{\Large 2024} 
    \end{center}
\end{titlepage}

% 
% Custom command to format text with two images in between and captions

\begin{titlepage}
	\begin{tabbing}
		\hspace{1cm} % Adjust this to control the space on the left side
		\= \kill % Set tab stops
		\begin{tabular}{ccc} % 3 columns with space between them
			\includegraphics[width=3cm]{0_frontmatter/figures/PNG/faculty_image.png} & \hspace{2.5cm} & \includegraphics[width=3cm]{0_frontmatter/figures/PNG/menoufia_logo.png} \\
			\multicolumn{1}{l}{Faculty of Computer and Information}                  & \hspace{2.5cm} & \multicolumn{1}{l}{Menoufia  University}                                 \\
		\end{tabular}
	\end{tabbing}
	\begin{center}
		\vspace{0.5cm}

		\textbf{\Huge \normalfont Learning in No Stationary Environment:Concept Drift handling techniques}
		\vspace{0.3cm}

		\textbf{\large \normalfont A thesis submitted to the Faculty of Computers and Information, Menoufia University, in partial fulfillment of the requirements for the degree of Master of Computers and Information} \\
		\vspace{0.2cm}

		\textbf{\large In} \\
		\vspace{0.2cm}

		\textbf{\Large [Information Systems]} \\
		\vspace{0.5cm}

		\textbf{\Large By} \\
		\vspace{0.3cm}

		\textbf{\LARGE  Ahmed Hamdy Madkour} \\
		\vspace{0.1cm}

		\textbf{\large \normalfont Teaching Assessment at Information Systems Department, Faculty of Computers and Information, Menoufia University} \\
		\vspace{0.5cm}

		\textbf{\Large Examination Committee}
		\vspace{0.0cm}

		\begin{tabbing}
			\begin{tabular}{ccc} % 3 columns with space between them
				\textbf{Prof. Hatem Mohamed}      &  & \textbf{Prof. Hatem Mohamed}              \\
				Professor of Information Systems,  &  & Professor of Information Systems,      \\
				Dean for Faculty of Computers and &  & Faculty of Computers and Information,
				\\
				Information, Menoufia University  &  & Menoufia University                   \\
				\hspace{0.1cm} [ \hspace{3.5cm} ] &  & \hspace{0.1cm} [ \hspace{3.5cm} ]\\
			\end{tabular}
		\end{tabbing}
		

        \begin{center} % Center the entire block on the page
            \begin{tabbing}
                \begin{tabular}{ccc} % Single column with centered content
                  \hspace{3cm}  &\textbf{Prof. Hatem Mohamed} &\\
                    & LProfessor of Information Systems, &\\
                    & Faculty of Computers and Information, &\\
                    & Menoufia University &\\
                    & \hspace{0.1cm} [ \hspace{3.5cm} ]&
                \end{tabular}
            \end{tabbing}
        \end{center}

		\vspace{0.2cm}
		\textbf{\Large 2024}

	\end{center}
\end{titlepage}



% The frontmatter text starts here
\frontmatter

% 
% Thesis Abstract -----------------------------------------------------


%\begin{abstractslong}    %uncommenting this line, gives a different abstract heading


\begin{alwayssingle} \pagestyle{empty}\begin{center}
    \vspace*{1.5cm}
    {\Large \bfseries  Abstract}
  \end{center}      %this creates the heading for the abstract page
    This thesis addresses challenges in machine learning models when dealing with multi-class imbalanced data streams in non-stationary environments. These challenges result in biases, unreliable predictions, and diminished model performance. To overcome these issues, the thesis introduces innovative approaches that combines Dynamic Ensemble Selection (DES) technique, an adaptive method for handling imbalanced multi-class data streams, a concept drift detector, eigen vector technique, and the K-Nearest Neighbors (KNN) algorithm to address class overlap.
    The main objective is to improve the classification of imbalanced multi-class drifted data streams. The adaptive oversampling method generates synthetic samples to mitigate imbalanced data stream issues, with KNN ensuring non-overlapping generated samples. A drift detector assists in deciding whether to keep existing classifiers or create new ones to handle incoming data streams. Dynamic Ensemble Selection (DES) optimizes the classification task by selecting the most appropriate classifier for incoming data. The proposed method offers an effective solution for achieving accurate and resilient classification in the context of imbalanced multi-class drifted data streams. 
    Additionally, the thesis discusses challenges posed by emerging new classes in dynamic data streams for incremental learning. Existing approaches struggle with high false positive rates, long prediction times, and the impractical assumption of having access to true labels for all instances when new classes emerge. To address these challenges, the thesis proposes a novel framework that integrates concept drift detection using the ADWIN method and ensemble stratified bagging. By leveraging true labels during the learning process, the framework effectively detects concept drifts and adapts to accommodate emerging new classes. 
    Furthermore, the thesis discusses challenges faced by machine learning models in heterogeneous multi-source streams and non-stationary conditions, which can introduce biases and unreliable predictions. In response, the thesis introduces a comprehensive framework that combines dynamic ensemble selection, eigenvector techniques, and a concept drift detector to enhance transfer learning of multi-class data streams subject to drift. During the training phase, the proposed weighted method assigns appropriate weights to each classifier, mitigating adverse learning effects. Eigenvectors handle heterogeneous multi-source streams, and Dynamic Ensemble Selection (DES) determines the most fitting classifiers for incoming data. A concept drift detector identifies and addresses drifted chunks within the data stream.
    The experimental results on various datasets, encompassing benchmark datasets, a real application stream dataset, and synthetic data streams, consistently demonstrate the superior performance of the proposed approach in effectively addressing challenges inherent in such dynamic data streams. Through evaluations conducted on both real-world and synthetic datasets, the framework exhibits exceptional accuracy, precision, recall, geometric mean, and F1-measure, all while maintaining an efficient runtime.
    This robust performance positions the proposed approach as a frontrunner in enabling incremental learning within real-time scenarios, particularly in the context of emerging new classes. The framework provides a comprehensive solution to the intricate complexities associated with incremental learning in dynamic data streams. Moreover, the experimental findings underscore the efficacy and superiority of the proposed approach, particularly when confronted with the unique challenges posed by imbalanced drifted streams, emerging new classes, and heterogeneous multi-source drifted data streams.
    Comparisons against state-of-the-art chunk-based methods consistently highlight the effectiveness of this approach in successfully managing the complexities inherent in these multifaceted data stream challenges. In essence, the proposed approach stands out as a powerful and versatile solution, showcasing its ability to outperform existing methods and addressing the evolving demands of imbalanced drifted streams and heterogeneous multi-source streams.
    Keywords: - Auto machine learning, Concept drift, Imbalanced Stream, Transfer Learning, Emerging new Classes, and Dynamic Ensemble Selection.
    
\end{alwayssingle}

% \begin{resumen}        %this creates the heading for the abstract page
%\selectlanguage{spanish}
% Motivación

% \end{resumen}


%\begin{laburpena}        %this creates the heading for the abstract page
%\selectlanguage{basque}
%% Jarri zure laburpena hemen.
%%
%
%
%\end{laburpena}

%\end{abstractlongs}


% ----------------------------------------------------------------------

% \begin{alwayssingle} \pagestyle{empty}\begin{center}
    \vspace*{1.5cm}
    {\Large \bfseries  Summary}
  \end{center}       %this creates the heading for the abstract page
The thesis proposes three innovative solutions to challenges faced by machine learning models in handling multi-class imbalanced data streams, emerging new classes, and heterogeneous muti-class streams in non-stationary environments. The primary issues include bias, unreliable predictions, and decreased overall model performance.
The first proposed approach integrates the Dynamic Ensemble Selection (DES), an adaptive oversampling method for handling imbalanced multi-class data streams, a concept drift detector, and the K-Nearest Neighbors (KNN) algorithm to address class overlap. The adaptive oversampling method generates synthetic samples, leveraging KNN to ensure non-overlapping samples. A drift detector helps decide whether to keep existing classifiers or create new ones for incoming data streams. Dynamic Ensemble Selection optimizes classifier performance. The approach demonstrates effectiveness in achieving accurate and resilient classification in imbalanced multi-class drifted data streams, as validated through experiments on various datasets.
Similarly, the second paper addresses the challenge of emerging new classes in dynamic data streams, affecting incremental learning. The second proposed approach integrates concept drift detection using the ADWIN method and ensemble stratified bagging. Leveraging true labels during the learning process, this approach detects concept drifts and adapts to emerging new classes. Evaluation using real and synthetic datasets shows superiority in accuracy, precision, recall, geometric mean, and F1 measure while maintaining efficient runtime. The framework promises to enable incremental learning in real-time scenarios, offering a comprehensive solution by combining concept drift detection (ADWIN) and ensemble stratified bagging.
Lastly, the third approach tackles challenges arising from heterogeneous multisource streams and non-stationary conditions in machine learning models. The third proposed approach combines DES, eigen vector techniques, and a concept drift detector to enhance transfer learning of multi-class data streams subject to drift. The weighted method assigns appropriate weights to each classifier during training, mitigating adverse learning effects. Eigenvectors handle heterogeneous multisource streams, and DES determines fitting classifiers for incoming data. The concept drift detector identifies and addresses drifted chunks within the data stream. Experiments on various datasets demonstrate the efficacy and superiority of the proposed approach in addressing challenges presented by transfer learning from heterogeneous multi-sources and concept drift, outperforming state-of-the-art chunk-based methods.

\end{alwayssingle}


% ----------------------------------------------------------------------

% Thesis Acknowledgements ------------------------------------------------


% Opening of the acknowledgements

%Short version
%this creates the heading for the acknowlegments
\begin{acknowledgements}
    First and foremost, I give my deep thanks to Allah for giving me the opportunity and the strength to accomplish this work.



    I would like to thank my supervisors Dr. Hatem Mohammed and Dr. Amgad Monir for their help and support during my work for creating a very inspiring research environment. They have been helpful with background information and have continually encouraged me and helped me with comments on my work. They always had time to discuss new ideas and give feedback on early ideas and research problems.
    
    I would like to thank my family who was a constant source of rising and supporting my spirit. Thanks go out especially to my father, my mother, my wife, and my brothers for support and encouragement.
    
    Finally, special thanks to my faculty, department, and my colleagues.
    
\begin{flushright}
\textit{Thank you everyone,}

Ahmed Madkour

% Moth and year
\monthname \ \the\year



% Signature figure

%\begin{figure}[htbp!]
%\end{figure}
%\includegraphics{signature}%



\end{flushright}



%Closing of the acknowledgements
%Short version
\end{acknowledgements}
% Long version
%\end{acknowledgementslong}

% ------------------------------------------------------------------------




% As abstract contains various languages we set the main language again
%\selectlanguage{british}
\selectlanguage{english}


%: ----------------------- contents ------------------------

\setcounter{secnumdepth}{5} % organisational level that receives a numbers
\setcounter{tocdepth}{5}    % print table of contents for level 3

\tableofcontents 
\listoffigures	% print list of figures
\listoftables  % print list of tables
\listofalgocfs
\printnomenclature % [] = distance between entry and description
\label{sec:glossary} % target name for links to glossary

\mainmatter
\pagestyle{fancy}


% =======================[chapters]=======================
% introduction

% this file is called up by thesis.tex
% content in this file will be fed into the main document


% this file is called up by thesis.tex
% content in this file will be fed into the main document

%: ----------------------- introduction file header -----------------------


\begin{savequote}[50mm]
Success is the sum of small efforts, repeated day-in and day-out.
\qauthor{Robert Collier}
\end{savequote}

\chapter{Introduction}
\label{cha:1_Introduction}

% the code below specifies where the figures are stored
\ifpdf
    \graphicspath{{1_introduction/figures/PNG/}{1_introduction/figures/PDF/}{1_introduction/figures/}}
\else
    \graphicspath{{1_introduction/figures/EPS/}{1_introduction/figures/}}
\fi


%-------------------------------------------------------------------------
%Chapter 1 contents:
%- Motivation of the research field: Context-aware systems -> LBS -> GNSS limitation -> Positioning techniques -> DR -> inertial PDR -> inertial PDR + wearables
%- Problem identification: smartphone not a wearable -> potentiality of wrist-worn wearables -> Problem: no wrist-worn PDRS
%- Goal of the thesis: tackle the problem -> how? Splitting it into sub-problems
%- Structure of the thesis
%-------------------------------------------------------------------------

Governments and companies are producing vast streams of data and require effective data analytics and machine learning methods to assist in making predictions and decisions promptly. One crucial aspect is the machine learning pipeline, which involves training a prepared dataset to construct a model and subsequently utilizing this model to predict new instance outputs. As depicted in Fig. (1.1), the process entails fetching historical data from the database during the training phase to construct the machine learning model. Then, the system can input new instances from the database to predict the output.

\begin{figure}[!ht]
    \centering
    \includegraphics[width=.9\textwidth]{1_introduction/figures/PNG/machine_flow.png}
    \caption{The research methodology of the thesis.}
    \label{ch1:research-emthodo}
\end{figure}

Nevertheless, when endeavoring to forecast outcomes for fresh instances sourced from an alternative database, as illustrated in Fig. (1.2), there frequently emerges a conspicuous decline in accuracy. This disparity accentuates the imperative for model developers to intervene and rectify the issue. Addressing this, developers must adjust and retrain the model utilizing datasets from the new environment to ameliorate accuracy. This iterative process aims to refine the model's precision and ensure its efficacy across diverse contexts, thereby bolstering the reliability of decision-making and predictive capabilities. To confront this challenge, the field of auto machine learning endeavors to facilitate online updates to the model without necessitating direct intervention from developers for modification.

\begin{figure}[!ht]
    \centering
    \includegraphics[width=.8\textwidth]{1_introduction/figures/PNG/wrong_machine_flow_1.png}\\
    (a) \\
    \includegraphics[width=.8\textwidth]{1_introduction/figures/PNG/wrong_machine_flow_2.png}\\
    (b)
    \caption{The research methodology of the thesis.}
    \label{ch1:research-emthodo}
\end{figure}



In recent years, the surge in high-speed data streams has posed notable challenges for machine learning models, particularly in the context of streaming data analysis. These data streams, characterized by continuous, dynamic, and high-volume data arrivals, demand adaptive learning algorithms that can effectively cope with their evolving nature [1] [2] [3]. Within these evolving data environments, two paramount challenges have emerged: concept drift, class imbalance, Emerging new class, and heterogenous transfer learning.

Concept drift, a phenomenon defined by the evolving statistical properties of a data generation process over time [4] [5]. introduces a dynamic element to the data, necessitating continuous adaptation of machine learning models. This shift can manifest as changes in underlying concepts, relationships between variables, or alterations in data distribution. Traditional models trained on historical data may suffer diminished accuracy or become inadequate when confronted with new data influenced by concept drift, highlighting the need for effective concept drift detection mechanisms. Addressing concept drift involves the utilization of concept drift detectors, which are methods capable of identifying changes in data stream distributions. These detectors rely on information related to classifier performance or incoming data items to signal the need for model updates, retraining, or even replacing the old model with a new one. The dynamic nature of concept drift necessitates ongoing monitoring and adaptation to maintain the model's efficacy.

Data streams also present challenges related to class imbalance, a condition characterized by uneven distribution among different classes [6] [7]. This scenario, especially prevalent in multi-class settings, poses a significant challenge for traditional classifiers. The risk of misclassifying minority class samples due to their limited representation demands specialized techniques to ensure accurate classification without sacrificing the performance of the majority class [8] [9] [10] [11]. To tackle class imbalance, three primary methods are commonly employed: sampling methods, algorithm adaptation methods, and hybrid methods. Sampling methods involve undersampling the majority class or oversampling the minority class to balance class distribution. Algorithm adaptation methods modify existing algorithms to handle imbalanced data [12] [13] [14], while hybrid methods combine data preprocessing with classification techniques, often utilizing ensemble classifiers to effectively mitigate class imbalance and enhance overall classifier performance [15] [16] [17] [18].

Another challenge arising in the context of class imbalance is class overlap, where instances from different classes share the same region in data space [17] [18]. This overlap complicates the task of distinguishing between representative instances of different classes, leading to performance challenges for traditional classifiers referred to as overlapping problems. Recent research introduces class-overlap undersampling methods to address this issue, leveraging local similarities among minority instances to identify potentially overlapping majority instances.

Therefore, both class imbalance and class overlap present significant hurdles in the realm of data stream analysis. Consequently, addressing class imbalance has become crucial in multi-class learning, leading to research efforts focusing on both concept drift and class imbalance challenges. Researchers have explored dynamic ensemble selection (DES) and multi-class oversampling techniques to tackle these issues. Dynamic classifier ensembles offer a unique ability to adapt their composition based on data characteristics, making them valuable in situations with evolving data conditions [19]. Researchers focus on the overproduce-and-select approach for classifier ensemble selection methods. The objective of classifier ensemble selection is to choose an optimal subset of classifiers from a larger ensemble. The selection process is guided by various criteria, including individual performance measures, diversity metrics, meta-learning techniques, and performance estimation approaches. This optimization is particularly important in scenarios where a balance between accuracy and computational resource constraints is critical. There are two distinct approaches: static and dynamic selection. Static selection involves assigning classifiers to predefined feature partitions, while dynamic selection adaptively selects classifiers based on their competency [20]. Dynamic selection offers two choices: individual models, known as Dynamic Classifier Selection (DCS), and ensemble models, called Dynamic Ensemble Selection (DES). DCS algorithms enable the selection of the most appropriate classifier for each data point based on its local competencies. In contrast, DES focuses on selecting the optimal classifiers for each instance based on their competence within localized regions [21] [22] [23] Competency assessment relies on a dynamic selection dataset (DSEL) containing labeled samples. Moreover, innovative techniques like the Randomized Reference Classifier introduce randomness into class supports to enhance adaptability in addressing challenges related to imbalanced data.

Additionally, transfer learning assumes a pivotal role in addressing the intricate challenges posed by dynamic data streams and inherent concept drift. This domain of research focuses on enhancing a model's learning performance within a target domain by harnessing knowledge gleaned from source domains [4] [5].Techniques in transfer learning include reducing domain gaps through instance re-weighting and feature matching, along with strategies to mitigate negative knowledge transfer by down-weighting irrelevant source data.

Lastly, in the study, the focus extends to the specific scenario of Streams with Emerging New Classes (SENC). This refers to situations where new classes, not present during the initial training of a learning model, emerge in the data stream. Traditional learning approaches, designed for fixed or predefined class distributions, face challenges in effectively recognizing and adapting to these novel classes in real-time. The need for adaptive learning mechanisms that can handle the emergence of new classes underscores the complexity of real-world data stream scenarios.
     

In this chapter, the motivation for this research along with the research questions
that naturally arise are discussed in  Section~\ref{sec:1_1_motivation}. After this, the objectives and contributions are presented in Sections \ref{sec:1_2_opportunity} and \ref{sec:1_3_1_goal}, respectively. Next, the
research methodology is summarised in Section \ref{sec:1_3_automl_and_tf}. Finally,
the research context and the outline of this thesis are presented in Sections \ref{ch1:research-context} and
\ref{sec:1_3_2_DissertationStructure}, respectively.
\section{Challenges}
\label{sec:1_introduction_challange}
Within the swiftly evolving realm of machine learning, the proliferation of high-speed data streams presents a significant hurdle — the proficient management of non-stationary data environments. This challenge is marked by dynamic fluctuations in statistical attributes, fundamental concepts, and data distributions over time. The crux of the issue emerges as models, initially trained on historical data, experience a decline in accuracy when confronted with new data influenced by concept drift. This includes scenarios such as:
\begin{itemize}
    \setlength{\itemsep}{0pt}
    \setlength{\parskip}{0pt}
    \item Multi-class imbalanced data streams
    \item Overlapping classes
    \item Emergence of new classes
    \item Heterogeneous transfer learning
\end{itemize}
However, addressing these challenges yields substantial benefits. By effectively managing non-stationary data environments, organizations can enhance the adaptability and resilience of their machine learning systems. This enables more robust decision-making processes, improved predictive capabilities, and heightened performance across diverse contexts. Moreover, it fosters innovation and agility, empowering organizations to stay ahead in dynamic markets and evolving scenarios.

\section{Thesis Motivations}
\label{sec:1_introduction_motivation}

The swift progress in machine learning, particularly in real-time data streams, introduces new challenges that demand innovative solutions. This thesis seeks to address the limitations of traditional models in dynamic, non-stationary data environments. The motivations behind this research are as follows:
\begin{itemize}
    \setlength{\itemsep}{0pt}
    \setlength{\parskip}{0pt}
    \item \textbf{Adapting to Emerging Classes in Real-Time:} New classes within data streams can cause a decline in accuracy. The goal is to develop adaptive mechanisms that quickly incorporate new classes, maintaining system relevance.
    \item \textbf{Proactive Management of Concept Drift:} Concept drift degrades model performance as data properties change. This research focuses on strategies to detect and manage concept drift to maintain model accuracy.
    \item \textbf{Dynamic Optimization of Classifier Ensembles:} To address emerging classes and shifting distributions, the goal is to develop techniques for the real-time optimization of classifier ensembles.
    \item \textbf{Addressing Multi-Class Imbalance:} Imbalanced multi-class distributions often lead to biased classification. The thesis aims to create methods that address class imbalances dynamically, ensuring fair classification.
    \item \textbf{Enhancing Transfer Learning in Non-Stationary Environments:} Transfer learning can suffer from negative transfer in non-stationary settings. The research focuses on frameworks that minimize negative transfer and enhance knowledge sharing across diverse domains.
\end{itemize}

\section{Thesis Objectives}
\label{sec:1_introduction_objectives}
The thesis objectives for our research stems from the imperative need to address the following key motivations:
\begin{itemize}
    \item \textbf{Objective 1}. Thandling Imbalanced Multiclass Drifted Data and overlapping classes streams.
    \item \textbf{Objective 2}. Addressing Emerging New Classes in Incremental Streams via Concept Drift and K-means Techniques.
    \item \textbf{Objective 3}. Addressing Heterogeneous Transfer Learning Problem in data streams via Concept Drift and Eigenvector Techniques.
\end{itemize}

\section{Thesis Questions}
\label{sec:1_introduction_questions}

\begin{itemize}
    % \setlength{\itemindent}{-.5in}
       \item $\pmb{Q_1}$. What is the impact of imbalanced stream on the performance of ML models in non-stationary Environment?
        \item $\pmb{Q_2}$. What is the impact of reducing overlapping classes stream on the performance of ML models in non-stationary environment?
        \item $\pmb{Q_3}$.How does the emergence of new classes in data streams affect the stability and performance of ML models?
        \item $\pmb{Q_4}$. how to employee concept drift to solve the emerging new classes problem?
        \item $\pmb{Q_5}$. How does the Heterogeneous sources in data streams affect the stability and performance of Transfer Learning Technique?
        \item $\pmb{Q_6}$. What approaches or techniques can be employed in heterogeneous transfer learning to facilitate knowledge transfer across diverse sources and domains?
    \end{itemize}
    
\section{Thesis Contributions}
\label{sec:1_introduction_contribution}

Our research focuses on developing advanced frameworks to manage non-stationary data streams with high accuracy and minimal computational complexity. The key contributions include:

\begin{itemize}
    \setlength{\itemsep}{0pt}
    \setlength{\parskip}{0pt}
    \item \textbf{Concept Drift Detection and Ensemble Classifier:} Integrating concept drift detection with ensemble classifiers for real-time adaptation in transfer learning.
    \item \textbf{Dynamic Classifier Ensemble for Emerging Classes:} A framework that dynamically adjusts classifiers to address emerging class issues in non-stationary streams.
    \item \textbf{Precise Weighting for Local Classifiers:} A novel method to refine local classifier contributions, enhancing overall ensemble accuracy.
    \item \textbf{Eigenvector-Based Framework for Heterogeneous Transfer Learning:} A framework using eigenvectors to facilitate knowledge transfer across diverse domains, improving performance.
    \item \textbf{Dynamic Adjustment for Multi-class Imbalanced Data:} A method combining drift detection and optimized ensemble selection to improve accuracy in imbalanced streams.
    \item \textbf{Adaptive Class Imbalance Method:} A dynamic approach to select oversampling methods, addressing class overlap in drifted data streams.
\end{itemize}

% 
\section{Publications}
\label{sec:1_introduction_publication}
During the research activities of this thesis, several international peer-reviewed
journal articles were published to disseminate the obtained results. The publications can be found in Table \ref{ch1.publication.list}. 

\begin{table}[!ht]
 \centering
 \caption{Publications in journals and conferences conducted during this thesis.}
\label{ch1.publication.list}
\renewcommand{\arraystretch}{1.1}
\begin{tabular}{l}
\hline
 \textbf{Title:} Vertical and Horizontal Data Partitioning for Classifier Ensemble Learning.\\
 \textbf{Authors:} AM Mohammed, E. ~Onieva, M.~Wo{\'{z}}niak.\\
 \textbf{Congress:} The 11th International Conference on Computer Recognition Systems,\\$\quad \quad \quad \quad$ 2019, Poland. \\ \line(1,0){400}\\
 \textbf{Title:} Training set selection and swarm intelligence for enhanced integration in \\ $\quad \quad \quad$ multiple classifier systems.\\
\textbf{Authors:} AM Mohammed, E. ~Onieva, M.~Wo{\'{z}}niak.\\
 \textbf{Journal:} Applied Soft Computing (Impact Factor $=$ 5.472 $\rightarrow$ Q1).\\
 \textbf{Status:} Published. \\ \line(1,0){400}\\

\textbf{Title:} Selective Ensemble of Classifiers Trained on Selective Samples.\\
 \textbf{Authors:} AM Mohammed, E. ~Onieva, M.~Wo{\'{z}}niak.\\
 \textbf{Journal:} Neurocomputing (Impact Factor $=$ 4.438 $\rightarrow$ Q1).\\
 \textbf{Status:} Under review. \\ \line(1,0){400}\\
 
\textbf{Title:} An Analysis of Heuristic Metrics For Classifier Ensemble Pruning Based on\\$\quad \quad \quad$ Ordered Aggregation.\\
\textbf{Authors:} AM Mohammed, E. ~Onieva, M.~Wo{\'{z}}niak, G Mart\'{i}nez-Mu\~{n}oz.\\
 \textbf{Journal:} Pattern Recognition (Impact Factor $=$ 7.196 $\rightarrow$ Q1).\\
 \textbf{Status:} Under review. \\ \line(1,0){400}\\ 
 
 
\end{tabular}
\end{table}
% \section{Thesis Plan}
\label{sec:1_introduction_methodology}
\begin{figure}[!ht]
    \centering
    \includegraphics[width=.8\textwidth]{1_introduction/figures/fig_research-methodo.pdf}
    \caption{The Research Plan of Thesis.}
    \label{ch1:research-emthodo}
\end{figure}

The research in this thesis is progressing rapidly due to technological advancements and continuous contributions in machine learning (ML). An iterative research methodology was followed, where each cycle builds upon the knowledge gained in the previous phase, leading to increasingly effective and original solutions as shown in Fig. \ref{ch1:research-emthodo}. The phases of this research methodology are as follows:

\begin{enumerate}
    \setlength{\itemsep}{0pt}
    \setlength{\parskip}{0pt}
    \item \textbf{Review of the current state-of-the-art:} Investigate existing research to identify challenges and inform the design of a solution.
    \item \textbf{Design and development:} Design a novel solution using updated knowledge to address the identified challenges.
    \item \textbf{Experimentation and evaluation:} Test the solution through experimentation, using established criteria for comparison.
    \item \textbf{Results analysis and comparison:} Analyze and compare results with state-of-the-art to determine the effectiveness of the solution, and disseminate the findings.
\end{enumerate}



\section{Structure of the Dissertation}
\label{sec:1_introduction_organizations}
The structure of the remainder of this thesis dissertation is outlined below.
\begin{description}	
	\item \textbf{Chapter~\ref{cha:2_background}} reviews background about concept drift, concept drift types, concept drift components, adapting types.  
	
	\item \textbf{Chapter~\ref{cha:3_State-of-the-art}} reviews state-of-the-art  concept drift, classifier ensemble selection, imbalanced data streams, Streams with Emerging New Classes (SENC), and transfer learning.  
	
	\item \textbf{Chapter~\ref{chapter:4_Imbalanced_Multiclass}
	} presents our first proposal to build an effective proposed approch for  handling Imbalanced Multi-class Drifted Data streams.
	
	\item \textbf{Chapter~\ref{chapter:4_Imbalanced_Multiclass}} provides our second proposal to adressing emerging new classes in incremental streams via concept drift techniques. 
	
	\item \textbf{Chapter~\ref{chapter:4_Imbalanced_Multiclass}} presents our third proposal to addressing heterogeneous transfer learning problem  in incremental streams via concept drift techniques.
	
	\item \textbf{Chapter~\ref{cha:7_Conclusions}} revisits the main goal and specific objectives posted earlier. In this chapter, we summarise the main contributions of this thesis and outline possible future research.


\end{description}

% background
% this file is called up by thesis.tex
% content in this file will be fed into the main document



% this file is called up by thesis.tex
% content in this file will be fed into the main document

%: ----------------------- introduction file header -----------------------


\begin{savequote}[50mm]
The good thing about science is that it's true whether or not you believe in it. 
\qauthor{Neil deGrasse Tyson}
% The beginning is the most important part of the work.
% \qauthor{Plato}
\end{savequote}


\chapter{Background}
\label{cha:2_background}

Amidst the surge of vast streaming data, governments and businesses find themselves in an urgent need for sophisticated data analysis and machine learning analytics approaches. These tools are indispensable for anticipating future trends and making well-informed decisions. However, the perpetual emergence of new goods, markets, and consumer behaviors introduces a formidable challenge known as concept drift [24]. This phenomenon involves the variation of statistical parameters of the target variable over time in unexpected ways, posing a substantial obstacle to accurate forecasting and optimal decision-making. The patterns derived from historical data may become obsolete when applied to new and evolving datasets.
The impact of concept drift extends across data-driven information systems, including decision support and early warning systems, diminishing their overall effectiveness. In the dynamic realm of big data, where data types and distributions are inherently unpredictable, the challenge of concept drift becomes even more pronounced. In response to this challenge, the field introduces a new subject: adaptive data-driven prediction/decision systems.

\section{Data Preprocessing}
\label{sec:2_1_DP}

Data preprocessing \cite{garcia2015} includes data preparation; (e.g., integration, cleaning, normalization, transformation) and data reduction; (e.g., instance selection, feature selection, discretization). The desired result is to get a cleaned, relevant, manageable, and meaningful dataset ready for analysis. Usually, there is a trade-off between time-complexity and accuracy to get the prepared data, which keeps the research ongoing for this area. 
\vspace{7mm}

\noindent \textbf{Data Preparation:}
Refers to a set of techniques to prepare data as an input for a certain DM algorithm. Usually, it is ignored by inexperienced practitioners, which may cause the model runtime crash. Even if the algorithm works, the expected results will not be optimistic. A set of preliminary steps can be followed before model training as described in \cite{garcia2015}: 
\begin{itemize}
    \item[-] \textbf{Data Cleaning:} The process of detecting and correcting (or removing) inaccurate records from data. Other tasks could be to detect irrelevant data fragments that do not make sense. The result will be a consistent, accurate, meaningful dataset \cite{rahm2000d}.
    
    \item[-] \textbf{Data Transformation and Data Integration:} Data transformation is the process to convert and consolidate the data to another format to improve model efficiency. This process is composed of sub-tasks; feature construction, feature aggregation, normalization, and more. While data integration is recommended for merging data that come from multiple data sources. The aim is to detect conflicts and to remove redundant and inconsistent data \cite{lenzerini2002,doan2012}. 
    \item[-] \textbf{Data Normalization:} This process is dedicated to unify the measurement unit to all the attributes. Under this schema, all the attributes are equally weighted. Statistically, to align the entire probability distribution of the adjusted values, to reduce the effects of certain gross influences \cite{pochon2008,pyle1999}.
    
   \item[-] \textbf{Missing Data Imputation and Noise Identification:} Training a model with a dataset that has many missing values can have a dramatic effect on the quality of the machine learning model. The imputation strategies can be (Mean/Median values, most frequent value, k-nearest neighbors, using deep learning, multivariate imputation by Chained Equation) \cite{buuren2010}. Noise identification is known as smoothing, to detect variances or random errors in a measured variable \cite{wang2010}. 
\end{itemize}


\noindent \textbf{Data Reduction:}
Not like data preparation, data reduction is an optional step. It provides a set of methods for obtaining a reduced version of the original data. It is the process of downsizing the data while maintaining the integrity of the complete dataset. However, it could be a crucial step as data preparation, to enable the DM algorithm when the data size exceeds. Following are some representative approaches: 

\begin{itemize}
    \item[-] \textbf{Feature Selection:} Data reduction can be accomplished by removing irrelevant or redundant attributes. The aim is to use the least number of features while keeping the output of the classification as similar as possible as if we were using the full feature set. Regarding that, the training speed can be boosted and the model performance can be elevated \cite{li2017, dash1997}. 
    \item[-] \textbf{Feature Extraction:} Lessening the amount of resources required for the representation of a large array of data. Analysis with a wide range of variables usually involves a large amount of memory and computational resources. In addition,  the classification algorithm could generalize to new samples badly. Feature extraction is a set of techniques that create combinations of variables to fix these issues, but also reflecting the data with sufficient precision. Many machine learning specialists believe that the properly configured extraction of features is the key to a successful model building \cite{geron2019}. 
    
    \item[-] \textbf{Instance Selection:} The process of reducing or eliminating samples intelligently without affecting the DM application. This process can be guided by heuristic rules to select horizontal subsets of data \cite{olvera2010,fu2013}. In addition, the process can be applied to adapt with a particular DM algorithm; like "selection of support vectors for support vector machine algorithm" \cite{liu2017}.   
      \item[-] \textbf{Discretization:} The mechanism by which quantitative data is converted into qualitative data; via converting numerical variables into discrete or nominal variables. For that, a huge spectrum of numeric values can be compacted or reduced into a subset of discrete values \cite{ding2010}.  
\end{itemize}

Finally, the benefits of data preprocessing can be among the following; (1) Adaptation to a particular machine learning algorithm. (2) Increasing predictive accuracy. (3) Enabling: data mining algorithms are negatively affected by the data size, and data reduction provides a solution for data choking. (4) Cleaning noisy, missing, and redundant data to improve data quality. (5) Focusing: to focus on relevant data instead of all available information.


%%%%%%%%%%%%%%%%%%%%%%%%%%%%%%%%%%%%%%%%%%%%%%%
%
%   Brief history of TF
%
%%%%%%%%%%%%%%%%%%%%%%%%%%%%%%%%%%%%%%%%%%%%%%%
\section{Data Mining}
\label{sec:2_2_SML}
A closely related field to machine learning is Data Mining, which is defined in \cite{lantz2013} as, "\textit{the generation of novel insights from large databases}". While in \cite{garcia2015}; data mining is defined as, "\textit{solving problems by analyzing data present in real databases}". While others \cite{chakrabarti2006,han2011,nisbet2009,clifton2010} view DM as, "\textit{the main steps of knowledge discovery in databases}". The main distinction between ML and DM is that; ML is dedicated to teach computers how to use data to solve problems, while DM is dedicated to teach computers how to identify patterns like a human to solve problems. It is interesting to mention that every DM involves the use of ML, but not all ML involves DM. Next, a brief description of data mining categories is presented:
 
\begin{itemize}
    \item [-] \textbf{Unsupervised Learning}: Those techniques deal with an unlabeled dataset, with a minimum of human intervention \cite{fisher2014}. Data with no pre-existing labels at our disposal and the purpose is to find associations, relationships, regularities, and similarities in the data. Cluster analysis is among the common models used in unsupervised learning. Cluster analysis is the process to segment/group datasets with shared attributes \cite{kaufman2009}. Unlabeled examples are given an implied cluster label from the relationships within the data entirely. Sometimes, the clustering task is referred to as "unsupervised classification", because it classifies unlabeled examples.   
    
    
\item[-] \textbf{Supervised Learning:}   Supervised learning is popular in the field of DM, commonly knows as prediction methods. In supervised learning, a relationship is to be learned between input space and target space. Based on the predicted target type, there are two common tasks; regression and classification. In regression, the numerical output to be predicted falls in a certain interval. Contrary to classification, the domain of the target is finite and categorical.   
\end{itemize}

    



Next, all the subsequent sections will be dedicated to the classification task in matching with the core of this thesis. In the classification problem, the input attributes (features) and the target attribute (class) are transparent. The aim is to learn a function that maps inputs to outputs. The learned function is called a model, and it is inferred from labeled training data. Let, 

\begin{equation}\label{eq:classification}
\textbf{x}=\left[x^{(1)},..., x^{(d)}\right]^T,\textrm{ and } \textbf{x}\in \mathcal{X}=\mathcal{X}^{(1)}\times ... \times \mathcal{X}^{(d)}
\end{equation}
 
\noindent where $\mathcal{X} $ denotes feature space and \textbf{x} is the sample, i.e., \textbf{x} is the so-called feature vector which informs about attribute values. We will assume that we have $d$ attributes at our disposal. The supervised classification model will assign a given object described by its features, \textbf{x}, into one of the predefined categories, also called labels. Let $\mathcal{M}=\{1,..., M\}$ stands for the set of class labels (decision regions). The classification algorithm (discrimination algorithm) is any learned function $\Psi$ with domain $\mathcal{X}$ and codomain $\mathcal{M}$ as clarified in Equation (\ref{mapper}). Where the target values in codomain are finite and categorical.  

\begin{equation}\label{mapper}
\Psi\: :\: \mathcal{X} \to \mathcal{M}.
\end{equation}

\begin{figure}[!ht]
    \centering
    \includegraphics[width=\textwidth]{2_Background/figures/fig_SML_models.pdf}
    \caption{Supervised classification algorithms.}
    \label{ch2:cl.sml}
\end{figure}

There is a large number of models with different inferring strategies to discover the hidden patterns from data. Next, I provide a short review of popular classification algorithms according to the division in Figure \ref{ch2:cl.sml}, inspired by \cite{garcia2015}.  

\begin{itemize}
    \item Regression Models: Logistic Regression (LR) is a statistical model, that returns a probability for each class level. The cutoff value is embedded to separate the upper and the lower probabilities to work as a binary classifier (binomial LR). While multinomial LR deals with more than two classes. In order to be enabled, the missing values should be handled. In addition, the high correlation among the predictors (variables) should be minimized. LR has been praised for its robustness and simplicity \cite{friedman2001}.  \item Artificial Neural Networks (ANNs): The excessive and growing formulations of ANNs from the theoretical and algorithmic depth \cite{carlini2017}, made them more influence on the field of pattern recognition. ANNs handle large and complex tasks due to their nested non-linear structure. While non-explanation is one of the pitfalls of those black-box models.  The computations of those models are based on the definition of neurons. ANNs are unstable and more sensitive to small changes in training data. Similar to LR, they require no missing values.  
    \item Bayesian Learning: Based on the probability theory to get rational decisions \cite{saritas2019}. The Na\"ive Bayes is the most popular algorithm in this category. The posterior probability for each class label is calculated, then the decision is promoted upon the maximum probability returned. Those methods only work with categorical attributes, cannot work with missing values, and are very sensitive to redundancy. The "Na\"ivety" comes from the assumption of the conditional independence between features. It can be viewed as an explainable model to give reasoning about the decisions. 
    \item Instance-based Learning: The prediction of a new unknown sample is based on a distance function with the past stored samples. Also called memorization techniques and lazy learners \cite{lopes2015}. The performance of those methods is affected by the used distance function, neighborhood size, and decision aggregation mechanism. K-Nearest Neighbor (KNN) is the most popular method in this category. Pitfalls of those methods can be mentioned as: high memory space for storage, delayed prediction response, sensitivity to noise.
    \item Support Vector Machines (SVM): Learning algorithms which are based on maximizing gab separation (margin) between different class samples to get correct decisions. They are suitable to work as linear and non-linear data separation. Only the class borders (support vectors) are important to optimize the margin where internal points can be removed to improve the efficiency \cite{nalepa2019}. They require no missing values and are commonly robust against noise. \item Rule Learning: Called divide-conquer algorithms \cite{furnkranz2012}. The data parts are divided based on one rule, then recursively conquer the divided parts. Those models are transparent or explainable to nonexperts in the form of logical structures. Available features are analyzed to find homogeneous groups, then an additional rule is built to drill down more. Small changes in the training data results in decision change. Rule learning techniques are affected by missing values, noisy samples, and outliers.
    \item Decision Trees (DT): This is a kind of indirect rule learning. Uses structure branching decisions to model the relationships among the features and the predicted class value. They are widely used and can model any type of data. The human-readable model is appropriate in applications where legal reasoning is required. Those models are vulnerable to overfitting, and the internal parameters should be tuned \cite{geron2019}. Unstable like ANNs and sensitive to change in the training data. Usually, they are biased towards the splits on features.               
\end{itemize}

For unseen pattern $\textbf{x}$, a class membership values are calculated as in Equation (\ref{membership}), then the labeling choice will be connected to the highest score.
\begin{equation}\label{membership}
\Psi(\textbf{x})=\text{Max}\{g_1(\textbf{x}),g_2(\textbf{x}),\dots,g_M(\textbf{x})\}
\end{equation}


 The decision region $R_1$ for the $1^{st}$ class, is the set of points for which $g_1(\textbf{x})$ has the highest score \cite{kuncheva2014}. While the \textit{classification boundaries} contain data points for which the membership values tie. If the decision region $R_1$ contains data points from the $2^{nd}$ class, then we have overlapped classes. From Figure \ref{ch2:overlapping}. (c) and as shown in \cite{galbusera2019}, the regions are nonoverlapping as the model learns all the details about the data. This case is known as overfitting, and the model will not perform properly to predict unseen samples. While  Figure \ref{ch2:overlapping}. (a) shows the optimal class separation boundary that guarantees minimum possible error with the future samples. Finally,  Figure \ref{ch2:overlapping}. (b) shows the the underfitting case when the model fails to capture relationships between a dataset’s features and a target variable during training. 
 
 
 \begin{figure}[!ht]
    \centering
    \includegraphics[width=\textwidth]{2_Background/figures/fig_overlapping.pdf}
    \caption{Trade-off between overlapping and overfitting, taken from \cite{galbusera2019}.}
    \label{ch2:overlapping}
\end{figure}


Each model has an accompanied error, we need to understand the different sources that cause this error. Equation (\ref{Gen.error}) represents the compound generalization error $E_G$ of a classifier $\Psi$ that is trained on dataset $D$. 


\begin{equation}\label{Gen.error}
E_G(\Psi,D)=E_A(D)+ E_M+E_B
\end{equation}

\noindent where $E_A(D)$ is the "approximation error" represents the variance due to using different training data, or non-deterministic training algorithm. Clarified as, the hyper-parameters of the model that affect its performance. The second term $E_M$ is the "model error" which represents the bias due to selecting a model in preference of another. The last term $E_B$ is the "irreducible error" coming from the insufficient representation of the data. This is commonly known as Bias/Variance Tradeoff \cite{geron2019,kuncheva2014}. Increasing a model’s complexity will typically increase its variance and reduce its bias.
Conversely, reducing a model’s complexity increases its bias and reduces its variance.
This is why it is called a tradeoff.    
%%%%%%%%%%%%%%%%%%%%%%%%%%%%%%%%%%%%%%%%%%%%%%%
%
%   Machine Learning Modelling Approaches of Traffic Forecasting
%
%%%%%%%%%%%%%%%%%%%%%%%%%%%%%%%%%%%%%%%%%%%%%%%
\section{Ensemble Data Mining}
\label{sec:2_3_EDM}

Ensemble learning is the strategy of using multiple learning algorithms in order to obtain greater predictive precision than all of the constituent learning algorithms alone \cite{polikar2006,polikar2012,rokach2010,dietterich2002,zhang2012,opitz1999,sagi2018,oza2001,krawczyk2017}. In addition, Wozniak. et al. defined ensemble models as hybrid intelligent systems \cite{wozniak2014} with the potentiality to cope with ambiguity, uncertainty, and complex problems. Thanks to their capabilities, ensembles received great attention in the applications related to data mining. For unsupervised learning, clustering performance could be significantly improved by ensemble methods \cite{zhou2006,vega2011,cornuejols2018}. Furthermore, ensembles are employed for unsupervised anomaly detection \cite{zhao2015}. While, for supervised learning, those systems are widely popular for regression tasks \cite{mendes2012,ren2015} and for classification tasks \cite{wozniak2014,sagi2018}.


Ensemble systems for pattern classification have been expanded in the literature under creative names as: consensus aggregation \cite{benediktsson1992}, stacked generalization \cite{naimi2018}, committees of neural networks \cite{cirecsan2011}, mixture of experts \cite{jacobs1991,jordan1994}, classifier ensembles \cite{yu2014,rodriguez2006}, classifier selection \cite{ruta2005,cruz2018}, multiple classifier systems \cite{wozniak2014, catal2017}, classifier fusion \cite{liu2017cl} and more.  Theoretical and empirical studies prove that ensemble systems are more accurate than any random classifier \cite{catal2017, tahir2012,polikar2006,fernandez2014}. The final decision is the accumulative decisions of all the classifier set. Let $\Pi$ denotes a pool of $T$ base classifiers $\Pi=\{\Psi_1, \Psi_2,\Psi_k ..., \Psi_T\}$ to be grouped by a combination function. The ensemble output $\hat{\Psi}$ is determined on the basis of the outputs of the base classifiers, i.e.,

\begin{equation}
\label{eq-generalfusion}
\hat{\Psi}(\textbf{x})=F(\Psi_1(\textbf{x}), \Psi_2(\textbf{x}), ..., \Psi_T(\textbf{x}))
\end{equation}

\textit{Intuitively, any classifier ensemble is in fact a classifier} (L. Kuncheva \cite{kuncheva2014}).


% \begin{figure}[!ht]
%    \centering
%    \includegraphics[width=.7\textwidth]{2_Background/figures/fig_ensemble-top.pdf}
 %   \caption{The simple topology for combining classifiers, taken from \cite{kuncheva2014}.}
  %  \label{ch2:ensemble-topology}
%\end{figure}










Each learning algorithm, Section \ref{sec:2_2_SML}, has a limit to discovering the hidden pattern. According to Wolpert's \emph{no free lunch} theorem \citep{wolpert2002}, there is no best classifier suitable for all problems, but each model has its own area of competence giving the design assumptions. For that, a set of learning models solving the same problem can be consolidated to generate a better composite global model \cite{ksieniewicz2018}. L. Kuncheva \cite{kuncheva2014} stated \textit{"the improvement of the ensemble over the single best classifier or even on the average of individual classifier accuracies is not guaranteed"}. From our perspective, the proper design of ensemble is conditioned by outperformance over the best individual classifier in the group.  



\subsection{Are we pursuing complexity?}
Why we accept complex systems instead of depending on a single classification algorithm?. The answer to this question is highlighted in the following points.
\begin{itemize}[nosep]
    \item[-] Ensemble models are the solution to deal with uncertainty. A solution from a single classifier can be boosted and trusted by aggregating a group of predictors (\textit{wisdom of crowds} \cite{rokach2010}).
    \item[-] There is no guide to design universal approximators, perfect model, i.e. it is difficult to set up ANNs or reaching their optimistic parameters.

    \item[-] Ensemble selection or pruning is an interesting research topic that aims to reduce ensemble complexity without deterioration in the performance.
\end{itemize}

In addition, a promising solution via ensemble learning can be achieved for the following scenarios:

\begin{itemize}[nosep]
    \item Imperfect Learning: Non-deterministic classifier can be considered as a local optimizer in terms of the training error, a safer option is to group several models that cover the solution space properly.
    \item Too much data: We are surrounded by too much data. In this case, data is split into chunks where similar or different learning algorithms can be trained on each part independently. Ensemble learning support parallelization and distributed computing for handling this scenario efficiently.
    \item Small-size data: In data shortage, stratified sampling with replacement can be applied, where several data replications will be obtained to train individual classifiers inside the ensemble. 
    \item Data fusion: The data pattern can be identified differently based on the data source. The availability of sensors strengthens decision making by analyzing different features. Instead of fusing all features and building a single classifier, it could be better to build a single classifier for each feature space and combine their decisions.  
    \item Complex hypothesis: Complex classification boundary can be approximated by combining several base classifiers.
    
\end{itemize}


\subsection{A Taxonomy of Classifier Ensemble Methods } \label{ch2.taxonomy}

A comprehensive review of the classifier ensemble methods, thorough discussion, and the development of further knowledge in this area was the core of many articles \cite{gonzalez2020,polikar2006,cruz2018, britto2014,re2012,sewell2008}, with a proposed taxonomy by L. Rokach \cite{rokach2009} who indicated the five dimensions to design this kind of powerful models. Figure \ref{ch2:ensemble-taxonomy} shows our perspective for the Multiple Classifier System (MCS) taxonomy as it has been proposed by L. Kuncheva \cite{kuncheva2014}, R. Cruz et al. \cite{cruz2018}, and L. Rokach \cite{rokach2009}. The two main phases in classifier ensembles are; Generation and Integration, while the selection is an intermediate/optional phase.   

\begin{figure}[!ht]
    \centering
    \includegraphics[width=.95\textwidth]{2_Background/figures/fig-taxonomy.pdf}
    \caption{The Taxonomy of Multiple Classifier System (MCS).}
    \label{ch2:ensemble-taxonomy}
\end{figure}



\subsubsection{Generation} \label{sub.generation}
The goal in this phase is to generate a pool of classifiers that are both diverse and accurate. This phase discusses strategies to handle data horizontally/vertically. In addition, what classifier type to accommodate, how to build the classifiers: dependent/independent manner, and how many classifiers to train (ensemble pool size). 

 \textbf{1) Diversity:}
Diversity is one of the main reasons for the effectiveness of ensemble methods \cite{zhou2012,gonzalez2020}. Diversified classifiers cause uncorrelated errors, which lead to improved classification accuracy \cite{hu2001}. In general, this can be achieved through the following six wide subcategories to promote diversity during the generation phase.
\begin{itemize}
    \item[-] \textbf{Different Parameters/Initialization:} \textit{Algorithm-level diversity}; via different parameters, the base classifiers can be generated by modifying the hyper-parameters of the learning algorithm; for example, controlling the number of $k$ neighbors, distance function in KNN model, and controlling the confidence level parameter in the decision tree. While, via different initialization and if the training process is initialization dependent, the model will be sensitive. In neural networks, different initial configurations of weights result in different decisions \cite{hansen1990}.        
  \item[-] \textbf{Different Architectures:}  \textit{Algorithm-level diversity}; this strategy is more suitable to multi-layer perceptron neural networks, where the number of hidden layers, the number of neurons, and the network topology affect the classifier domain space. For example; in the Addemup algorithm \cite{opitz1996}, the genetic algorithm is used to choose the required network topology to compose the ensemble according to a measure of diversity.     
  \item[-]\textbf{Different Classifier types:} \textit{Algorithm-level diversity}; each classifier model build its inference with a capability to discover a hidden pattern differently. Each model, Section \ref{sec:2_2_SML}, contains explicit or implicit bias that leads to a better generalization accuracy through combination. In addition, as in \cite {woods1997}, the divide and conquer mechanism is best implemented by calculating the local accuracy in the feature space to choose between four different classifier types.  
  \item[-]\textbf{Data Partitioning:} \textit{Data-level diversity}; Partitioning means the division into smaller disjoint components of the initial training. A different classifier will be trained on each part to gain different and accurate decision \cite{maimon2005}. This mechanism enables data mining algorithms to handle massive datasets by managing memory size and computational resources perfectly. The data can be partitioned horizontally or vertically. \textit{In horizontal partitioning}, several parts will be formed; each part will contain sub-samples while sharing the complete feature set. Sub-samples can reflect the entire dataset by selecting the instances from all the formed clusters of the dataset; known as cluster-based concurrent decomposition (CBCD) \cite{rokach2005}. The mixture of experts (ME) \cite{nowlan1991}   splits the input space into several subspaces and assign an expert, classifier, to each subspace. In \cite{cohen2007} a decision tree framework is employed to divide the input space into mutual exclusive partitions, then a new unseen pattern will be classified by a dedicated classifier that is learned from the space to which the instance belongs. \textit{In vertical partitioning}, The feature space is partitioned into several subsets keeping the same number of samples, then each classifier can be trained on a different projection. This mechanism is more suitable for a high-dimensional dataset without affection by the feature selection drawbacks \cite{tumer2003}, and the accuracy could be improved by the less correlation among classifiers. This division results in a high-speed classification algorithm \cite{bryll2003}, and solves the problem of class under-representation that exist with instance-based sampling.      
  
  \item[-]\textbf{Resampling:} \textit{Data-level diversity}; the diversity could be promoted by manipulating the training set. Each base model is trained over a different sample from the training. Bagging \cite{breiman1996,skurichina1998} and Boosting \cite{freund1997,freund1999} are the popular strategies in that paradigm. Bagging ensembles achieve a reasonable diversity level by creating different bootstrap samples to train each base model independently, then the final decision is adopted by a simple majority voting-based aggregation. Moreover, the non-sensitivity of bagging and robustness under diverse noise conditions makes it more attractive \cite{dietterich2000}. Contrary to sequential ensembles, \textit{Boosting}, the individual members are generated in the sequential schema by the learning algorithm \cite{freund1997}. The sequential mechanism of boosting encourages the complementariness between ensemble members, by focusing on previously misclassified samples. However, the performance in boosting is more sensitive to noisy samples \cite{dietterich2000,caruana2006} and sometimes overfitting can be observed for large pool size \cite{ratsch2001}. 
  
  \item[-]\textbf{Differnet label targets:} \textit{Data-level diversity}; manipulating the target attribute is very interesting mechanism to promote diversity. Instead of building complex classifier, several classifiers with usually simpler representations, about the target attribute, will be trained. For instance, to handle the multi-class dataset, the original target attribute can be replaced by a simpler and smaller target domain. Among those strategies, One versus all (OVA) \cite{anand1995} which divides the $M$-class classification problem into $M$ two-class classification tasks. While one versus one (OVO) \cite{lu1999} divides the $M$-class classification problem into $M(M-1)/2$ two-class classification tasks, so the complex decision boundary can be simplified. Minimal classification method (MCM) \cite{sivalingam2005} converts the $M$-class classification task to the minimal binary classification tasks. MCM requires $\log_2(M)$ classification tasks in the form of a separation between groups of multiple classes. Furthermore, the error-correcting output coding (ECOC) algorithm, uses a code matrix to decompose the multiclass problem into multiple binary problems \cite{dietterich1994}. In addition, label switching algorithms \cite{breiman2000,martinez2005} change the labels of samples picked randomly. 
\end{itemize}


\textbf{2) Building:}
This part discusses the building schema to generate a pool of classifiers, whether to be dependent or independent. In the dependent framework (sequential/incremental), the performance of a classifier affects the creation of the next classifier in the chain. Instead, in the independent framework (simultaneous), the classifiers are generated independently. 

\begin{itemize}
    \item[-]\textbf{Simultaneous:} The independent methods to generate a set of classifiers mainly depends on forming different training samples from the original training set. The training samples could be mutually exclusive (disjointed) or overlapping. The reason behind preferring this schema is to improve the predictive accuracy or to speed up the generation step as this schema supports parallel implementation. \textit{Bagging} (Bootstrap AGGregatING) \cite{breiman1996} is the popular method in this category, with a replacement from the original training set, each classifier is trained on data samples. The sample size will be equivalent to the original training size; randomly some samples will be duplicated and some samples will be ignored.  
     \item[-]\textbf{Incremental:} Sometimes called dependent, there is a kind of interconnection during the learning process. The model generation depends on the accuracy of the previous model/committee. The training process is done in iterative form, where the learning process will be directed to focus on the previously misclassified samples. \textit{Boosting} \cite{freund1996} (also known as arcing- Adaptive Resampling and Combining) is the popular method in this category. The training process mainly depends on assigning weights to all the training samples. In the beginning, all the samples will be equally weighted (having the same importance), but throughout the iterations, the weights of correctly classified samples are lowered while the weights of misclassified samples are raised. As a consequence, the base model is directed to focus on the hard samples in the training set.          
\end{itemize}


\textbf{3) Universality:}
This part discusses the universality of the ensemble model. Some ensembles techniques could be designed to work for any classifier, while other ensembles have been designed to work with specific classifiers. In other meaning, the relation between the ensemble technique and the used classifier type. 

\begin{itemize}
    \item[-]\textbf{Specific Inducer:} Known as inducer-dependent ensembles, where the effectiveness of the ensemble could be degraded if applied for other classifier types. For example, \cite{hansen1990,lu1999} those ensembles are explicitly designed for neural networks. Additional schemas \cite{tao2006} are perfectly suited for SVM.
    
     \item[-]\textbf{Any Inducer:} Known as inducer-independent ensembles, Those implementations can be extended to a wide variety of classifier types without affecting the generalization accuracy. 
    
\end{itemize}


\textbf{4) Ensemble Size:}
This part discusses the aspect of pool size. How many classifiers should be trained?. The main four factors that concern the ensemble size as determined by L. Rokach \cite{rokach2009} are; (A) \textit{Sufficient Accuracy}; the appropriate accuracy of the ensemble could be reached by aggregating 10 models \cite{hansen1990}, while other empirical studies show that this level of accuracy is correlated with large-size ensembles containing 25 models \cite{opitz1999}. (B) \textit{Computational Cost}; the more classifiers are generated the more computational resources are consumed. As a consequence, the user may predetermine the pool size to match the computational cost limits. (C) \textit{Nature of Task}; the ensemble size could be problem-dependent, as we stated before with the ensemble size of OVO and OVA strategies for handling multi-class classification tasks. (D) \textit{Number of processor cores}; for independent ensembles, the number of internal cores can be the upper bound to control how many classifiers can be trained in parallel mode.                      

\begin{itemize}
    \item[-]\textbf{Fixed in Advance:} This is the simple form to predetermine how many iterations should be considered. Most Bagging and Boosting software packages give the flexibility to control the number of iterations.  
    
     \item[-]\textbf{Determined during Training:} The best ensemble size can be determined during train time. The contribution of a new classifier to the ensemble performance is to be checked if it is significant or not. To estimate the unbiased error of the test sample, Random Forests uses the out-of-bag (OOB) procedure. The OOB error estimation is used in \cite{banfield2006} to determine the sufficient number of classifiers. As the maximum accuracy no longer increases, the training procedure stops.       
    
\end{itemize}



\subsubsection{Selection}
There is no value from combining similar classifiers' decisions. The effectiveness of the ensemble is conditioned by the diversity and the correctness of the base classifiers. Ensemble Selection is one of the strategies that can be used to handle this challenge. Ensemble Selection has been known in the literature as ensemble pruning \cite{lu2010,zhang2006,ykhlef2017}, ensemble thinning \cite{banfield2005} and ensemble reduction \cite{diao2013,zhang2018}. ES can be considered as an intermediate process between building the ensemble and aggregating the decisions. Specifically, ES is the strategy of optimizing and selecting the number and the type of individual classifiers in-advance. Collecting the decisions from a reduced number of models speeds up the classification systems and relieves memory storage. In the literature, the selection process can be performed offline, static selection \cite{tsoumakas2009}, or online, dynamic selection \cite{cruz2018}.
 
 
 \begin{itemize}
    \item[-]\textbf{Static Ensemble Selection:} Known in the literature review as ensemble pruning. This process is done during the training and before the real test. A subset of classifiers that optimize a predefined function/metric are selected, and the estimation of that metric is done over a pruning set. The most common metrics/selection criteria are; diversity \cite{aksela2003,brown2010}, classification accuracy \cite{dos2009,ruta2005}, instance margin \cite{yang2012, guo2013,hernandez2008}. However, it is not trivial to find the optimal subset of classifiers from a large ensemble as the complexity increases exponentially with the pool size. Researchers agreed in common that ES is a combinatorial search problem with $2^{T}-1$ nonempty subsets to be evaluated from pool size, $T$, to find the best subset \cite{tamon2000, tsoumakas2009}. To handle this complexity, several attempts ranging from optimized search \cite{diao2013,adnan2016,ruta2001b}, clustering techniques \cite{onan2017,zyblewski2020}, and greedy algorithms \cite{cao2018,martinez2009,margineantu1997, partalas2010, mao2011} have been applied over decades. More details about ordering-based pruning will be discussed in Section \ref{ch5_ordering_pruning}.        
    
     \item[-]\textbf{Dynamic Ensemble Selection:} This process is done during the test time. A single classifier or subset of classifiers are selected based on the unseen sample. In addition, dynamic weights are assigned according to the competition among individual classifiers over the local region of an unknown sample $\textbf{x}$. As a consequence, the selected subset is changeable for each test pattern \cite{KO2008}. This process consists of three steps; (1) Definition of the local region surrounding the query sample $\textbf{x}$, (2) Determination of the selection criteria to estimate the competence level of the classifiers, and (3) Determination of the selection mechanism, whether to select a single classifier or ensemble of classifiers. Without debate, dynamic selection techniques can outperform the static selection methods as experimentally been proved in \cite{cruz2018} since the selection is optimized for each test sample independently. The rationale in dynamic selection is that each classifier is an expert in a different local region within the feature space. However, in the dynamic selection, there is a computational overhead for selecting subensemble for each test sample. Besides, those techniques flood the memory space as all individual classifiers have to be retained in memory. Additionally, the dynamic selection is affected by the outlier instances around the query sample in the feature space \cite{CRUZ2015}.
    
\end{itemize}

Cluster and pick \cite{kuncheva2000} is a poor variant of dynamic selection. Initially, the input space is split into disjoint regions via clustering the training samples. The best classifier is then defined and chosen for each cluster. Cluster and select is in-between static and dynamic strategies; the classifiers are selected dynamically depending on which region the input sample belongs to, but the regions are static, determined in advance during the training \cite{ruta2005}.

 
 
 
\subsubsection{Integration}
Also known as the combiner function. The combiner balances the deviation between the diversity and the bias, also alleviates the errors that certain models have made \cite{SOARES2013}. This process concerns the methodology by which the outputs of the base classifiers will be aggregated.  A vital phase in building a group of classifiers is the use of a suitable fusion strategy to aggregate response decisions \cite{zhou2012,kuncheva2014}. The responses of individual classifiers restrict the fusion method and enhance or degrade the composite prediction. Those responses may be on the \textit{abstract level} \cite{xu1992}, where each classifier specifies a class name as a decision. Additionally, the responses may be \textit{ranked levels} \cite{PARKER2001}, where each classifier outputs a ranked subset of class labels, or even  \textit{measurement values} \cite{kuncheva2014a,niu2007}, where each classifier specifies a posterior probability.


\begin{itemize}
    \item[-]\textbf{Non-trainable:} 
    The combination function does not require any training from the classifiers' decision space. Many combination rules have been proposed; for measurement values (Sum, Product, Maximum, Minimum, Median, and Decision Templates \cite{Kuncheva2001}), for abstract level (Majority voting \cite{Kittler1998}, Behavior Knowledge Space \cite{Huang1995}, and Naive Bayes \cite{xu1992}), and for ranked levels (Borda count \cite{Ho1994}). The majority voting, Equation(\ref{majority}), will be effective if the base classifiers are independent. While, the weighted sum rule, Equation(\ref{weighted.sum}), produces good results if the classifiers perform the same task and have comparable success or when we would like to avoid over-fitting or long training time \cite{rokach2009}. The weights, $w_k$, are calculated to be proportional to the individuals' performance over a validation dataset.  In addition, weighted voting is preferred for highly imbalanced datasets \cite{kuncheva2014a,krawczyk2014}.  
    
\begin{equation}
\label{majority}
\hat{\Psi}(\textbf{x})=\mathop{\arg\max}\limits_{c \in \mathcal{M}} \mathop{\sum}\limits_{k=1}^T \left[\Psi_k(\textbf{x})=c\right] 
\end{equation}    

\begin{equation}
\label{weighted.sum}
\hat{\Psi}(\textbf{x})=\mathop{\arg\max}\limits_{c \in \mathcal{M}} \mathop{\sum}\limits_{k=1}^T \left[\Psi_k(\textbf{x})=c\right]w_k
\end{equation}
    
    
     \item[-]\textbf{Trainable:}  
The combination function is to be configured specifically to the classification task. The aggregation function will be trained over a validation dataset from the domain, \textit{base classifiers outputs}, and co-domain, \textit{real attribute output}. For example; the classifiers' fusion weights can be optimized by evolutionary algorithms (EA).  In \cite{krawczyk2014}, the authors tuned the weights for the selection and fusion of multiple cost-sensitive decision trees to handle imbalanced data using the evolutionary genetic algorithm (GA) \cite{mitchell1998}. Each chromosome from the genetic population simulates a weighted ensemble as $\left[w_1, w_2, \ldots, w_k, \ldots w_T\right]^T$, $ \forall w_k \in [0,1]$. The ensemble performance is estimated for each chromosome, and genetic operators evolve the next generations. In addition, GA has been applied in \cite{sikora2015} to tune the weights of heterogeneous classifiers in consideration of the above chromosome encoding. While, a bird-flock based optimization algorithm has been applied in \cite{cordeiro2011, kausar2010} to enhance the fusion process.          
      
\item[-]\textbf{Meta-classifier:}   
Meta-learner is another important fusion mechanism. It is a trainable method, where the aggregation function is to be learned based on the base classifiers outputs. \textit{Stacking} \cite{Wolpert1992} is the most popular meta-learning method, where the predictions from the base classifiers become meta-features/inputs to a new classifier. The original target attribute from the training set remains as it is. Stacking
is generally used to combine heterogeneous models, the term refers to stacking layers of classifiers. In stacking, the correctness of the base classifiers will be learned indirectly through the meat-learner. In \textit{Grading} \cite{seewald2001}, the predictions of base classifiers are transformed into true or false. After that, one meta-classifier will be trained on the transformed decisions of one base model to learn when it errors. Therefore, grading can be seen as a generalization of cross-validation selection, by using only those classifiers that correctly predict specific instances. 



\item[-]\textbf{Dynamic weighting:} Similar to dynamic ensemble selection. The classifiers' fusion weights are determined dynamically based on the local competence of the classifier in the region where the unknown $\textbf{x}$ is located. A higher weight value is assigned to the most competent classifier  \cite{cruz2018}. For example, dynamic integration of classifiers in \cite{Tsymbal2008}. In addition, the dynamic ensemble selection and the dynamic ensemble weighting can be hybrid as in \cite{KO2008}.       
\end{itemize}


Finally, Figure \ref{ch2:four-levels} shows the questions that should be answered during the four levels of constructing MCS \cite{kuncheva2014}.     

\begin{figure}[!ht]
    \centering
    \includegraphics[width=\textwidth]{2_Background/figures/fig-four-levels.pdf}
    \caption{ Four level questions while building MCS.}
    \label{ch2:four-levels}
\end{figure}


\subsection{Diversity and Uncorrelated Errors}
Indeed the two basic facets of enhancing the efficacy of the committee are often referred to as decision optimization and coverage optimization \cite{ho2002}. \textit{Decision optimization} refers to methods for choosing and optimizing the combination method. \textit{Coverage optimization} refers to the techniques used to construct a diverse classifier set, assuming a fixed combiner. In Section \ref{sub.generation}, the prospective methods to generate diverse base models were reviewed. 
If a classifier makes errors on some objects, then it is better to complement it with another classifier that compensates that error. The ensemble performance is restricted by a compromise between an individual's accuracy and group diversity. Confirmed as, neither the individual's accuracy \cite{rogova1994} nor the diversity \cite{ruta2001a} on their own provide reliable ensemble to outperform the best individual classifier. Here, we agree with  L. Kuncheva \cite{kuncheva2014}, ''\textit{good estimate can be obtained from arbitrarily inaccurate estimators as long as their deviations cancel one another}''. However, it is so difficult to engineer this ensemble design.   


There is no benefit from combining redundant classifiers, and the system will be more complex. A large number of classifiers increase the ambiguity, risk, and add more complexity to the selection procedure which could lead to weak generalization \cite{ruta2005}. While for critical systems the useful evidence is so important and should not be wasted. Regarding that, more attempts have been made to select classifiers based on diversity measures. In \cite{giacinto2001}, the authors used a double fault measure to cluster classifier outputs, then they picked a single classifier from each cluster. While in \cite{kim1997} a similarity measure is used to pick a classifier from a pool of five classifiers. Many diversity measures have been presented \cite{kuncheva2003,ruta2001a} with the conclusion that diversity guided search could be invaluable. In addition, the diversity has been intensively used with the combiner performance \cite{ruta2005,li2012,brown2010} to improve the efficiency. The system design does not end with the selection process, as the selected classifiers should be next combined. For that, there should be a correlation between diversity measure value and committee performance. Diversity initiatives that represent at least some clear association trends, with generalization, have the potential to become appropriate criteria for selection. Diversity measurement is not a trivial task in ensemble methods. 


Matti et.al \cite{aksela2006} stated that diversity can be measured from two perspectives, Figure \ref{ch2:diversity-measures}, based on the population. (a) \textit{The data-based approach:} we have $N$ populations with $T$ objects. The ensemble diversity of the $T$ classifiers is calculated for each sample $\textbf{x}_i$, then the overall average is considered over $N$ samples. Here the diversity of all member classifiers is evaluated simultaneously (Non-Pairwise measurement). (b) \textit{The classifier-based approach:} in this case we have $T$ populations each contain $N$ objects. The diversity is to be calculated based on the classifiers' outputs for all input data. Each pair of classifiers are considered for measuring the diversity, then the overall diversity is computed by averaging over the number of pairs (Pairwise measurement).       

\begin{figure}[!ht]
    \centering
    \includegraphics[width=.5\textwidth]{2_Background/figures/fig-diveristy-measure.pdf}
    \caption{Diversity measure approaches for MCS: (a) data-based, and (b) classifier-based, taken from \cite{aksela2006}.}
    \label{ch2:diversity-measures}
\end{figure}




Next, a set of popular measures are presented based on the correct/incorrect outputs. First, the output of $\Psi_k$ can be represented as N-dimensional binary vector $y_k= \left[y_{1k}, y_{2k}, \ldots, y_{Nk}\right]^T$, such that $y_{ik}$=1, if $\Psi_k$ recognizes $\textbf{x}_i$ correctly, and 0 otherwise, $k=1,2,\dots,T$. Moreover, the relationship between a pair of classifiers ( $\Psi_j$, $\Psi_{k}$ ) can be represented as in Table \ref{classifier.relation}. 

  \begin{table}[!ht]
 \centering \scriptsize
 \caption{2 $\times$ 2 table of the relationship between a pair of classifiers.}
\label{classifier.relation}
\begin{tabular}{l|cc}
\hline
 & $\Psi_k$ correct (1) & $\Psi_k$ wrong (0)  .\\ \hline

$\Psi_{j}$ correct (1)  & $N^{11}$ & $N^{10}$  \\
$\Psi_{j}$ wrong (0)  & $N^{01}$& $N^{00}$ \\ \hline
\multicolumn{2}{c}{Total, $N=N^{00} + N^{01} + N^{10}+ N^{11} $} & \\

\hline
\end{tabular}
\end{table}   


\subsubsection{Pairwise diversity measures}

\begin{enumerate}
    \item \textit{The Q statistics :} Is a various statistic measure to measure the similarity of two classifier outputs. The \textit{Q} statistics for two classifiers, $\Psi_j$ and $\Psi_{k}$ is: 

\begin{equation}
\label{Q.statistics}
 Q_{j,k}=\frac{N^{11}N^{00}-N^{01}N^{10}}{N^{11}N^{00}+N^{01}N^{10}}
\end{equation}

Symbols are explained in Table \ref{classifier.relation}. \textit{Q} varies between -1 and 1. Classifiers that agree on the same objects will have positive \textit{Q} values, and those that make mistakes on different objects will have negative \textit{Q} values. The averaged statics over all pairs of classifiers is:


\begin{equation}
\label{averaged.qstatistics}
Q_{avg}=\frac{2}{T(T-1)} \mathop{\sum}\limits_{j=1}^{T-1} \mathop{\sum}\limits_{k=j+1}^T Q_{j,k}   \end{equation}
    
 \item \textit{The correlation coefficient 	$\rho$ :}  
 The correlation between two binary classifier outputs can be calculated as:   
   

\begin{equation}
\label{correlation.cofficient}
 \rho_{j,k}=\frac{N^{11}N^{00}-N^{01}N^{10}}{\sqrt{(N^{11} + N^{10})(N^{01} + N^{00})(N^{11} + N^{01})(N^{10} + N^{00})}}
\end{equation}    
    
For any two classifiers, $Q$ and $\rho$ have the same sign, and it can be proved that $|\rho|$ $\le |Q|$. 
  
  
\item \textit{The disagreement measure :}    
This measure has been used by HO \cite{ho1998} to measure the diversity in decision forests. It is the ratio between the number of samples on which one classifier disagree with another, to the total number of samples.

\begin{equation}
\label{disagreement.metr}
 Dis_{j,k}=\frac{N^{01} + N^{10}}{N^{11}+ N^{10}+N^{01}+N^{00}}
\end{equation}  
    
    
\item \textit{The double-fault measure :}     
It is defined as the proportion of samples on which both classifiers makes error, i.e.,

\begin{equation}
\label{default.measure}
 DF_{j,k}=\frac{N^{00}}{N^{11}+ N^{10}+N^{01}+N^{00}}
\end{equation}  

    
For all pairwise measures, the averaged values are calculated similarly to Equation (\ref{averaged.qstatistics}).       
\end{enumerate}
 
 
 
 
\subsubsection{Non-Pairwise diversity measures}

\begin{enumerate}
   
 \item \textit{The entropy measure E :}    The highest diversity among classifiers for a particular $\textbf{x}_i \in \textbf{X}$ is proved by $\floor{T/2}$ of the votes for $\textbf{x}_i$ with the same value (0 or 1) and the other $T - \floor{T/2}$ with the alternative value. If all decisions are 0's or all are 1's, then there is no diverse. Denote by $t(\textbf{x}_i$) the number of classifiers that correctly classify $\textbf{x}_i$, i.e,  $t(\textbf{x}_i$)=$\sum_{k=1}^{T} \Psi_{ik}$. On the basis of this concept, the diversity can be measured as:
    
   \begin{equation}
\label{Entropy.measure}
 E=\frac{1}{N} \sum_{i=1}^{N} \frac{1}{(T-\ceil{T/2})} \text{min}\{t(\textbf{x}_i),T-t(\textbf{x}_i)\}
\end{equation}   
  
$E$ ranges between 0 and 1, where 0 implies no difference, whereas high diversity is measured by 1.  
  
\item \textit{Kohavi-Wolpert variance :} It measures the average variance between the binomial distributions of the outputs of each classifier. This measure can be simply calculated by :
 
   \begin{equation}
\label{Kohavi.measure}
 KW=\frac{1}{NT^2} \sum_{i=1}^{N} t(\textbf{x}_i)(T-t(\textbf{x}_i))
\end{equation}   
  
  
 \item \textit{Measurement of interrater agreement $\kappa$ :} It is used to measure the level of agreement while correcting the chance. let's denote $\Bar{p}$ as the average individual classification accuracy, i.e.,
 
\begin{equation}
\label{individual.accuracy}
 \Bar{p}=\frac{1}{NT} \sum_{i=1}^{N}\sum_{k=1}^{T} \Psi_{ik}
\end{equation}   
  
 
 then, the interrater agreement could be formulated as:
 
\begin{equation}
\label{interrater.agreement}
 \kappa=1-\frac{\sum_{i=1}^{N}t(\textbf{x}_i)(T-t(\textbf{x}_i))}{NT(T-1)\Bar{p}(1-\Bar{p})} 
\end{equation} 
    
\end{enumerate}





 
\subsubsection{Uncorrelated Errors } \label{ch2_uncorrelated}

Standard diversity measures do not take into account that for classification purposes, identical correct decisions are preferred over identical erroneous decisions. It may be useful to analyze, in particular, the errors made by committee members. In this case, we will focus on the error types and whether this error occurs or not by the base models. If two classifiers incorrectly classify a sample into two separate categories, then this case is known as \textit{uncorrelated errors/diversity of errors}, as the predicted class is not the same despite the fact that both make mistake. The most difficult samples are the cases where all classifiers agree to the same incorrect class. Diversity measures that capture the type of error are initiatives and powerful as they can solve the following challenges:


    


\begin{itemize}[nosep]
\item[-] There are a few correct recognition results in most recognition tasks, while it is hard for the combination function to predict the correct output from this whole set of incorrect predictions. This can be solved by knowing the class of error.
\item[-] Classifiers that agree on the correct outcome should be credited rather than eliminated when selecting classifiers, however, this is conflicting with the principle of diversity maximization. 
\end{itemize}
 
\noindent Naturally, It is best if all classifiers make a correct prediction and it is better if we have fewer classifiers make a mistake. Even, for those mistakes, it is exceedingly helpful if the errors are different as often as possible, i.e. maximum diverse errors. Hence, the oracle-level (binary) classifier outputs should be interconnected with the predicted class category to identify the diversity of errors.


As stated in the introduction of this part, the difference in the mistakes made by the member classifiers really affects the performance. Given the following notations; $N_{same}^{00}$ indicates the number of samples when both classifiers are inaccurate and suggest the same decision. $N_{diff}^{00}$ stands for the number of samples when both classifiers are incorrect, but suggest different decisions. Next, let's present metrics \cite{aksela2006} that could be used to measure the diversity of errors: 

\begin{enumerate}
   
 \item \textit{Same-fault measure :} In an extension of the Double-Fault measure, the simultaneous fault could be restricted to measure when both classifiers are inaccurate  and suggest the same output. This can be measured for two classifiers $\Psi_j$ and $\Psi_{k}$ as:
 
 
 \begin{equation}
\label{samefault.measure}
 SF_{j,k}=\frac{N_{same}^{00}}{N^{11}+ N^{10}+N^{01}+N^{00}}
\end{equation}  
 
Then the averaged pairwise measures could be calculated similarly to Equation (\ref{averaged.qstatistics}). The optimal classifier set is picked based on the minimum measure.


\item \textit{Weighted count of errors and correct results:} It is designed to consider information about correct prediction. The number of samples that match specific cases can be weighted; based on classifier outputs, "both correct" is favorable and classifiers of this type are highly selected via increasing the weights. While "both incorrect and same" are highly penalized by assigning high negative weight, this can be defined for two classifiers $\Psi_j$ and $\Psi_{k}$ as: 
 
 
 \begin{equation}
\label{weighted.count.errors}
WCEC_{j,k}=N^{11}+\frac{1}{2}(N^{10}+N^{01}) -N_{diff}^{00}-5N_{same}^{00}
\end{equation}   
 
Moreover, based on the combiner's performance over the training set, the weights above could be optimized. The averaged pairwise measures are calculated similarly to Equation (\ref{averaged.qstatistics}), and the optimal classifier set is picked based on the maximum measure.  

\item \textit{Exponential error count :} As more errors are encountered the classifier capability will be hindered. Here, this concept is emphasized by counting the number of errors and assigning a weight in an exponential fashion. Assume $\Psi_{same}^{k \times 0}$ denote the number of errors made by $k$ classifiers to the same class, then the measure can be defined as:

\begin{equation}
\label{exponential.error.count}
EEC_{(\Psi_1,\dots,\Psi_T)}= \frac{\sum_{k=1}^{T} (\Psi_{same}^{k \times 0})^k }{N^{T \times 1}+1}
\end{equation} 
 
This measure considers all classifiers set. In addition, the correct classification is considered by scaling the exponential sum with $N^{T \times 1}$ (the number of samples for which every member classifier was correct). The optimal classifier set is picked based on the minimum measure.    
 
 \end{enumerate}






%%%%%%%%%%%%%%%%%%%%%%%%%%%%%%%%%%%%%%%%%%%%%%%
%
%   Machine Learning Modelling Approaches of Traffic Forecasting
%
%%%%%%%%%%%%%%%%%%%%%%%%%%%%%%%%%%%%%%%%%%%%%%%
\section{Soft Computing}
\label{sec:2_4_soft-computing}
Soft computing is one of the pioneer computing paradigms which resembles the human mind's remarkable capacity to think, understand, and solve difficult real-life problems \cite{das2013,konar2018}. Soft computing exploits the tolerance for imprecision, partial truth, and uncertainty to achieve robustness, tractability, low solution cost, and very high-performance \cite{zhu2015}. Modern machinery is more complex to be controlled and stabilized. The reasons for this difficulty is the lack of numerical models that describe exactly how they work, and the existence of many nonlinear and time-variant plants. Soft computing methods support intelligent control, nonlinear programming, optimization, and decision making support. Among those methods; fuzzy logic, genetic algorithms, artificial neural network. Soft computing has become popular with their wide applications to many research fields as speech recognition, communication, pattern recognition, signal processing, automatic control engineering. In their connectivity with the area of MCS, we present several soft computing methods that have been used to improve efficiency:


 \textbf{Fuzzy Logic:}
In domains and environments that are realistic, incomplete evidence inevitably emerges. During experiments, noise corruption or instrument errors may give rise to data when a certain value is measured, leading to incomplete data. In other scenarios, collecting the correct information can be excessively costly or unviable. In addition, using extra information from an expert, which is usually given by fuzzy logic and fuzzy sets, can be useful. Typically, data has a certain degree of vagueness. If the imprecision is significant, then the imprecise values must be handled in all the phases of learning and classification. To address the uncertainty of decision trees with minor disruptions, fuzzy sets and their underlying approximate reasoning capacities have been paired with decision trees in \cite{yuan1995,olaru2003}. The resulting trees exhibit increased tolerance to noise, and extend applicability to ambiguous or unclear circumstances. In \cite{bonissone2010}, a fuzzy random forest was suggested to increase the diversity of the trees through randomness, with the versatility of handling incomplete data. The numeric attributes are discretized through a fuzzy partition, so each internal node in the fuzzy decision tree constructs a child node for each fuzzy set of the partition. Then the membership degree of the examples to different fuzzy sets is determined to optimize the node-split attribute. In addition, a fuzzy combination method for one-class classifiers has been proposed in \cite{wilk2012}.             

\textbf{Evolutionary Algorithms:} Evolutionary algorithms uses mechanisms inspired by biological evolution, such as reproduction, mutation, recombination, and selection to solve complex problems. For example, genetic algorithm (GA) \cite{Holland1992} emulates the natural and human evolution, where common genes from parents can be transferred to their children. GA evolves an initial population of chromosomes; each chromosome represents one solution in the search space. The algorithm depends on applying crossover and mutation operators to explore and exploit different regions from the search space, which probably contains promising solutions. The process continues for a specific number of generations or till reaching threshold error or if the algorithm is trapped in a local minimum without the ability to find more accurate solutions. Ensemble systems are very complex architectures that need to be optimized. GA has been applied in \cite{ruta2005} to control the ensemble size (classifier subset selection) via optimizing several criterias. While in \cite{sikora2015} GA has been considered as a trainable combiner to tune the classifiers' weights for the fusion process. Furthermore,  Kuncheva et al. \cite{kuncheva2000d} applied two versions of GA for selecting feature subsets to be used by base models. While tsymbal et al. \cite{tsymbal2005} applied GA to optimize the diversity of the best-collected feature subsets.            

\textbf{Artificial Neural Networks:}
Neural networks were proved to be universal approximators with unlimited flexibility.  In any number of dimensions, they could approximate any classification boundary. However, this capability comes at a price. There is a need to train large systems with a huge number of parameters. Then, an acceptable architecture with the tuned parameters will be trusted for all future classifications. In the face of several local minima, all global optimization methods yield "optimal" parameters ($w$) that differ significantly from one run of the algorithm to the next. This reveals a great deal of randomness arising from various initial weights ($w^0$) and various sequencing of training examples. This randomness appears to distinguish between network errors. The final weights correspond to various ways of identifying the training pattern. Hence, the collective decision created by several ANNs may, therefore, be much less fallible than any network. In \cite{kim2010}, the ANNs in the form of base classifiers were generated to form bagging and boosting ensembles to predict the bankruptcy. While in \cite{plesinger2018} a convolutional neural network is linked to a shallow neural network to classify arrhythmia in a Holter ECG signal.


\textbf{Swarm Intelligence:}  SI algorithms are defined as \textit{"nature-inspired algorithms that concern the collective, emerging behavior of multiple, interacting agents that follow some simple rules"} \cite{hassanien2018}. These algorithms mimic the social behavior of swarms/groups of creatures in nature.  In \cite{mirjalili2014}, the authors discussed the benefits of SI over evolutionary algorithms.  Most SI methods consider exploration and exploitation in their working mechanisms \cite{mirjalili2010}. The popularity of those algorithms returns to; the simplicity of inspiration, flexibility, derivative-free mechanism, and local optimum avoidance \cite{mirjalili2014}. Firefly algorithm has been applied in \cite{krawczyk2015} to combine the ensemble pruning with a weighted classifier fusion module. In addition, the Ant colony algorithm has been incorporated to optimize the decision forest \cite{kozak2015} to provide self-adaptability with the classification task. Recent SI algorithms will be discussed in Sections \ref{Sec:4_3_5-MFO}, \ref{GWOalgo}, and \ref{Sec:4_3_5-WOA}.










% state-of-the-art
\chapter{State-of-the-art}
\label{cha:3_State-of-the-art}

This chapter provides an in-depth review of recent advancements in stream classification, addressing key challenges in dynamic data environments, such as concept drift, imbalanced multiclass scenarios, class overlap, ensemble selection, new class emergence, and transfer learning. Section \ref{sec:3_1_concept_drift} explores methodologies for real-time concept drift detection and adaptation to maintain classification accuracy. Section \ref{sec:3_2_ensemble} focuses on dynamic classifier ensemble selection strategies to enhance performance in evolving data streams. Section \ref{sec:3_3_imbalanced} tackles imbalanced multiclass problems, discussing oversampling techniques to balance minority classes in drifted streams. Section \ref{sec:3_4_emergence} examines methods for integrating new classes into existing systems to improve adaptability, while Section \ref{sec:3_5_transfer_learning} highlights the role of transfer learning in leveraging knowledge from related tasks to improve performance, particularly in scenarios with limited labeled data. Section \ref{sec:3_6_comparsion} provides a comparative analysis of recent works, evaluating contributions, limitations, and research gaps, and Section \ref{sec:3_7_remartks} concludes the chapter by outlining critical challenges and directions for future research.



%%%%%%%%%%%%%%%%%%%%%%%%%%%%%%%%%%%%%%%%%%%%%%%
%
%   Taxonomies in the Traffic Forecasting Field
%
%%%%%%%%%%%%%%%%%%%%%%%%%%%%%%%%%%%%%%%%%%%%%%%
\section{Concept Drift}
\label{sec:3_1_concept_drift}
Concept drift refers to changes in the underlying data distribution over time, which can reduce the accuracy of previously trained machine learning models \cite{baena2006early, madkour2023historical, tan2022information}. Detecting and responding to concept drift is crucial for maintaining model performance. Several detection methods have been proposed to address this challenge. The Drift Detection Method (DDM) \cite{gama2004learning, bifet2009new} uses a statistical test to identify significant error rate increases, signaling concept drift. The Early Drift Detection Method (EDDM) \cite{gama2004learning, adams2023explainable} extends DDM by considering a moving window of recent data. ADWIN \cite{gama2004learning, adams2023explainable} employs a sliding window to monitor statistical differences and adjusts the window size to adapt to drift patterns. The Kolmogorov-Smirnov windowing method (KSWIN) \cite{adams2023explainable} calculates the Kolmogorov-Smirnov distance to detect drift, while Hoeffding's bounds with moving average test (HDDMA) and its variant HDDMW \cite{gama2004learning, bifet2009new} compute bounds for the true mean to detect distribution changes. Lastly, the Page-Hinkley method \cite{page1954continuous} tracks the cumulative sum of errors, detecting drift when the sum exceeds a threshold. These methods enable machine learning models to adapt to evolving data streams, enhancing model performance and robustness.

%%%%%%%%%%%%%%%%%%%%%%%%%%%%%%%%%%%%%%%%%%%%%%%
%
%   General-purpose Automated Machine Learning
%
%%%%%%%%%%%%%%%%%%%%%%%%%%%%%%%%%%%%%%%%%%%%%%%
\section{Classifier Ensemble Selection}

\label{sec:3_2_ensemble}
This study focuses on the overproduce-and-select approach for classifier ensemble selection methods [23] [88] [20]. The primary objective of classifier ensemble selection is to identify the optimal subset of classifiers from a larger ensemble, considering various criteria such as performance measures, diversity metrics, meta-learning techniques, and performance estimation approaches. This selection process aims to reduce computational complexity, enhance efficiency, and improve overall ensemble performance, making it highly valuable for real-world applications. By carefully selecting a smaller subset of classifiers, ensemble selection strikes a balance between accuracy and computational resources, adapting to the evolving nature of the data stream. This approach leverages the strengths of different classifiers and adjusts the ensemble composition to handle changing conditions effectively. The goal is to enhance the accuracy, robustness, and overall performance of classification models in dynamic and challenging scenarios. There are two main approaches to the selection process: static and dynamic selection. Static selection assigns classifiers to specific partitions of the feature space, while dynamic selection chooses a classifier specifically for each unknown data sample based on its local competencies. Dynamic Ensemble Selection (DES) is a widely recognized approach that selects the best classifiers for each test instance, considering their competence within the local region of competence. The Randomized Reference Classifier proposed by Woloszynski and Kurzynski [21] stands out among various approaches. This classifier introduces randomness through beta distribution, enhancing adaptability and robustness. By considering the stochastic nature of class supports, the Randomized Reference Classifier can potentially improve classification performance in concept drift scenarios. However, it is important to note that employing diversity measures during the classifier selection process, as demonstrated by Lysiak [22], may lead to smaller ensembles but does not necessarily enhance classification accuracy. Overall, the overproduce-and-select approach for classifier ensemble selection methods offers a comprehensive framework for addressing the challenges associated with concept drift. By dynamically adapting the ensemble composition and leveraging the competencies of individual classifiers, this approach aims to improve classification performance, efficiency, and adaptability in dynamic and challenging scenarios.
%%%%%%%%%%%%%%%%%%%%%%%%%%%%%%%%%%%%%%%%%%%%%%%
%
%   Machine Learning for Traffic Forecasting
%
%%%%%%%%%%%%%%%%%%%%%%%%%%%%%%%%%%%%%%%%%%%%%%%
\section{Imbalanced data Streams}
\label{sec:3_3_imbalanced}
In imbalanced data classification, three main approaches have been identified \cite{yin2022graph}, with this study emphasizing the first category, which addresses imbalanced data streams through sampling methods, particularly oversampling \cite{ren2023grouping}. This technique creates synthetic instances to achieve balanced class distributions \cite{nitesh2002smote, han2005borderline, bunkhumpornpat2009safe, maciejewski2011local}. Imbalances can arise in both binary and multi-class settings, with this research concentrating on multi-class oversampling strategies.

To address multi-class imbalances, Multi-Label SMOTE (MLSMOTE) \cite{charte2015mlsmote} extends SMOTE to multi-class learning by generating synthetic examples for minority class labels and ensuring their proper assignment. A more recent technique, Multi-Label Synthetic Oversampling based on Local Label Imbalance (MLSOL) \cite{yin2022graph}, improves upon MLSMOTE by targeting local imbalances within multi-class classification. MLSOL uses distinct sampling strategies for each label, offering superior performance in classification accuracy and other metrics. It generates synthetic samples from minority class instances in a restricted neighborhood, improving computational efficiency and reducing overfitting, making it a promising technique that outperforms MLSMOTE in several areas.
\section{Streams with Emerging New Classes (SENC)}
\label{sec:3_4_emergence}
Existing approaches have been proposed to detect and handle the emergence of new classes in streaming data. Clustering-based methods, such as SACCOS \cite{gao2020saccos}, ECSMiner \cite{masud2010classification}, and SAND \cite{haque2016sand}, employ clustering techniques to identify new class emergence. However, these methods require access to true labels for either parts or all instances, limiting their practical applicability. Similarly, SENC-MaS \cite{mu2017streaming} uses matrix sketches for detecting emerging new classes but assumes the availability of true label information for all instances. In contrast, tree-based methods like SENCForest \cite{mu2017classification} and SEEN \cite{zhu2020semi} utilize anomaly detection techniques to identify new classes, often with limited or no label information. However, these methods often suffer from high false positive rates and runtime inefficiencies. Another approach, SENNE \cite{cai2019nearest}, focuses on exploiting local information using the nearest neighbor ensemble for improved detection performance.  Nevertheless, the absence of an effective model retirement mechanism in SENNE results in longer runtimes than alternative methods. The k-nearest Neighbor Ensemble-based method (KNNENS) \cite{zhang2022knnens} method emerges as a promising solution for the challenges of streaming emerging new class problems. By effectively utilizing a k-nearest neighbor-based hypersphere ensemble and incorporating model updates, the KNNENS approach tackles the issues of new class detection and known class classification within a unified framework. It is worth noting that an explicit limitation of existing methods is their lack of utilization of concept drift techniques for detecting emerging new classes and retraining the classification model. This limitation highlights the need for approaches that can effectively handle concept drift while addressing the emergence of new classes in streaming data.

\section{Transfer Learning} 
\label{sec:3_5_transfer_learning}
Transfer learning has gained significant attention for addressing distribution disparities between source and target domains, with methods falling into two categories: instance re-weighting and feature matching \cite{long2013transfer}. Instance re-weighting methods, such as Kernel Mean Matching (KMM) \cite{long2014transfer}, Kullback-Leibler Importance Estimation Procedure (KLIEP) \cite{sun2011two}, and TrAdaBoost \cite{freund1996experiments}, adjust the weights of source instances to align with the target domain. These methods have been extended to multisource transfer learning (e.g., MsTrAdaBoost \cite{sun2016return}). Feature matching approaches, including Transfer Component Analysis (TCA) \cite{sun2016return} and CORAL \cite{rahman2020correlation}, focus on aligning feature representations between domains, often through transformations that minimize distribution differences. However, negative transfer, where transferred knowledge harms performance, is a challenge addressed by techniques like Transfer Joint Matching (TJM) \cite{zhong2009cross}. Additionally, methods such as HE-CDTL \cite{powers2020evaluation} handle concept drift in transfer learning by incorporating historical knowledge and class-wise weighted ensembles. Finally, online transfer learning techniques like Melanie \cite{sun2016return} manage non-stationary environments by dynamically adjusting model weights to accommodate concept drift.

% Define colors
\definecolor{headerColor}{HTML}{4F81BD}
\definecolor{rowColor1}{HTML}{B8CCE4}
\definecolor{rowColor2}{HTML}{DCE6F1}
\definecolor{textColor}{HTML}{000000}
\definecolor{highlightColor}{HTML}{00B0F0}
\section{Comparsion} 
\label{sec:3_6_comparsion}

This section presents a critical comparison of closely related works addressing the challenges of imbalanced multiclass streams \ref{sec:3_6_1_related_work_imbalanced}, the emergence of new classes \ref{sec:3_6_2_related_work_emergence}, and the integration of transfer learning \ref{sec:3_6_2_related_work_transfer} within streaming environments. The increasing complexity of real-world data streams necessitates advanced methodologies that can effectively manage the intricacies of these challenges. By examining various approaches in the literature, the goal is to highlight their contributions, strengths, and limitations in dealing with imbalanced data distributions, adapting to new class occurrences, and leveraging transfer learning techniques. This comparative analysis highlights the current state of research while emphasizing specific gaps and unresolved challenges, paving the way for more robust and adaptive solutions in streaming data classification.


\subsection{Imbalanced Stream}
\label{sec:3_6_1_related_work_imbalanced}

Addressing class imbalances is critical in multi-class classification. Multi-Label SMOTE (MLSMOTE) \cite{charte2015mlsmote} enhances classifier performance by generating synthetic examples for minority class labels using neighboring examples in the feature space. However, Multi-Label Synthetic Oversampling based on Local label imbalance (MLSOL) \cite{liu2022multi} improves upon this by employing tailored sampling strategies for each label to address local imbalances. Research shows that MLSOL outperforms MLSMOTE in classification accuracy and computational efficiency by generating synthetic samples from minority instances within restricted neighborhoods, resulting in a more compact and efficient dataset while reducing overfitting. Figure \ref{fig:mlsmote_mlsol} illustrates that MLSOL is more likely to select $x1$ as a seed instance because it is surrounded by more neighbors of the opposite class for $l3$. MLSMOTE assigns the label vector [0,1,0] to all synthetic instances based on their neighbors. In contrast, MLSOL creates more diverse instances by assigning labels according to their location. Moreover, synthetic instances $c2$ and $c3$ generated by MLSMOTE introduce noise, whereas MLSOL copies the labels of the nearest instance to the new examples. In summary, MLSMOTE tends to generate new instances biased toward the dominant class in the local area, whereas MLSOL effectively explores and exploits both the feature and label space.
\begin{figure*}[!ht]

    \begin{center}
      \includegraphics[width=1\textwidth]{3_State-of-the-art/fig/mlsmote_mlsol.png}
    \end{center}

    \caption{Comparsion of MLSMOTE and MLSOL Generated Instances \cite{liu2022multi}.\\ \textcolor{gray}{\fontsize{10}{0}\selectfont DOI: 110.1016/j.patcog.2021.108294}}
    \label{fig:mlsmote_mlsol}

    \end{figure*}
    
    Table \ref{table:imbalanced} compares MLSMOTE and MLSOL, two methods for addressing imbalanced data in multi-class classification. MLSMOTE generates synthetic examples for minority classes to balance distributions but may blur class distinctions and struggles with overlapping classes, leading to misclassification. MLSOL improves upon this by using localized sampling strategies to address local imbalances more effectively. However, both methods face limitations with overlapping class boundaries, which impact their overall classification accuracy. Despite MLSOL's precision in handling local imbalances, the challenge of overlapping classes remains significant for both approaches.
\begin{table*}[!ht]

    \centering
    \caption{Comparison of the MLSMOTE and MLSOL Methods.}
    \label{table:imbalanced}

    \small % Reduce font size
    \renewcommand{\arraystretch}{1} % Reduce cell padding
    \setlength{\tabcolsep}{4pt} % Reduce cell padding
    \setlength{\arrayrulewidth}{0.15mm}

    \begin{tabularx}{\textwidth}{|>{\centering\arraybackslash\bfseries}p{2cm}|
                                       >{\raggedright\arraybackslash}X|
                                       >{\raggedright\arraybackslash}X|
                                       >{\raggedright\arraybackslash}X|}
    \hline
    \textbf{Method} & \textbf{Theory} & \textbf{Advantages} & \textbf{Limitations} \\ 
    \hline
    \textbf{MLSMOTE \cite{charte2015mlsmote}} & 
    MLSMOTE significantly enhances classifier performance by generating synthetic examples for each minority class label. & 
    Generating synthetic examples for each minority class label. & 
    \begin{itemize}[leftmargin=*]
        \item Random synthetic samples may be related to the majority class.
        \item Overlapping classes.
    \end{itemize} \\ 
    \hline
    \textbf{MLSOL \cite{liu2022multi}} & 
    MLSOL systematically combats local imbalances within the domain of multi-class classification by employing distinct sampling strategies for each label. & 
    Generating synthetic examples for each minority class label within a restricted neighborhood. & 
    \begin{itemize}[leftmargin=*]
        \item Overlapping classes.
    \end{itemize} \\
    \hline
    \end{tabularx}
    \end{table*}

\subsection{Emergence of new classes}
\label{sec:3_6_2_related_work_emergence}


Detecting and adapting to new classes in streaming data is essential for maintaining classification accuracy. Tree-based methods like SENCForest \cite{mu2017classification} and SEEN \cite{zhu2020semi} use anomaly detection but suffer from high false positive rates and inefficiencies. SENNE improves detection using a nearest neighbor ensemble but has longer runtimes due to ineffective model retirement. The k-Nearest Neighbor Ensemble-based method (KNNENS) \cite{zhang2022knnens} enhances new class detection and known class classification through hypersphere ensembles and dynamic model updates. However, all these methods struggle to handle concept drift effectively, which is critical for detecting new classes and updating classification models.

Figures \ref{fig:SENCForest}, \ref{fig:SENNE}, and \ref{fig:KENNE} illustrate key approaches for classifying emerging and known classes. SENCForest divides the space into three regions (normal, outlying, and anomaly) and detects new classes using threshold path lengths. SENNE uses hyperplanes in three dimensions ($x1$, $x2$, and $x3$) to classify instances as emerging or known based on class rankings. KNNENS employs hyperplanes for all class samples and uses a voting mechanism to classify instances as emerging or known. These visualizations emphasize the differences in how SENNE and KNNENS handle class classification.

\begin{figure*}[!ht]
    \centering
    \includegraphics[width=0.80\textwidth]{3_State-of-the-art/fig/SENCForst.png}
    \caption{Overview of SENCForest Detection Flow Diagram \cite{mu2017classification}. \\
    \textcolor{gray}{\fontsize{10}{0}\selectfont DOI: 10.1109/TKDE.2017.2691702}}
    \label{fig:SENCForest}
\end{figure*}
    

\begin{figure*}[!ht]
    \begin{center}
        \includegraphics[width=.45\textwidth]{3_State-of-the-art/fig/senne0.png} 
        \includegraphics[width=.45\textwidth]{3_State-of-the-art/fig/senne.png} 
        (a)\hspace{6.5cm}(b)
    \end{center}
    \caption{Overview of the Stream Emerging Nearest Neighbor Ensemble (SENNE) \cite{zhu2020semi}.}
    \label{fig:SENNE}
    \end{figure*}
    \vline
    \begin{figure*}[!ht]    
        \begin{center}
            \includegraphics[width=.45\textwidth]{3_State-of-the-art/fig/kenne0.png} 
            \includegraphics[width=.45\textwidth]{3_State-of-the-art/fig/kenne.png}
            (a)\hspace{6.5cm}(b)
            \end{center}
    
        \caption{Overview of the k-nearest Neighbor Ensemble-based \cite{zhang2022knnens} (KENNE).}
        \label{fig:KENNE}
        \end{figure*}
        
        Table \ref{table:emerging} compares three emerging class detection methods: SENCForest, SENNE, and KNNENS. SENCForest uses iForest \cite{wang2010negative} for anomaly detection and a threshold path for identifying new classes, serving as both an unsupervised detector and supervised classifier, but it is prone to false positives and relies on a complex threshold mechanism. SENNE utilizes a nearest neighbor-based hypersphere ensemble to analyze local neighborhood information, effectively handling varying geometric distances between classes, though it assumes static class distributions and has lengthy update times. KNNENS improves by using a hypersphere ensemble for all classes, reducing false positives and enabling updates without true labels, but it shares SENNE's limitation of assuming unchanged known class distributions.
\begin{table*}[!ht]

    \centering
    \caption{Comparison of the SENCForest, SENNE, and KENNE methods.}
    \label{table:emerging}
    \small % Reduce font size
    \renewcommand{\arraystretch}{1} % Reduce cell padding
    \setlength{\tabcolsep}{4pt} % Reduce cell padding
    \setlength{\arrayrulewidth}{0.15mm}
    \begin{tabularx}{\textwidth}{|>{\centering\arraybackslash\bfseries}p{2cm}|
                                       >{\raggedright\arraybackslash}X|
                                       >{\raggedright\arraybackslash}X|
                                       >{\raggedright\arraybackslash}X|}
    \hline
    \textbf{Method} & \textbf{Theory} & \textbf{Advantages} & \textbf{Limitations} \\ 
    \hline
    \textbf{SENCForst \cite{mu2017classification}} & 
    employs anomaly detection method iForest for a new class detection and then applies threshold path to detect the anomalies. & 
    SENCForest serves as both an unsupervised anomaly detector and a supervised classifier.&
    \begin{itemize}[leftmargin=*]
        \item Potential for High false positives.
        \item Dependency on path length threshold (more complexity).
    \end{itemize} \\ 
    \hline
    \textbf{SENNE \cite{zhu2020semi}} & 
    nearest neighbor-based hypersphere of one class ensemble to explore local neighborhood information and sort distance to calculate distance. & 
    SENNE is able to handle both the low and high geometric distance between two classes in the feature space. & 
    \begin{itemize}[leftmargin=*]
        \item Assumes that the distribution of known classes remains unchanged.
        \item Take long time for update.
    \end{itemize} \\ 
    \hline
    \textbf{KENNE \cite{zhang2022knnens}} & 
    nearest neighbor-based hypersphere of all class ensemble to explore local neighborhood information. & 
    KNNENS to reduce false positives for the new class. KNNENS does not require true labels to update the model. & 
    \begin{itemize}[leftmargin=*]
        \item Assumes that the distribution of known classes remains unchanged.
    \end{itemize} \\
    \hline
    \end{tabularx}
    \end{table*}



\subsection{Transfer Learning}
\label{sec:3_6_2_related_work_transfer}

In transfer learning, CORAL, Melanie, and HE-CDTL are key approaches related to the third proposed method \ref{chapter:6_transfer_learning}. CORAL \cite{sun2016return} aligns sub-space bases through second-order statistics using a learned transformation matrix, minimizing domain discrepancies and negative transfer. Melanie \cite{dong2019multistream} employs an online ensemble learning strategy to address non-stationary environments by incrementally training models from source and target domains, dynamically adjusting weights, and combining models via a weighted-sum approach. HE-CDTL extends these concepts specifically for Concept Drift Transfer Learning (CDTL), leveraging historical and source domain knowledge through a class-wise weighted ensemble and AW-CORAL to reduce domain disparities. Experiments show HE-CDTL outperforms baseline methods, demonstrating its efficacy in managing transfer learning under concept drift.
Table \ref{table:transfer} compares CORAL, Melanie, and HE-CDTL for transfer learning. CORAL minimizes domain discrepancies and reduces negative transfer by projecting source data into the target domain using a transformation matrix and Singular Value Decomposition (SVD) but struggles with non-stationary and heterogeneous data. Melanie addresses non-stationary environments in online learning by dynamically training and combining models from source and target domains, yet faces challenges with the complexity of online learning and data heterogeneity. HE-CDTL reduces domain shifts by aligning second-order statistics and leveraging historical knowledge but depends on source domain quality and also encounters issues with heterogeneous data.
\begin{figure*}[!ht]

    \begin{center}
        \includegraphics[width=.80\textwidth]{3_State-of-the-art/fig/coral.png} 
    \end{center}
    \caption{Overview of CORrelation ALignment (CORAL)\cite{sun2016return}.}
    \label{coral_fig}
    \end{figure*}
    \begin{figure*}[!ht]    
        \begin{center}
            \includegraphics[width=.80\textwidth]{3_State-of-the-art/fig/cdtl.png} 
        \end{center}
        \caption{Overview of Concept Drift Transfer Learning (CDTL) \cite{sun2016return}.}
        \label{cdtl_fig}

        \end{figure*}

\begin{table*}[!ht]
    \centering
    \caption{Comparison of the CORAL, Malanie, and CDTL Methods.}
    \label{table:transfer}
    \small % Reduce font size
    \renewcommand{\arraystretch}{1} % Reduce cell padding
    \setlength{\tabcolsep}{4pt} % Reduce cell padding
    \setlength{\arrayrulewidth}{0.15mm}
    \begin{tabularx}{\textwidth}{|>{\centering\arraybackslash\bfseries}p{2cm}|
                                       >{\raggedright\arraybackslash}X|
                                       >{\raggedright\arraybackslash}X|
                                       >{\raggedright\arraybackslash}X|}
    \hline
    \textbf{Method} & \textbf{Theory} & \textbf{Advantages} & \textbf{Limitations} \\ 
    \hline
    \textbf{CORAL \cite{sun2016return}} & 
    Correlation Alignment (CORAL) uses a learned transformation matrix and Singular Value Decomposition (SVD)  to project the source instances into the target domain. & 
    CORAL can minimize domain discrepancy across
source and target domains, meanwhile reducing the negative
knowledge transfer. & 
    \begin{itemize}[leftmargin=*]
        \item Non-stationary environments.
        \item Heterogenous multisource.
    \end{itemize} \\ 
    \hline
    \textbf{Melanie \cite{dong2019multistream}} & 
    Multi-sourcE onLine TrAnsfer
learning for Non-statIonary Environments (Melanie). utilize the class-wise weighted . & 
It considers
an online problem in which the data in source and target
domains are generated from non-stationary environments. & 
    \begin{itemize}[leftmargin=*]
        \item Based on the online learning  only.
        \item Heterogenous multisource.
    \end{itemize} \\
    \hline
    \textbf{HE-CDTL \cite{sun2016return}} & 
    HE-CDTL uses the class-wise weighted and domain wise ensemble for historical knowledge and reduce the disparities between the source and target domains . & 
    HE-CDTL minimizes domain shift by aligning the second-order statistics of source and target distributions. & 
    \begin{itemize}[leftmargin=*]
        \item Depend on source domain quality.
        \item Heterogenous multisource.
    \end{itemize} \\
    \hline
    \end{tabularx}
    \end{table*}

\section{Related Works Challenges} 
\label{sec:3_7_remartks}

By comparing the literature on ensemble learning for classification tasks, the proposals in this thesis differ from other studies in several ways:

\begin{enumerate}
    
    \item [-] As evident from our literature review on imbalanced streams, most studies have concentrated on generating synthetic samples while ignoring class overlap. \textit{To address this challenge}, we propose an approach to generate non-overlapping classes in imbalanced streams.

    \item [-] Oversampling techniques often perform inefficiently in the presence of concept drift. \textit{To tackle this issue}, we introduce a methodology that selects the oversampling technique based on the current and historical distribution of the stream chunks.
    
    \item [-] Our literature review on non-stationary environments reveals that most works focus on detecting emerging new classes while overlooking distribution changes. \textit{To overcome this challenge}, we propose a combined approach utilizing Dynamic Ensemble Selection (DES) to select the best classifier for each chunk based on stream distribution, k-means clustering, and concept drift to address both emerging new class detection and distribution changes.
    
    \item [-] In our literature review on transfer learning, we observed that most studies focus on homogeneous multisource transfer and neglect heterogeneous multisources in non-stationary environments. \textit{To resolve this issue}, we propose a combined approach integrating Dynamic Ensemble Selection (DES), Concept Drift Transfer Learning (CDTL), eigenvector techniques, and concept drift to address heterogeneous transfer learning in non-stationary environments.

\end{enumerate}



% imbalanced
% this file is called up by thesis.tex
% content in this file will be fed into the main document


\chapter{Dynamic Classification Ensembles for Handling Imbalanced
  Multiclass Drifted Data Streams}
  \label{chapter:4_Imbalanced_Multiclass}
  
  In recent years, the explosion of high-speed data streams has presented new challenges for machine learning models. Three critical
  issues that have emerged are concept drift, class imbalance, and class overlap. Concept drift refers to the phenomenon where the
  statistical properties of a data generation process change over time \cite{yang2021concept, dong2019multistream}. This signifies that the underlying concepts, relationships
  between variables, or data distribution can change, leading to a fundamental shift in the nature of data. Dealing with concept drift
  poses a fundamental challenge in machine learning and data mining. This can cause models trained on historical data to become
  inadequate when applied to new data affected by concept drift, leading to a decline in the model performance \cite{dong2019multistream}. To address this issue,
  concept drift detectors are used to identify changes in data stream distributions by leveraging information associated with classifier
  performance or the incoming data items themselves. These signals frequently prompt model updates, retraining, or substitution of an
  old model with a new one.
  In addition, class imbalance \cite{dong2019multistream, pan2009survey}, which is characterized by uneven class distribution, poses a challenge for traditional classifiers
  \cite{zhuang2020comprehensive}, particularly in multiclass scenarios where minority class samples are at risk of misclassification owing to their limited representation \cite{wang2018systematic}. Addressing imbalanced data classification requires specialized techniques to ensure accurate minority class classification without compromising the performance of the majority class \cite{sun2009classification, charte2015addressing, charte2015mlsmote}. This challenge becomes more daunting when minority class instances are scattered in unknown configurations, thereby increasing the likelihood of overfitting during the learning process. To
  address this issue, three primary methods are employed in the context of imbalanced data classification, which are effective in both
  binary and multiclass imbalance scenarios \cite{daniels2017addressing}. The first approach involves sampling methods that address class imbalance by either
  reducing the number of majority class instances (undersampling) or generating artificial minority class instances (oversampling)
  \cite{liu2018making,lopez2012analysis}. The second and third groups encompass adaptive algorithms and hybrid methods, respectively. Adaptive algorithms include
  one-class and cost-sensitive classification \cite{zhang2020towards}. Hybrid methods merge data preprocessing with classification techniques, often utilizing ensemble classifiers to effectively mitigate class imbalance and enhance classifier performance \cite{chawla2003smoteboost, wang2010negative, bhowan2012evolving}.
  Class overlap occurs when instances from different classes inhabit the same region in the data space \cite{galar2011review, cruz2018dynamic}. This overlap complicates the task of distinguishing between representative instances of various classes and posing performance challenges for traditional classifiers. This issue is commonly referred to as class overlap. Researchers have proposed class overlap undersampling
  techniques to address class imbalance problems \cite{kuncheva2000clustering}. These techniques aim to leverage local similarities among minority instances to
  identify potentially overlapping majority instances. Although these methods have demonstrated promising results in improving model
  performance on specific datasets, many of them rely heavily on nearest-neighbor approaches to detect overlapping regions around
  minority instances. This approach often neglects the global similarity within the overlapping domain, which can lead to local optimal
  values getting stuck during calculations. Furthermore, determining appropriate parameters for these models poses a significant
  challenge. If the parameter selection is too extensive, it may lead to the exclusion of valuable instances, while conservative parameters
  may overlook instances with overlap. This issue can significantly affect classifier performance, particularly when handling streams
  containing both a minority class and instances with class overlap. Therefore, both class imbalance and class overlap present significant
  challenges in the realm of data stream analyses. Consequently, addressing class imbalance is crucial in multiclass learning, leading to
  research efforts that focus on both concept drift and class imbalance challenges. Researchers have explored techniques such as DES and
  multiclass oversampling to address these issues.
  Dynamic classifier ensembles offer the unique ability to adapt their composition based on data characteristics, making them
  valuable in situations with evolving data conditions \cite{woloszynski2011probabilistic}. The aim of classifier ensemble selection is to identify the optimal subset of
  classifiers from a larger ensemble. A prominent approach in classifier ensemble selection is the overproduce-and-select strategy. This
  selection process is guided by diverse criteria, including individual performance measures, diversity metrics, meta-learning techniques,
  and performance-estimation approaches. Such optimization is particularly crucial in scenarios in which striking a balance between
  accuracy and computational resource constraints is paramount. There are two distinct approaches: static and dynamic approaches.
  Static selection involves assigning classifiers to predefined feature partitions, whereas dynamic selection adaptively selects classifiers
  based on their competency \cite{lysiak2014optimal}. Dynamic selection offers two choices: Dynamic Classifier Selection (DCS) and Dynamic Ensemble
  Selection (DES). DCS algorithms enable the selection of the most appropriate classifier for each data point, based on its local competencies. In contrast, DES focuses on selecting the optimal classifiers for each instance based on their competence within localized
  regions \cite{cruz2017meta, widmer1996learning, lu2016concept}. Competency assessment relies on a Dynamic SELection dataset (DSEL) containing labeled samples. Moreover,
  innovative techniques, such as the randomized reference classifier, introduce randomness into class supports to enhance adaptability
  in addressing the challenges related to imbalanced data.
  The main goal of this study is to formulate a precise classification approach that addresses changing conditions. Specifically, the
  proposed approach aims to address scenarios where there is an uneven distribution among several classes, overlapping instances of
  classes, and instances where the fundamental concept of data evolves. To address these challenges, we employed dynamic classifier
  ensembles. These ensembles utilize oversampling techniques, implemented either on a global or local scale, as a preprocessing step to
  address class imbalance. Furthermore, we enhanced multiclass learning techniques to counteract class imbalance through method
  adaptation. 
  
  The remainder of this chapter is organized as follows: In Section \ref{sec:4_2_motivation}, we present the motivations and the contributions. The proposed framework are discussed in detail in Section \ref{sec:4_first_proposed_approach}. The  experimental results and the discussion are presented in Sections \ref{sec:4_5_Expsetup}. Finally, the conclusions of this study and future research are discussed in Section \ref{sec:4_8_Conclusions}. 
  
  
  \section{Motivations and Contributions} \label{sec:4_2_motivation}
  \begin{enumerate}[nosep]
    \item We introduce a classification approach that dynamically adjusts to multiclass imbalanced data, incorporates mechanisms for
    detecting concept drift, and optimizes classifier ensemble selection. The objective is to enhance the classification accuracy, specifically for multiclass imbalanced non-stationary streams.
   \item Additionally, we propose an adaptive method for the class imbalance issue, considering the data distributions and historical instances of class imbalance. This is particularly relevant in cases where class overlap occurs within the multiclass and drifted datan performance by selecting the most suitable oversampling method based on the unique characteristics of the data stream. 
    \end{enumerate}
\section{Proposed Methodology}
\label{sec:4_first_proposed_approach}

In this section, we present the primary phases of our study, comprising three distinct stages. Following this introduction, we delve into the synthetic data-generator method, providing a comprehensive breakdown of the four essential steps. Our proposal aims to develop a robust approach designed to overcome the challenging domain of multiclass classification for imbalanced and drifting data streams. In pursuit of this goal, our proposal addresses the four primary challenges inherent in constructing our approach.
\begin{itemize}
	\item \textbf{Multi-class imbalanced streams:} This study focuses on addressing the widespread problem of imbalanced data streams in the context of multiple classes. We aim to address this issue using well-known techniques, notably MLSMOTE and MLSOL.
	\item \textbf{Overlapping class:} We also address a critical challenge involving class overlapping, a factor known to substantially affect model performance \cite{cruz2017meta, widmer1996learning}. To address this issue, we introduce an adaptive method that generates nonoverlapping synthetic instances, thereby enhancing the overall performance of the model.
	\item \textbf{Drifted streams:} First proposed approach integrates a concept drift detector to identify shifts in the underlying data distribution. This dynamic detection mechanism enables the model to promptly recognize changes and adjust its classifiers, thereby ensuring its effectiveness in handling drifting stream.
	\item \textbf{Classifier performance:} To enhance the classifier performance, our approach employs Dynamic Ensemble Selection (DES). This technique creates a pool of classifiers and dynamically selects the most suitable classifier for each incoming data point, further improving classification accuracy and robustness.
\end{itemize}

\subsection{Approach Overall Details}

First proposed approach is designed with three distinct phases that work together to improve its performance in managing multiclass imbalanced and drifting data streams. 
\begin{itemize}
	\item \textbf{DES phase (dynamic ensemble selection phase):} The first phase, known as the dynamic ensemble selection (DES) phase, is responsible for selecting the most appropriate classifier for the incoming data. This ensures that the selected classifier is well-suited for the current data chunk.
	\item \textbf{Drift detector phase:} The second phase of our approach is the drift detector phase, which operates in real-time to continuously monitor the data stream. Its primary function is to identify any signs of concept drift, which indicates shifts in the underlying data distribution over time.
	\item \textbf{Synthetic data generator phase:} The final phase of our approach is the synthetic data generator phase, which is dedicated to generating synthetic data for the minority classes. This step is crucial for addressing class imbalance by producing additional samples for underrepresented classes, thereby significantly enhancing the model's ability to accurately classify instances from minority classes.
\end{itemize}

Specifically, as shown in Fig. \ref{fig:4_first_proposal_step_1}, the DES phase retrieves the current data chunk from the stream and applies the DES technique to select the most suitable classifiers for the received chunk. The selected classifiers were then passed to the second phase, where they were employed to predict the class of each instance within the received data chunk. Simultaneously, detectors like ADWIND or DDM are employed to monitor any occurrence of concept drift. If the discrepancy between the class frequency and standard deviation of the current chunk is significant, as described in reference \cite{gama2004learning}, and the imbalance ratio exceeds the average imbalance ratio, the current chunk is forwarded to the third phase, as indicated by the red rectangle in Fig. 1. In the third phase, first proposed method uses a set of equations \ref{eq:4_first_proposal_1}\ref{eq:4_first_proposal_2}\ref{eq:4_first_proposal_3}
to identify minority classes. The initial equation computes the frequency of each class within the current chunk, where $i$ represents the current chunk, $c$ denotes the input class, and $y_i$ refers to the predicted class. The second equation determines the optimal frequency for each class based on the size of the chunk ($n$) and the number of classes in the current chunk($C$). Finally, the third equation identifies the classes as minority classes if their frequency deviates significantly from the standard deviation of the current chunk, where $sd_c$ represents the standard deviation of the class instances, and $frq_i$ refers to the standard deviation of the current chunk.
As shown in Fig. \ref{fig:4_first_proposal_step_2}, phase These identified minority classes are then fed into the synthetic data generator phase, which increases the minority class samples to balance any imbalanced chunks. This ensures optimal performance for the new classifiers. Algorithm  \ref{alg:4_alg_1} provides a comprehensive outline of the process of the first proposed approach, which is the main contribution of our study and is designed to effectively address multiclass imbalanced and drifting data streams, uses streaming data as input, and systematically executes each step within the approach. The outcome of this process is the classification prediction generated using the first proposed approach.


\subsection{Synthetic Data Generator}

Fig. \ref{fig:4_first_proposal_step_2} presents a comprehensive overview of the synthetic data generator phase, which is an essential component responsible for generating synthetic samples by considering both data distribution and historical chunk behaviors. This phase has several advantages and can perform a wide range of tasks.
\begin{itemize}
	\item \textbf{Similar chunk analysis:} Initially, the phase analyzes the current chunk distribution and identifies a similar chunk from historical data. This analysis forms the basis for generating synthetic samples that align with prevailing distribution patterns.
	\item \textbf{Oversampling method selection:} This phase utilizes the knowledge of the oversampling technique applied to the identified similar chunk. Consequently, it employs an alternative oversampling technique, using MLSMOTE and MLSOL, to create the most effective synthetic data. This step is designed not only to optimize the current classification but also to preemptively address potential drifts in similar future chunks.
	\item \textbf{Class overlap validation:} This step involves generating synthetic samples for the minority classes to effectively address the issue of minority classes and consequently enhance the classifier performance. The process utilizes the K-Nearest Neighbor (KNN) algorithm, as applied in \cite{lu2016concept}, to identify overlaps between newly generated data instances and existing instances. If overlaps are detected, the first proposed approach iteratively removes these samples because their presence can potentially diminish the overall classifier performance \cite{cruz2017meta, widmer1996learning}. Consequently, the first proposed approach generates alternative samples to address this challenge and preserve the primary objectives of the synthetic data generator step.
	\item \textbf{Continuous refinement:} This process iterates until it successfully generates high-quality synthetic data that aligns with the data distribution, minimizes overlap, and mitigates potential concept drift. The generated data were subsequently utilized in the training phase to improve classifier accuracy.
\end{itemize}

\begin{figure}[!ht]
	\centering
	\includegraphics[width=1\linewidth]{4_Imbalanced/figures/approach_step_1.png}
	\caption{First Proposed Approach (PA1) for imbalanced multi-class drifted data streams.}
	\label{fig:4_first_proposal_step_1}
\end{figure}
\begin{figure}[!ht]
	\centering
	\includegraphics[width=1\linewidth]{4_Imbalanced/figures/approach_step_2.png}
	\caption{Flow of the synthetic data generator.}
	\label{fig:4_first_proposal_step_2}
\end{figure}

\begin{equation}
	\label{eq:4_first_proposal_1}
    frq_{c} = \sum_{i=1}^{\text{chunk size}} \begin{cases} 
    1, & \text{if } y_i = c \\
    0, & \text{otherwise}
    \end{cases}, \quad i = 1, 2, 3, \dots \text{chunk size}\;
\end{equation}

\begin{equation}
	\label{eq:4_first_proposal_2}
    \text{best } freq_{n} = \frac{|n|}{|C|}
\end{equation}

\begin{equation}
	\label{eq:4_first_proposal_3}
    \text{classes type}{\text{chunk}} = \sum{c=1}^{C} \begin{cases} 
    \text{Minority,} & \text{if } diff(sd_c - frq_i) > \text{best } freq_{\text{chunk}} \\
    \text{Majority,} & \text{otherwise}
    \end{cases}
\end{equation}


\begin{algorithm}[H]
\caption{First proposed approach algorithm for imbalanced multi-class drifted data streams.}
\label{alg:4_alg_1}
\KwIn{data stream, maximum classifiers pool size $\kappa$}
% \Parameter{current chunk $a$, synthetic data $b$, classifiers pool $\Psi$, drifted pool $\psi$, classes frequency $\Omega$, best frequency $\omega$, minority classes $\mu$}
\KwOut{Prediction $P$}
\BlankLine
$\psi, \Psi, \Omega, \mu \gets \emptyset$\;
$\omega \gets 0$\;
\For{stream have chunk}{
    \eIf{$a$ is the First chunk}{
        $k \gets$ \texttt{trainingNewClassifier}($a$)\;
        $P \gets$ \texttt{getPrediction}($a, k$)\;
    }{
        $k \gets$ \texttt{DES}($a, \Psi$)\;
        $P \gets$ \texttt{getPrediction}($a, k$)\;
        $\psi \gets$ \texttt{conceptDriftDetector}($P$)\;
        \If{$\psi > 0$}{
            $\Omega \gets$ get classes frequency according to Eq.1\;
            $\omega \gets$ best frequency according to Eq.2\;
            $\mu \gets$ get minority classes according to Eq.3\;
            $b \gets$ utilize $a$ and $\mu$ to get the synthetic data according to Algorithm \ref{alg:4_alg_2}\;
            trainingData $\gets a + b$\;
            $k \gets$ \texttt{trainingNewClassifier}(trainingData)\;
            $\Psi \gets \Psi + k$\;
            \If{$\Psi > \kappa$}{
                \texttt{removeWorstClassifier}($\Omega$)\;
            }
        }
        $P \gets$ \texttt{getPrediction}($a, k$)\;
    }
}
\Return{$P$}
\end{algorithm}

\vspace{1cm}

In Algorithm \ref{alg:4_alg_2}, also known as the Synthetic Data Generator, we input three essential elements: the minority class samples, the current data chunk, and the desired size for generating synthetic data. The primary objective of this algorithm is to produce synthetic data samples. To achieve this, we employ the KNN algorithm to identify any overlapping instances within the current chunk (Line 3). This algorithm utilizes two specific techniques, MLSMOTE \cite{gama2004learning} and MLSOL \cite{liu2017regional}. MLSMOTE was chosen for its introduction of randomness during instance generation, reducing its reliance on the local characteristics and distribution of the minority class. This randomization diminishes the likelihood of producing overlapping instances, particularly in cases where minority class instances are situated in overlapping regions. In contrast, MLSOL considers the local behavior of minority classes, resulting in synthetic points that closely resemble the minority class. This approach significantly improves the accuracy of the classifier (lines 4-10). Additionally, in this algorithm, specifically from lines 11 to 17, these lines are dedicated to generating synthetic instances that ensure non-overlap with existing classes. This procedure depends on the selected oversampling method and utilizes the KNN algorithm to guarantee that the generated instances do not overlap with the existing ones.
\vspace{1cm}

\begin{algorithm}[H]
	\caption{Synthetic data generator.}
	\label{alg:4_alg_2}
	\KwIn{Minority classes $\mu$, current chunk $a$, sample size $\eta$, historical chunks $h$}
	\KwOut{Generated data $b$}
	$b \gets \emptyset$\;
	$f \gets \text{MLSMSOTE}$\;
	$knn \gets \text{kNearestNeighbor}(a)$\;
	$chunk \gets \text{similarChunk}(a, h)$\;
	$f \gets \text{similarChunkOverSamplingMethod}(chunk)$\;
	\If{$f = \text{MLSMSOTE}$}{
		$f \gets \text{MLSOL}$\;
	}
	\Else{
		$f \gets \text{MLSMSOTE}$\;
	}
	\While{$|b| < \eta$}{
		$p \gets \text{generateSyntheticPoint}(\mu, f)$\;
		$similarPointsClass \gets \text{KNN.getKneighbor}(b)$\;
		\If{$similarPointsClass = \mu$}{
			$b \gets b \cup \{p\}$\;
		}
	}
	\Return $b$\;
	\end{algorithm}

%%%%%%%%%%%%%%%%%%%%%%%%%%%%%%%%%%%%%%%%%%%%%%%
%
%   Families of Traffic Forecasting Problems
%
%%%%%%%%%%%%%%%%%%%%%%%%%%%%%%%%%%%%%%%%%%%%%%%

%\usepackage{cite}
%\usepackage{multirow}
%\usepackage{rotating} 
%\usepackage[table,xcdraw]{xcolor}
%\usepackage{float}
%\usepackage[utf8]{inputenc}
%\usepackage{amsmath,amssymb,amsfonts}
%\usepackage{algorithmic}
%\usepackage{graphicx}
%\usepackage{textcomp}
%\usepackage{rotating}
%\usepackage{verbatim}


\section{Experimental Results}
\label{sec:4_5_Expsetup}

The objective of the experiments conducted in this study was to evaluate the efficacy of multiclass oversampling techniques in enhancing the performance of the proposed method in imbalanced drifted multiclass classification streams. Our primary goal was to develop a novel approach that combines Dynamic Ensemble Selection (DES) to improve classification accuracy and robustness in such streams. These experiments yielded valuable insights that could further refine the performance of the proposed approach and its ability to effectively handle imbalanced data streams. These findings provide a better understanding of the capabilities of the proposed approach and offer insights into an optimal strategy for tackling minority class issues and concept drift in imbalanced data streams. This study contributes to the advancement of stream mining techniques for generating more accurate and robust classification models in dynamic data stream environments. By addressing the challenges posed by minority classes and concept drift, this study offers valuable insights for improving the performance of the proposed approach and enhancing the overall efficiency of stream mining. The two main questions to be answered are:

\begin{itemize}
  \setlength{\itemindent}{-.5in}
  
      \item $\pmb{Q_1}$: What is the impact of reduced and consistent data on the performance of ensemble learning?
      \item  $\pmb{Q_2}$: Is it possible with the search capability of swarm intelligence to enhance the combination of classifiers? 
  \end{itemize}

\subsection{Experimental setup}
The evaluation of the proposed method incorporated the utilization of various metrics such as recall, precision, specificity, f1 score, balanced accuracy score (BAC), and geometric mean score (G-mean) \cite{bu2016pdf}. The experimental protocol utilized for evaluation was the test-then-train approach \cite{venkatasubramanianinformation}, where the classification classifier was trained on a specific data chunk and subsequently evaluated on the subsequent one. The chunk size was standardized for all utilized data streams to 2,000 instances. We employed four classification classifiers as base estimators: K-Nearest Neighbor (KNN), Support Vector Machine (SVM), Gaussian Naive Bayes (GNB), and Hoeffding Tree (HT), as implemented in scikit-learn \cite{frias2014online}. A pool of classifiers was constructed with a maximum size of L = 8, where the DES selected the best classifier for each chunk. If the pool surpassed the set threshold (L), the classifier with the lowest performance was eliminated. The experiments were conducted using Python programming language, and the source code was publicly available on GitHub\footnote{\url{https://github.com/Amadkour/dynamic__classification_ensembles_for_handling_imbalanced_multi-class_drifted_data_streams.git}} . We conducted a comparison between multiclass oversampling techniques (MLSMOTE and MLSOL) and our proposed approach to demonstrate the effectiveness of our contribution. Additionally, we conducted these experiments using two different concept drift detectors, ADWIN \cite{storkey2008training} and DDM \cite{losing2016knn}, to demonstrate the adaptability and robustness of our proposed approach across varying drift detectors.

\subsection{Data Streams}
In this study, the proposed approach was assessed using various datasets including benchmark datasets, a real application stream dataset, and synthetic data streams. The Stream-learn Python library was used to conduct the evaluations \cite{dries2009adaptive}. Table \ref{tab:4_first_proposal_result_table_1} illustrates the benchmark dataset employed in this study, which consists of the Covertype dataset containing 40 features, seven classes, and 581,010 instances. For real application stream evaluation, the Sensor stream dataset was used, which consisted of five features, 58 classes, and 392,600 instances. This represents a real-world application scenario and provides valuable insights into the performance of the proposed approach in practical settings. Synthetic datasets were generated using Scikit-learn Python library to evaluate the performance of the proposed approach. The synthetic dataset was designed to simulate data streams and comprised 10 features and four classes divided into 200 chunks of 2,000 instances each. The performance of the proposed approach was systematically evaluated using these datasets and a stream-learn library. These evaluations provided insights into the effectiveness of the proposed approach in handling different types of data streams, including benchmark datasets, real application streams, and synthetic data streams.

\begin{table}[h!]
  \centering
  \resizebox{\textwidth}{!}{
  \begin{tabular}{|l|c|c|c|}
  \hline
  \textbf{Dataset} & \textbf{Number of Features} & \textbf{Number of Classes} & \textbf{Number of Instances} \\ \hline
  Covertype dataset\footnote{\url{http://archive.ics.uci.edu/dataset/31/covertype}} & 40 & 7 & 581,010 \\ \hline
  Sensor Stream dataset\footnote{\url{https://www.cse.fau.edu/~xqzhu/Stream/sensor.arff}} & 5 & 58 & 392,600 \\ \hline
  Synthetic stream & 8 & 3 & 200,000 \\ \hline
  \end{tabular}
  }
  \caption{Characteristics of the datasets used in the experimentation.}
  \label{tab:4_first_proposal_result_table_1}
  \end{table}

\subsection{Analysis of Experimental Results}
The performance of the proposed framework was comprehensively assessed on multiple data streams, considering two distinct concept drift detectors, ADWIN and DDM. To ensure thorough evaluation, six key performance metrics—F1 score, recall, precision, G-mean, specificity, and balanced accuracy—were carefully presented using two visualization diagrams: radar and line. A radar diagram was strategically utilized to provide an overview that effectively depicted the performance of each algorithm across the six metrics. The mean value of each metric was calculated to present the overall performance of each method (MLSMOTE, MLSOL, PA). It is important to note that the PA is represented by red lines.
\subsubsection{Results on the Benchmark Stream}
The results of the mentioned methods (MLSMOTE, MLSOL, PA) applied to the Covertype dataset are presented in Fig. \ref{fig:4_first_proposal_result_exp_1}
, utilizing ADWIN as the drift detector. The radar diagram shows that the metric values were between 0.8 and 1.0. MLSOL exhibited the highest precision, whereas MLSOTE and PA had nearly identical values. However, PA excels in other metrics, whereas MLSMOTE has the lowest values. The line diagram in Fig. \ref{fig:4_first_proposal_result_exp_1}
shows the classification accuracy across 100 data chunks using the specificity metric for the methods in each chunk. Notably, all methods exhibit suboptimal accuracy during the first 20 chunks. However, a noticeable improvement was observed beyond this initial phase. The key factor driving this improvement was the expansion of the classifier pool, which now encompasses a growing number of classifiers. This expansion enables the Dynamic Ensemble Selection (DES) technique to become more proficient in selecting the most suitable classifier for each incoming chunk. Consequently, accuracy experienced a significant boost in later chunks, reflecting the adaptability and effectiveness of the ensemble approach. From chunk 20 to the last chunk, PA achieved the highest accuracy, whereas MLSMOTE recorded the lowest accuracy. PA's superior performance of the PA is credited to its use of historical chunks to generate optimal nonoverlapping samples, thereby effectively training the pool classifiers. In Fig. \ref{fig:4_first_proposal_result_exp_2}
, the same dataset is employed with the DDM functioning as the drift detector. The radar plot illustrates results comparable to those of the prior experiment, whereas the line diagram underscores PA's supremacy in most chunks. However, it also reveals that all approaches demonstrate diminished performance, in contrast to Fig. \ref{fig:4_first_proposal_result_exp_1} (ADWIN), suggesting that ADWIN surpasses DDM when utilized in the Covertype data stream.


\begin{figure}[!ht]
	\centering
	\includegraphics[width=1\linewidth]{4_Imbalanced/figures/exp_1.png}
	\caption{Synthetic Data Generator Flow.}
	\label{fig:4_first_proposal_result_exp_1}
\end{figure}

\begin{figure}[!ht]
	\centering
	\includegraphics[width=1\linewidth]{4_Imbalanced/figures/exp_2.png}
	\caption{Synthetic Data Generator Flow.}
	\label{fig:4_first_proposal_result_exp_2}
\end{figure}

\subsubsection{Results on the Real Application Stream}
Fig. \ref{fig:4_first_proposal_result_exp_3} illustrates the outcomes of employing the three methods on the Sensor data stream using ADWIN as the drift detector. The radar graph depicts metric values ranging from 0.6 to 0.8, which indicate the pronounced drift and imbalance of the Sensor stream. In terms of the precision and recall metrics, the three methods exhibited almost identical values, whereas PA stood out in the other metrics. In contrast, MLSMOTE and MLSOL displayed similar values. By examining the line graph in Fig. \ref{fig:4_first_proposal_result_exp_3}
, it is evident that during the initial 30 chunks, PA's performance of PA might be suboptimal in some instances because of the limited number of classifiers in the pool. However, after the first 30 chunks, the performance significantly improved with the addition of more classifiers. In the same graph, PA consistently achieved the highest performance across all chunks, whereas MLSMOTE exhibited lower performance in certain chunks, and MLSOL exhibited lower performance in the other chunks. In Fig. \ref{fig:4_first_proposal_result_exp_4}, using the same dataset with the DDM as the drift detector, the radar plot metrics indicate lower results compared to the previous experiment. Nonetheless, the line graph underscores PA's dominance of PA in most chunks, although all methods exhibit nearly identical performance across the entire set of chunks. Overall, the performance of all the methods in Fig. \ref{fig:4_first_proposal_result_exp_4} is inferior to that in Fig. \ref{fig:4_first_proposal_result_exp_3} (ADWIN), which highlights ADWIN's superiority of ADWIN over DDM when applied to the Sensor data stream. 


\begin{figure}[!ht]
	\centering
	\includegraphics[width=1\linewidth]{4_Imbalanced/figures/exp_3.png}
	\caption{Synthetic Data Generator Flow.}
	\label{fig:4_first_proposal_result_exp_3}
\end{figure}

\begin{figure}[!ht]
	\centering
	\includegraphics[width=1\linewidth]{4_Imbalanced/figures/exp_4.png}
	\caption{Synthetic Data Generator Flow.}
	\label{fig:4_first_proposal_result_exp_4}
\end{figure}


\subsubsection{Results on the Synthetic Stream}
The results of applying the same methods on the synthetic data stream using ADWIN as the drift detector are presented in Fig. \ref{fig:4_first_proposal_result_exp_5}. The radar diagram indicates metric values ranging from 0.6 to 0.8, suggesting that the synthetic stream is prone to frequent drifts. While MLSOL exhibited the highest precision, MLSOTE exhibited the lowest values. Conversely, PA performed well in other metrics, and MLSMOTE demonstrated the least favorable values. Upon examining the line diagram in Fig. \ref{fig:4_first_proposal_result_exp_5}, it becomes evident that during the initial ten chunks, the PA's performance might be suboptimal because of the limited number of classifiers in the pool. However, after the first ten chunks, the performance improved significantly with the inclusion of more classifiers. In the same diagram, PA consistently achieves the highest performance across all chunks, whereas MLSOL maintains satisfactory performance, and MLSMOTE exhibits lower performance throughout. In Fig. \ref{fig:4_first_proposal_result_exp_6}, using the same synthetic data stream but with the DDM as the drift detector, the radar plot metrics produce similar results to the previous experiment. However, the line diagram emphasizes the same outcomes as those in the previous experiment. Nevertheless, MLSOL and MLSMOTE achieved lower values than the previous experiment. These findings confirmed ADWIN's superiority of ADWIN over DDM when applied to synthetic data streams. These results highlight the versatility and robustness of the algorithm, demonstrating its effectiveness in handling concept drift across diverse datasets, including the challenging synthetic dataset, Covertype dataset, and Sensor dataset, regardless of whether ADWIN or DDM is used as the concept drift detector.

\begin{figure}[!ht]
	\centering
	\includegraphics[width=1\linewidth]{4_Imbalanced/figures/exp_5.png}
	\caption{Synthetic Data Generator Flow.}
	\label{fig:4_first_proposal_result_exp_5}
\end{figure}

\begin{figure}[!ht]
	\centering
	\includegraphics[width=1\linewidth]{4_Imbalanced/figures/exp_6.png}
	\caption{Synthetic Data Generator Flow.}
	\label{fig:4_first_proposal_result_exp_6}
\end{figure}


\subsection{Analysis of Class Overlap Factor between Proposed Approach (PA), MLSMOTE, and MLSOL Techniques.}
In our experimental study, we aimed to identify the critical factors that influence the selection of an optimal approach for minority classes in imbalanced drifted streams. These factors include various aspects, such as dataset characteristics, the choice of concept drift detectors, and the effectiveness of the algorithm in handling class overlap. To compare the overlapping class behavior of our proposed approach (PA) with MLSMOTE and MLSOL, we systematically designed experiments organized into groups of ten chunks. The results were visualized in bar diagrams, contrasting PA with other methods (MLSMOTE and MLSOL). Figures \ref{fig:4_first_proposal_result_exp_7}\ref{fig:4_first_proposal_result_exp_8}\ref{fig:4_first_proposal_result_exp_9} present ten groups, each with three bars representing MLSOTE, MLSOL, and PA, respectively. Each bar visually represents overlapped samples for ten chunks of several data streams, considering distinct concept drift detectors, specifically ADWIN and DDM.
Fig. \ref{fig:4_first_proposal_result_exp_7} shows two diagrams for the ADWIN and DDM detectors. In the ADWIN diagram, the third bar group (chunks 20-30) lacks overlapping samples because this chunk range does not have drifts, and no synthetic samples are generated for the training step. The sixth and tenth groups have the highest overlapped samples because these groups experience many drifts, leading to the generation of several samples and, consequently, the data indicates that MLSMOTE displays the highest number of overlapping samples across all groups, while PA exhibits very few, except for the last group (90-100), which has a small number of overlapping samples. However, the DDM diagram has more overlapping samples than the ADWIN diagram because the DDM detector detects fewer drifts than ADWIN. Specifically, the last value on the Y-axis of the ADWIN diagram is 3,000 samples, while the last value on the Y-axis of the DDM diagram is 2,000 samples. Fig. \ref{fig:4_first_proposal_result_exp_8} and Fig. \ref{fig:4_first_proposal_result_exp_9} display the overlapping samples of the three methods on the Sensor and synthetic data streams, respectively. In Fig. \ref{fig:4_first_proposal_result_exp_8}, the ADWIN diagram demonstrates fewer overlapped samples than in Fig. \ref{fig:4_first_proposal_result_exp_8}, indicating a lower number of drifts in the Sensor stream. The overall diagram indicates that our proposed approach achieves fewer overlapped samples than MLSOL and MLSMOTE, with MLSMOTE having the highest number of overlapped samples. In the DDM diagram of Fig. 10, the number of overlapped samples is lower than in Fig. \ref{fig:4_first_proposal_result_exp_7}, with our proposed approach consistently achieving the lowest number of overlapped samples across most group bars.
In Fig. \ref{fig:4_first_proposal_result_exp_9}, the ADWIN and DDM diagrams exhibit the greatest number of overlapped samples compared to the earlier figures. This suggests that the synthetic stream underwent frequent drifts and was more susceptible to noise. Consequently, the last value on the Y-axis in both the ADWIN and DDM diagrams was recorded for 7,000 samples. Our proposed approach consistently achieves the lowest number of overlapped samples, whereas MLSMOTE consistently attains the highest number of overlapped samples across most group bars in both the ADWIN and DDM diagrams. In conclusion, our thorough examination of overlapping class instances across MLSMOTE, MLSOL, and our proposed approach consistently reveals minimal overlapped samples compared to other methods, regardless of whether DDM or ADWIN is employed as drift detectors

\begin{figure}[!ht]
	\centering
	\includegraphics[width=1\linewidth]{4_Imbalanced/figures/exp_7.png}
	\caption{Synthetic Data Generator Flow.}
	\label{fig:4_first_proposal_result_exp_7}
\end{figure}

\begin{figure}[!ht]
	\centering
	\includegraphics[width=1\linewidth]{4_Imbalanced/figures/exp_8.png}
	\caption{Synthetic Data Generator Flow.}
	\label{fig:4_first_proposal_result_exp_8}
\end{figure}

\begin{figure}[!ht]
	\centering
	\includegraphics[width=1\linewidth]{4_Imbalanced/figures/exp_9.png}
	\caption{Synthetic Data Generator Flow.}
	\label{fig:4_first_proposal_result_exp_9}
\end{figure}

\subsection{Analyzing Runtime Factor Between the Proposed Approach, MLSMOTE, and MLSOL Techniques}
The results of our experiments indicate that the choice of the best algorithm for multiclass imbalanced streams depends on various factors, including dataset characteristics, the concept drift detector used, the presence of an overlapping class problem, and the algorithm's runtime demands. To investigate the runtimes of MLSMOTE, MLSOL, and our proposed approach, we conducted experiments, as shown in Table \ref{tab:4_first_proposal_result_table_2}. Our findings reveal that our proposed approach is highly efficient, regardless of whether ADWIN or DDM is used as the concept drift detector. Specifically, when ADWIN was used, the proposed approach algorithm took 8344 s to train and predict the Covertype stream for ensemble classifiers, whereas it took 8019 s when using the DDM detector. In contrast, MLSMOTE and MLSOL require more time for the same task. Notably, our proposed approach maintains efficiency, even when using the DDM concept detector. Additionally, our proposed approach demonstrates shorter processing times in the Sensor stream and synthetic data compared with other methods. Consistently, the DDM detector achieved less time across all experiments owing to its lower detection of drifts compared to the ADWIN detector. This is because there are fewer instances that trigger training for pool classifiers when using the DDM. The bold highlighting in Table \ref{tab:4_first_proposal_result_table_2} further emphasizes the efficiency of our proposed approach with both the ADWIN and DDM detectors across all dataset streams. Because our proposal generates fewer overlapped samples, leading to an overall decrease in the running time.

\begin{table}[h!]
  \centering
  \resizebox{\textwidth}{!}{
  \begin{tabular}{|l|l|c|c|c|}
  \hline
  \textbf{Stream} & \textbf{Concept Drift Detector} & \textbf{MLSMSOTE} & \textbf{MLSOL} & \textbf{PA} \\ \hline
  \multirow{2}{*}{Benchmark} & ADWIN & 9559 & 8655 & \textbf{8344} \\ \cline{2-5} 
   & DDM & 8388 & 8031 & \textbf{8019} \\ \hline
  \multirow{2}{*}{Real Application} & ADWIN & 1291 & 1310 & \textbf{1102} \\ \cline{2-5} 
   & DDM & 585 & 607 & \textbf{521} \\ \hline
  \multirow{2}{*}{Synthetic} & ADWIN & 12870 & 4866 & \textbf{4834} \\ \cline{2-5} 
   & DDM & 12397 & 4958 & \textbf{4687} \\ \hline
  \end{tabular}
  }
  \caption{Runtime of MLSMOTE, MLSOLE, and Proposed Approach (PA)}
  \label{tab:4_first_proposal_result_table_2}
  \end{table}

\subsection{Analyzing Non-parametric Tests between the Proposed Approach, MLSMOTE, and MLSOL Techniques}
We conducted a thorough series of statistical analyses encompassing 12 comparisons across three diverse datasets, three methods, and two drift detectors. These analyses were rigorously evaluated using a non-parametric test, specifically the Kruskal-Wallis test. The results of this test were striking, revealing substantial variations in the G-mean measurements across most experiments. Importantly, these differences were not due to random chance \cite{yamada2013change}. Upon closer examination of the assessments for the three methods, PA, MLSMOTE, and MLSOTE, as detailed in Table \ref{tab:4_first_proposal_result_table_3}, we found compelling evidence supporting the acceptance of the null hypothesis (H0).
This implies that the expected and observed data exhibited statistically significant disparities. H0 is rejected when significant differences are not observed and H0 is accepted if the P-value falls below the critical value. Underlining the significance of our analysis, the Kruskal-Wallis test was conducted with a 95\% confidence level, and the P-value was rounded to the first three digits after the decimal point. Nevertheless, it is important to note that a similarity in the performances of the methods emerged in the third and fourth experiments, resulting in the rejection of H0. This similarity can be attributed to the fact that the DDM detected fewer drifts in these experiments.


\begin{table}[h!]
  \centering
  \resizebox{\textwidth}{!}{
  \begin{tabular}{|c|c|c|c|c|c|}
  \hline
  \multirow{2}{*}{Dataset} & \multirow{2}{*}{Drift detector} & \multirow{2}{*}{Comparison} & \multirow{2}{*}{P-value} & \multirow{2}{*}{Critical value} & \multirow{2}{*}{H0} \\ 
                           &                                 &                             &                           &                                 &  \\
  \hline
  \multirow{4}{*}{Covertype stream} & \multirow{2}{*}{ADWIN} & PA - MLSSMOTE & 0.001 & 0.05 & Accept \\ \cline{3-6}
                                    &                        & PA - MLSOL    & 0.014 & 0.05 & Accept \\ \cline{2-6}
                                    & \multirow{2}{*}{DDM}   & PA - MLSSMOTE & 0.361 & 0.05 & Rejected \\ \cline{3-6}
                                    &                        & PA - MLSOL    & 0.401 & 0.05 & Rejected \\ 
  \hline
  \multirow{4}{*}{Sensor stream} & \multirow{2}{*}{ADWIN} & PA - MLSSMOTE & 0.001 & 0.05 & Accept \\ \cline{3-6}
                                 &                        & PA - MLSOL    & 0.001 & 0.05 & Accept \\ \cline{2-6}
                                 & \multirow{2}{*}{DDM}   & PA - MLSSMOTE & 0.001 & 0.05 & Accept \\ \cline{3-6}
                                 &                        & PA - MLSOL    & 0.001 & 0.05 & Accept \\ 
  \hline
  \multirow{4}{*}{Synthetic stream} & \multirow{2}{*}{ADWIN} & PA - MLSSMOTE & 0.001 & 0.05 & Accept \\ \cline{3-6}
                                    &                        & PA - MLSOL    & 0.001 & 0.05 & Accept \\ \cline{2-6}
                                    & \multirow{2}{*}{DDM}   & PA - MLSSMOTE & 0.001 & 0.05 & Accept \\ \cline{3-6}
                                    &                        & PA - MLSOL    & 0.001 & 0.05 & Accept \\ 
  \hline
  \end{tabular}
  }
  \caption{Kruskal-Wallis test between MLSMOTE, MLSOLE, and Proposed Approach (PA)}
  \label{tab:4_first_proposal_result_table_3}
  \end{table}
% %%%%%%%%%%%%%%%%%%%%%%%%%%%%%%%%%%%%%%%%%%%%%%%
%
%   Conclusions
%
%%%%%%%%%%%%%%%%%%%%%%%%%%%%%%%%%%%%%%%%%%%%%%%

%%%%%%%%%%%%%%%%%%%%%%%% Here it is needed to re-frame the conclusions

\section{Conclusion and Future Works}
\label{sec:4_8_Conclusions}

Our study in this chapter presents a comprehensive methodology designed to facilitate incremental learning in drifted streams, with a particular focus on addressing the challenges associated with imbalanced data streams including minority and overlapping classes. The proposed methodology integrates our proposed oversampling method, concept drift detection strategies, and Dynamic Ensemble Selection (DES) to select the most suitable ensemble classifier. Extensive experimentation across various datasets, including benchmark datasets, real application streams, and synthetic data, validated the effectiveness of this contribution. The critical role of the concept drift detector in our approach lies in its capacity to promptly detect concept drifts, allowing our methodology to adapt by training new base classifiers to maintain relevance and performance in real-time scenarios. Oversampling techniques were utilized to mitigate the minority class problem, and the KNN algorithm was employed to prevent the generation of overlapped class instances. The DES technique was utilized to intelligently select the best base classifiers to ensure optimal performance. Our evaluation approach employs various performance measures, showcasing its proficiency in addressing multiclass imbalanced stream problems, particularly its exceptional performance in classifying data streams with evolving class distributions. In addition to its performance and performance, our approach has several advantages, including adaptability, efficiency, and scalability. Dynamic model updates driven by incoming data instances enable continuous adaptation to changing data distributions, thereby ensuring reliability and relevance in real-time scenarios. However, it is essential to recognize the specific limitations of the proposed approach. One limitation is the extended time required to generate nonoverlapping synthetic instances. Moreover, the efficacy of our contribution depends on the performance of the MLSMOTE and MLSOL techniques. Consequently, future research efforts should be directed toward enhancing our contributions. Potential avenues for improvement include the development of advanced oversampling techniques to avoid generating overlapping synthetic instances. Additionally, meta-learning methods have been explored to calculate imbalanced multiclass ratios and minority classes.


%emerging
% this file is called up by thesis.tex
% content in this file will be fed into the main document
 
% this file is called up by thesis.tex
% content in this file will be fed into the main document

%: ----------------------- introduction file header -----------------------


\begin{savequote}[50mm]
  You cannot teach a man anything; you can only help him discover it in himself.
  \qauthor{Galileo}
  % The beginning is the most important part of the work.
  % \qauthor{Plato}
  \end{savequote}
  
  
  \chapter{Dynamic Classification Ensembles for Handling Imbalanced
  Multiclass Drifted Data Streams}
  \label{chapter:4_Imbalanced_Multiclass}
  
  In various real-world applications, data streams have introduced new challenges to learning algorithms. Data streams are continuous with high volumes of data arriving rapidly and dynamically. This data deluge poses unprecedented challenges for learning algorithms because they must adapt to the dynamic nature of the data environment [1], [2], [3]. Among the various research areas in machine learning, incremental learning under data Streams with Emerging New Classes (SENC) has garnered considerable attention because of its practical relevance and unique challenges [4], [5], [6]. SENC refers to a scenario in which new classes that were not present during the initial training of a learning model emerged in the data stream. This poses a significant challenge for traditional learning approaches that are typically designed to handle fixed or predefined class distributions. The ability to effectively recognize and adapt to these novel classes in real-time is crucial for maintaining accurate and up-to-date models. Furthermore, the inherent limitations of data streams, such as limited memory and storage constraints, impose additional complexities in the learning process. Learning algorithms must operate efficiently within these resource constraints to ensure real-time processing and to avoid overwhelming computational overhead.
  To address the challenges presented by data streams containing emerging new classes, dynamic ensemble selection (DES) [7], [8], [9]. Dynamic ensemble selection involves utilizing multiple classifiers in machine learning to make collective predictions or classifications of data. Dynamic ensemble selection is distinguished by its ability to dynamically adapt the ensemble based on the characteristics of the data. Instead of relying on a fixed ensemble of classifiers, dynamic ensemble selection continuously assesses the performance of individual classifiers and selects the subset that demonstrates the highest competence for the current data. This adaptability enables the ensemble to enhance its performance over time by incorporating the most suitable classifiers for prevailing data conditions. Furthermore, if a classifier becomes ineffective owing to concept drift or the emergence of new classes, it can be excluded from the ensemble to prevent it from negatively affecting the overall performance.
   Adaptive Windowing (ADWIN) is another widely employed method to address concept drift. Concept drift refers to the phenomenon in which the statistical properties of data change over time, leading to a decline in the learning algorithm performance [10], [11], [12]. ADWIN continuously monitors incoming data and detects changes or drifts in data distribution. It achieves this by maintaining sliding windows of variable sizes and monitoring statistical measures such as the mean or variance within these windows. Upon detecting a significant change or drift, the ADWIN triggers an update in the ensemble or classifier configuration. This update may involve retraining the classifiers with new data or incorporating new classifiers that are better suited to the updated data distribution. By adapting the ensemble to changing data conditions, ADWIN ensures that the classifier system remains accurate and up-to-date even in the presence of concept drift or the emergence of new classes. The primary objective of utilizing these approaches, such as dynamic ensemble selection and ADWIN, is to establish a flexible and effective classification system that is capable of handling data streams containing emerging new classes. By dynamically adjusting the ensemble based on the data characteristics and detecting and adapting to concept drift, these approaches enable the system to maintain accurate predictions over time. Adaptability and responsiveness are crucial for addressing the unique challenges posed by the dynamic nature of data streams and the emergence of new classes within them.  
  This research proposes efficient algorithms for classifying data streams in real-time scenarios, focusing on addressing the challenges posed by emerging new classes in data distributions. 
  The subsequent sections of this paper adhere to a well-structured organization. Section 2 presents a comprehensive review of the relevant literature encompassing concept drift, dynamic classifier ensembles, and emergency class detection methods. Section 3 introduces the proposed framework, providing intricate explanations of its constituents, including dynamic classifier ensembles, concept drift handling, and emergency class identification, and the adaptive proposed method of the emerging pool size. Section 4 outlines the experimental setup and presents the results, providing details regarding the employed datasets, evaluation metrics, and procedures. Finally, in Section 5, we offer concluding remarks, summarizing the key findings, discussing their implications, and proposing potential avenues for future research.
  
  
  The remainder of this chapter is organized as follows: In Section \ref{sec:5_2_motivation}, we present the motivations and the contributions. The proposed framework and combination via SI algorithms are discussed in detail in Section \ref{sec:5_4_proposed}. The  experimental results and the discussion are presented in Sections \ref{sec:5_5_Expsetup} and \ref{sec:5_7_Discussion}, respectively. Finally, the conclusions of this study and future research are discussed in Section \ref{sec:5_8_Conclusions}. 
  
  
  \section{Motivations and Contributions} \label{sec:5_2_motivation}
  We aim to develop novel algorithms that can effectively handle the emergence of new classes issue in nonstationary data streams, achieve high classification accuracy, and minimize computational complexity. The key contributions of this study are outlined as follows:
   
  \begin{enumerate}[nosep]
    \item Our first contribution involves utilizing the ADWIN, DES, and K-means techniques in combination with the ensemble stratified bagging technique to detect and adapt to emerging new classes. This approach allows us to dynamically update the classification model to accommodate an evolving data environment.
   \item The second contribution is the introduction of an adaptive method to adapt the emerging pool size depend on the stream distribution.
    \end{enumerate} 
   
     
  
  
 
   



\section{Proposed Methodology}\label{sec:5_first_proposed_approach}

In this section, we present the primary phases of our study, which consist of three distinct steps designed to address the challenges of handling emerging new classes in drifted streams. Our proposed approach aims to develop a robust framework capable of tackling these complex challenges through a comprehensive, three-step methodology.
\begin{itemize}
	\item \textbf{Emerging New Classes:} Our study addresses the widespread issue of emerging new classes in drifted streams using well-known techniques, including concept drift detection and K-means clustering.
	\item \textbf{Drifted Streams: } To handle changes in the data distribution, our approach integrates a concept drift detector that dynamically identifies shifts, allowing the model to promptly adapt its classifiers to maintain effectiveness.
	\item \textbf{Classifier Performance:} The final phase identifies unknown classes in drifted streams, creating new classifiers for emerging classes to improve the model's ability to classify new instances accurately.
\end{itemize}

\subsection{THE PROPOSED APPROACH Flow}

Our proposed approach is designed with three distinct phases that work together to improve its performance in managing multiclass imbalanced and drifting data streams. 
\begin{itemize}
	\item \textbf{DES Phase (dynamic ensemble selection phase):} The first phase, known as the dynamic ensemble selection (DES) phase, is responsible for selecting the most appropriate classifier for the incoming data. This ensures that the selected classifier is well-suited for the current data chunk.
	\item \textbf{Drift detector phase:} The second phase of our approach is the drift detector phase, which operates in real-time to continuously monitor the data stream. Its primary function is to identify any signs of concept drift, which indicates shifts in the underlying data distribution over time.
	\item \textbf{Emerging New Class Identifier Phase:} The final phase of our approach is the synthetic data generator phase, which is dedicated to generating synthetic data for the minority classes. This step is crucial for addressing class imbalance by producing additional samples for underrepresented classes, thereby significantly enhancing the model's ability to accurately classify instances from minority classes.
\end{itemize}

As depicted in Fig. \ref{fig:5_first_proposal_step_1}, the DES phase retrieves the current data chunk and applies the DES technique to select the best classifiers for the chunk (black box of Fig.  \ref{fig:5_first_proposal_step_1}). These classifiers are used in the second phase to predict class labels, while drift detectors like ADWIN or DDM monitor for concept drift. If a drift is detected (highlighted by the red rectangle), the chunk is forwarded to the third phase, where new classes are identified (see Fig.  \ref{fig:5_first_proposal_step_2}). The overall approach is implemented in Algorithm  \ref{alg:5_first_proposal_1}, which takes a data stream and a DES Pool threshold as inputs. Key components include training the initial ensemble (Lines 4-6), using DES to select the best classifier for the current chunk (Line 8), detecting drifted chunks (Line 10), creating new classifiers for emerging classes (Line 12), and removing the worst classifier if the DES Pool threshold is exceeded (Lines 14-16).


\subsection{Emerging New Classes Phase Details}

In this section we present the Emerging new classes phase details. As shown in Fig.2, The emerging phase contains two primary steps: 
\begin{itemize}
	\item aaa 
	\item ee
	The Emerging New Classes phase is implemented in Algorithm  \ref{alg:5_first_proposal_2}, which utilizes drifted instances and current changes as inputs. Key steps include applying K-means (Line 1), calculating cosine similarity between centroids (Line 2), determining the maximum centroid distance (Line 3), setting the adaptive pool threshold (Lines 4-7), using KNN to find the nearest neighbor (Line 9), storing instances in the emerging pool if the distance exceeds the threshold (Lines 11-12), and creating a new classifier if the pool size surpasses the adaptive limit (Lines 14-17). A simulated scenario is illustrated in Figures 3 and 4. In Fig. \ref{fig:5_scenario1}(a), the drifted chunk is divided into three clusters using the K-means algorithm. Fig. \ref{fig:5_scenario1}(b) shows the calculation of the maximum distance between cluster centroids, which serves as a threshold for identifying new classes. The procedure for classifying instances is as follows: when a drifted instance is detected, it is considered a new class if the distance between the instance and its nearest centroid exceeds the maximum inter-centroid distance (Fig. \ref{fig:5_scenario2} a). If the distance is within the threshold, the instance is classified as a known class (Fig. \ref{fig:5_scenario2} b).
\end{itemize}

\begin{figure}[!ht]
	\centering
	\includegraphics[width=1\linewidth]{5_Emerging/figures/algorithm1.png}
	\caption{Proposed Approch Flow}
	\label{fig:5_first_proposal_step_1}
\end{figure}
\begin{figure}[!ht]
	\centering
	\includegraphics[width=1\linewidth]{5_Emerging/figures/algorithm2.png}
	\caption{Emerging Phase Flow}
	\label{fig:5_first_proposal_step_2}
\end{figure}
\begin{figure}[!ht]
	\centering
	\includegraphics[width=1\linewidth]{5_Emerging/figures/scenario1.png}
	\caption{Proposed Approch Flow}
	\label{fig:5_scenario1}
\end{figure}
\begin{figure}[!ht]
	\centering
	\includegraphics[width=1\linewidth]{5_Emerging/figures/scenario2.png}
	\caption{Emerging Phase Flow}
	\label{fig:5_scenario2}
\end{figure}

\begin{equation}
	\label{eq:5_first_proposal_1}
    d_c = \arg\max_d \sum_{i=1}^{i} \sum_{j=i+1}^{i} d = ED(c_i, c_j)
\end{equation}

\begin{equation}
	\label{eq:5_first_proposal_2}
	d_x = \arg\min_i \sum_{i=1}^{i} ED(x, c_i)
\end{equation}

\begin{equation}
	\label{eq:5_first_proposal_3}
    \begin{cases}
		EC & \text{if } d_x > d_c \\
		P_{DES} & \text{otherwise}
	\end{cases}
\end{equation}

\begin{algorithm}[H]
	\caption{Proposed Framework Algorithm for Imbalanced Multi-Class Drifted Data Streams}
	\KwIn{data stream, maximum classifiers pool size $\kappa$}
	% \Parameter{current chunk $a$, synthetic data $b$, classifiers pool $\Psi$, drifted pool $\psi$, classes frequency $\Omega$, best frequency $\omega$, minority classes $\mu$}
	\KwOut{Prediction $P$}
	\BlankLine
	$\psi, \Psi, \Omega, \mu \gets \emptyset$\;
	$\omega \gets 0$\;
	\For{stream have chunk}{
		\eIf{$a$ is the First chunk}{
			$k \gets$ \texttt{trainingNewClassifier}($a$)\;
			$P \gets$ \texttt{getPrediction}($a, k$)\;
		}{
			$k \gets$ \texttt{DES}($a, \Psi$)\;
			$P \gets$ \texttt{getPrediction}($a, k$)\;
			$\psi \gets$ \texttt{conceptDriftDetector}($P$)\;
			\If{$\psi > 0$}{
				$\Omega \gets$ get classes frequency according to Eq.1\;
				$\omega \gets$ best frequency according to Eq.2\;
				$\mu \gets$ get minority classes according to Eq.3\;
				$b \gets$ utilize $a$ and $\mu$ to get the synthetic data according to Algorithm 2\;
				trainingData $\gets a + b$\;
				$k \gets$ \texttt{trainingNewClassifier}(trainingData)\;
				$\Psi \gets \Psi + k$\;
				\If{$\Psi > \kappa$}{
					\texttt{removeWorstClassifier}($\Omega$)\;
				}
			}
			$P \gets$ \texttt{getPrediction}($a, k$)\;
		}
	}
	\Return{$P$}
	\end{algorithm}
	
	\vspace{1cm}
	
	\begin{algorithm}[H]
		\caption{Synthetic data generator}
		\label{alg:4_first_proposal_2}
		\KwIn{Minority classes $\mu$, current chunk $a$, sample size $\eta$, historical chunks $h$}
		\KwOut{Generated data $b$}
		$b \gets \emptyset$\;
		$f \gets \text{MLSMSOTE}$\;
		$knn \gets \text{kNearestNeighbor}(a)$\;
		$chunk \gets \text{similarChunk}(a, h)$\;
		$f \gets \text{similarChunkOverSamplingMethod}(chunk)$\;
		\If{$f = \text{MLSMSOTE}$}{
			$f \gets \text{MLSOL}$\;
		}
		\Else{
			$f \gets \text{MLSMSOTE}$\;
		}
		\While{$|b| < \eta$}{
			$p \gets \text{generateSyntheticPoint}(\mu, f)$\;
			$similarPointsClass \gets \text{KNN.getKneighbor}(b)$\;
			\If{$similarPointsClass = \mu$}{
				$b \gets b \cup \{p\}$\;
			}
		}
		\Return $b$\;
		\end{algorithm}
\section{Experimental Results}
\label{sec:results}
The experiments conducted in this study were designed to assess the effect of concept drift on the performance of the proposed framework in detecting emerging new classes. The primary goal was to identify the machine learning algorithm in conjunction with DES, Dynamic Ensemble Selection technique, which improves the performance of classification when faced with the emergence of new classes. By conducting these experiments, valuable insights were gained, leading to potential enhancements in the effectiveness of the proposed approach in managing imbalanced data streams and its overall performance. These experiments provide valuable insights into the capabilities of the framework and shed light on the optimal approach for handling emerging new classes in the presence of concept drift. The results provide valuable guidance for selecting the most suitable classification machine learning algorithm and DES configuration aims to enhance both classification accuracy and robustness. The experiments detailed in this study offer crucial insights into improving the proposed approach's performance and tackling challenges related to emerging new classes and concept drifts in incremental drifted streams. These insights are instrumental in advancing stream mining techniques and developing more precise and resilient classification models for dynamic and evolving data-stream environments. The two main questions to be answered are:

\begin{itemize}
  \setlength{\itemindent}{-.5in}
  
  \item $\pmb{Q_1}$.How does the emergence of new classes in data streams affect the stability and performance of ML models?
  \item $\pmb{Q_2}$. how can ML models be adapted to accommodate such changes?
  \item $\pmb{Q_3}$. how to employee concept drift to solve the emerging new classes problem? 
  \end{itemize}

\subsection{Experimental Setup}
\label{sec:setup}
The evaluation of the proposed approach involves a comparison with SENCForest \cite{mu2017classification}, SENNE \cite{yang2021concept}, and KENNES \cite{zhang2022knnens}. The evaluation utilizes several metrics, including recall, precision, F1 score \cite{sasaki2007truth}, BAC \cite{brodersen2010balanced}, and G-mean \cite{kubat1997addressing}. The procedure for experimentation followed a test-then-train technique \cite{krawczyk2017ensemble}, where the classifier is first trained on a specific chunk and then evaluated on the subsequent chunk. Each chunk was standardized to 2000 instances. Four different classification models served as base estimators: Gaussian Naive Bayes (GNB) technique, K-Nearest Neighbors (KNN), the Support Vector Classifier (SVC), and the Hoeffding Tree (HT), all features are implemented using scikit-learn \cite{ksieniewicz2022stream}. We establish an ensemble classifier pool with a set limit of L = 8, wherein each ensemble consists of N = 4 base models. While these constraints remained fixed across all our experiments, the threshold for the pool classifier in each approach was maintained at eight. Consequently, if the threshold is surpassed, the least-performing classifier is systematically eliminated. ADWIN \cite{adams2023explainable} and DDM \cite{gama2004learning} are employed as concept drift detectors as implemented in the River library \footnote{\url{https://riverml.xyz/0.21.0/introduction/basic-concepts/}} , to identify concept drifts, as they are considered more accurate techniques for use in incremental data streams \cite{gama2004learning}\cite{adams2023explainable}\cite{madkour2023historical}\cite{baena2006early}. This configuration was applied consistently across all approaches to ensure fair engagement. The experiments were conducted using Python, and the source code is available publicly on GitHub \footnote{\url{https://github.com/Amadkour/transfer_learning_with_concept_drift.git}}.
\subsection{Data Streams}
\label{sec:data_stream}
In this section, the performance of the proposed approach (PA), was evaluated using several datasets, including synthetic data streams, a real-world application stream, and benchmark datasets. To conduct the evaluations, the stream-learn and River python libraries \cite{ksieniewicz2022stream} were used. As detailed in Table \ref{table:table_1}, the Covertype dataset stream is used as the benchmark dataset in the study. This dataset comprises 52 features, 7 classes, and a total of 581,010 instances. It serves as a standard benchmark dataset widely used in stream mining research. For the evaluation of real application streams, the Sensor stream dataset was utilized. This dataset includes 5 features, 58 classes, and 392,600 instances in total, reflecting a real-world application scenario. Synthetic dataset stream was generated using the scikit-learn Python package \cite{ksieniewicz2022stream}. This synthetic stream was generated to mimic data streams and assess the performance of the proposed method. It included 10 features and four classes, divided into 200 chunks, each with a size of 2,000. The effectiveness of the proposed method was rigorously assessed using these datasets along with a stream-learn library. This thorough evaluation offered important insights into how well the approach manages various types of data streams, encompassing benchmark datasets, synthetic data streams, and real-world application streams.

\begin{table}[t]
	\centering
  \resizebox{\textwidth}{!}{
	\begin{tabular}{|l|c|c|c|c|}
	\hline
	\textbf{Data stream}     & \textbf{Features} & \textbf{Classes} & \textbf{Instances} & \textbf{Chunk Size} \\ \hline
	Sensor stream \footnote{\url{https://www.cse.fau.edu/~xqzhu/Stream/sensor.arff}}    & 5               & 58            & 392600            & 2000                \\ \hline
	Covertype stream\footnote{\url{http://archive.ics.uci.edu/dataset/31/covertype}.} & 52              & 7             & 581010            & 2000                \\ \hline
	Synthetic stream      & 10              & 4             & 200000            & 2000                \\ \hline
	\end{tabular}
  }
  \caption{The datasets utilized in the experimentation exhibited various characteristics.}
	\label{table:table_1}
	\end{table}

\subsection{Compared Approaches}
\label{sec:compared_approaches}
In the following section, we present a comparative analysis between our proposed GNB methodology and three established benchmark techniques, detailed as follows:
\begin{enumerate}
	\item \textbf{SENCForest \cite{mu2017classification}:} The SENCForest method employs anomaly detection techniques to identify new classes and is founded on entirely random trees. It constructs isolation trees for both classification and detection by subsampling from each class, utilizing 20 random trees and a subsample size of 20. Although it can operate with incomplete or no label information, it frequently suffers from elevated false positive rates and suboptimal runtime efficiency in practical applications.
	\item \textbf{SENNE \cite{zhu2020semi}:} Utilizing a hypersphere ensemble mechanism, SENNE calculates scores to identify both new and known classes. It operates with a buffer capacity of 100 and sets a new class-score threshold of 0.5.
	\item \textbf{KENNES \cite{zhang2022knnens}:} The KNNENS method addresses the dual challenge of detecting new classes and classifying known classes within a unified framework. It was configured with a buffer size of 100 and a new class score threshold of 0.5.
\end{enumerate}



\subsection{Examination of Experimental Findings}
\label{sec:finding}
This section offers an in-depth analysis of the proposed approach's (PA) performance across various data streams. To ensure a thorough evaluation, five performance metrics— recall, F1 score, precision, recall, G-mean, and BAG—are displayed through two types of visual diagrams: radar and line charts. The radar chart effectively summarizes the performance of each algorithm by highlighting their performance across the six key metrics. Additionally, to evaluate the overall performance of each method—including PA, SENCForest, SENNE, and KENNES—the average value for each metric was computed. Additionally, a line diagram using the G-mean metric was used to compare these methods for each chunk across 200 chunks. A radar diagram provided a detailed and nuanced evaluation of the proposed approach’s performance across various experimental scenarios.

\subsubsection{Results On The Benchmark Dataset}
\label{sec:covertype}
This experiment aimed to evaluate the performance of the proposed framework on the Covertype data stream using different buffer sizes with ADWIN as the concept drift detector. The findings are visually represented using scatter plots in Fig. \ref{fig:res1}, which contains four subplots for emerging buffer sizes of 5, 10, 20, and adaptive sizes, respectively. In Fig. \ref{fig:res1}(b), the G-mean accuracy of SENCForest, SENNE, KENNES, and PA is shown across 200 chunks, specifically focusing on a buffer size of 5 instances for emerging new classes. At first, all methods exhibit less-than-ideal accuracy in the initial 20 chunks. However, a marked improvement is observed after this period, as the classifier pool grows. This growth bolsters the DES, dynamic ensemble selection, technique, allowing it to choose the most suitable classifiers for subsequent chunks. Consequently, there are significant gains in accuracy from chunk 20 onward, PA consistently achieved the highest accuracy, while KENNES and SENNE recorded the lowest. Additional experiments with buffer sizes of 10 and 20, shown in Fig. \ref{fig:res1}(b) and Fig. \ref{fig:res1}(c), revealed slightly lower accuracy compared to the buffer size of 5, likely due to fewer updates when using larger buffer sizes. Finally, the adaptive buffer size experiment in Fig. \ref{fig:res1}(d) highlights the PA algorithm's superior performance across all chunks, with the adaptive emerging pool size delivering the best results.

\begin{figure}[!ht]
	\centering
	\includegraphics[width=1\linewidth]{5_Emerging/images/res1.png}
	\caption{Covertype stream for various emerging buffer sizes: (a) buffer size of 5, (b) buffer size of 10, (c) buffer size of 20, and (d) our adaptive buffer size.}
	\label{fig:res1}
\end{figure}				

\subsubsection{Results On Real Application Stream}
\label{sec:sensor}
This experiment focused on assessing the accuracy of SENCForest, SENNE, KENNES, and PA across 200 chunks of a sensor data stream, utilizing varying buffer sizes similar to the previous experiment. The results, shown in Fig. \ref{fig:res2}, were visualized using scatter plots. Fig. \ref{fig:res2} (a) displays the classification accuracy for each algorithm with a buffer size of 5. The PA algorithm achieved the highest performance overall, except for chunks 43 to 48, where some instances were incorrectly assumed as outliers rather than emergent. In contrast, SENCForest and SENNE had the lowest accuracy. Additional experiments with buffer sizes of 10 and 20 (Fig. \ref{fig:res2}(b) and Fig. \ref{fig:res2}(c)) demonstrated reduced accuracy compared to the buffer size of 5, attributed to fewer updates with larger buffer sizes. The adaptive buffer size experiment in Fig. \ref{fig:res2}(d) reaffirmed the effectiveness of the PA algorithm, which outperformed others across all chunks, with the adaptive emerging pool size providing optimal results.

\begin{figure}[!ht]
	\centering
	\includegraphics[width=1\linewidth]{5_Emerging/images/res2.png}
	\caption{Sensor stream for various emerging buffer sizes: (a) buffer size of 5, (b) buffer size of 10, (c) buffer size of 20, and (d) our adaptive buffer size.}
	\label{fig:res2}
\end{figure}				

\subsubsection{Results On Synthetic Stream}
\label{sec:synthetic}
This experiment aimed to assess the performance of the previously tested methods on a synthetic data stream. Results, depicted in Fig. \ref{fig:res3}, were analyzed using radar and scatter plots. Fig. \ref{fig:res3}(a) shows the performance of various algorithms using a buffer size of 5, with the radar plot highlighting PA's superior performance. Scatter plots illustrate that PA consistently delivered the highest accuracy when applied to the synthetic stream with buffer sizes of 5, 10, and 20, as shown in Fig. \ref{fig:res3}(a), Fig. \ref{fig:res3}(b), and Fig. \ref{fig:res3}(c), respectively. Fig. \ref{fig:res3}(d) demonstrates that the adaptive pool size yielded the best accuracy on the synthetic stream, underscoring the adaptability and efficiency of the PA algorithm.

\begin{figure}[!ht]
	\centering
	\includegraphics[width=1\linewidth]{5_Emerging/images/res3.png}
	\caption{Synthetic stream for various emerging buffer sizes: (a) buffer size of 5, (b) buffer size of 10, (c) buffer size of 20, and (d) our adaptive buffer size.}
	\label{fig:res3}
\end{figure}				

\subsection{Runtime Analysis Of The Best Algorithms}
\label{sec:running}
Our experimental results revealed that selecting the optimal algorithm for detecting emerging new classes is influenced by various factors, including dataset characteristics, buffer length, and the algorithm's runtime requirements. We conducted a series of experiments to evaluate the runtime performance of different algorithms and observed that the PA algorithm demonstrated outstanding efficiency. Specifically, with the adaptive pool size approach, the PA algorithm trained bagging classifiers 150 times within 2131 seconds when using the Covertype stream, outperforming others despite SENCForest achieving the lowest runtime but with fewer updates compared to PA as shown in Table \ref{table:table_2}. On the Sensor stream, PA delivered the best runtime for 66 updates, while SENCForest, KENNES, and SENNE required more time to complete the same or fewer iterations. Similarly, PA exhibited superior runtime performance on the Synthetic stream. These results underscore the PA algorithm’s efficiency, making it an ideal choice for time-constrained scenarios. The superior runtime performance of PA is further highlighted in bold in Table \ref{table:table_2}.
	
\begin{table}[t]
	\centering
  \resizebox{\textwidth}{!}{
	\begin{tabular}{|l|l|c|c|}
	\hline
	\textbf{Stream}    & \textbf{Algorithm} & \textbf{Training Times} & \textbf{Time (seconds)} \\ \hline
	Covertype          & SENCForest         & 142                     & \textbf{1997}           \\ \cline{2-4} 
					   & SENNE              & 5                       & 2167                    \\ \cline{2-4} 
					   & KENNES             & 3                       & 1742                    \\ \cline{2-4} 
					   & PA                 & 150                     & 2131                    \\ \hline
	Sensor             & SENCForest         & 23                      & 1244                    \\ \cline{2-4} 
					   & SENNE              & 17                      & 3825                    \\ \cline{2-4} 
					   & KENNES             & 152                     & 2109                    \\ \cline{2-4} 
					   & PA                 & 66                      & \textbf{1075}           \\ \hline
	Synthetic          & SENCForest         & 12                      & 107                     \\ \cline{2-4} 
					   & SENNE              & 26                      & 224                     \\ \cline{2-4} 
					   & KENNES             & 149                     & 202                     \\ \cline{2-4} 
					   & PA                 & 47                      & \textbf{91}             \\ \hline
	\end{tabular}
  }
  \caption{Running time of SENCForest, SENNE, KENNES, and PA.}
	\label{table:table_2}
	\end{table}

\subsection{Comparison between GNB, KNN, and HT, SVC}
\label{sec:compared_base_calssfier}
In this section, we compare various algorithms (KNN, SVC, GNB, and HT) as base classifiers for the bagging technique. The comparison, illustrated in Fig.\ref{fig:res4}, focuses on the Covertype dataset stream, where GNB (blue line) and HT (red line) consistently achieve the highest scores across most chunks. Both classifiers also excel in radar plots across key performance metrics, including accuracy, precision, recall, and F1-score. Based on these results, GNB and HT emerge as the most suitable base classifiers for the bagging technique. We further compared GNB and HT in terms of runtime and update frequency, as shown in Table \ref{table:table_3}. The results indicate that GNB has the best runtime performance, while HT demonstrates superior accuracy, likely due to its higher number of updates compared to GNB, as reflected in Table \ref{table:table_3} and Fig. \ref{fig:res4}.

\begin{figure}[t]
	\centering
	\includegraphics[width=1\linewidth]{5_Emerging/images/res4.png}
	\caption{Covertype stream for concept drift detector (ADWIN, DDM) via several metrics (recall, precision, balanced accuracy, -mean, and $f1_score$)}
	\label{fig:res4}
\end{figure}				

	
\begin{table}[t]
	\centering
	\begin{tabular}{|l|c|c|c|c|}
	\hline
	\textbf{Algorithm}     & \textbf{KNN} & \textbf{SVC} & \textbf{GNB} & \textbf{HT} \\ \hline
		Training Times         & 140          & 150          & \textbf{139} & 148         \\ \hline
		Time (seconds)         & 586          & 878          & \textbf{291} & 1169        \\ \hline
	\end{tabular}
  \caption{Running time of KNN, SVC, GNB, and HT as a base classifier.}
	\label{table:table_3}
	\end{table}


\subsection{Comparison between ADWIN and DDM}
\label{sec:compared_drift_detector}
In this section, we compare two concept drift detection methods: the Drift Detection Method (DDM) \cite{gama2004learning} and the Adaptive Window (ADWIN) \cite{gama2004learning} \cite{adams2023explainable}. These methods are widely recognized for their excellent performance in handling incremental drifted streams \cite{gama2004learning}\cite{adams2023explainable}\cite{madkour2023historical}\cite{baena2006early}. As shown in Fig. \ref{fig:res5}, using the Covertype stream with GNB as the base classifier, ADWIN consistently outperforms DDM in both radar and line plots across all performance metrics, including accuracy. Furthermore, when comparing their runtime and update frequency, as presented in Table \ref{table:table_4}, ADWIN proves to be the superior detector for drifted streams. It identifies the highest number of drifts (139) in the least amount of time (272 seconds), demonstrating its efficiency and effectiveness.
\begin{figure}[!ht]
	\centering
	\includegraphics[width=1\linewidth]{5_Emerging/images/res4.png}
	\caption{Covertype stream for concept drift detector (ADWIN, DDM) via several metrics (recall, precision, balanced accuracy, mean, and $f1_score$)}
	\label{fig:res5}
\end{figure}
	
\begin{table}[!ht]
	\centering
	\begin{tabular}{|l|c|c|}
		\hline
	\textbf{Algorithm}     & \textbf{ADWIN} & \textbf{DDM}  \\ \hline
Training Times         & 139          & 55                   \\ \hline
Time (seconds)         & 272          & 545                   \\ \hline
	
	\end{tabular}

  \caption{RUNNING TIME ADWIN and DDM as a drift detector.}
	\label{table:table_4}
	\end{table}
% Conclusion
\section{Conclusion And Future Directions}
\label{sec:5_8_Conclusions}
Our research introduces an extensive framework designed for incremental learning within data streams with emerging new classes. By integrating concept drift detection using the ADWIN algorithm, the K-means technique, Gaussian Naive Bayes (GNB) as the base classifier, and ensemble stratified bagging, our approach effectively addresses the challenges associated with this problem. Comprehensive testing on real-world application streams, benchmark datasets, and synthetic data demonstrates the effectiveness of our approach. The ADWIN algorithm is crucial in our approach, enabling timely detection of concept drifts and new classes, making it highly recommended for incremental streams due to its superior performance and lower runtime. This feature enables the proposed method to develop new base classifiers, maintaining their relevance and precision in real-time situations. The use of ensemble stratified bagging further enhances predictive performance and robustness by combining the outputs of various base classifiers. Dynamic Ensemble Selection (DES) is employed to select the best classifiers, ensuring optimal performance. Evaluated through various performance metrics, our approach effectively addresses SENC challenges, demonstrating exceptional performance in the precise classification of changeable streams with changing class distributions, especially when employing GNB as the base classifier. Beyond its notable performance, our approach provides key benefits such as adaptability, efficiency, and scalability. By updating the model dynamically in response to incoming data instances, our approach can continuously adjust to evolving data distributions, thereby maintaining its reliability and relevance in real-time situations. The integration of concept drift detection and ensemble stratified bagging techniques strengthens the framework’s ability to handle dynamic data environments and emerging new classes. Future work can explore further enhancements, such as investigating advanced concept drift detection algorithms, exploring alternative ensemble strategies, and incorporating deep learning techniques. Additionally, our framework can be expanded to address a broader range of real-world applications.


%transfer
% this file is called up by thesis.tex
% content in this file will be fed into the main document

% 
% this file is called up by thesis.tex
% content in this file will be fed into the main document

%: ----------------------- introduction file header -----------------------


\begin{savequote}[50mm]
Success is the sum of small efforts, repeated day-in and day-out.
\qauthor{Robert Collier}
\end{savequote}

\chapter{Introduction}
\label{cha:1_Introduction}

% the code below specifies where the figures are stored
\ifpdf
    \graphicspath{{1_introduction/figures/PNG/}{1_introduction/figures/PDF/}{1_introduction/figures/}}
\else
    \graphicspath{{1_introduction/figures/EPS/}{1_introduction/figures/}}
\fi


%-------------------------------------------------------------------------
%Chapter 1 contents:
%- Motivation of the research field: Context-aware systems -> LBS -> GNSS limitation -> Positioning techniques -> DR -> inertial PDR -> inertial PDR + wearables
%- Problem identification: smartphone not a wearable -> potentiality of wrist-worn wearables -> Problem: no wrist-worn PDRS
%- Goal of the thesis: tackle the problem -> how? Splitting it into sub-problems
%- Structure of the thesis
%-------------------------------------------------------------------------

Governments and companies are producing vast streams of data and require effective data analytics and machine learning methods to assist in making predictions and decisions promptly. One crucial aspect is the machine learning pipeline, which involves training a prepared dataset to construct a model and subsequently utilizing this model to predict new instance outputs. As depicted in Fig. (1.1), the process entails fetching historical data from the database during the training phase to construct the machine learning model. Then, the system can input new instances from the database to predict the output.

\begin{figure}[!ht]
    \centering
    \includegraphics[width=.9\textwidth]{1_introduction/figures/PNG/machine_flow.png}
    \caption{The research methodology of the thesis.}
    \label{ch1:research-emthodo}
\end{figure}

Nevertheless, when endeavoring to forecast outcomes for fresh instances sourced from an alternative database, as illustrated in Fig. (1.2), there frequently emerges a conspicuous decline in accuracy. This disparity accentuates the imperative for model developers to intervene and rectify the issue. Addressing this, developers must adjust and retrain the model utilizing datasets from the new environment to ameliorate accuracy. This iterative process aims to refine the model's precision and ensure its efficacy across diverse contexts, thereby bolstering the reliability of decision-making and predictive capabilities. To confront this challenge, the field of auto machine learning endeavors to facilitate online updates to the model without necessitating direct intervention from developers for modification.

\begin{figure}[!ht]
    \centering
    \includegraphics[width=.8\textwidth]{1_introduction/figures/PNG/wrong_machine_flow_1.png}\\
    (a) \\
    \includegraphics[width=.8\textwidth]{1_introduction/figures/PNG/wrong_machine_flow_2.png}\\
    (b)
    \caption{The research methodology of the thesis.}
    \label{ch1:research-emthodo}
\end{figure}



In recent years, the surge in high-speed data streams has posed notable challenges for machine learning models, particularly in the context of streaming data analysis. These data streams, characterized by continuous, dynamic, and high-volume data arrivals, demand adaptive learning algorithms that can effectively cope with their evolving nature [1] [2] [3]. Within these evolving data environments, two paramount challenges have emerged: concept drift, class imbalance, Emerging new class, and heterogenous transfer learning.

Concept drift, a phenomenon defined by the evolving statistical properties of a data generation process over time [4] [5]. introduces a dynamic element to the data, necessitating continuous adaptation of machine learning models. This shift can manifest as changes in underlying concepts, relationships between variables, or alterations in data distribution. Traditional models trained on historical data may suffer diminished accuracy or become inadequate when confronted with new data influenced by concept drift, highlighting the need for effective concept drift detection mechanisms. Addressing concept drift involves the utilization of concept drift detectors, which are methods capable of identifying changes in data stream distributions. These detectors rely on information related to classifier performance or incoming data items to signal the need for model updates, retraining, or even replacing the old model with a new one. The dynamic nature of concept drift necessitates ongoing monitoring and adaptation to maintain the model's efficacy.

Data streams also present challenges related to class imbalance, a condition characterized by uneven distribution among different classes [6] [7]. This scenario, especially prevalent in multi-class settings, poses a significant challenge for traditional classifiers. The risk of misclassifying minority class samples due to their limited representation demands specialized techniques to ensure accurate classification without sacrificing the performance of the majority class [8] [9] [10] [11]. To tackle class imbalance, three primary methods are commonly employed: sampling methods, algorithm adaptation methods, and hybrid methods. Sampling methods involve undersampling the majority class or oversampling the minority class to balance class distribution. Algorithm adaptation methods modify existing algorithms to handle imbalanced data [12] [13] [14], while hybrid methods combine data preprocessing with classification techniques, often utilizing ensemble classifiers to effectively mitigate class imbalance and enhance overall classifier performance [15] [16] [17] [18].

Another challenge arising in the context of class imbalance is class overlap, where instances from different classes share the same region in data space [17] [18]. This overlap complicates the task of distinguishing between representative instances of different classes, leading to performance challenges for traditional classifiers referred to as overlapping problems. Recent research introduces class-overlap undersampling methods to address this issue, leveraging local similarities among minority instances to identify potentially overlapping majority instances.

Therefore, both class imbalance and class overlap present significant hurdles in the realm of data stream analysis. Consequently, addressing class imbalance has become crucial in multi-class learning, leading to research efforts focusing on both concept drift and class imbalance challenges. Researchers have explored dynamic ensemble selection (DES) and multi-class oversampling techniques to tackle these issues. Dynamic classifier ensembles offer a unique ability to adapt their composition based on data characteristics, making them valuable in situations with evolving data conditions [19]. Researchers focus on the overproduce-and-select approach for classifier ensemble selection methods. The objective of classifier ensemble selection is to choose an optimal subset of classifiers from a larger ensemble. The selection process is guided by various criteria, including individual performance measures, diversity metrics, meta-learning techniques, and performance estimation approaches. This optimization is particularly important in scenarios where a balance between accuracy and computational resource constraints is critical. There are two distinct approaches: static and dynamic selection. Static selection involves assigning classifiers to predefined feature partitions, while dynamic selection adaptively selects classifiers based on their competency [20]. Dynamic selection offers two choices: individual models, known as Dynamic Classifier Selection (DCS), and ensemble models, called Dynamic Ensemble Selection (DES). DCS algorithms enable the selection of the most appropriate classifier for each data point based on its local competencies. In contrast, DES focuses on selecting the optimal classifiers for each instance based on their competence within localized regions [21] [22] [23] Competency assessment relies on a dynamic selection dataset (DSEL) containing labeled samples. Moreover, innovative techniques like the Randomized Reference Classifier introduce randomness into class supports to enhance adaptability in addressing challenges related to imbalanced data.

Additionally, transfer learning assumes a pivotal role in addressing the intricate challenges posed by dynamic data streams and inherent concept drift. This domain of research focuses on enhancing a model's learning performance within a target domain by harnessing knowledge gleaned from source domains [4] [5].Techniques in transfer learning include reducing domain gaps through instance re-weighting and feature matching, along with strategies to mitigate negative knowledge transfer by down-weighting irrelevant source data.

Lastly, in the study, the focus extends to the specific scenario of Streams with Emerging New Classes (SENC). This refers to situations where new classes, not present during the initial training of a learning model, emerge in the data stream. Traditional learning approaches, designed for fixed or predefined class distributions, face challenges in effectively recognizing and adapting to these novel classes in real-time. The need for adaptive learning mechanisms that can handle the emergence of new classes underscores the complexity of real-world data stream scenarios.
     

In this chapter, the motivation for this research along with the research questions
that naturally arise are discussed in  Section~\ref{sec:1_1_motivation}. After this, the objectives and contributions are presented in Sections \ref{sec:1_2_opportunity} and \ref{sec:1_3_1_goal}, respectively. Next, the
research methodology is summarised in Section \ref{sec:1_3_automl_and_tf}. Finally,
the research context and the outline of this thesis are presented in Sections \ref{ch1:research-context} and
\ref{sec:1_3_2_DissertationStructure}, respectively.
\chapter{Dynamic Classification Ensembles for Handling Imbalanced
Multiclass Drifted Data Streams}
\label{chapter:6_transfer_learning}
Transfer learning is a powerful approach for addressing the challenges of dynamic data streams and concept drift. Its primary objective is to improve learning performance in a target domain by utilizing knowledge from one or more source domains. The source domain typically comprises data or tasks with abundant labeled information, whereas the target domain often faces constraints such as limited or unlabeled data. This distinction underscores the significance of transfer learning in bridging knowledge gaps and enhancing learning in resource-constrained environments \cite{pan2009survey, wang2019characterizing}.  
The field of transfer learning focuses on maximizing positive knowledge transfer while minimizing negative transfer. To achieve this, researchers have developed strategies such as instance re-weighting and feature matching to reduce the domain gap \cite{zadrozny2004learning, cortes2008sample, pan2010domain, sun2016return}. Additionally, mitigating negative knowledge transfer often involves down-weighting irrelevant data sources to prevent the incorporation of misleading information \cite{wang2019characterizing}. While much of the research has been dedicated to static environments where data distributions remain stable, real-world scenarios like financial forecasting, energy demand prediction, and climate analysis often feature dynamic environments. These scenarios present the concept drift problem, where evolving data distributions challenge traditional models \cite{li2015learning, cao2019learning}.  
In dynamic environments, transfer learning must go beyond static methodologies to adapt continuously to changing data. A notable technique is Dynamic Ensemble Selection (DES), which employs multiple classifiers to make predictions. DES dynamically evaluates and selects the most competent classifiers for the prevailing data conditions, ensuring adaptability and resilience against concept drift. It excludes underperforming classifiers while progressively improving overall performance \cite{cruz2017meta, jackowski2014improved, kuncheva2000clustering}.  
Another critical approach is Adaptive Windowing (ADWIN), which detects concept drift by monitoring statistical changes, such as mean or variance, within dynamically adjusted sliding windows. When substantial shifts occur, ADWIN reconfigures the ensemble by retraining or introducing classifiers suited to the updated data distribution \cite{madkour2023historical}. Similarly, the Streaming Ensemble Algorithm (SEA) manages real-time data streams by continuously adapting classifier ensembles, maintaining robustness and performance even as data conditions shift and new classes emerge \cite{gama2004learning, adams2023explainable, madkour2023historical}.  
This study introduces efficient algorithms for classifying data streams in real-time, specifically addressing the challenges posed by heterogeneous transfer learning in non-stationary environments. The proposed framework integrates DES, ADWIN, and SEA to tackle dynamic data streams effectively. Its objectives include achieving high classification performance, adapting to evolving data distributions, and minimizing computational complexity. By leveraging these advanced techniques, the framework provides a robust, scalable, and efficient solution to the challenges of heterogeneous transfer learning, ensuring reliable performance in dynamic real-world scenarios.
The subsequent sections of this paper adhere to a well-structured organization. Section \ref{sec:6_2_motivation} presents the motivations and contributions. Section \ref{sec:proposed_methodology} introduces the third proposed approach, providing intricate explanations of its constituents, including dynamic classifier ensembles, concept drift handling, and heterogeneous multisource transformation. Section \ref{sec:4_5_Expsetup} outlines the experimental setup and presents the results, providing details regarding the employed datasets, evaluation metrics, and procedures. Finally Section \ref{sec:6_summary} presents a summary of this chapter.

\section{Motivations and Contributions of this Chapter} \label{sec:6_2_motivation}
This paper presents a novel contribution to real-time streaming scenarios by addressing the challenges of heterogeneous multisource streams, with key contributions that can be summarized as follows:
\begin{enumerate}[nosep]
  \setlength{\itemsep}{0pt}
  \setlength{\parskip}{0pt}
  \item The primary innovation involves incorporating a concept drift detection method in conjunction with an ensemble classifier, enabling real-time adaptation and refinement of the third proposed approach in response to transfer learning in non-stationary environments. This methodology ensures continuous evolution of the classification model in accordance with the changing data landscape.
  \item The second significant advancement is the introduction of a precise weighting method to assess the significance of each local classifier within the ultimate classifier.
 \item The third significant contribution is the development of an innovative approach that employs the eigenvector technique to facilitate the transfer of knowledge from heterogeneous source domains to the target domain.
  \end{enumerate} 
 
   


\section{Proposed Methodology}\label{sec:6_third_proposed_approach}

In this section, we introduce the Heterogeneous Transfer Learning (HTL) algorithm tailored for non-stationary environments. HTL adeptly assimilates knowledge from both heterogeneous and homogeneous sources, operating within the framework of data streams subject to concept drift. Leveraging an online learning inductive parameter transfer strategy, HTL achieves seamless knowledge transfer. We delineate the core components and workflow of the HTL algorithm. To address the challenges intrinsic to our approach, we focus on three key aspects:
\begin{itemize}
	\item \textbf{Heterogeneous Sources:} Traditional transfer learning methods often assume homogeneity in the data distribution across source and target domains. However, in real-world scenarios, data sources may vary significantly in terms of feature space. HTL addresses this challenge by allowing knowledge transfer from sources with different dimensionalities. This means that the algorithm can effectively learn from diverse dimentionality sources, enabling a more comprehensive knowledge transfer process.
	\item \textbf{Drifted Streams:} In dynamic environments, data distributions may change over time, leading to concept drift. This phenomenon poses a significant challenge for machine learning algorithms, as models trained on historical data may become obsolete as the underlying data distribution shifts. HTL incorporates a concept drift detector that continuously monitors the incoming data stream. Upon detecting a shift in the data distribution, the algorithm adapts its classifiers accordingly to ensure continued effectiveness in handling drifting streams. This dynamic adjustment mechanism allows HTL to maintain high performance even in the presence of concept drift.
	\item \textbf{Classifier Performance:} Classifier performance is crucial for the overall effectiveness of the transfer learning process. HTL employs Dynamic Ensemble Selection (DES) to enhance classifier performance. DES creates a diverse ensemble of classifiers, each trained on different subsets of the data. When presented with a new data point, DES dynamically selects the most suitable classifiers from the ensemble based on its performance on similar chunk. This adaptive selection process ensures that the most appropriate classifier is chosen for each data point, leading to improved classification accuracy and robustness. By leveraging DES, HTL maximizes the utility of available classifiers, resulting in superior performance in non-stationary environments with heterogeneous data sources.
\end{itemize}

\subsection{HTL Overall Details}

The aim of this section is to harness knowledge from diverse dimensional multisource domain streams. Following a similar framework to CDTL \cite{yang2021concept}, this approach employs a class-wise and domain-weighted strategy. However, HTL enhances the weight function of CDTL in a class-wise manner, as demonstrated in Equation 1. This equation factors in both correct and incorrect predictions for each classifier class, with K representing the number of classifiers in the current chunk, C denoting the number of classes in the current chunk, and i indicating the current chunk. The components of HTL operate synergistically, as depicted in Fig. \ref{fig:6_alg1}, encompassing three phases:
\begin{itemize}
	\item \textbf{Dynamic Ensemble Selection phase (DES Phase):}: The primary objective DES phase is to identify the optimal classifier for incoming data. This is crucial for ensuring that the selected classifier effectively aligns with the unique characteristics of the current data segment.
	\item \textbf{Drift detector phase:} After the DES phase, our method advances to the drift detector phase, where it operates in real-time to continually monitor the data stream. This phase employs ADWIN and DDM techniques, which play a pivotal role in swiftly identifying any signs of concept drift. These techniques are designed to detect changes in the underlying data distribution over time, thus enabling the algorithm to adapt to evolving data patterns.
	\item \textbf{Feature Scaling} The concluding phase of our approach is featuring scaling, operating in real-time to harmonize the diverse dimensionalities of the multisource streams with the target dimensionality. Leveraging the eigen vector technique, this phase facilitates the transformation of data into a unified dimensionality, essential for creating new classifiers tailored to the target and domain streams. By ensuring compatibility with the learning framework, this process enhances the algorithm's effectiveness in handling diverse dimensionality streams.
\end{itemize}
\subsection{Classifier Creation Details }

Fig. \ref{fig:6_alg2} provides a comprehensive overview of the classifier creation phase, a crucial component responsible for generating new classifiers based on the current chunk of the target domain, previous classifiers, and source domain classifiers. This phase offers several advantages and perform three tasks. \begin{itemize}
	\item \textbf{Source Projection:} This step involves projecting the source domain classifiers onto the current chunk of the target domain. The projection likely employs the source weight function, as described in CDTL \cite{yang2021concept}, to assign a weight to each classifier based on its relevance to the current chunk of data. This weighting mechanism ensures that classifiers from the source domains contribute appropriately to the creation of the new classifier for the target domain.
	\item \textbf{Class-Wise Weights:} This block computes class-specific weights for each classifier using Equations \ref{eq:6_eq_1} and \ref{eq:6_eq_2} . The process involves analyzing the current chunk of the target stream (denoted as \emph{D}) and considering both correct and incorrect predictions made by each classifier where $prediction_i$ refers to the predicted class of the current instance and $c$ to the income class. These weights are crucial for determining the significance of individual classifiers across different classes. The equations facilitate the calculation of weights that reflect the classifiers' performance on specific classes, enabling the creation of a well-balanced ensemble classifier.
	\item \textbf{Classifier Combination:} In this phase, the class-wise weights, along with the current chunk of data and the projected source domain classifiers, are combined to derive the final classifier. The combination process follows the details outlined in Eq. \ref{eq:6_eq_3} , where historical classifiers (denoted as \emph{H}), projected data classifiers (denoted as \emph{P}), and the classifier of the current chunk (denoted as \emph{K}) are integrated. This integration ensures that the resulting classifier incorporates contributions from both historical and projected data sources, leveraging the strengths of each to enhance predictive performance. The resulting classifier is then utilized to make predictions based on the data chunk, effectively leveraging the insights gleaned from both historical and current data sources.
\end{itemize}

\subsection{HTL Detailed Algorithm}

As illustrated in Algorithm \ref{alg:6_alg_1}, the HTL algorithm involves several stages, beginning with training a classifier for the target stream and proceeding through various steps related to heterogeneous multisource preprocessing, classifier weighting, concept drift detection, and classifier management to ensure the accuracy and adaptability of the final classifier. In this section, detailed explanations of each individual step within the HTL algorithm are provided.
\begin{itemize}
	\item \textbf{Converting heterogeneous multisource to homogeneous multisource:} In the initial step, the algorithm unifies the various data sources that might have different characteristics (heterogeneous multisource). This is achieved using eigenvectors and the feature count of the current chunk (line 7).
	\item \textbf{Training a new classifier for the first target chunk:} In line 8, the new classifier is trained specifically for the first chunk of the target stream. This classifier was used to predict the initial chunk of the target stream.
	\item \textbf{Calculate heterogeneous multisource weights:} The algorithm computes the weights for each heterogeneous data source. These weights are determined based on the characteristics of the data from each source and the target classifier. This weighting process helps prioritize more relevant data sources (line 9).
	\item \textbf{Calculating class-wise weights:} In line 27, the algorithm calculates class-wise weights using Equations 1 and 2. These weights were essential for determining the contribution of each class to the final classifier.
	\item \textbf{Converting homogeneous multisource to projected data source:} Line 11 involves using the weights calculated in the third step to transform the homogeneous multisource representation to a projected data source. This transformation likely uses the source weight function of CDTL \cite{yang2021concept}.
	\item \textbf{Prediction of the current chunk:} In line 16, the algorithm determines the output class using Equation 3.
	\item \textbf{Monitoring for concept drift:} The algorithm continuously monitors the accuracy of its predictions to detect any concept drift, a situation in which the underlying data distribution changes, potentially leading to a degradation in model performance. Line 19 has been used for this purpose.
	\item \textbf{Updating the Projected Multisource classifiers:}Lines 29–32 are dedicated to updating the classifiers associated with the projected multisource. It is necessary to adapt to changes in the data or to maintain the accuracy and relevance of the classifiers over time.
	\item \textbf{Managing the pool of classifiers:} If the pool of classifiers exceeds a predefined maximum size threshold, lines 23 and 24 indicate that a new classifier is trained, and the worst-performing classifier is removed. This process helps to maintain a manageable and effective set of classifiers.
\end{itemize}
The heterogeneous Transfer Learning (HTL) algorithm represents a significant advancement in addressing the complexities of non-stationary environments prone to concept drift. By integrating heterogeneous and homogeneous sources within dynamic data streams, HTL demonstrates remarkable adaptability and effectiveness. Through its meticulously designed workflow, HTL can efficiently handle the challenges posed by evolving data landscapes. From the initial reception of environmental data to the nuanced processing of new data chunks and vigilant monitoring of concept drift, HTL embodies a comprehensive approach to knowledge transfer and adaptation. Owing to its ability to compute class-specific weights and leverage vital classifiers, HTL offers a robust solution for navigating non-stationary environments with confidence and precision.


\begin{figure}[!ht]
	\centering
	\includegraphics[width=1\linewidth]{6_transfer_learning/figures/alg1.png}
	\caption{Proposed Approch Flow.}
	\label{fig:6_alg1}
\end{figure}
\begin{figure}[!ht]
	\centering
	\includegraphics[width=1\linewidth]{6_transfer_learning/figures/alg2.png}
	\caption{Approach for Classifier Creation.}
	\label{fig:6_alg2}
\end{figure}

\begin{equation}
	\label{eq:6_eq_1}
	\text{classWeight}{c,\mathcal{D}} = \sum{i=1}^{|\mathcal{D}|} \frac{|\text{prediction}_i = c|}{|\mathcal{D}|} \times \frac{|\text{prediction}_i \neq c|}{|\mathcal{D}|} \quad
\end{equation}

\begin{equation}
	\label{eq:6_eq_2}
	\text{classifierWeight} = \sum_{k \in K} \sum_{c \in C} \text{classWeight}_{k,c,\mathcal{D}} \quad \text{where } \mathcal{D} = 1,2,3 \dots N
\end{equation}

\begin{equation}
	\label{eq:6_eq_3}
    \text{Prediction}{p,l,k,\text{chunk}} = \sum{p} \text{prediction}^{p}{\text{chunk}} + \sum{l} \text{prediction}^{l}{\text{chunk}} + \text{prediction}^{k}{\text{chunk}}
\end{equation}

\begin{algorithm}[ht]
	\DontPrintSemicolon
	\KwIn{Target domain stream \emph{stream}, heterogeneous multisource domain $\Psi$, pool of classifiers, threshold $\ell$}
		\KwData{Current chunk \emph{a}, classifiers \emph{k}, source classifiers $\psi$, projected source domain \emph{S}, target domain weights $\omega$, source domain weights $\lambda$}
		\KwOut{Prediction}
		  \For{$ \emph{stream}\ have\ chunk$ }{
				  \eIf{$\emph{a}\ is\ the\ First\ chunk$}{
				    $\Psi \gets convertSourcesToTargetDim(\Psi,\ \emph{a})$\;
				    $\emph{k} \gets  trainingNewClassifier(\emph{a})$\;
				    $\lambda \gets SourcesDomainWeights(\Psi,\ \emph{k})$\;
						 $\omega \gets classWiseWeights(\emph{K},\ \emph{a})$\;
				    $\emph{S} \gets projectedSourceDomain(\Psi,\ \lambda)$\;
				   \For{$\emph{source}\ in\ \emph{S}$}{
						 $\emph{newClassifier} \gets  trainingNewClassifier(\emph{source})$\;
				    $\psi \gets  \psi \cup\  \emph{newClassifier}$\;
			  
				   }
				   $\emph{prediction} \gets getPrediction(\emph{a},\ \emph{k},\ \psi,\ \lambda,\ \omega)$ \;
				   }{
				   $\emph{prediction} \gets getPrediction(\emph{a},\ \emph{k},\ \psi)$ \;
					   $driftResult \gets conceptDriftDetector(\emph{prediction})$ \;
			  
					  \If{ $driftResult\ have\ drift $}{
					    $\emph{newClassifier} \gets trainingNewClassifier(\emph{a})$\;
				    $\emph{k} \gets \emph{k}\ \cup \emph{newClassifier}$\;
						\If{$size(\emph{K}) \geq \ell$}{
						 $removeWorstClasssifier(\emph{k})$\;
						}
				    $\lambda \gets SourcesDomainWeights(\Psi,\ \emph{k})$\;
				    $\omega \gets classWiseWeights(\emph{K},\ \emph{a})$\;
			  
				    $\emph{S} \gets projectedSourceDomain(\Psi,\ \lambda)$\;
				   \For{$\emph{source}\ in\ \emph{S}$}{
						 $\emph{newClassifier} \gets  trainingNewClassifier(\emph{source})$\;
				    $\psi \gets  \psi \cup\  \emph{newClassifier}$\;
			  
				   }
				  
				   $\emph{prediction} \gets getPrediction(\emph{a},\ \emph{k},\ \psi,\ \lambda,\ \omega)$ \;
					 }
				  }
				  }
	\Return{$\emph{prediction}$}\;
	\caption{Flow of the Emerging Phase.}
	\label{alg:6_alg_1}

\end{algorithm}
%%%%%%%%%%%%%%%%%%%%%%%%%%%%%%%%%%%%%%%%%%%%%%%
%
%   Families of Traffic Forecasting Problems
%
%%%%%%%%%%%%%%%%%%%%%%%%%%%%%%%%%%%%%%%%%%%%%%%

%\usepackage{cite}
%\usepackage{multirow}
%\usepackage{rotating} 
%\usepackage[table,xcdraw]{xcolor}
%\usepackage{float}
%\usepackage[utf8]{inputenc}
%\usepackage{amsmath,amssymb,amsfonts}
%\usepackage{algorithmic}
%\usepackage{graphicx}
%\usepackage{textcomp}
%\usepackage{rotating}
%\usepackage{verbatim}


\section{Experimental Results}
\label{sec:4_5_Expsetup}

In this section, the proposed HTL, CORAL \cite{sun2016return}, and CDTL \cite{yang2021concept}, and Melanie \cite{dong2019multistream} algorithms are compared. Three subsections are introduced: the first section focuses on the experimental setup; second, a discussion on data streams to apply our experiments on heterogeneous and homogeneous source domains, including a comparison to evaluate the runtime factor between all methods; and finally, a comparison is made between online learning and chunk-based concept drift.


\subsection{Experimental setup}
The evaluation of Heterogeneous Taxonomy Learning (HTL) involves a comparison with CORAL \cite{sun2016return}, CDTL \cite{yang2021concept}, and Melanie \cite{dong2019multistream}  which employs multiple metrics such as precision, recall, F1 score \cite{sasaki2007truth}, BAC \cite{brodersen2010balanced}, and G-mean \cite{kubat1997addressing}. The experimental protocol employed for evaluation followed the test-then-train approach \cite{krawczyk2017ensemble}, where the classification model is trained on a specific data chunk and subsequently evaluated on the next chunk. The chunk size was standardized to 2000 instances. Four different classification models were used as base estimators: k-Neighbors (KNN) algorithm, Support Vector Machine (abbreviated as SVM), Gaussian Naive Bayes abbreviated as (GNB), and Hoeffding Tree (abbreviated as HT), as implemented in scikit-learn \cite{frias2014online}. We establish an ensemble classifier pool with a set limit of L = 8, wherein each ensemble consists of N = 4 base models. While these constraints remained fixed across all our experiments, the threshold for the pool classifier in each approach was maintained at eight. Consequently, if the threshold is surpassed, the least-performing classifier is systematically eliminated. This configuration was applied consistently across all approaches to ensure fair engagement. The experiments were carried out using the Python programming language, with the source code publicly accessible on GitHub . When dealing with heterogeneous multisource streams, it is often impractical or not applicable to truly heterogeneous experiences. Consequently, the eigenvector technique was utilized in all related approaches. This decision stems from the fact that the algorithms in related work primarily focus on homogeneous source domains and do not address the challenges posed by heterogeneous multi-source data. Therefore, heterogeneous experiments were conducted.

\subsubsection{Data Streams}
In this study, the performance of the proposed approach was evaluated using various datasets, including synthetic data streams, a real application stream, and dataset benchmark datasets. To conduct the evaluations, the stream-learn Python library \cite{ksieniewicz2022stream}\cite{madkour} was used. As detailed in Table 1, the benchmark dataset employed in the study is the Covertype dataset. This dataset comprises 52 features, 7 classes, and a total of 581,010 instances. It serves as a standard benchmark dataset widely used in stream mining research. For the evaluation of real application streams, the Sensor Stream dataset was utilized. This dataset includes 5 features, 58 classes, and a total of 392,600 instances, representing a real-world application scenario. Synthetic datasets were generated using the scikit-learn Python package. This synthetic dataset was created to simulate data streams and evaluate the performance of the framework. The synthetic datasets consisted of 10 features and four classes and were divided into 200 chunks. Each chunk had a size of 2,000. The performance of the proposed framework was systematically evaluated using these datasets and a stream-learn library. These evaluations provided valuable insights into the effectiveness of the framework in handling different types of data streams, including benchmark datasets, real application streams, and synthetic data streams.

\begin{table}[h!]
  \centering
  \begin{tabular}{|l|c|c|c|}
  \hline
  \textbf{Dataset} & \textbf{Number of Features} & \textbf{Number of Classes} & \textbf{Number of Instances} \\ \hline
  Covertype dataset$^2$ & 40 & 7 & 581,010 \\ \hline
  Sensor Stream dataset$^3$ & 5 & 58 & 392,600 \\ \hline
  Synthetic stream-52 & 52 & 5 & 200,000 \\ \hline
  Synthetic stream-8 & 8 & 4 & 200,000 \\ \hline
  \end{tabular}
  \caption{Characteristics of the datasets used in the experimentation.}
  \label{table:6_table1}
  \end{table}

  \subsubsection{Compared Approaches}
  In the subsequent section, we introduce the established benchmark techniques as outlined below:
  \begin{itemize}
    \item \textbf{AW- CORAL \cite{sun2016return}:} The AW- CORAL (Adaptive Weighted CORrelation ALignment) method is a simple domain adaptation technique designed to align the distributions of both source and target features without supervision. This is accomplished by aligning second-order statistics, focusing specifically on covariance, to match the distributions between the domains.
    \item \textbf{HE-CDTL \cite{sun2016return}:} The HE-CDTL approach tackles Concept Drift Transfer Learning (CDTL) by integrating knowledge from source domains and historical time steps in the target domain to improve learning performance. HE-CDTL features class-wise weighted ensemble for independent selection of historical knowledge by each class, and AW-CORAL to mitigate domain discrepancy and reduce negative knowledge transfer.
    \item \textbf{Melanie \cite{dong2019multistream}:} Multisource Online Transfer Learning for Non-stationary Environments, known as Melanie, represents the inaugural method capable of transferring knowledge across various data streaming sources in non-stationary environments. Melanie constructs numerous sub-classifiers to grasp diverse facets from distinct source and target concepts dynamically. It identifies sub-classifiers that align closely with the prevailing target concept, assembling them into an ensemble for predicting instances originating from the target concept.
  \end{itemize}

While Melanie extends online learning through chunk-based learning akin to CDTL, it exhibits two limitations: 
\begin{itemize}
  \item utilizing a global weight for each classifier, disregarding performance variation across different locations of a data chunk
  \item combining learned classifiers from these domains directly with the target classifier, which may impede effective knowledge transfer due to domain discrepancies. Hence, we compare the proposed approach (HTL) with AW- CORAL \cite{sun2016return} and HE-CDTL yang2021concept.
\end{itemize}

\subsection{Analysis of Experimental Results}
This section presents a comprehensive evaluation of the performance of the proposed approach across multiple data streams. To guarantee a thorough assessment, five performance metrics—F1 score, precision, recall, G-mean, and BAG—were presented using two visualization diagrams: radar and line. A radar chart was cleverly utilized to provide a comprehensive summary, precisely demonstrating the performance of each algorithm across the six key metrics. By calculating the average value of each metric, the overall performance of each approach (HTL, CORAL, and CDTL) was determined. On the other hand, the line diagram utilizes the G-mean metric to compare these methods for each chunk across 100 chunks. A line diagram was used in the first five experiments, whereas a combination of radar and line diagrams was employed in the last two experiments. This approach ensured a detailed and nuanced evaluation of the performance of the proposed approach in diverse experimental scenarios.

\subsubsection{Results on Target Domains Dataset without Source Domain}
Fig. \ref{fig:6_exp1} showcases the results of applying the aforementioned methods to the Covertype dataset, where no stream is utilized as a source domain. The line diagram specifically highlights the classification accuracy measured by the G-mean metric across 100 data chunks for each method. Notably, all methods display suboptimal accuracy during the initial 10 chunks. However, a significant improvement is evident in subsequent phases. This improvement can be attributed to the expansion of the classifier pool, encompassing an increasing number of classifiers. This expansion allows the DES technique to become more adept at selecting the most appropriate classifier for each incoming chunk. Consequently, accuracy experiences a notable upsurge in later chunks, underscoring the adaptability and efficacy of the ensemble approach. From chunk 20 to the final chunk, HTL consistently achieves the highest accuracy across most chunks. In contrast, CORAL shows lower performance in certain chunks, while CDTL displays lower performance in others. The superior performance of HTL can be credited to its utilization of the proposed weighting method, which assigns a weight to each historical classifier. This approach contributes to the effective training of the pool classifiers, thereby enhancing the overall performance of the approach.

\begin{figure}[!ht]
	\centering
	\includegraphics[width=1\linewidth]{6_transfer_learning/figures/exp1_0.png}
	\caption{Result of the Covertype Stream as the Target Domain without the Presence of the Source Domain.}
	\label{fig:6_exp1}
\end{figure}
\begin{figure}[!ht]
	\centering
	\includegraphics[width=1\linewidth]{6_transfer_learning/figures/exp1_1.png}
	\caption{Result of the Covertype Stream as the Target Domain and Homogeneous Source Domains.}
	\label{fig:6_exp2}
\end{figure}

\subsubsection{Results on the Homogeneous Source Domains Dataset}
Fig. \ref{fig:6_exp2} depicts the results of applying the compared methods to the Covertype data stream as the target domain and Synthetic-52 as a homogenous source domain, which has the same dimensionality as the Covertype stream (52 features and seven classes). Upon examining the line diagram, it becomes evident that HTL's performance in the first eight chunks may be suboptimal due to the insufficient number of classifiers available. However, following the initial eight chunks, the addition of more classifiers results in an improvement in HTL's performance. In contrast, CDTL maintains satisfactory performance throughout the chunks, while MLSMOTE consistently displays lower performance. It is important to note that HTL consistently achieves the highest performance across all chunks in the diagram. Notably, the CDTL and HTL methods in Fig. \ref{fig:6_exp2} demonstrated superiority over the same methods in Fig. \ref{fig:6_exp1}. This is attributed to the experiment shown in Fig. \ref{fig:6_exp2}, which incorporates Synthetic-52 as a source domain, thereby enhancing the performance of the methods. In contrast to the experiment in Fig. \ref{fig:6_exp1}, which lacks a source domain.

\begin{figure}[!ht]
	\centering
	\includegraphics[width=1\linewidth]{6_transfer_learning/figures/exp2_0.png}
	\caption{Result of the Covertype Stream as the Target Domain and One Heterogeneous Source Domain.}
	\label{fig:6_exp3}
\end{figure}
\begin{figure}[!ht]
	\centering
	\includegraphics[width=1\linewidth]{6_transfer_learning/figures/exp2_1.png}
	\caption{Result of the Covertype Stream as the Target Domain and Multiple Heterogeneous Source Domains.}
	\label{fig:6_exp4}
\end{figure}
\begin{figure}[!ht]
	\centering
	\includegraphics[width=1\linewidth]{6_transfer_learning/figures/exp3.png}
	\caption{Result of the Sensor Stream as the Target Domain and Multiple Heterogeneous Source Domains.}
	\label{fig:6_exp5}
\end{figure}

\subsubsection{Results on One Heterogeneous Source Domain}
In this experiment, the Synthetic-8 stream was utilized as a heterogeneous source domain. Considering the nature of the HTL algorithm, which can operate with heterogeneous sources, specific adjustments were implemented to enable CORAL and CDTL, which typically do not operate with such sources, to function in a heterogeneous domain environment. The eigenvector technique was applied to CORAL and CDTL to facilitate compatibility with the heterogeneous sources. Fig. \ref{fig:6_exp3} shows the results of applying the compared methods to the Covertype data stream as the target domain, with Synthetic-8 acting as the heterogeneous source domain. Synthetic-8 has a different dimensionality (eight features and four classes) compared with the Covertype stream. An analysis of the line diagram reveals that the performance of the compared approaches may vary across different chunks depending on the source domain and the positive knowledge it contributes. HTL consistently outperforms, achieving the highest performance across almost all chunks, whereas CDTL and CORAL show lower performance.

\begin{figure}[!ht]
	\centering
	\includegraphics[width=1\linewidth]{6_transfer_learning/figures/exp4.png}
	\caption{Online learning and detector chunk-based result in Covertype stream via ADWIN detector}
	\label{fig:6_exp6}
\end{figure}

\begin{figure}[!ht]
	\centering
	\includegraphics[width=1\linewidth]{6_transfer_learning/figures/exp5.png}
	\caption{Online learning and detector chunk-based result in Covertype stream via DDM detector}
	\label{fig:6_exp7}
\end{figure}


\subsubsection{Results on All Heterogeneous Source Domains Stream and Covertype Stream as the Target Domain}
Fig. \ref{fig:6_exp5} presents the results of applying the compared methods to the Covertype data stream as the target domain, with the Synthetic-8, Synthetic-52, and Sensor streams serving as heterogeneous source domains. From the analysis, the performance in this figure exceeds that in Fig. \ref{fig:6_exp5}. This improvement is likely due to the larger number of source domains in this experiment, which leads to an incremental increase in positive knowledge with each added source domain, consequently improving method performance. In Fig. \ref{fig:6_exp6}, the same experimental setup was employed, but the target domain was switched from the cover type to the sensor stream. Additionally, the Covertype stream was included as one of the source domains in this configuration. This modification was intended to evaluate the performance of the HTL across the various target streams. Fig. \ref{fig:6_exp6} shows that the performance of all methods might be lower than that of the Covertype stream, mainly because of the increased noise in the sensor stream and numerous drifts. However, HTL consistently outperforms, achieving the highest performance across all chunks, whereas CORAL and CDTL display nearly identical values in most chunks.

\subsubsection{Results on One Heterogeneous Source Domain}
This section presents the overall performances of the compared methods (CORAL, CDTL, and HTL) across five key performance metrics: F1 score, precision, recall, G-mean, and BAG. Table 2 showcases the performance of each method across different experiments for the five metrics. The table illustrates the average value of each metric, which represents the overall performance of each method for 100 chunks. The emphasis on the bold highlighting in the table underscores the superiority of the HTL algorithm in various source domain configurations. In the first experiment (without the source domain), CORAL showed the best performance in terms of precision, while HTL demonstrated the best performance in the other metrics. Furthermore, in most experiments, HTL outperformed the other methods across most metrics, except for the recall metric in the third experiment, in which CDTL achieved the best performance. Notably, each subsequent experiment demonstrated improved performance compared with the previous experiment. This trend arose from the incremental nature of the source domain in each experiment. The incremental addition of the source domain enhances the positive knowledge of each experiment, leading to an increased performance of the compared methods in most experiments.

\begin{table}[h]
  \centering
  \caption{The Comprehensive Performance of CORAL, CDTL, and HTL Across All Previous Experiments}
  \label{table:6_table2}
  \begin{tabular}{|l|l|c|c|c|}
  \hline
  \textbf{Experiment}                            & \textbf{Metric} & \textbf{CORAL} & \textbf{CDTL} & \textbf{HTL} \\ \hline
  \multirow{4}{*}{Without source domain}         & BAC             & 0.723          & 0.723         & \textbf{0.725} \\ \cline{2-5} 
                                                 & G-mean          & 0.652          & 0.654         & \textbf{0.717} \\ \cline{2-5} 
                                                 & F1-score        & 0.875          & 0.874         & \textbf{0.890} \\ \cline{2-5} 
                                                 & Precision       & 0.827          & 0.823         & \textbf{0.851} \\ \cline{2-5} 
                                                 & Recall          & \textbf{0.950} & 0.944         & 0.939          \\ \hline
  \multirow{4}{*}{Homogenous source domain}      & BAC             & 0.731          & 0.730         & \textbf{0.755} \\ \cline{2-5} 
                                                 & G-mean          & 0.658          & 0.653         & \textbf{0.717} \\ \cline{2-5} 
                                                 & F1-score        & 0.876          & 0.875         & \textbf{0.858} \\ \cline{2-5} 
                                                 & Precision       & 0.830          & 0.829         & \textbf{0.851} \\ \cline{2-5} 
                                                 & Recall          & 0.950          & 0.950         & \textbf{0.953} \\ \hline
  \multirow{4}{*}{Single heterogeneous source domain} & BAC         & 0.732          & 0.732         & \textbf{0.757} \\ \cline{2-5} 
                                                 & G-mean          & 0.661          & 0.657         & \textbf{0.717} \\ \cline{2-5} 
                                                 & F1-score        & 0.875          & 0.875         & \textbf{0.859} \\ \cline{2-5} 
                                                 & Precision       & 0.835          & 0.835         & \textbf{0.851} \\ \cline{2-5} 
                                                 & Recall          & 0.947          & 0.950         & \textbf{0.953} \\ \hline
  \multirow{4}{*}{Multi-source heterogeneous domain (Covertype stream)} & BAC & 0.732  & 0.732         & \textbf{0.757} \\ \cline{2-5} 
                                                 & G-mean          & 0.661          & 0.661         & \textbf{0.717} \\ \cline{2-5} 
                                                 & F1-score        & 0.875          & 0.875         & \textbf{0.859} \\ \cline{2-5} 
                                                 & Precision       & 0.835          & 0.835         & \textbf{0.851} \\ \cline{2-5} 
                                                 & Recall          & 0.947          & 0.950         & \textbf{0.953} \\ \hline
  \multirow{4}{*}{Multi-source heterogeneous domain (Sensor stream)} & BAC & 0.998  & 0.998         & \textbf{0.998} \\ \cline{2-5} 
                                                 & G-mean          & 0.971          & 0.971         & \textbf{0.992} \\ \cline{2-5} 
                                                 & F1-score        & 0.997          & 0.997         & \textbf{0.997} \\ \cline{2-5} 
                                                 & Precision       & 0.998          & 0.998         & \textbf{0.998} \\ \cline{2-5} 
                                                 & Recall          & 0.998          & 0.998         & \textbf{0.998} \\ \hline
  \end{tabular}
  \end{table}

\subsection{4.3.	Analysis Runtime between CORAL, CDTL, and HTL Techniques}
Our experimental results highlight that the selection of the optimal algorithm for handling heterogeneous transfer learning is influenced by various factors. These factors include the dataset characteristics and runtime requirements of the algorithm. In the experiments, the focus was on analyzing the runtimes of different algorithms, and the results were insightful. Notably, the HTL algorithm was exceptionally efficient. For example, when considering the training of the first experience (homogeneous source domain), the HTL algorithm completed 100 iterations within 304 seconds. In contrast, the CORAL and CDTL algorithms require more time to achieve the same number of training iterations. Furthermore, the HTL algorithm consistently demonstrated remarkable efficiency, particularly when the source domain settings were absent, single heterogeneous, or multisource heterogeneous. These findings highlight the superior runtime performance of the HTL algorithm, positioning it as an attractive choice for scenarios with stringent time constraints. The bold highlighting in Table 3 further accentuates the efficiency of the HTL algorithm across various source domain settings, including homogeneous, absent, single heterogeneous, and multi-source heterogeneous settings, making it a compelling option for applications where time efficiency is of paramount importance.

\begin{table}[h]
  \centering
  \caption{Runtimes (in seconds) for CORAL, CDTL, and HTL}
  \label{table:6_table3}
  \begin{tabular}{|l|l|l|c|c|c|}
  \hline
  \textbf{Experience}                              & \textbf{Target domain} & \textbf{Source Domain}                      & \textbf{CORAL} & \textbf{CDTL} & \textbf{HTL} \\ \hline
  without source domain                            & Covertype              & None                                         & 312            & 312           & \textbf{304} \\ \hline
  Homogenous source domain                         & Covertype              & Synthetic-52                                 & 6165           & 614           & \textbf{606} \\ \hline
  Single heterogeneous source domain               & Covertype              & Synthetic-8                                  & 6060           & 4703          & \textbf{4475} \\ \hline
  Multi-source heterogeneous domain                & Covertype              & Sensor, Synthetic-52, and Synthetic-8        & 23011          & 14043         & \textbf{13824} \\ \hline
  Multi-source heterogeneous domain                & Sensor                 & Covertype, Synthetic-52, and Synthetic-8     & 14524          & 3052          & \textbf{3005} \\ \hline
  \end{tabular}
  \end{table}
  
\subsection{4.4.	Analysis accuracy and runtime of HTL for online learning and chunk-based concept drift}
We conducted a thorough series of statistical analyses encompassing 12 comparisons across three diverse datasets, three methods, and two drift detectors. These analyses were rigorously evaluated using a non-parametric test, specifically the Kruskal-Wallis test. The results of this test were striking, revealing substantial variations in the G-mean measurements across most experiments. Importantly, these differences were not due to random chance \cite{yamada2013change}. Upon closer examination of the assessments for the three methods, PA, MLSMOTE, and MLSOTE, as detailed in Table \ref{tab:4_first_proposal_result_table_3}, we found compelling evidence supporting the acceptance of the null hypothesis (H0).
This implies that the expected and observed data exhibited statistically significant disparities. H0 is rejected when significant differences are not observed and H0 is accepted if the P-value falls below the critical value. Underlining the significance of our analysis, the Kruskal-Wallis test was conducted with a 95\% confidence level, and the P-value was rounded to the first three digits after the decimal point. Nevertheless, it is important to note that a similarity in the performances of the methods emerged in the third and fourth experiments, resulting in the rejection of H0. This similarity can be attributed to the fact that the DDM detected fewer drifts in these experiments.

\begin{table}[h]
  \centering
  \caption{Runtimes (in seconds) for Online Learning, ADWIN, and EDMM drift detector}
  \label{table:6_table4}
  \begin{tabular}{|l|c|c|}
  \hline
  \textbf{Technique}       & \textbf{Runtime} & \textbf{Learning Times} \\ \hline
  Online Learning          & 3102             & 100                     \\ \hline
  ADWIN detector           & 2257             & 63                      \\ \hline
  EDMM detector            & 1948             & 40                      \\ \hline
  \end{tabular}
  \end{table}
  
\section{Summary}
\label{sec:6_summary}
This study presents a novel framework for incremental learning in data streams, with a focus on heterogeneous transfer learning. The approach effectively integrates concept drift detection via the ADWIN algorithm, eigenvectors for knowledge transfer, and ensemble classifiers. Extensive experiments on benchmark, real-world, and synthetic datasets validate the framework's efficacy in adapting to evolving data patterns. The use of Dynamic Ensemble Selection (DES) optimizes classifier performance, enhancing predictive performance and robustness. The framework excels in handling heterogeneous multi-source domains, providing high adaptability, efficiency, and scalability to real-time data streams while maintaining reliable and relevant performance.


% Conclusions

% this file is called up by thesis.tex
% content in this file will be fed into the main document

\chapter{Conclusions and Future Work}
\label{chapter:7_Conclusions}
This chapter summarizes the key findings of this study on incremental learning in data streams and presents potential directions for future research to further improve and expand the methodology.

\section{Conclusions}
\label{section:7_1}
\begin{itemize}
    \setlength{\itemsep}{1.5pt}
    \setlength{\parskip}{1.5pt}
    \item This study introduces a novel approach integrating oversampling techniques, concept drift detection, and Dynamic Ensemble Selection (DES) to identify the most suitable ensemble classifiers.
    \item We demonstrated the effectiveness of the methodology through extensive experiments across various datasets, including benchmarks, real-world streams, and synthetic data.
    \item A key feature is the rapid detection of concept drift, enabling the system to adapt quickly, ensuring continued relevance and performance in real-time scenarios.
    \item We addressed the minority class problem using advanced oversampling and KNN algorithms to prevent overlapping class instances.
    \item The DES technique was pivotal in selecting the most effective classifiers, optimizing overall performance.
    \item Our methodology excels in handling multiclass imbalanced stream challenges, maintaining high performance as class distributions evolve.
    \item The approach is marked by its adaptability, efficiency, and scalability, offering dynamic model updates to ensure real-time reliability.
    \item Despite its advantages, the methodology has limitations such as the significant time required to generate non-overlapping synthetic instances, and its reliance on the performance of MLSMOTE and MLSOL techniques.
\end{itemize}

\section{Future Work}
\label{section:7_2}
\begin{itemize}
    \setlength{\itemsep}{1.5pt}
    \setlength{\parskip}{1.5pt}
    \item Future research should focus on developing more advanced oversampling techniques that avoid overlapping synthetic instances, thus reducing the computational burden.
    \item Exploring meta-learning methods to better assess imbalanced multiclass ratios and improve minority class identification could enhance the effectiveness of the approach.
    \item Further development of advanced concept drift detection algorithms is needed to improve the timely recognition of drift and better handle the evolving data distributions.
    \item The incorporation of alternative ensemble strategies and deep learning techniques could lead to further improvements in both drift detection and classifier performance.
    \item Expanding the methodology to address a broader range of real-world applications, including streaming from diverse domains, will enhance its applicability and robustness.
    \item Future work should also focus on refining classifier selection and prediction methods, especially for handling multisource domains in real-time environments.
    \item Investigating the use of deep learning methods to predict future drifts and optimize classifier performance is a promising direction for improving model adaptation and performance.
\end{itemize}


% In this chapter, we presented a comprehensive methodology for incremental learning in data streams, particularly addressing the complexities of imbalanced data, including minority and overlapping classes \ref{section:7_1}, Emerging new classes in the incremental drifted streams \ref{section:7_2} and the heterogeneous transfer learning in the drifted streams \ref{section:7_2}.

% \section{Imalanced multiclass stream}
% \label{section:7_1}

% This study integrates a novel oversampling technique, concept drift detection, and Dynamic Ensemble Selection (DES) to identify the most suitable ensemble classifier. Extensive experimentation across various datasets, including benchmarks, real-world streams, and synthetic data, demonstrated the effectiveness of our methodology.
% A key feature of our approach is the rapid detection of concept drifts, which allows the system to adapt quickly by training new base classifiers, ensuring continued relevance and accuracy in real-time scenarios. We addressed the minority class problem through advanced oversampling, while the KNN algorithm was employed to prevent the creation of overlapping class instances. The DES technique played a crucial role in selecting the most effective classifiers, optimizing overall performance.

% The results of our evaluation, based on multiple performance metrics, show that our methodology is particularly effective in handling multiclass imbalanced stream challenges. It excels in maintaining high accuracy as class distributions evolve. Beyond accuracy, the approach is marked by its adaptability, efficiency, and scalability. The capability for dynamic model updates driven by incoming data allows for continuous adaptation to changing data distributions, ensuring the model's reliability and relevance in real-time applications.

% However, our methodology does have certain limitations. The time required to generate non-overlapping synthetic instances is significant, and the approach's success heavily depends on the accuracy of MLSMOTE and MLSOL techniques. Future research should focus on developing more advanced oversampling techniques that avoid overlapping synthetic instances. Additionally, exploring meta-learning methods to better assess imbalanced multiclass ratios and minority class identification could further improve the approach.

% \section{Emerging new classes}
% \label{section:7_2}

% This section, a robust framework for incremental learning in data streams with emerging new classes. By integrating the ADWIN algorithm for concept drift detection, K-means clustering, and Gaussian Naive Bayes (GNB) as the base classifier within an ensemble stratified bagging framework, our method effectively tackles the challenges posed by such dynamic data streams. The ADWIN algorithm is particularly important, enabling the prompt detection of concept drifts and new classes, which is critical for training new classifiers and maintaining accuracy over time.

% The use of ensemble stratified bagging combined with DES significantly enhances predictive performance and robustness. Our evaluations confirm that the framework is highly effective in classifying data streams with evolving distributions, especially when GNB serves as the base classifier. In addition to its high accuracy, the framework's strengths lie in its adaptability, efficiency, and scalability. The dynamic updates to the model, based on incoming data, ensure continuous adaptation to shifting data patterns, supporting real-time reliability and relevance.

% Looking ahead, future work could explore several avenues for improvement. These include the development of advanced concept drift detection algorithms, the exploration of alternative ensemble strategies, and the incorporation of deep learning techniques. Additionally, expanding the framework to address a broader range of real-world applications and refining the selection and prediction methods for classifiers will be crucial for further enhancing the methodology's effectiveness.

% \section{Heterogeneous transfer learning}
% \label{section:7_3}

% This study introduces a comprehensive Approach for incremental learning in data streams, particularly focusing on heterogeneous transfer learning. The framework effectively addresses the challenges associated with this context by integrating concept drift detection through the ADWIN algorithm, employing eigenvectors, and utilizing ensemble classifiers. The efficacy of this approach was validated through extensive experiments on a variety of datasets, including benchmark datasets, real-world application streams, and synthetic data.

% A key strength of this approach is the pivotal role played by the ADWIN algorithm, which enables the timely detection of concept drifts and changes in data patterns. This allows the framework, referred to as HTL, to adapt by training new base classifiers, ensuring their continued relevance and accuracy in real-time scenarios. The use of ensemble classifiers further enhances predictive performance and robustness by aggregating predictions from multiple base classifiers. Dynamic Ensemble Selection (DES) is utilized to select the most suitable base classifiers, optimizing overall performance.

% Our thorough evaluation using various performance measures confirms the approach’s effectiveness in addressing the challenges of heterogeneous multisource domains. The framework excels in accurately classifying data streams with evolving class distributions, particularly when leveraging ADWIN as the drift detector, eigenvectors for knowledge transfer across heterogeneous domains, and DES for classifier selection.

% Beyond its strong accuracy and performance, the framework offers significant advantages in terms of adaptability, efficiency, and scalability. The ability to update the model dynamically, based on incoming data instances, ensures continuous adaptation to changing data distributions, maintaining the framework's reliability and relevance in real-time scenarios.

% Looking ahead, future research could focus on further enhancing this approach. Potential improvements include developing advanced concept drift detection methods, refining classifier selection by focusing on positive target and multisource classifiers, and incorporating deep learning techniques to predict future drifts and optimize classifier performance. These advancements could further bolster the framework’s capability to manage complex, evolving data streams in diverse real-world applications.



\backmatter
\include{0_frontmatter/summary-arabic}

% =======================[references]=======================

\bibliographystyle{Latex/StyleBST/IEEEtran} % Defines the bibliography style
\bibliography{11_backmatter/references} % adjust this to fit your BibTex file

\end{document}
