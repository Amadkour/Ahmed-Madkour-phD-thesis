\section{Transfer Learning} 
\label{sec:3_5_transfer_learning}
In recent years, transfer learning has received significant attention due to its growing importance in addressing disparities between source and target domain distributions. Bridging the gap in distribution disparities is vital for optimizing performance in transfer learning, resulting in the development of diverse approaches, which can be broadly categorized into instance re-weighting and feature matching \cite{long2013transfer}. Instance re-weighting methods focus on aligning domain distributions by adjusting the weights of source instances, enabling the reuse of those source instances that closely align with the target domain. Notably, there's a strong emphasis on estimating these instance weights. For example, Huang et al. \cite{long2014transfer} introduced the Kernel Mean Matching (KMM) technique, which calculates weights by minimizing the mean differences between instances from the source and target domains within a Reproducing Kernel Hilbert Space (RKHS). Sugiyama et al. \cite{sun2011two} put forth the Kullback-Leibler Importance Estimation Procedure (KLIEP), utilizing Kullback-Leibler distance as a metric for assessing domain distribution dissimilarity, which includes a model selection step. Building on these instance re-weighting methods, Sun et al. \cite{dai2007boosting} introduced the 2-Stage Weighting Framework for Multisource Domain Adaptation (2SW-MDA) to address challenges in multisource transfer learning. It simultaneously adjusts the weights of source domains and their instances to reduce both marginal and conditional distribution disparities, akin to KMM, while leveraging the smoothness assumption for domain weighting. TrAdaBoost \cite{freund1996experiments}, a variation of the AdaBoost framework \cite{yao2010boosting}, operates by iteratively adjusting the weights of training data. In each iteration, it trains a classifier on a mix of source and target data and uses this classifier to make predictions on the training data. If a source instance is incorrectly predicted, its weight is reduced, diminishing its influence on the classifier. Conversely, the weights of misclassified target instances are increased to amplify their impact. An extension of TrAdaBoost, known as Multisource TrAdaBoost (MsTrAdaBoost) \cite{sun2016return}, is employed to address multisource transfer learning challenges. MsTrAdaBoost combines each source and target dataset, training a separate classifier for each. Subsequently, it selects the classifier with the least error on the target data to update the instance weights.
On a different note, feature matching aims to establish a shared feature representation space between source and target domains, which can be achieved through either symmetric or asymmetric transformations. A typical example of a symmetric transformation is the Transfer Component Analysis (TCA) method by Pan et al. \cite{sun2016return}, employing Maximum Mean Discrepancy (MMD) \cite{zhu2020deep, long2018transferable}  \cite{long2013transfer} to minimize differences in marginal distribution between source and target domains within an RKHS. Expanding on TCA, Joint Distribution Adaptation (JDA) \cite{wang2020transfer} has been introduced to address both marginal and conditional distribution disparities. Recognizing the varying importance of marginal and conditional distribution differences across different problems, Wang et al. \cite{wang2020transfer, fernando2013unsupervised} introduced the Balanced Distribution Adaptation (BDA) approach, which introduces a balancing factor. Subspace Alignment (SA) \cite{pearson1901liii} focuses on aligning domain distributions in a lower-dimensional subspace, selecting crucial eigenvectors using principal component analysis \cite{pearson1901liii} and learning a linear transformation matrix to minimize differences in eigenvectors between domains. In contrast, the Distribution Alignment between Two Subspaces (SDA-TS) \cite{pan2010domain} was proposed to align both bases and distributions. Correlation Alignment (CORAL) \cite{rahman2020correlation}, \cite{long2014transfer}, an asymmetric transformation approach, is designed to align sub-space bases and employs second-order statistics. CORAL uses a learned transformation matrix to project source instances into the target domain. While there is a wide range of feature matching techniques in transfer learning, it is imperative to prevent the negative transfer, which occurs when transferred knowledge hinders the performance of target tasks. One of the reasons for negative transfer is the inclusion of unrelated or detrimental source samples in the target domain. To mitigate this, transfer joint matching (TJM) \cite{zhong2009cross} introduces sparsity regularization in the feature transformation matrix, aligning features and re-weighting instances simultaneously. Zhong et al. \cite{cao2018partial} have also developed strategies to mitigate the impact of unrelated source instances and ensure positive transfer. To tackle the challenges posed by partial transfer learning scenarios, where the source domain contains more classes than the target domain, prior works \cite{cao2019learning, li2020dual, yang2021concept} introduce instance-level re-weighting and class-level re-weighting mechanisms. These mechanisms are employed to reduce the influence of outlier classes from the source domains. Additionally, another approach presented in \cite{shan2018online} utilizes an adversarial neural network to align domain distributions. In this method, lower weights are assigned to the source samples that are deemed distant from the discriminator. This weighting reflects the perception that such instances have weaker relevance to the target domain. Yang et al. \cite{powers2020evaluation} introduce the HE-CDTL approach for Concept Drift Transfer Learning (CDTL). HE-CDTL leverages knowledge from both source domains and historical time steps within the target domain to improve learning performance. Its key advantages include the utilization of the class-wise weighted ensemble for historical knowledge and the implementation of AW-CORAL for knowledge extraction from source domains. The class-wise weighted ensemble empowers individual classes in the current learning process to select historical knowledge independently. AW-CORAL serves to minimize domain disparities between source and target domains while mitigating negative knowledge transfer. Extensive experiments demonstrate that HE-CDTL outperforms baseline methods in addressing transfer learning challenges in the context of concept drift.
Melanie addresses the challenge of non-stationary environments by considering an online scenario where data in both source and target domains are generated. This method employs an online ensemble to learn models from each domain, subsequently combining these models using a weighted-sum approach. The models are trained incrementally, with their weights dynamically adjusted to handle concept drift. Generally, Melanie can be adapted to address CDTL by substituting the online learning ensemble with an ensemble designed for chunk-based concept drift.
