\chapter{State-of-the-art}
\label{cha:3_State-of-the-art}


This chapter presents a comprehensive review of recent advancements in stream classification, focusing on several critical challenges that arise in dynamic data environments. The rapid evolution of data streams introduces complexities such as concept drift, imbalanced multiclass scenarios, class overlap, classifier ensemble selection, emergence of new classes, and the incorporation of transfer learning. This introduction outlines the structure of the chapter and highlights the significance of each topic in advancing stream classification methodologies.
The first section (Setion \ref{sec:3_1_concept_drift}) focuses on concept drift, a phenomenon where the statistical properties of the data change over time. Various methodologies developed for real-time detection and adaptation to concept drift in streaming scenarios are explored. This includes a review of techniques that enable systems to identify shifts in data patterns promptly, ensuring sustained classification accuracy. By analyzing the strengths of these approaches, the necessity for robust drift detection mechanisms in evolving data streams is highlighted. Next, Section \ref{sec:3_2_ensemble} explores classifier ensemble selection in the context of streaming data. As the characteristics of data streams evolve, selecting the most appropriate ensemble of classifiers becomes crucial.Algorithms are presented that are designed to dynamically select the best-performing classifiers based on the current data distribution. This section emphasizes the importance of adaptive ensemble strategies in enhancing classification performance amidst changing data landscapes.
Section \ref{sec:3_3_imbalanced} addresses the challenges posed by imbalanced multiclass scenarios, particularly in the presence of overlapping classes.Specialized oversampling techniques are discussed to counter the effects of class imbalance in streaming data. Insights into effective strategies for balancing the representation of minority classes are provided, laying the groundwork for developing robust frameworks that maintain performance in imbalanced drifted streams.
As new classes emerge in streaming systems, traditional classifiers often struggle to adapt effectively. Section \ref{sec:3_4_emergence} explores methodologies that specifically focus on integrating new classes within existing frameworks. This section highlights innovative approaches that enhance the adaptability of stream classification systems, ensuring they remain effective as new data patterns emerge.
Section \ref{sec:3_5_transfer_learning} examines the role of transfer learning in stream classification. This section discusses how transfer learning techniques leverage knowledge from related tasks to enhance classification performance in dynamic environments. It explores current methodologies that facilitate the transfer of information across different domains, particularly in scenarios where labeled data is limited. The insights from this discussion are crucial for understanding how transfer learning integrates with other strategies in the proposed framework. In Section \ref{sec:3_6_comparsion}, a comparative analysis of recent works is presented, focusing on addressing specific challenges. This analysis evaluates their contributions, highlights limitations, and identifies research gaps that this study seeks to address. Finally, Section \ref{sec:3_7_remartks} concludes this chapter by identifying key challenges and gaps, which inform the research direction for the subsequent chapters.


