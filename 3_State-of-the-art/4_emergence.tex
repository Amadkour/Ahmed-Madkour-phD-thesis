\section{Streams with Emerging New Classes (SENC)}
\label{sec:3_4_emergence}
Existing methods for detecting new class emergence in streaming data include clustering-based approaches like SACCOS \cite{gao2020saccos}, ECSMiner \cite{masud2010classification}, and SAND \cite{haque2016sand}, which require true labels, limiting their practical use. Similarly, SENC-MaS \cite{mu2017streaming} uses matrix sketches but also needs label information for all instances. Tree-based methods, such as SENCForest \cite{mu2017classification} and SEEN \cite{zhu2020semi}, employ anomaly detection but suffer from high false positives and inefficiencies. SENNE \cite{cai2019nearest} improves detection using nearest neighbor ensembles but lacks model retirement mechanisms, causing longer runtimes. The KNNENS \cite{zhang2022knnens} method combines a k-nearest neighbor ensemble with model updates to address new class detection and classification. However, existing methods generally overlook the use of concept drift techniques, highlighting the need for approaches that can handle both concept drift and new class emergence.
