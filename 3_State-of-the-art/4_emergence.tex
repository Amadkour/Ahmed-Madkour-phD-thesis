\section{Streams with Emerging New Classes (SENC)}
\label{sec:3_4_emergence}
Existing approaches have been proposed to detect and handle the emergence of new classes in streaming data. Clustering-based methods, such as SACCOS [89], ECSMiner [90], and SAND [91], employ clustering techniques to identify new class emergence. However, these methods require access to true labels for either parts or all instances, limiting their practical applicability. Similarly, SENC-MaS [92] uses matrix sketches for detecting emerging new classes but assumes the availability of true label information for all instances. In contrast, tree-based methods like SENCForest [93] and SEEN [94] utilize anomaly detection techniques to identify new classes, often with limited or no label information. However, these methods often suffer from high false positive rates and runtime inefficiencies. Another approach, SENNE [95], focuses on exploiting local information using the nearest neighbor ensemble for improved detection performance.  Nevertheless, the absence of an effective model retirement mechanism in SENNE results in longer runtimes than alternative methods. The k-nearest Neighbor Ensemble-based method (KNNENS) [96] method emerges as a promising solution for the challenges of streaming emerging new class problems. By effectively utilizing a k-nearest neighbor-based hypersphere ensemble and incorporating model updates, the KNNENS approach tackles the issues of new class detection and known class classification within a unified framework. It is worth noting that an explicit limitation of existing methods is their lack of utilization of concept drift techniques for detecting emerging new classes and retraining the classification model. This limitation highlights the need for approaches that can effectively handle concept drift while addressing the emergence of new classes in streaming data.
