\section{Related Works Challenges} 
\label{sec:3_7_remartks}

By comparing the literature on ensemble learning for classification tasks, the proposals in this thesis differ from other studies in several ways:

\begin{enumerate}
    
    \item [-] As evident from our literature review on imbalanced streams, most studies have concentrated on generating synthetic samples while ignoring class overlap. \textit{To address this challenge}, we propose an approach to generate non-overlapping classes in imbalanced streams.

    \item [-] Oversampling techniques often perform inefficiently in the presence of concept drift. \textit{To tackle this issue}, we introduce a methodology that selects the oversampling technique based on the current and historical distribution of the stream chunks.
    
    \item [-] Our literature review on non-stationary environments reveals that most works focus on detecting emerging new classes while overlooking distribution changes. \textit{To overcome this challenge}, we propose a combined approach utilizing Dynamic Ensemble Selection (DES) to select the best classifier for each chunk based on stream distribution, k-means clustering, and concept drift to address both emerging new class detection and distribution changes.
    
    \item [-] In our literature review on transfer learning, we observed that most studies focus on homogeneous multisource transfer and neglect heterogeneous multisources in non-stationary environments. \textit{To resolve this issue}, we propose a combined approach integrating Dynamic Ensemble Selection (DES), Concept Drift Transfer Learning (CDTL), eigenvector techniques, and concept drift to address heterogeneous transfer learning in non-stationary environments.

\end{enumerate}
