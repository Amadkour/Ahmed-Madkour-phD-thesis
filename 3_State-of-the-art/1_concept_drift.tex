%%%%%%%%%%%%%%%%%%%%%%%%%%%%%%%%%%%%%%%%%%%%%%%
%
%   Taxonomies in the Traffic Forecasting Field
%
%%%%%%%%%%%%%%%%%%%%%%%%%%%%%%%%%%%%%%%%%%%%%%%
\section{Concept Drift}
\label{sec:3_1_concept_drift}
Concept drift refers to changes in the underlying data distribution over time, which can reduce the accuracy of previously trained machine learning models \cite{baena2006early, madkour2023historical, tan2022information}. Detecting and responding to concept drift is crucial for maintaining model performance. Several detection methods have been proposed to address this challenge. The Drift Detection Method (DDM) \cite{gama2004learning, bifet2009new} uses a statistical test to identify significant error rate increases, signaling concept drift. The Early Drift Detection Method (EDDM) \cite{gama2004learning, adams2023explainable} extends DDM by considering a moving window of recent data. ADWIN \cite{gama2004learning, adams2023explainable} employs a sliding window to monitor statistical differences and adjusts the window size to adapt to drift patterns. The Kolmogorov-Smirnov windowing method (KSWIN) \cite{adams2023explainable} calculates the Kolmogorov-Smirnov distance to detect drift, while Hoeffding's bounds with moving average test (HDDMA) and its variant HDDMW \cite{gama2004learning, bifet2009new} compute bounds for the true mean to detect distribution changes. Lastly, the Page-Hinkley method \cite{page1954continuous} tracks the cumulative sum of errors, detecting drift when the sum exceeds a threshold. These methods enable machine learning models to adapt to evolving data streams, enhancing model performance and robustness.
