%%%%%%%%%%%%%%%%%%%%%%%%%%%%%%%%%%%%%%%%%%%%%%%
%
%   Machine Learning for Traffic Forecasting
%
%%%%%%%%%%%%%%%%%%%%%%%%%%%%%%%%%%%%%%%%%%%%%%%
\section{Imbalanced data Streams}
\label{sec:3_3_imbalanced}
In imbalanced data classification, three primary approaches have been identified \cite{yin2022graph}, with our study focusing on the first category that addresses imbalanced data streams through sampling methods, specifically oversampling \cite{ren2023grouping}. This method generates synthetic instances to balance class distributions \cite{nitesh2002smote, han2005borderline, bunkhumpornpat2009safe, maciejewski2011local}. Class imbalance can occur in binary or multi-class scenarios, and our research specifically focuses on multi-class oversampling techniques.

To address multi-class imbalances, Multi-Label SMOTE (MLSMOTE) \cite{charte2015mlsmote} extends SMOTE to multi-class learning by generating synthetic examples for minority class labels and ensuring their proper assignment. A more recent technique, Multi-Label Synthetic Oversampling based on Local Label Imbalance (MLSOL) \cite{yin2022graph}, improves upon MLSMOTE by targeting local imbalances within multi-class classification. MLSOL uses distinct sampling strategies for each label, offering superior performance in classification accuracy and other metrics. It generates synthetic samples from minority class instances in a restricted neighborhood, improving computational efficiency and reducing overfitting, making it a promising technique that outperforms MLSMOTE in several areas.