%%%%%%%%%%%%%%%%%%%%%%%%%%%%%%%%%%%%%%%%%%%%%%%
%
%   Machine Learning for Traffic Forecasting
%
%%%%%%%%%%%%%%%%%%%%%%%%%%%%%%%%%%%%%%%%%%%%%%%
\section{Imbalanced data Streams}
\label{sec:3_3_imbalanced}
In the context of imbalanced data classification, as previously discussed, researchers have categorized three primary methodologies\cite{yin2022graph}. Our study primarily focuses on the first category, which pertains to addressing imbalanced data streams, particularly through sampling methods, specifically generating synthetic instances to rectify the class imbalance. This process is commonly referred to as oversampling\cite{ren2023grouping} and aims to balance instance quantities across both classes \cite{nitesh2002smote,han2005borderline,bunkhumpornpat2009safe,maciejewski2011local} . It's important to note that class imbalance can manifest in two scenarios: binary imbalanced classes, featuring skewed distribution between two classes, and multi-class imbalanced situations. Our study is specifically centered on multi-class oversampling techniques. In addressing class imbalances within multi-class contexts, Multi-Label SMOTE (MLSMOTE) \cite{charte2015mlsmote} has extended the principles of SMOTE to cater to imbalanced multi-class learning scenarios. MLSMOTE significantly enhances classifier performance by generating synthetic examples for each minority class label. This approach thoughtfully considers neighboring examples in the feature space, ensuring that synthetic examples are appropriately assigned to their respective minority classes. Moreover, a recent advancement known as Multi-Label Synthetic Oversampling based on Local label imbalance (MLSOL) \cite{yin2022graph} has emerged as a promising technique. MLSOL systematically combats local imbalances within the domain of multi-class classification by employing distinct sampling strategies for each label. Empirical research confirms MLSOL's superiority by showcasing its capability to outperform existing methods in terms of classification accuracy and other evaluation metrics, particularly in effectively addressing local imbalance. Importantly, MLSOL offers a potentially effective approach to enhancing the performance of multi-class classification models, surpassing MLSMOTE in several aspects. Notably, MLSOL constructs synthetic samples exclusively from minority class instances within a restricted neighborhood, resulting in a more compact synthetic dataset compared to MLSMOTE. This attribute bears potential advantages for computational efficiency and helps mitigate concerns related to overfitting.