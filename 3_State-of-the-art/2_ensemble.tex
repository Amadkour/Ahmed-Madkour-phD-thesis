%%%%%%%%%%%%%%%%%%%%%%%%%%%%%%%%%%%%%%%%%%%%%%%
%
%   General-purpose Automated Machine Learning
%
%%%%%%%%%%%%%%%%%%%%%%%%%%%%%%%%%%%%%%%%%%%%%%%
\section{Classifier Ensemble Selection}

\label{sec:3_2_ensemble}
This study focuses on the overproduce-and-select approach for classifier ensemble selection methods \cite{cruz2017meta}\cite{kuncheva2000clustering}\cite{jackowski2014improved}. The primary objective of classifier ensemble selection is to identify the optimal subset of classifiers from a larger ensemble, considering various criteria such as performance measures, diversity metrics, meta-learning techniques, and performance estimation approaches. This selection process aims to reduce computational complexity, enhance efficiency, and improve overall ensemble performance, making it highly valuable for real-world applications. By carefully selecting a smaller subset of classifiers, ensemble selection strikes a balance between accuracy and computational resources, adapting to the evolving nature of the data stream. This approach leverages the strengths of different classifiers and adjusts the ensemble composition to handle changing conditions effectively. The goal is to enhance the accuracy, robustness, and overall performance of classification models in dynamic and challenging scenarios. There are two main approaches to the selection process: static and dynamic selection. Static selection assigns classifiers to specific partitions of the feature space, while dynamic selection chooses a classifier specifically for each unknown data sample based on its local competencies. Dynamic Ensemble Selection (DES) is a widely recognized approach that selects the best classifiers for each test instance, considering their competence within the local region of competence. The Randomized Reference Classifier proposed by Woloszynski and Kurzynski \cite{woloszynski2011probabilistic} stands out among various approaches. This classifier introduces randomness through beta distribution, enhancing adaptability and robustness. By considering the stochastic nature of class supports, the Randomized Reference Classifier can potentially improve classification performance in concept drift scenarios. However, it is important to note that employing diversity measures during the classifier selection process, as demonstrated by Lysiak \cite{lysiak2014optimal}, may lead to smaller ensembles but does not necessarily enhance classification accuracy. Overall, the overproduce-and-select approach for classifier ensemble selection methods offers a comprehensive framework for addressing the challenges associated with concept drift. By dynamically adapting the ensemble composition and leveraging the competencies of individual classifiers, this approach aims to improve classification performance, efficiency, and adaptability in dynamic and challenging scenarios.